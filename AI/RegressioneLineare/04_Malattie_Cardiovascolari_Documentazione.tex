\documentclass[12pt]{article}
\usepackage[italian]{babel}
\usepackage[utf8]{inputenc}
\usepackage{amsmath}
\usepackage{graphicx}
\usepackage{listings}
\usepackage{xcolor}
\usepackage{tcolorbox}
\usepackage{hyperref}

\definecolor{codegray}{gray}{0.95}
\definecolor{keywordcolor}{blue!60!black}
\definecolor{stringcolor}{red!70!black}
\definecolor{commentcolor}{green!50!black}

\lstset{
	backgroundcolor=\color{codegray},
	basicstyle=\ttfamily\small,
	keywordstyle=\color{keywordcolor}\bfseries,
	stringstyle=\color{stringcolor},
	commentstyle=\color{commentcolor},
	frame=single,
	breaklines=true,
	language=Python,
	numbers=left,
	numberstyle=\tiny,
	stepnumber=1,
	numbersep=5pt,
	captionpos=b
}

\title{Il mio primo intelligente: un dottore che impara dai dati}
\author{Prof. Fedeli Massimo}
\date{}

\begin{document}
	
	\maketitle
	
	\section*{Introduzione}
	Immagina di avere un foglio Excel con le cartelle cliniche di 300 pazienti: età, pressione, colesterolo, ecc.  
	Il nostro obiettivo è \textbf{insegnare al computer a riconoscere} se un nuovo paziente è sano o a rischio di infarto, \textbf{senza scrivergli regole}, ma facendogli \textbf{imparare} dai casi passati.
	
	\section*{1. I dati: il ``libro'' delle cartelle cliniche}
	Scarichiamo un file dal web che contiene 14 colonne:
	\begin{center}
		\begin{tabular}{ll}
			age    & età (anni) \\
			sex    & sesso (0 = donna, 1 = uomo) \\
			cp     & tipo di dolore al petto \\
			trestbps & pressione a riposo \\
			chol   & colesterolo \\
			fbs    & glicemia a digiuno \\
			...    & ... \\
			target & grado di malattia (0 = sano, 1-4 = malato)
		\end{tabular}
	\end{center}
	
	Alcune celle hanno il simbolo ``?'' al posto del numero: sono \textbf{dati mancanti}.  
	Il computer non può lavorare con i buchi, quindi \textbf{cancelliamo le righe incomplete}:
	\begin{lstlisting}
		df = df.dropna()
	\end{lstlisting}
	
	\section*{2. Trasformiamo il problema in ``sì o no''}
	Nel file originale ci sono 5 livelli di malattia. Noi vogliamo solo due risposte:
	\begin{itemize}
		\item 0 = \textbf{sano}
		\item 1 = \textbf{malato}
	\end{itemize}
	Convertiamo quindi tutti i valori 1-4 in 1:
	\begin{lstlisting}
		df['target'] = (df['target'] > 0).astype(int)
	\end{lstlisting}
	
	\section*{3. Dividiamo la classe e gli esercizi}
	\begin{itemize}
		\item \textbf{X} = tutte le colonne tranne ``target'' (le \emph{feature})
		\item \textbf{y} = solo la colonna ``target'' (la \emph{classe})
	\end{itemize}
	Ora spacchiamo il foglio in due:
	\begin{itemize}
		\item \textbf{Training set} (80 \%) = foglio su cui il computer \textbf{studia}
		\item \textbf{Test set} (20 \%) = foglio su cui \textbf{interroghiamo} il computer
	\end{itemize}
	\begin{lstlisting}
		X_train, X_test, y_train, y_test = train_test_split(
		X, y, test_size=0.2, random_state=42, stratify=y)
	\end{lstlisting}
	L'opzione \texttt{stratify} fa sì che in entrambi i fogli ci sia la stessa percentuale di sani e malati.
	
	\section*{4. Standardizzare: mettere tutti ``sulla stessa riga''}
	Le colonne hanno unità diverse: età (0-100), colesterolo (100-600), ecc.  
	Per non far vincere il ``più grande'', \textbf{centriamo e rimpiccioliamo} tutti i numeri:
	\begin{itemize}
		\item media = 0
		\item deviazione standard = 1
	\end{itemize}
	\begin{lstlisting}
		scaler = StandardScaler()
		X_train = scaler.fit_transform(X_train)  # impara la media e la scala
		X_test  = scaler.transform(X_test)     # applica la stessa trasformazione
	\end{lstlisting}
	Importante: impariamo la media e la scala \textbf{solo sul training}, altrimenti ``bariamo''.
	
	\section*{5. L'apprendimento: il computer fa i compiti}
	Usiamo un algoritmo chiamato \textbf{Regressione Logistica}.  
	Non è una retta, ma una \textbf{curva a S} che restituisce una \textbf{probabilità}:
	\begin{center}
		$P(\text{malato}) = \sigma(w_1 x_1 + w_2 x_2 + \dots + w_n x_n)$
	\end{center}
	$\sigma$ è la funzione logistica; $w_i$ sono i ``pesi'' che il computer aggiusta per avvicinarsi alla risposta corretta.
	\begin{lstlisting}
		model = LogisticRegression(max_iter=1000)
		model.fit(X_train, y_train)
	\end{lstlisting}
	
	\section*{6. L'interrogazione: quanto è bravo?}
	Facciamo rispondere il computer sul foglio che \textbf{non ha mai visto}:
	\begin{lstlisting}
		y_pred = model.predict(X_test)
		accuracy = accuracy_score(y_test, y_pred)
	\end{lstlisting}
	L'accuratezza è la percentuale di risposte esatte.  
	Un valore tipico con questi dati è $\approx 85$\,\%.  
	Non è perfetto, ma è \textbf{meglio di un lancio di moneta}!
	
	\section*{7. Prova con un nuovo paziente}
	Creiamo un paziente immaginario:
	\begin{lstlisting}
		paziente = {'age':60, 'sex':1, 'cp':0, 'trestbps':150,
			'chol':250, 'fbs':1, 'restecg':1, 'thalach':130,
			'exang':1, 'oldpeak':2.3, 'slope':1, 'ca':1, 'thal':3}
	\end{lstlisting}
	Lo standardizziamo con la stessa ``scheda'' usata prima e chiediamo la previsione:
	\begin{lstlisting}
		paziente_scaled = scaler.transform(pd.DataFrame([paziente]))
		rischio = model.predict_proba(paziente_scaled)[0][1]
	\end{lstlisting}
	Il computer dice: ``C'è un 73\,\% di probabilità che questo paziente sia a rischio.''
	
	\section*{8. Cosa abbiamo imparato?}
	\begin{itemize}
		\item Il computer \textbf{non} ha bisogno di regole scritte a mano: impara \emph{da solo} dai dati.
		\item I dati vanno \textbf{puliti} (no buchi) e \textbf{preparati} (scale simili).
		\item \textbf{Mai valutare} sullo stesso foglio su cui si è studiato: si ``barerebbe''.
		\item L'accuratezza è una \textbf{prima misura}, ma in medicina servono anche altri controlli (sensibilità, specificità, ecc.).
	\end{itemize}
	
	\section*{9. Provaci tu!}
	\begin{enumerate}
		\item Cambia il paziente di esempio: più giovane, donna, colesterolo basso...  
		\item Osserva come cambia la probabilità.
		\item Prova a modificare \texttt{test\_size} o \texttt{random\_state}: l'accuratezza cambia?
	\end{enumerate}
	
	\bigskip
	\begin{tcolorbox}[colback=blue!5!white,colframe=blue!75!black,title=Link utili]
		\centering
		\url{https://archive.ics.uci.edu/ml/datasets/heart+Disease} (dataset)\\
		\url{https://scikit-learn.org} (documentazione di scikit-learn)
	\end{tcolorbox}
	
\end{document}