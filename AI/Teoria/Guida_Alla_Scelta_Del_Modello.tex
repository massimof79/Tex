\documentclass[a4paper,12pt]{article}
\usepackage[italian]{babel}
\usepackage[utf8]{inputenc}
\usepackage{geometry}
\usepackage{hyperref}
\geometry{margin=2.5cm}

\title{Guida alla scelta di un modello di Machine Learning}
\author{Prof. Fedeli Massimo - Tutti i diritti riservati}
\date{}

\begin{document}
	

	\maketitle
		\newpage

		
		\section{Introduzione}
		
		Nel Machine Learning una delle domande più importanti è: \textit{quale modello devo usare per il mio problema?}
		
		Non esiste un algoritmo perfetto per ogni situazione. La scelta dipende da:
		\begin{itemize}
			\item il tipo di problema
			\item la quantità di dati
			\item la complessità delle relazioni nei dati
			\item la necessità di spiegare le decisioni del modello
		\end{itemize}
		
		In questa guida vediamo \textbf{quali modelli usare}, \textbf{perché usarli}, \textbf{quali librerie Python permettono di implementarli} e un \textbf{esempio concreto} per ciascuno.
		
		\section{Problemi di classificazione}
		
		Il modello deve assegnare un elemento a una categoria.
		
		\subsection{Regressione Logistica}
		
		\textbf{Perché usarla:} è semplice, veloce e funziona bene quando le classi sono separabili in modo abbastanza regolare.  
		\textbf{Quando usarla:} problemi con due classi e relazioni non troppo complesse.  
		\textbf{Libreria Python:} \texttt{sklearn.linear\_model.LogisticRegression}  
		
		\textbf{Esempio concreto:} prevedere se uno studente sarà \textit{promosso} o \textit{non promosso} in base a ore di studio, assenze e media dei voti.
		
		\subsection{Albero Decisionale}
		
		\textbf{Perché usarlo:} prende decisioni tramite regole semplici del tipo “se... allora...”, quindi è facile da interpretare.  
		\textbf{Quando usarlo:} quando vuoi un modello comprensibile e i dati hanno relazioni non lineari.  
		\textbf{Libreria Python:} \texttt{sklearn.tree.DecisionTreeClassifier}  
		
		\textbf{Esempio concreto:} stabilire se una richiesta di prestito va accettata o rifiutata in base a reddito, età e storico dei pagamenti.
		
		\subsection{Random Forest}
		
		\textbf{Perché usarla:} combina molti alberi decisionali, riducendo errori e migliorando l’accuratezza.  
		\textbf{Quando usarla:} problemi con relazioni complesse e un buon numero di dati.  
		\textbf{Libreria Python:} \texttt{sklearn.ensemble.RandomForestClassifier}  
		
		\textbf{Esempio concreto:} riconoscere il tipo di pianta (specie) a partire da misure di foglie e fiori.
		
		
		\subsection{Reti Neurali}
		
		\textbf{Perché usarle:} riescono a modellare relazioni molto complesse.  
		\textbf{Quando usarle:} immagini, audio, testo o dataset molto grandi.  
		\textbf{Librerie Python:} \texttt{TensorFlow/Keras}, \texttt{PyTorch}  
		
		\textbf{Esempio concreto:} riconoscere se in una foto è presente un cane o un gatto.
		
		\section{Problemi di regressione}
		
		Il modello deve prevedere un valore numerico continuo.
		
		\subsection{Regressione Lineare}
		
		\textbf{Perché usarla:} è il modello più semplice e spesso sorprendentemente efficace.  
		\textbf{Quando usarla:} relazioni quasi lineari tra variabili.  
		\textbf{Libreria Python:} \texttt{sklearn.linear\_model.LinearRegression}  
		
		\textbf{Esempio concreto:} prevedere il prezzo di una casa in base a metri quadrati, numero di stanze e zona.
		
		\subsection{Decision Tree Regressor}
		
		\textbf{Perché usarlo:} cattura relazioni non lineari senza trasformazioni complicate.  
		\textbf{Quando usarlo:} dati con soglie e comportamenti “a scalini”.  
		\textbf{Libreria Python:} \texttt{sklearn.tree.DecisionTreeRegressor}  
		
		\textbf{Esempio concreto:} stimare il tempo di consegna di un pacco in base a distanza, traffico e tipo di spedizione.
		
		\subsection{Random Forest Regressor}
		
		\textbf{Perché usarlo:} più stabile e preciso di un singolo albero.  
		\textbf{Libreria Python:} \texttt{sklearn.ensemble.RandomForestRegressor}  
		
		\textbf{Esempio concreto:} prevedere il consumo elettrico di un edificio in base a temperatura esterna, numero di persone presenti e orario.
		
		\subsection{Gradient Boosting Regressor}
		
		\textbf{Perché usarlo:} spesso tra i modelli più accurati per dati tabellari.  
		\textbf{Librerie Python:} \texttt{xgboost.XGBRegressor}, \texttt{lightgbm.LGBMRegressor}  
		
		\textbf{Esempio concreto:} prevedere il punteggio finale di uno studente usando molte variabili (voti, assenze, partecipazione, studio a casa).
		
		\subsection{Reti Neurali per regressione}
		
		\textbf{Perché usarle:} utili con grandi quantità di dati e relazioni molto complesse.  
		\textbf{Librerie Python:} \texttt{TensorFlow/Keras}, \texttt{PyTorch}  
		
		\textbf{Esempio concreto:} prevedere l’andamento futuro della temperatura usando dati meteorologici storici molto dettagliati.
		
		\section{Problemi di clustering}
		
		\subsection{k-Means}
		
		\textbf{Perché usarlo:} semplice e veloce.  
		\textbf{Quando usarlo:} gruppi abbastanza compatti e simili tra loro.  
		\textbf{Libreria Python:} \texttt{sklearn.cluster.KMeans}  
		
		\textbf{Esempio concreto:} suddividere clienti di un negozio in gruppi in base a quanto spendono e quanto spesso acquistano.
		
		\subsection{DBSCAN}
		
		\textbf{Perché usarlo:} trova gruppi di forma irregolare e individua dati anomali.  
		\textbf{Quando usarlo:} dati con densità variabile.  
		\textbf{Libreria Python:} \texttt{sklearn.cluster.DBSCAN}  
		
		\textbf{Esempio concreto:} individuare gruppi di sensori che registrano valori simili e isolare quelli che funzionano in modo anomalo.
		
		\section{Strategia pratica consigliata}
		
		\begin{enumerate}
			\item Parti sempre da un modello semplice con \texttt{scikit-learn}.
			\item Valuta le prestazioni.
			\item Passa a Random Forest se serve maggiore accuratezza.
			\item Usa reti neurali solo quando il problema è davvero complesso o i dati sono moltissimi.
		\end{enumerate}
		
	\end{document}