\documentclass[10pt]{beamer}

\usepackage[utf8]{inputenc}
\usepackage[T1]{fontenc}
\usepackage[italian]{babel}
\usepackage{amsmath}
\usepackage{graphicx}

% Tema
\usetheme{Madrid}
\usecolortheme{seahorse}

\title{Classificazione del Superamento dell'Esame}
\subtitle{Un esempio con Albero Decisionale}
\author{Prof. Fedeli Massimo}
\date{ }

\begin{document}
	
	%------------------------------------------------
	\begin{frame}
		\titlepage
	\end{frame}
	
	%------------------------------------------------
	\begin{frame}{Obiettivo del problema}
		L’obiettivo è costruire un modello di \textbf{Machine Learning supervisionato} capace di prevedere l’esito di uno studente:
		\begin{itemize}
			\item \textbf{Promosso}
			\item \textbf{Bocciato}
		\end{itemize}
		
		La previsione si basa su tre variabili osservabili:
		\begin{itemize}
			\item Ore di studio settimanali
			\item Percentuale di presenza alle lezioni
			\item Voto medio annuale
		\end{itemize}
	\end{frame}
	
	%------------------------------------------------
	\begin{frame}{Rappresentazione dei dati}
		Ogni studente è rappresentato come un vettore numerico:
		\[
		X = [\text{ore\_studio}, \text{presenza}, \text{voto\_medio}]
		\]
		
		L’etichetta associata è binaria:
		\begin{itemize}
			\item 1 = Promosso
			\item 0 = Bocciato
		\end{itemize}
		
		Il dataset contiene 20 esempi, ciascuno con caratteristiche e risultato noto.
	\end{frame}
	
	%------------------------------------------------
	\begin{frame}{Tipo di apprendimento}
		Il problema rientra nell’\textbf{apprendimento supervisionato}:
		\begin{itemize}
			\item gli input sono noti
			\item l’output corretto è fornito durante l’addestramento
		\end{itemize}
		
		Il compito specifico è una \textbf{classificazione binaria}.
	\end{frame}
	
	%------------------------------------------------
	\begin{frame}{Suddivisione del dataset}
		I dati vengono divisi in:
		\begin{itemize}
			\item \textbf{Training set (80\%)}: utilizzato per addestrare il modello
			\item \textbf{Test set (20\%)}: utilizzato per valutare le prestazioni
		\end{itemize}
		
		Questa separazione consente di stimare la capacità di generalizzazione del modello.
	\end{frame}
	
	%------------------------------------------------
	\begin{frame}{Scelta del modello}
		Il modello utilizzato è un \textbf{Albero Decisionale}.
		
		Motivazioni principali:
		\begin{itemize}
			\item struttura basata su regole \textit{if--then}
			\item elevata interpretabilità
			\item adatto a dataset piccoli
			\item coerente con il ragionamento umano e didattico
		\end{itemize}
	\end{frame}
	
	%------------------------------------------------
	\begin{frame}{Come funziona un albero decisionale}
		Un albero decisionale:
		\begin{itemize}
			\item seleziona la variabile più informativa a ogni nodo
			\item suddivide i dati tramite soglie
			\item riduce progressivamente l’impurità dei gruppi
		\end{itemize}
		
		Ogni percorso dalla radice a una foglia rappresenta una regola decisionale.
	\end{frame}
	
	%------------------------------------------------
	\begin{frame}{Profondità massima dell’albero}
		Nel modello è impostata una \textbf{profondità massima pari a 3}.
		
		Ciò significa:
		\begin{itemize}
			\item al massimo tre decisioni consecutive
			\item modello semplice e leggibile
			\item controllo del rischio di overfitting
		\end{itemize}
	\end{frame}
	
	%------------------------------------------------
	\begin{frame}{Perché profondità 3}
		Una profondità limitata:
		\begin{itemize}
			\item evita di adattarsi eccessivamente ai dati di training
			\item migliora la generalizzazione
			\item mantiene il modello coerente con il contesto educativo
		\end{itemize}
		
		Tre livelli sono sufficienti per catturare relazioni significative senza perdere interpretabilità.
	\end{frame}
	
	%------------------------------------------------
	\begin{frame}{Valutazione del modello}
		Il modello viene valutato confrontando:
		\begin{itemize}
			\item risultato reale
			\item risultato previsto
		\end{itemize}
		
		La metrica principale utilizzata è:
		\begin{itemize}
			\item \textbf{accuratezza} (percentuale di previsioni corrette)
		\end{itemize}
	\end{frame}
	
	%------------------------------------------------
	\begin{frame}{Previsione su nuovi dati}
		Una volta addestrato, il modello può essere utilizzato per:
		\begin{itemize}
			\item stimare l’esito di nuovi studenti
			\item supportare decisioni didattiche
			\item simulare scenari ipotetici
		\end{itemize}
		
		Il processo decisionale rimane sempre spiegabile e tracciabile.
	\end{frame}
	
	%------------------------------------------------
	\begin{frame}{Conclusioni}
		L’albero decisionale rappresenta una scelta:
		\begin{itemize}
			\item tecnicamente adeguata
			\item didatticamente efficace
			\item facilmente interpretabile
		\end{itemize}
		
		La limitazione della profondità garantisce equilibrio tra semplicità, accuratezza e robustezza del modello.
	\end{frame}
	
\end{document}
