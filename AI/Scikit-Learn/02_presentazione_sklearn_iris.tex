\documentclass[aspectratio=169]{beamer}
\usepackage[utf8]{inputenc}
\usepackage[italian]{babel}
\usepackage{listings}
\usepackage{xcolor}
\usepackage{tikz}
\usepackage{graphicx}
\usepackage{amsmath}

% Tema
\usetheme{Madrid}
\usecolortheme{default}

% Configurazione listing Python
\lstset{
	language=Python,
	basicstyle=\ttfamily\small,
	keywordstyle=\color{blue}\bfseries,
	commentstyle=\color{gray}\itshape,
	stringstyle=\color{red},
	showstringspaces=false,
	breaklines=true,
	frame=single,
	numbers=left,
	numberstyle=\tiny\color{gray},
	backgroundcolor=\color{gray!10}
}

% Informazioni
\title{Machine Learning con Scikit-Learn}
\subtitle{Classificazione dei Fiori Iris}
\author{Prof. Massimo Fedeli}
\institute{IIS Fermi Sacconi Ceci - Ascoli Piceno}
\date{\today}

\begin{document}
	
	% Slide 1: Titolo
	\begin{frame}
		\titlepage
	\end{frame}
	
	% Slide 2: Indice
	\begin{frame}{Contenuti}
		\tableofcontents
	\end{frame}
	
	% Slide 3: Cos'è il Machine Learning
	\section{Introduzione al Machine Learning}
	\begin{frame}{Cos'è il Machine Learning?}
		\begin{block}{Definizione}
			Il \textbf{Machine Learning} (apprendimento automatico) è la capacità di un computer di \textit{imparare} dai dati senza essere esplicitamente programmato.
		\end{block}
		
		\vspace{0.5cm}
		
		\begin{columns}
			\column{0.5\textwidth}
			\textbf{Programmazione tradizionale:}
			\begin{itemize}
				\item Regole esplicite
				\item If-then-else
				\item Logica predefinita
			\end{itemize}
			
			\column{0.5\textwidth}
			\textbf{Machine Learning:}
			\begin{itemize}
				\item Impara dai dati
				\item Trova pattern
				\item Fa predizioni
			\end{itemize}
		\end{columns}
	\end{frame}
	
	% Slide 4: Scikit-learn
	\begin{frame}[fragile]{Scikit-Learn: La Libreria per ML in Python}
		\begin{block}{Cos'è Scikit-Learn?}
			Libreria Python per Machine Learning, semplice ed efficiente per analisi dati e data mining.
		\end{block}
		
		\vspace{0.3cm}
		
		\textbf{Caratteristiche principali:}
		\begin{itemize}
			\item Open source e gratuita
			\item Facile da usare
			\item Integrata con NumPy e Pandas
			\item Molti algoritmi già implementati
		\end{itemize}
		
		\vspace{0.3cm}
		
		\textbf{Installazione:}
		\begin{lstlisting}[numbers=none]
			pip install scikit-learn
		\end{lstlisting}
	\end{frame}
	
	% Slide 5: Il Dataset Iris
	\section{Il Dataset Iris}
	\begin{frame}{Il Dataset Iris}
		\begin{columns}
			\column{0.6\textwidth}
			\textbf{Dataset storico} creato da Ronald Fisher nel 1936.
			
			\vspace{0.3cm}
			
			\textbf{Contenuto:}
			\begin{itemize}
				\item 150 fiori di iris
				\item 3 specie diverse (50 per specie)
				\item 4 misurazioni per fiore
			\end{itemize}
			
			\vspace{0.3cm}
			
			\textbf{Le 3 specie:}
			\begin{enumerate}
				\item Iris Setosa
				\item Iris Versicolor
				\item Iris Virginica
			\end{enumerate}
			
			\column{0.4\textwidth}
			\begin{center}
				\begin{tikzpicture}[scale=0.8]
					% Fiore stilizzato
					\draw[fill=purple!30] (0,0) ellipse (0.3 and 0.5);
					\draw[fill=purple!30] (0.4,0.2) ellipse (0.3 and 0.5);
					\draw[fill=purple!30] (-0.4,0.2) ellipse (0.3 and 0.5);
					\draw[fill=yellow!50] (0,0.3) circle (0.2);
					\draw[green!50!black, thick] (0,-0.5) -- (0,-2);
					\draw[green!50!black, thick] (0,-1.2) -- (-0.5,-1.7);
					\draw[green!50!black, thick] (0,-1.5) -- (0.5,-2);
				\end{tikzpicture}
			\end{center}
		\end{columns}
	\end{frame}
	
	% Slide 6: Features del Dataset
	\begin{frame}{Le 4 Caratteristiche (Features)}
		\begin{center}
			\begin{tikzpicture}
				\node[draw, rectangle, fill=blue!20, minimum width=8cm, minimum height=1cm] at (0,3) 
				{\textbf{Features (X)} - Cosa misuriamo};
				
				\node[draw, rectangle, fill=green!20, minimum width=3.5cm] at (-2.5,1.5) 
				{Lunghezza Sepalo};
				\node[draw, rectangle, fill=green!20, minimum width=3.5cm] at (2.5,1.5) 
				{Larghezza Sepalo};
				\node[draw, rectangle, fill=orange!20, minimum width=3.5cm] at (-2.5,0.5) 
				{Lunghezza Petalo};
				\node[draw, rectangle, fill=orange!20, minimum width=3.5cm] at (2.5,0.5) 
				{Larghezza Petalo};
				
				\node[draw, rectangle, fill=red!20, minimum width=8cm, minimum height=1cm] at (0,-0.5) 
				{\textbf{Target (y)} - Cosa prevediamo: la specie (0, 1, o 2)};
			\end{tikzpicture}
		\end{center}
		
		\vspace{0.5cm}
		
		\textbf{Esempio di un fiore:}
		\begin{center}
			\begin{tabular}{|c|c|c|c||c|}
				\hline
				Lung. Sepalo & Larg. Sepalo & Lung. Petalo & Larg. Petalo & Specie \\
				\hline
				5.1 cm & 3.5 cm & 1.4 cm & 0.2 cm & Setosa (0) \\
				\hline
			\end{tabular}
		\end{center}
	\end{frame}
	
	% Slide 7: Il Processo di ML
	\section{Il Processo di Machine Learning}
	\begin{frame}{Il Processo di Machine Learning}
		\begin{center}
			\begin{tikzpicture}[node distance=2cm, auto]
				\node[draw, rectangle, fill=blue!20] (data) {1. Carica Dati};
				\node[draw, rectangle, fill=green!20, right of=data, node distance=3cm] (split) {2. Dividi Train/Test};
				\node[draw, rectangle, fill=orange!20, below of=split] (train) {3. Addestra Modello};
				\node[draw, rectangle, fill=purple!20, left of=train, node distance=3cm] (predict) {4. Fai Predizioni};
				\node[draw, rectangle, fill=red!20, below of=predict] (evaluate) {5. Valuta Risultati};
				
				\draw[->, thick] (data) -- (split);
				\draw[->, thick] (split) -- (train);
				\draw[->, thick] (train) -- (predict);
				\draw[->, thick] (predict) -- (evaluate);
			\end{tikzpicture}
		\end{center}
		
		\vspace{0.5cm}
		
		\begin{alertblock}{Obiettivo}
			Creare un modello che impari a classificare i fiori in base alle loro misure.
		\end{alertblock}
	\end{frame}
	
	% Slide 8: Train/Test Split
	\begin{frame}{Dividere i Dati: Train e Test}
		\begin{block}{Perché dividere i dati?}
			Per verificare se il modello funziona su dati \textit{mai visti prima}.
		\end{block}
		
		\vspace{0.3cm}
		
		\begin{center}
			\begin{tikzpicture}
				% Dataset completo
				\draw[fill=blue!30] (0,2) rectangle (8,3);
				\node at (4,2.5) {\textbf{150 fiori (100\%)}};
				
				% Training set
				\draw[fill=green!30] (0,0.5) rectangle (6.4,1.5);
				\node at (3.2,1) {\textbf{Training Set (80\%)} - 120 fiori};
				
				% Test set
				\draw[fill=orange!30] (6.4,0.5) rectangle (8,1.5);
				\node at (7.2,1) {\scriptsize\textbf{Test} (20\%)};
				\node at (7.2,0.7) {\scriptsize 30 fiori};
				
				\draw[->, thick] (4,2) -- (3.2,1.5);
				\draw[->, thick] (4,2) -- (7.2,1.5);
			\end{tikzpicture}
		\end{center}
		
		\vspace{0.3cm}
		
		\textbf{Training Set:} il modello "studia" qui \\
		\textbf{Test Set:} il modello viene "interrogato" qui
	\end{frame}
	
	% Slide 9: Codice - Importazioni
	\section{Implementazione in Python}
	\begin{frame}[fragile]{Passo 1: Importare le Librerie}
		\begin{lstlisting}
			from sklearn.datasets import load_iris
			from sklearn.model_selection import train_test_split
			from sklearn.tree import DecisionTreeClassifier
			from sklearn.metrics import accuracy_score
		\end{lstlisting}
		
		\vspace{0.5cm}
		
		\begin{itemize}
			\item \texttt{load\_iris}: carica il dataset
			\item \texttt{train\_test\_split}: divide train/test
			\item \texttt{DecisionTreeClassifier}: il modello
			\item \texttt{accuracy\_score}: valuta l'accuratezza
		\end{itemize}
	\end{frame}
	
	% Slide 10: Codice - Caricamento dati
	\begin{frame}[fragile]{Passo 2: Caricare i Dati}
		\begin{lstlisting}
			# Carica il dataset Iris
			iris = load_iris()
			X = iris.data      # Features (150 righe x 4 colonne)
			y = iris.target    # Target (150 valori: 0, 1, o 2)
		\end{lstlisting}
		
		\vspace{0.5cm}
		
		\textbf{X contiene le misure:}
		\begin{lstlisting}[numbers=none]
			[[5.1, 3.5, 1.4, 0.2],
			[4.9, 3.0, 1.4, 0.2],
			...]
		\end{lstlisting}
		
		\textbf{y contiene le specie:}
		\begin{lstlisting}[numbers=none]
			[0, 0, 0, ..., 1, 1, 1, ..., 2, 2, 2]
		\end{lstlisting}
	\end{frame}
	
	% Slide 11: Codice - Split e Training
	\begin{frame}[fragile]{Passo 3 e 4: Split e Addestramento}
		\begin{lstlisting}
			# Dividi i dati (80% training, 20% test)
			X_train, X_test, y_train, y_test = train_test_split(
			X, y, test_size=0.2, random_state=42
			)
			
			# Crea il modello (Decision Tree)
			model = DecisionTreeClassifier(random_state=42)
			
			# Addestra il modello
			model.fit(X_train, y_train)
		\end{lstlisting}
		
		\vspace{0.3cm}
		
		\begin{block}{Cosa fa \texttt{fit()}?}
			Il modello "studia" i dati di training per imparare la relazione tra features (X) e specie (y).
		\end{block}
	\end{frame}
	
	% Slide 12: Codice - Predizioni
	\begin{frame}[fragile]{Passo 5: Fare Predizioni}
		\begin{lstlisting}
			# Fai predizioni sul test set
			y_pred = model.predict(X_test)
		\end{lstlisting}
		
		\vspace{0.5cm}
		
		\textbf{Cosa succede?}
		\begin{enumerate}
			\item Il modello riceve le features di 30 fiori (\texttt{X\_test})
			\item Per ogni fiore, predice la specie
			\item Restituisce un array con 30 predizioni
		\end{enumerate}
		
		\vspace{0.3cm}
		
		\begin{exampleblock}{Esempio}
			Input: \texttt{[[5.1, 3.5, 1.4, 0.2]]} \\
			Output: \texttt{[0]} $\rightarrow$ Setosa
		\end{exampleblock}
	\end{frame}
	
	% Slide 13: Codice - Valutazione
	\begin{frame}[fragile]{Passo 6: Valutare il Modello}
		\begin{lstlisting}
			# Calcola l'accuratezza
			accuracy = accuracy_score(y_test, y_pred)
			print(f"Accuratezza: {accuracy:.2%}")
		\end{lstlisting}
		
		\vspace{0.5cm}
		
		\begin{block}{Cos'è l'accuratezza?}
			\[
			\text{Accuratezza} = \frac{\text{Predizioni Corrette}}{\text{Predizioni Totali}} \times 100
			\]
		\end{block}
		
		\vspace{0.3cm}
		
		\textbf{Esempio:}
		\begin{itemize}
			\item Test set: 30 fiori
			\item Predizioni corrette: 30
			\item Accuratezza: $\frac{30}{30} = 100\%$
		\end{itemize}
	\end{frame}
	
	% Slide 14: Random State
	\begin{frame}[fragile]{Il Parametro \texttt{random\_state}}
		\begin{block}{Cos'è?}
			Un numero "seme" che controlla la casualità per rendere i risultati \textbf{riproducibili}.
		\end{block}
		
		\vspace{0.3cm}
		
		\textbf{Senza random\_state:}
		\begin{lstlisting}[numbers=none]
			train_test_split(X, y, test_size=0.2)
			
			# Risultati diversi ogni volta!
			# Esecuzione 1: accuratezza 96%
			# Esecuzione 2: accuratezza 100%
			# Esecuzione 3: accuratezza 93%
		\end{lstlisting}
		
		\textbf{Con random\_state=42:}
		\begin{lstlisting}[numbers=none]
			train_test_split(X, y, test_size=0.2, random_state=42)
			
			# Sempre lo stesso risultato!
			# Esecuzione 1: accuratezza 100%
			# Esecuzione 2: accuratezza 100%
			# Esecuzione 3: accuratezza 100%
		\end{lstlisting}
		
		\begin{alertblock}{Nota}
			Il valore 42 non è speciale, puoi usare qualsiasi numero (0, 1, 100, ecc.)
		\end{alertblock}
	\end{frame}
	
	% Slide 15: Conclusioni
	\section{Conclusioni}
	\begin{frame}{Riepilogo e Prossimi Passi}
		\textbf{Cosa abbiamo imparato:}
		\begin{itemize}
			\item Cos'è il Machine Learning
			\item Come usare scikit-learn
			\item Il processo completo: dati $\rightarrow$ training $\rightarrow$ predizioni
			\item Come valutare un modello
		\end{itemize}
		
		\vspace{0.5cm}
		
		\textbf{Esercizi per casa:}
		\begin{enumerate}
			\item Prova altri modelli: \texttt{KNeighborsClassifier}, \texttt{LogisticRegression}
			\item Cambia il \texttt{random\_state} e osserva i risultati
			\item Visualizza la matrice di confusione
			\item Prova con altri dataset di scikit-learn
		\end{enumerate}
		
		\vspace{0.5cm}
		
		\begin{center}
			\Large{\textbf{Domande?}}
		\end{center}
	\end{frame}
	
\end{document}