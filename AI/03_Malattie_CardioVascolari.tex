\documentclass[10pt]{beamer}

\usepackage[utf8]{inputenc}
\usepackage[T1]{fontenc}
\usepackage[italian]{babel}
\usepackage{amsmath}
\usepackage{graphicx}

% Tema Beamer
\usetheme{Madrid}
\usecolortheme{seahorse}

\title{Predizione del Rischio Cardiovascolare}
\subtitle{Un approccio essenziale con Regressione Logistica}
\author{Prof. Fedeli Massimo}
\date{}

\begin{document}
	
	%------------------------------------------------
	\begin{frame}
		\titlepage
	\end{frame}
	
	%------------------------------------------------
	\begin{frame}{Obiettivo del sistema}
		L’algoritmo realizza un sistema di \textbf{Machine Learning supervisionato} per la predizione del rischio cardiovascolare.
		
		Obiettivo principale:
		\begin{itemize}
			\item classificare ogni paziente come \textbf{sano} o \textbf{a rischio}
		\end{itemize}
		
		Il modello utilizzato è la \textbf{regressione logistica}, scelta per la sua semplicità e interpretabilità.
	\end{frame}
	
	%------------------------------------------------
	\begin{frame}{Dataset clinico}
		Il sistema utilizza un dataset di pazienti reali contenente:
		\begin{itemize}
			\item dati anagrafici (età, sesso)
			\item parametri clinici (pressione sanguigna, colesterolo)
			\item risultati di test diagnostici
		\end{itemize}
		
		I dati sono caricati da una sorgente online e organizzati in un \textbf{DataFrame}.
	\end{frame}
	
	%------------------------------------------------
	\begin{frame}{Gestione dei valori mancanti}
		Nel dataset originale sono presenti valori mancanti indicati con un simbolo speciale.
		
		La procedura adottata è:
		\begin{itemize}
			\item conversione dei simboli in valori nulli
			\item eliminazione delle osservazioni incomplete
		\end{itemize}
		
		Questa scelta, seppur semplice, è adeguata a una versione didattica dell’algoritmo.
	\end{frame}
	
	%------------------------------------------------
	\begin{frame}{Definizione della variabile target}
		La variabile di interesse rappresenta il grado di malattia cardiaca.
		
		Nel dataset originale:
		\begin{itemize}
			\item il livello di malattia è espresso con più valori interi
		\end{itemize}
		
		L’algoritmo semplifica il problema trasformando il target in:
		\begin{itemize}
			\item 0 = assenza di malattia
			\item 1 = presenza di malattia
		\end{itemize}
	\end{frame}
	
	%------------------------------------------------
	\begin{frame}{Classificazione binaria}
		Questa trasformazione consente di:
		\begin{itemize}
			\item ricondurre il problema a una \textbf{classificazione binaria}
			\item rendere il modello più interpretabile
			\item utilizzare in modo naturale la regressione logistica
		\end{itemize}
		
		Il focus diventa la stima del rischio, non la gravità clinica.
	\end{frame}
	
	%------------------------------------------------
	\begin{frame}{Preparazione dei dati}
		Il dataset viene suddiviso in:
		\begin{itemize}
			\item variabili di input (feature cliniche)
			\item variabile di output (stato di salute)
		\end{itemize}
		
		Successivamente i dati sono separati in:
		\begin{itemize}
			\item training set
			\item test set
		\end{itemize}
	\end{frame}
	
	%------------------------------------------------
	\begin{frame}{Suddivisione stratificata}
		La suddivisione del dataset è \textbf{stratificata}.
		
		Questo significa che:
		\begin{itemize}
			\item la proporzione tra pazienti sani e malati è mantenuta
			\item si evitano distorsioni nella valutazione del modello
		\end{itemize}
		
		La valutazione risulta così più affidabile.
	\end{frame}
	
	%------------------------------------------------
	\begin{frame}{Standardizzazione delle variabili}
		Le variabili cliniche hanno scale molto diverse:
		\begin{itemize}
			\item età in anni
			\item valori di laboratorio
			\item indicatori discreti
		\end{itemize}
		
		Prima dell’addestramento viene applicata la \textbf{standardizzazione}.
	\end{frame}
	
	%------------------------------------------------
	\begin{frame}{Perché standardizzare}
		La standardizzazione:
		\begin{itemize}
			\item centra le variabili sulla media
			\item imposta deviazione standard unitaria
			\item rende le feature confrontabili tra loro
		\end{itemize}
		
		Questo passaggio è particolarmente importante per la regressione logistica, sensibile alle differenze di scala.
	\end{frame}
	
	%------------------------------------------------
	\begin{frame}{Regressione logistica}
		La regressione logistica è un algoritmo di classificazione che:
		\begin{itemize}
			\item stima la probabilità di appartenenza alla classe “a rischio”
			\item utilizza una funzione sigmoide
		\end{itemize}
		
		Il risultato è una probabilità compresa tra 0 e 1.
	\end{frame}
	
	%------------------------------------------------
	\begin{frame}{Addestramento del modello}
		Durante l’addestramento:
		\begin{itemize}
			\item il modello apprende dai dati di training
			\item stima un peso per ciascuna variabile clinica
		\end{itemize}
		
		I pesi rappresentano l’influenza di ogni fattore sul rischio cardiovascolare.
	\end{frame}
	
	%------------------------------------------------
	\begin{frame}{Valutazione del modello}
		Il modello viene testato su pazienti mai visti in precedenza.
		
		Le previsioni sono confrontate con i valori reali per calcolare:
		\begin{itemize}
			\item \textbf{accuratezza}
		\end{itemize}
		
		L’accuratezza indica la percentuale di classificazioni corrette.
	\end{frame}
	
	%------------------------------------------------
	\begin{frame}{Considerazioni finali}
		L’accuratezza fornisce una prima valutazione dell’efficacia del sistema.
		
		Tuttavia:
		\begin{itemize}
			\item non esaurisce tutte le metriche clinicamente rilevanti
			\item rappresenta un buon punto di partenza didattico
		\end{itemize}
		
		Il modello è semplice, interpretabile e adatto a scopi formativi.
	\end{frame}
	
\end{document}
