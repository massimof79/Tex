\documentclass[a4paper,11pt]{article}
\usepackage[italian]{babel}
\usepackage[T1]{fontenc}
\usepackage[utf8]{inputenc}
\usepackage{amsmath}
\usepackage{amssymb}
\usepackage{geometry}
\geometry{margin=2.5cm}
\usepackage{listings}
\usepackage{xcolor}

\lstset{
	basicstyle=\ttfamily\small,
	keywordstyle=\color{blue},
	commentstyle=\color{gray},
	stringstyle=\color{red},
	frame=single,
	breaklines=true,
	showstringspaces=false
}

\title{Alberi decisionali Gestione intelligente delle richieste di assistenza}
\author{Prof. Fedeli Massimo}
\date{}

\begin{document}
	\maketitle
	
	\newpage
	\section{Introduzione}
	
	In questo documento viene illustrato il funzionamento di un \textbf{albero decisionale} attraverso un esempio realistico e vicino all'esperienza quotidiana: la classificazione delle richieste di assistenza tecnica in base alla loro priorità.
	
	L'obiettivo è spiegare come un algoritmo di \emph{machine learning} possa supportare decisioni organizzative, rendendole più rapide, coerenti e motivate.
	
	\section{Il problema affrontato}
	
	In molti contesti, come scuole, aziende o uffici informatici, arrivano quotidianamente richieste di assistenza. Non tutte hanno la stessa importanza: alcune possono attendere, altre richiedono un intervento immediato.
	
	Il problema consiste nel determinare automaticamente la \textbf{priorità} di una richiesta sulla base di alcune informazioni fornite dall'utente.
	
	Ogni richiesta è descritta tramite quattro caratteristiche:
	\begin{itemize}
		\item tipo di problema (software, hardware, rete);
		\item numero di utenti coinvolti;
		\item impatto sul servizio;
		\item urgenza dichiarata.
	\end{itemize}
	
	A queste informazioni è associata un'etichetta, chiamata \emph{Priorità}, che rappresenta la decisione finale da prendere.
	
	\section{Classificazione supervisionata}
	
	Il problema rientra nella \textbf{classificazione supervisionata}. Il modello viene addestrato su un insieme di esempi per i quali la priorità corretta è già nota.
	
	L'algoritmo analizza questi esempi e impara una serie di regole che collegano le caratteristiche della richiesta alla priorità finale.
	
	\section{Preparazione dei dati}
	
	Il dataset è memorizzato in un file CSV e viene caricato tramite la libreria \texttt{pandas}. Poiché le informazioni sono espresse in forma testuale, è necessario trasformarle in valori numerici prima di poterle utilizzare.
	
	Questa operazione viene svolta tramite il \textbf{Label Encoding}. Ogni valore testuale viene sostituito da un numero intero.
	
	Ad esempio:
	\begin{quote}
		software -- 0, hardware  1,  rete  2
	\end{quote}
	
	Per ogni colonna viene salvato il relativo codificatore, così da poter tradurre correttamente anche i dati inseriti successivamente dall'utente.
	
	\section{Suddivisione dei dati}
	
	Il dataset viene diviso in due parti:
	\begin{itemize}
		\item \textbf{training set} (70\%), utilizzato per addestrare il modello;
		\item \textbf{test set} (30\%), utilizzato per valutarne le prestazioni.
	\end{itemize}
	
	Questa separazione permette di verificare se l'albero decisionale è in grado di generalizzare, cioè di funzionare correttamente anche su dati mai visti prima.
	
	\section{L'albero decisionale}
	
	Il modello utilizzato è un \texttt{DecisionTreeClassifier}. Durante l'addestramento, l'albero costruisce una sequenza di decisioni del tipo:
	\begin{quote}
		Se l'impatto è alto e l'urgenza è alta, allora la priorità è elevata.
	\end{quote}
	
	A ogni nodo l'algoritmo sceglie la condizione che separa meglio le richieste, riducendo l'\textbf{impurità} dei gruppi ottenuti. In questo caso viene utilizzato il criterio di Gini.
	
	\section{Profondità dell'albero}
	
	La profondità massima dell'albero è limitata a 4 livelli. Questa scelta evita il fenomeno dell'\textbf{overfitting}, che si verifica quando il modello impara regole troppo specifiche e perde la capacità di generalizzare.
	
	Un albero con profondità limitata rappresenta un buon compromesso tra:
	\begin{itemize}
		\item capacità di descrivere situazioni diverse;
		\item semplicità e leggibilità delle decisioni;
		\item robustezza del modello.
	\end{itemize}
	
	\section{Valutazione del modello}
	
	Dopo l'addestramento, il modello viene testato sui dati di test. La prestazione viene misurata tramite l'\textbf{accuratezza}, cioè la percentuale di previsioni corrette.
	
	Questo valore fornisce un'indicazione quantitativa dell'affidabilità del sistema.
	
	\section{Salvataggio e riutilizzo del modello}
	
	Una caratteristica importante del programma è il salvataggio del modello addestrato su file. In questo modo:
	\begin{itemize}
		\item l'addestramento non deve essere ripetuto ogni volta;
		\item il modello può essere utilizzato in momenti diversi;
		\item il sistema diventa più efficiente.
	\end{itemize}
	
	Insieme al modello vengono salvati anche i codificatori utilizzati per i dati.
	
	\section{Previsione di una nuova richiesta}
	
	L'utente può inserire manualmente i dati di una nuova richiesta di assistenza. Il programma:
	\begin{enumerate}
		\item codifica i valori testuali;
		\item applica il modello addestrato;
		\item riconverte il risultato in forma testuale;
		\item mostra la priorità prevista.
	\end{enumerate}
	
	Questo processo rende evidente come il modello possa essere utilizzato in un contesto reale.
	
	\section{Struttura del programma}
	
	Il programma è organizzato in funzioni, ciascuna con un compito preciso:
	\begin{itemize}
		\item caricamento e preparazione dei dati;
		\item addestramento del modello;
		\item previsione della priorità;
		\item gestione del menù.
	\end{itemize}
	
	Questa struttura migliora la leggibilità del codice e ne facilita la manutenzione.
	
	\section{Conclusione}
	
	L'albero decisionale si dimostra uno strumento efficace e facilmente interpretabile per la gestione delle priorità nelle richieste di assistenza.
	
\end{document}
