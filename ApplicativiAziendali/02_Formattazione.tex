\documentclass{beamer}
\usetheme{Madrid}
\usecolortheme{default}

\title{Formattazione del Testo e Gestione dei Paragrafi}
\subtitle{Competenze di Elaborazione Testi}
\author{Prof. Fedeli Massimo - Fabbrica Digitale 4.0}
\date{}

\begin{document}
	
	\frame{\titlepage}
	
	%------------------------------------------------
	
	\begin{frame}{Obiettivi della lezione}
		\begin{itemize}
			\item Applicare correttamente le principali formattazioni al testo
			\item Gestire paragrafi, spaziature e allineamenti in modo professionale
			\item Utilizzare elenchi, bordi e sfondi
			\item Applicare e copiare stili di formattazione
		\end{itemize}
	\end{frame}
	
	%------------------------------------------------
	
	\begin{frame}{Carattere: tipo e dimensione}
		La scelta del carattere influisce su leggibilità e stile del documento.
		
		È possibile:
		\begin{itemize}
			\item Cambiare il \textbf{tipo di carattere} (font)
			\item Modificare la \textbf{dimensione} del testo
		\end{itemize}
		
		Regola generale: usare pochi font e dimensioni coerenti.
	\end{frame}
	
	%------------------------------------------------
	
	\begin{frame}{Grassetto, corsivo, sottolineato}
		Queste formattazioni servono per evidenziare parti importanti del testo.
		
		\begin{itemize}
			\item \textbf{Grassetto}: mette in risalto concetti chiave
			\item \textit{Corsivo}: usato per termini stranieri o titoli di opere
			\item \underline{Sottolineato}: da usare con moderazione
		\end{itemize}
		
		Un uso eccessivo riduce l'efficacia visiva.
	\end{frame}
	
	%------------------------------------------------
	
	\begin{frame}{Apice e pedice}
		Utilizzati in ambito scientifico, matematico e tecnico.
		
		Esempi:
		\begin{itemize}
			\item Apice: $x^2$, m\textsuperscript{2}
			\item Pedice: H\textsubscript{2}O, CO\textsubscript{2}
		\end{itemize}
		
		Servono a rappresentare potenze, formule chimiche e notazioni tecniche.
	\end{frame}
	
	%------------------------------------------------
	
	\begin{frame}{Colore del testo}
		Il \textbf{colore} aiuta a distinguere le informazioni, ma va usato con criterio.
		
		Buone pratiche:
		\begin{itemize}
			\item Mantenere un alto contrasto con lo sfondo
			\item Usare pochi colori coerenti
			\item Evitare combinazioni difficili da leggere
		\end{itemize}
	\end{frame}
	
	%------------------------------------------------
	
	\begin{frame}{Maiuscole e minuscole}
		Il comando di conversione consente di modificare rapidamente il testo:
		
		\begin{itemize}
			\item MAIUSCOLO
			\item minuscolo
			\item Iniziali Maiuscole
		\end{itemize}
		
		Utile per uniformare titoli e intestazioni.
	\end{frame}
	
	%------------------------------------------------
	
	\begin{frame}{Sillabazione automatica}
		La \textbf{sillabazione} divide automaticamente le parole a fine riga.
		
		Vantaggi:
		\begin{itemize}
			\item Migliora l'allineamento del testo giustificato
			\item Riduce spazi bianchi irregolari
		\end{itemize}
		
		Particolarmente utile nei documenti con colonne strette.
	\end{frame}
	
	%------------------------------------------------
	
	\begin{frame}{Collegamenti ipertestuali}
		Un \textbf{collegamento ipertestuale} permette di:
		\begin{itemize}
			\item Aprire una pagina web
			\item Saltare a una parte del documento
			\item Aprire un file o un messaggio di posta elettronica
		\end{itemize}
		
		Può essere inserito, modificato o rimosso in qualsiasi momento.
	\end{frame}
	
	%------------------------------------------------
	
	\begin{frame}{Il concetto di paragrafo}
		Un \textbf{paragrafo} è un blocco di testo separato da un ritorno a capo.
		
		Ogni paragrafo può avere:
		\begin{itemize}
			\item Allineamento
			\item Rientri
			\item Spaziature
			\item Bordi e sfondi
		\end{itemize}
	\end{frame}
	
	%------------------------------------------------
	
	\begin{frame}{Creare e unire paragrafi}
		\begin{itemize}
			\item Premendo Invio si crea un nuovo paragrafo
			\item Eliminando il segno di paragrafo si uniscono due paragrafi
		\end{itemize}
		
		Attenzione: non confondere il paragrafo con una semplice interruzione di riga.
	\end{frame}
	
	%------------------------------------------------
	
	\begin{frame}{Interruzioni di riga}
		L'\textbf{interruzione di riga} va a capo senza creare un nuovo paragrafo.
		
		Serve quando:
		\begin{itemize}
			\item Si vuole andare a capo mantenendo la stessa formattazione
			\item Si scrivono indirizzi o testi poetici
			\item Per inserire interruzioni di riga si devono premere contemporaneamente i tasti \textbf{Maiusc + Invio}
			\item Il simbolo dell'interruzione di riga è una freccia piegata verso sinistra di 90 gradi ed è visibile cliccando il pulsante "Mostra Tutto" nella barra dei menù standard.
		\end{itemize}
	\end{frame}
	
	%------------------------------------------------
	
	\begin{frame}{Disposizione corretta del testo}
		Per allineare il testo in modo preciso bisogna usare:
		\begin{itemize}
			\item Allineamenti
			\item Rientri
			\item Tabulazioni
		\end{itemize}
		
\underline{		Evitare l’uso di spazi ripetuti: creano disallineamenti.}
	\end{frame}
	
	%------------------------------------------------
	
\begin{frame}{Allineamento del testo}
	Un \textbf{paragrafo} può essere disposto in modi diversi a seconda dell’effetto grafico desiderato.
	
	\begin{itemize}
		\item \textbf{Allineato a sinistra}: il testo è allineato al margine sinistro ed è la modalità più comune nei documenti
		\item \textbf{Centrato}: il testo è disposto al centro della riga, spesso usato per titoli o intestazioni
		\item \textbf{Allineato a destra}: il testo segue il margine destro, utile per date, firme o elementi brevi
		\item \textbf{Giustificato}: il testo è allineato sia a sinistra sia a destra, creando un bordo uniforme ai lati
	\end{itemize}
	
	La scelta dipende dal tipo di documento e dall’effetto di leggibilità che si vuole ottenere.
\end{frame}

	
	%------------------------------------------------
	
\begin{frame}{Rientri dei paragrafi}
	I rientri modificano la distanza del testo rispetto ai margini della pagina e aiutano a organizzare visivamente i contenuti.
	
	Tipologie:
	\begin{itemize}
		\item \textbf{Rientro sinistro e destro}: spostano l’intero paragrafo verso l’interno rispetto ai margini, utile per citazioni o blocchi di testo evidenziati
		\item \textbf{Rientro della prima riga}: solo la prima riga del paragrafo è rientrata, tipico dei testi narrativi o delle relazioni
		\item \textbf{Rientro sporgente}: tutte le righe tranne la prima sono rientrate, usato spesso negli elenchi o nelle bibliografie
	\end{itemize}
\end{frame}

	
	%------------------------------------------------
\begin{frame}{Tabulazioni}
	Le \textbf{tabulazioni} permettono di allineare il testo in modo preciso creando colonne ordinate senza usare spazi ripetuti.
	
	Tipi principali:
	\begin{itemize}
		\item \textbf{Sinistra}: il testo inizia dal punto di tabulazione e si estende verso destra
		\item \textbf{Centro}: il testo viene centrato rispetto alla posizione della tabulazione
		\item \textbf{Destra}: il testo termina in corrispondenza della tabulazione ed è allineato verso sinistra
		\item \textbf{Decimale}: allinea i numeri rispetto alla virgola o al punto decimale, utile per dati numerici
	\end{itemize}
	
	Le tabulazioni sono particolarmente utili per elenchi strutturati, moduli e dati organizzati in colonne.
\end{frame}
\begin{frame}{Tabulazioni: concetto}
	Le \textbf{tabulazioni} sono marcatori di allineamento orizzontale che permettono di organizzare il testo in colonne ordinate.
	
	Non si basano su spazi ripetuti, ma su posizioni precise lungo il righello del documento.
	
	\medskip
	\textbf{Obiettivo:} ottenere un allineamento stabile, pulito e facilmente modificabile.
\end{frame}

\begin{frame}{Perché non usare gli spazi}
	L’uso della barra spaziatrice per allineare il testo è scorretto perché:
	
	\begin{itemize}
		\item la larghezza dei caratteri varia a seconda del font
		\item eventuali modifiche al testo rompono l’allineamento
		\item il risultato non è tipograficamente preciso
	\end{itemize}
	
	Le tabulazioni risolvono questi problemi in modo strutturato.
\end{frame}

\begin{frame}{Tipi di tabulazione}
	I principali tipi di tabulazione sono:
	
	\begin{itemize}
		\item \textbf{Sinistra} — il testo parte dalla tabulazione e si estende verso destra
		\item \textbf{Centro} — il testo viene centrato rispetto al punto impostato
		\item \textbf{Destra} — il testo termina in corrispondenza della tabulazione
		\item \textbf{Decimale} — allinea i numeri rispetto al separatore decimale
	\end{itemize}
\end{frame}

\begin{frame}{Quando usare i diversi tipi}
	\begin{itemize}
		\item \textbf{Sinistra}: elenchi descrittivi o voci testuali
		\item \textbf{Centro}: titoli brevi o intestazioni di colonne
		\item \textbf{Destra}: prezzi, totali, valori numerici interi
		\item \textbf{Decimale}: tabelle con numeri decimali da confrontare
	\end{itemize}
\end{frame}

\begin{frame}{Impostare le tabulazioni dal righello}
	Nei programmi di videoscrittura come \textbf{Microsoft Word}:
	
	\begin{enumerate}
		\item Seleziona il tipo di tabulazione dal selettore del righello
		\item Fai clic sul righello orizzontale nel punto desiderato
		\item Ripeti per ogni colonna che vuoi creare
	\end{enumerate}
\end{frame}

\begin{frame}{Finestra di dialogo Tabulazioni}
	Con un doppio clic su una tabulazione nel righello si apre la finestra di configurazione, dove puoi:
	
	\begin{itemize}
		\item impostare con precisione la posizione (in cm)
		\item scegliere il tipo di allineamento
		\item aggiungere un \textbf{carattere di riempimento} (puntini, linee, trattini)
	\end{itemize}
\end{frame}

\begin{frame}{Usare il tasto TAB}
	Dopo aver impostato le tabulazioni:
	
	\begin{itemize}
		\item premi \textbf{TAB} per spostare il cursore alla tabulazione successiva
		\item ogni pressione inserisce un unico carattere di controllo, non spazi
		\item il testo si allinea automaticamente alla colonna definita
	\end{itemize}
\end{frame}

\begin{frame}{Le tabulazioni sono proprietà del paragrafo}
	Le tabulazioni non appartengono al singolo testo, ma al \textbf{paragrafo}.
	
	\medskip
	Se selezioni più paragrafi e imposti le tabulazioni, tutti avranno la stessa struttura di allineamento.
\end{frame}

\begin{frame}{Modificare o rimuovere una tabulazione}
	\begin{itemize}
		\item Trascina il simbolo della tabulazione fuori dal righello per eliminarla
		\item Oppure usa la finestra Tabulazioni e scegli \textbf{Cancella} o \textbf{Cancella tutto}
	\end{itemize}
	
	Questo permette di aggiornare facilmente la struttura del documento.
\end{frame}

\begin{frame}{Vantaggi delle tabulazioni}
	\begin{itemize}
		\item Allineamento preciso e professionale
		\item Maggiore leggibilità dei dati
		\item Facilità di modifica
		\item Struttura coerente del documento
	\end{itemize}
\end{frame}
	%------------------------------------------------
	
\begin{frame}{Spaziatura tra paragrafi}
	Per separare visivamente i paragrafi si utilizza la spaziatura prima e dopo il paragrafo, evitando di premere il tasto Invio più volte.
	
	Vantaggi:
	\begin{itemize}
		\item Garantisce un’impaginazione uniforme in tutto il documento
		\item Permette modifiche rapide e coerenti agendo sulle impostazioni del paragrafo
		\item Evita errori di formattazione difficili da individuare
		\item Mantiene la struttura del testo ordinata e professionale
	\end{itemize}
\end{frame}

	%------------------------------------------------
	
\begin{frame}{Interlinea}
	L'\textbf{interlinea} è la distanza verticale tra le righe all’interno di uno stesso paragrafo e incide direttamente sulla leggibilità del testo.
	
	Valori comuni:
	\begin{itemize}
		\item \textbf{Singola}: compatta, adatta a documenti brevi o con poco testo
		\item \textbf{1,5 righe}: equilibrio tra compattezza e leggibilità, spesso usata in relazioni e compiti scolastici
		\item \textbf{Doppia}: aumenta lo spazio tra le righe, utile per revisioni o annotazioni a mano
	\end{itemize}
	
	Una scelta corretta dell’interlinea rende il testo più chiaro e meno affaticante da leggere.
\end{frame}

	%------------------------------------------------
	
\begin{frame}{Elenchi puntati e numerati}
	Gli elenchi servono a organizzare le informazioni in modo chiaro e facilmente leggibile.
	
	\begin{itemize}
		\item Gli \textbf{elenchi puntati} sono adatti quando l’ordine degli elementi non è importante
		\item Gli \textbf{elenchi numerati} si usano quando è necessario indicare una sequenza o una gerarchia
		\item È possibile inserire o rimuovere punti o numeri in qualsiasi momento
		\item Si possono scegliere stili diversi tra quelli predefiniti
		\item Gli elenchi possono avere più livelli per rappresentare sottoargomenti
	\end{itemize}
\end{frame}

	%------------------------------------------------
	
\begin{frame}{Bordi e sfondi del paragrafo}
	Un paragrafo può essere messo in evidenza attraverso bordi e colori di sfondo.
	
	\begin{itemize}
		\item I \textbf{bordi} permettono di racchiudere il testo con linee di diverso stile e spessore
		\item È possibile applicare il bordo a tutti i lati o solo ad alcuni (ad esempio solo sopra o sotto)
		\item I \textbf{colori di sfondo} aiutano a distinguere visivamente un blocco di testo dal resto del documento
		\item Questi strumenti sono utili per evidenziare avvisi, definizioni o parti importanti
		\item È consigliabile usare effetti grafici con moderazione per non appesantire l’impaginazione
	\end{itemize}
\end{frame}

	
	%------------------------------------------------
	
\begin{frame}{Stili di carattere}
	Uno \textbf{stile di carattere} applica contemporaneamente più formattazioni a una porzione di testo, garantendo coerenza grafica.
	
	Può includere:
	\begin{itemize}
		\item Colore del testo
		\item Dimensione del carattere
		\item Grassetto, corsivo o altre varianti tipografiche
		\item Eventuali effetti come sottolineature o evidenziazioni
	\end{itemize}
	
	Vantaggi:
	\begin{itemize}
		\item Assicura uniformità visiva in tutto il documento
		\item Permette di modificare rapidamente più elementi aggiornando lo stile una sola volta
		\item Riduce errori dovuti a formattazioni applicate manualmente
	\end{itemize}
\end{frame}

	
	%------------------------------------------------
	
\begin{frame}{Stili di paragrafo}
	Uno \textbf{stile di paragrafo} applica in un’unica operazione tutte le principali impostazioni di formattazione del paragrafo.
	
	Può includere:
	\begin{itemize}
		\item Allineamento del testo
		\item Spaziature prima e dopo il paragrafo
		\item Rientri sinistri, destri e della prima riga
		\item Tipo, dimensione e stile del carattere
	\end{itemize}
	
	Vantaggi:
	\begin{itemize}
		\item Garantisce uniformità grafica in tutto il documento
		\item Permette di aggiornare rapidamente l’aspetto di più paragrafi modificando un solo stile
		\item È fondamentale nei documenti lunghi e strutturati, come relazioni, tesi e manuali
	\end{itemize}
\end{frame}

	
	%------------------------------------------------
	
\begin{frame}{Copia formato}
	Lo strumento \textbf{Copia} formato consente di trasferire rapidamente la formattazione da un testo a un altro senza doverla reimpostare manualmente.
	
	\underline{Funzionamento}:
	\begin{itemize}
		\item Si seleziona il testo con la formattazione desiderata
		\item Si attiva lo strumento Copia formato
		\item Si applica la stessa formattazione al nuovo testo
	\end{itemize}
	
	\underline{Vantaggi}:
	\begin{itemize}
		\item Risparmio di tempo nelle operazioni ripetitive
		\item Maggiore coerenza grafica nel documento
		\item Riduzione degli errori dovuti a impostazioni manuali diverse
	\end{itemize}
\end{frame}

	%------------------------------------------------
	
	\begin{frame}{Conclusione}
		Una buona formattazione:
		\begin{itemize}
			\item Migliora la leggibilità
			\item Rende il documento professionale
			\item Aiuta a comunicare in modo chiaro
		\end{itemize}
		
		La cura dell'impaginazione è parte integrante della qualità del contenuto.
	\end{frame}
	
\end{document}
