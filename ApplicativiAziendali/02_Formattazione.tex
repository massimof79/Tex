\documentclass{beamer}
\usetheme{Madrid}
\usecolortheme{default}

\title{Formattazione del Testo e Gestione dei Paragrafi}
\subtitle{Competenze di Elaborazione Testi}
\author{Prof. Fedeli Massimo - Fabbrica Digitale 4.0}
\date{}

\begin{document}
	
	\frame{\titlepage}
	
	%------------------------------------------------
	
	\begin{frame}{Obiettivi della lezione}
		\begin{itemize}
			\item Applicare correttamente le principali formattazioni al testo
			\item Gestire paragrafi, spaziature e allineamenti in modo professionale
			\item Utilizzare elenchi, bordi e sfondi
			\item Applicare e copiare stili di formattazione
		\end{itemize}
	\end{frame}
	
	%------------------------------------------------
	
	\begin{frame}{Carattere: tipo e dimensione}
		La scelta del carattere influisce su leggibilità e stile del documento.
		
		È possibile:
		\begin{itemize}
			\item Cambiare il \textbf{tipo di carattere} (font)
			\item Modificare la \textbf{dimensione} del testo
		\end{itemize}
		
		Regola generale: usare pochi font e dimensioni coerenti.
	\end{frame}
	
	%------------------------------------------------
	
	\begin{frame}{Grassetto, corsivo, sottolineato}
		Queste formattazioni servono per evidenziare parti importanti del testo.
		
		\begin{itemize}
			\item \textbf{Grassetto}: mette in risalto concetti chiave
			\item \textit{Corsivo}: usato per termini stranieri o titoli
			\item \underline{Sottolineato}: da usare con moderazione
		\end{itemize}
		
		Un uso eccessivo riduce l'efficacia visiva.
	\end{frame}
	
	%------------------------------------------------
	
	\begin{frame}{Apice e pedice}
		Utilizzati in ambito scientifico, matematico e tecnico.
		
		Esempi:
		\begin{itemize}
			\item Apice: $x^2$, m\textsuperscript{2}
			\item Pedice: H\textsubscript{2}O, CO\textsubscript{2}
		\end{itemize}
		
		Servono a rappresentare potenze, formule chimiche e notazioni tecniche.
	\end{frame}
	
	%------------------------------------------------
	
	\begin{frame}{Colore del testo}
		Il colore aiuta a distinguere informazioni, ma va usato con criterio.
		
		Buone pratiche:
		\begin{itemize}
			\item Mantenere alto il contrasto con lo sfondo
			\item Usare pochi colori coerenti
			\item Evitare combinazioni difficili da leggere
		\end{itemize}
	\end{frame}
	
	%------------------------------------------------
	
	\begin{frame}{Maiuscole e minuscole}
		Il comando di conversione consente di modificare rapidamente il testo:
		
		\begin{itemize}
			\item MAIUSCOLO
			\item minuscolo
			\item Iniziali Maiuscole
		\end{itemize}
		
		Utile per uniformare titoli e intestazioni.
	\end{frame}
	
	%------------------------------------------------
	
	\begin{frame}{Sillabazione automatica}
		La sillabazione divide automaticamente le parole a fine riga.
		
		Vantaggi:
		\begin{itemize}
			\item Migliora l'allineamento del testo giustificato
			\item Riduce spazi bianchi irregolari
		\end{itemize}
		
		Particolarmente utile nei documenti con colonne strette.
	\end{frame}
	
	%------------------------------------------------
	
	\begin{frame}{Collegamenti ipertestuali}
		Un collegamento ipertestuale permette di:
		\begin{itemize}
			\item Aprire una pagina web
			\item Saltare a una parte del documento
			\item Aprire un file o un'email
		\end{itemize}
		
		Può essere inserito, modificato o rimosso in qualsiasi momento.
	\end{frame}
	
	%------------------------------------------------
	
	\begin{frame}{Il concetto di paragrafo}
		Un paragrafo è un blocco di testo separato da un ritorno a capo.
		
		Ogni paragrafo può avere:
		\begin{itemize}
			\item Allineamento
			\item Rientri
			\item Spaziature
			\item Bordi e sfondi
		\end{itemize}
	\end{frame}
	
	%------------------------------------------------
	
	\begin{frame}{Creare e unire paragrafi}
		\begin{itemize}
			\item Premendo Invio si crea un nuovo paragrafo
			\item Eliminando il segno di paragrafo si uniscono due paragrafi
		\end{itemize}
		
		Attenzione: non confondere paragrafo con semplice interruzione di riga.
	\end{frame}
	
	%------------------------------------------------
	
	\begin{frame}{Interruzioni di riga}
		L'interruzione di riga va a capo senza creare un nuovo paragrafo.
		
		Serve quando:
		\begin{itemize}
			\item Si vuole andare a capo mantenendo la stessa formattazione
			\item Si scrivono indirizzi o poesie
		\end{itemize}
	\end{frame}
	
	%------------------------------------------------
	
	\begin{frame}{Disposizione corretta del testo}
		Per allineare il testo in modo preciso bisogna usare:
		\begin{itemize}
			\item Allineamenti
			\item Rientri
			\item Tabulazioni
		\end{itemize}
		
		Non usare spazi ripetuti: creano disallineamenti.
	\end{frame}
	
	%------------------------------------------------
	
	\begin{frame}{Allineamento del testo}
		Un paragrafo può essere:
		\begin{itemize}
			\item Allineato a sinistra
			\item Centrato
			\item Allineato a destra
			\item Giustificato
		\end{itemize}
		
		La scelta dipende dal tipo di documento.
	\end{frame}
	
	%------------------------------------------------
	
	\begin{frame}{Rientri dei paragrafi}
		I rientri modificano la distanza del testo dai margini.
		
		Tipologie:
		\begin{itemize}
			\item Rientro sinistro e destro
			\item Rientro prima riga
			\item Rientro sporgente
		\end{itemize}
	\end{frame}
	
	%------------------------------------------------
	
	\begin{frame}{Tabulazioni}
		Le tabulazioni permettono allineamenti precisi in colonne.
		
		Tipi principali:
		\begin{itemize}
			\item Sinistra
			\item Centro
			\item Destra
			\item Decimale
		\end{itemize}
	\end{frame}
	
	%------------------------------------------------
	
	\begin{frame}{Spaziatura tra paragrafi}
		Per separare visivamente i paragrafi si usa la spaziatura, non il tasto Invio ripetuto.
		
		Vantaggi:
		\begin{itemize}
			\item Impaginazione uniforme
			\item Modifiche più rapide e coerenti
		\end{itemize}
	\end{frame}
	
	%------------------------------------------------
	
	\begin{frame}{Interlinea}
		L'interlinea è la distanza tra le righe di un paragrafo.
		
		Valori comuni:
		\begin{itemize}
			\item Singola
			\item 1,5 righe
			\item Doppia
		\end{itemize}
		
		Influisce molto sulla leggibilità.
	\end{frame}
	
	%------------------------------------------------
	
	\begin{frame}{Elenchi puntati e numerati}
		Gli elenchi servono a organizzare le informazioni.
		
		Si può:
		\begin{itemize}
			\item Inserire o rimuovere punti o numeri
			\item Scegliere stili diversi predefiniti
		\end{itemize}
	\end{frame}
	
	%------------------------------------------------
	
	\begin{frame}{Bordi e sfondi del paragrafo}
		Un paragrafo può avere:
		\begin{itemize}
			\item Contorni con diversi stili e spessori
			\item Colori di sfondo per evidenziare contenuti
		\end{itemize}
		
		Utili per richiamare l’attenzione su parti importanti.
	\end{frame}
	
	%------------------------------------------------
	
	\begin{frame}{Stili di carattere}
		Uno stile di carattere applica più formattazioni insieme, ad esempio:
		\begin{itemize}
			\item Colore
			\item Dimensione
			\item Grassetto o corsivo
		\end{itemize}
		
		Permette uniformità e rapidità di modifica.
	\end{frame}
	
	%------------------------------------------------
	
	\begin{frame}{Stili di paragrafo}
		Uno stile di paragrafo include:
		\begin{itemize}
			\item Allineamento
			\item Spaziature
			\item Rientri
			\item Carattere
		\end{itemize}
		
		Fondamentale nei documenti lunghi e strutturati.
	\end{frame}
	
	%------------------------------------------------
	
	\begin{frame}{Copia formato}
		Lo strumento copia formato consente di trasferire rapidamente la formattazione da un testo a un altro.
		
		Vantaggi:
		\begin{itemize}
			\item Risparmio di tempo
			\item Coerenza grafica
			\item Riduzione degli errori
		\end{itemize}
	\end{frame}
	
	%------------------------------------------------
	
	\begin{frame}{Conclusione}
		Una buona formattazione:
		\begin{itemize}
			\item Migliora la leggibilità
			\item Rende il documento professionale
			\item Aiuta a comunicare in modo chiaro
		\end{itemize}
		
		La cura dell'impaginazione è parte integrante della qualità del contenuto.
	\end{frame}
	
\end{document}
