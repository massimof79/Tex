\documentclass{beamer}
\usetheme{Madrid}
\usecolortheme{default}

\title{Elaborazione Testi: Competenze di Base}
\subtitle{Programma di videoscrittura}
\author{Prof. Fedeli Massimo - Fabbrica Digitale 4.0}
\date{}

\begin{document}
	
	\frame{\titlepage}
	
	%------------------------------------------------

	
	%------------------------------------------------
	\section{Lavorare con i documenti}
	
	\begin{frame}{Aprire e chiudere il programma}
		\begin{itemize}
			\item			Un programma di elaborazione testi come \textbf{Microsoft Word} può essere avviato in vari modi: dal menu delle applicazioni, da un’icona sul desktop oppure aprendo direttamente un file di testo.
			
			\item 		Quando chiudi il programma usando \textbf{File → Esci}, il sistema controlla se ci sono documenti non salvati. In quel caso ti chiede se vuoi salvare: ignorare questo avviso può significare perdere tutto il lavoro fatto.
			
	\end{itemize}
		
	\end{frame}
	
	\begin{frame}{Aprire e chiudere documenti}
		\begin{itemize}
			\item 		È importante distinguere tra \textbf{programma} e \textbf{documento}.  
				 
				 Il programma è lo strumento.
			     
			     
			     ll documento è il file su cui stai lavorando.
			\item 		Con \textbf{File → Apri} puoi cercare documenti già esistenti nelle cartelle del computer o nel cloud. Puoi anche chiudere solo il documento, lasciando aperto il programma per lavorare su un altro file.			
					\end{itemize}
		
	\end{frame}
	
\begin{frame}{Creare un nuovo documento}
	Quando inizi un nuovo lavoro puoi:
	\begin{itemize}
		\item \textbf{Creare un documento vuoto}
		\item Partire da un \textbf{modello}
	\end{itemize}
	
	\vspace{0.5cm}
	
	I \textbf{modelli} sono documenti già strutturati (per esempio lettere, relazioni, curriculum) che ti aiutano a non partire da zero.
	
	\begin{itemize}
		\item Permettono di lavorare più rapidamente
		\item Offrono una formattazione già ordinata e professionale
	\end{itemize}
\end{frame}

\begin{frame}{Salvare un documento}
	Salvare significa memorizzare il lavoro in modo permanente.
	
	\begin{itemize}
		\item La prima volta si utilizza \textbf{Salva con nome}
		\begin{itemize}
			\item Scegli la cartella di destinazione
			\item Inserisci il nome del file
			\item Decidi la posizione di salvataggio (computer o spazio online)
		\end{itemize}
		\item Dopo il primo salvataggio si usa \textbf{Salva}
		\begin{itemize}
			\item Aggiorna il file esistente
			\item Registra le ultime modifiche apportate
		\end{itemize}
	\end{itemize}
\end{frame}

\begin{frame}{Salvare con un altro nome}
	Il comando \textbf{Salva con nome} non si usa solo la prima volta.
	
	\begin{itemize}
		\item Permette di creare una copia del documento con un nome diverso
		\item È utile per conservare versioni successive dello stesso lavoro
		\begin{itemize}
			\item relazione\_v1
			\item relazione\_v2
			\item versione\_definitiva
		\end{itemize}
		\item Aiuta a non modificare o rovinare il file originale
	\end{itemize}
\end{frame}

\begin{frame}{Salvare in formati diversi}
	Un documento non deve essere salvato solo nel formato del programma.
	
	\begin{itemize}
		\item Può essere esportato in \textbf{PDF}
		\begin{itemize}
			\item Utile per l’invio ad altre persone
			\item Ideale per la stampa
			\item Evita modifiche accidentali al contenuto
		\end{itemize}
		\item Può essere salvato come \textbf{testo semplice (.txt)}
		\begin{itemize}
			\item Mantiene solo il contenuto testuale
			\item Rimuove ogni formattazione
		\end{itemize}
		\item La scelta del formato dipende dallo scopo del documento
	\end{itemize}
\end{frame}

	
\begin{frame}{Spostarsi tra documenti aperti}
	Spesso si lavora con più documenti aperti contemporaneamente.
	
	\begin{itemize}
		\item Utile, ad esempio, per copiare informazioni da un documento all’altro
		\item È possibile passare da un documento all’altro tramite:
		\begin{itemize}
			\item La barra delle applicazioni
			\item Il menu del programma
			\item Scorciatoie da tastiera come \textbf{Alt+Tab}
		\end{itemize}
		\item Sapersi spostare rapidamente tra le finestre aumenta la produttività
	\end{itemize}
\end{frame}

	%------------------------------------------------
	\section{Migliorare la produttività}
	
\begin{frame}{Impostazioni di base}
	Ogni programma permette di personalizzare alcune \textbf{impostazioni}.
	
	\begin{itemize}
		\item Inserire il proprio nome come autore dei documenti
		\item Scegliere cartelle predefinite per aprire e salvare file
		\item Impostare preferenze che semplificano il lavoro quotidiano
		\item Queste configurazioni fanno risparmiare tempo durante l’utilizzo del programma
	\end{itemize}
\end{frame}

	
\begin{frame}{Usare la Guida}
	La \textbf{guida integrata} è uno strumento spesso sottovalutato.
	
	\begin{itemize}
		\item Permette di cercare funzioni specifiche del programma
		\item Aiuta a capire a cosa serve un comando
		\item Fornisce procedure spiegate passo passo
		\item Favorisce l’autonomia, riducendo la necessità di chiedere aiuto
	\end{itemize}
\end{frame}

	
\begin{frame}{Zoom e ingrandimento}
	Lo \textbf{zoom} modifica solo la visualizzazione sullo schermo, non la dimensione reale del testo.
	
	\begin{itemize}
		\item Permette di ingrandire il documento per leggere meglio
		\item Consente di ridurre la visualizzazione per avere una visione d’insieme
		\item È utile durante la revisione di documenti lunghi
		\item Non altera il contenuto né la formattazione in stampa
	\end{itemize}
\end{frame}

	
\begin{frame}{Barre degli strumenti e barra multifunzione}
	L’interfaccia del programma può essere adattata alle proprie esigenze.
	
	\begin{itemize}
		\item È possibile mostrare o nascondere le \textbf{barre degli strumenti}
		\item Si può ridurre o espandere la barra multifunzione
		\item Avere più spazio a disposizione migliora la visibilità del documento
		\item Un ambiente di lavoro ordinato aiuta a mantenere la concentrazione sul contenuto
	\end{itemize}
\end{frame}

\begin{frame}{Navigare nel documento}
	Nei documenti lunghi non è efficiente scorrere pagina per pagina.
	
	\begin{itemize}
		\item Si possono usare scorciatoie da tastiera per spostarsi rapidamente
		\item Il \textbf{riquadro di navigazione} permette di visualizzare la struttura del testo
		\item L’\textbf{elenco dei titoli }consente di raggiungere subito le diverse sezioni
		\item Questo metodo è particolarmente utile in relazioni, tesine e documenti articolati
	\end{itemize}
\end{frame}

	
\begin{frame}{Lo strumento Vai}
	Il comando \textbf{Vai} permette di raggiungere rapidamente parti specifiche del documento.
	
	\begin{itemize}
		\item Consente di spostarsi direttamente a una pagina precisa
		\item Permette di raggiungere una sezione o un punto specifico del testo
		\item Evita la ricerca manuale scorrendo il documento
		\item Basta inserire un numero o un riferimento per arrivare subito alla destinazione desiderata
	\end{itemize}
\end{frame}

	
	%------------------------------------------------
	\section{Creazione di un documento}
	
\begin{frame}{Modalità di visualizzazione}
	Un \textbf{documento} può essere visualizzato in modi diversi a seconda del lavoro che si sta svolgendo.
	
	\begin{itemize}
		\item \textbf{Layout di stampa}
		\begin{itemize}
			\item Mostra il documento come apparirà su carta
			\item Permette di controllare margini, intestazioni e impaginazione
		\end{itemize}
		\item \textbf{Bozza}
		\begin{itemize}
			\item Presenta una visualizzazione più semplice
			\item Utile per concentrarsi principalmente sul testo
		\end{itemize}
		\item \textbf{Lettura}
		\begin{itemize}
			\item Ottimizzata per consultare il contenuto
			\item Riduce gli elementi di modifica a favore della leggibilità
		\end{itemize}
	\end{itemize}
\end{frame}

	
\begin{frame}{Cambiare visualizzazione}
	Dal menu \textbf{Visualizza} è possibile passare rapidamente da una modalità all’altra.
	
	\begin{itemize}
		\item Consente di scegliere la visualizzazione più adatta alla fase di lavoro
		\begin{itemize}
			\item \underline{Scrittura del testo}
			\item \underline{Revisione del documento}
			\item \underline{Lettura e consultazione}
		\end{itemize}
		\item Rendere adeguata la visualizzazione migliora comfort ed efficacia operativa
	\end{itemize}
\end{frame}

	
\begin{frame}{Inserire testo}
	Per inserire testo è necessario posizionare il cursore nel punto desiderato.
	
	\begin{itemize}
		\item Il cursore indica la posizione esatta in cui compariranno i nuovi caratteri
		\item È sufficiente iniziare a digitare per aggiungere contenuto
		\item Il testo può essere inserito in qualsiasi parte del documento
		\item Non è obbligatorio scrivere solo alla fine del testo esistente
	\end{itemize}
\end{frame}

	
\begin{frame}{Caratteri speciali e simboli}
	Non tutti i simboli sono direttamente presenti sulla tastiera.
	
	\begin{itemize}
		\item È possibile inserirli tramite il menu \textbf{Inserisci → Simbolo}
		\item Permette di aggiungere caratteri come ©, ® e ™
		\item Questi simboli sono comuni in documenti ufficiali, tecnici e commerciali
		\item Consente di arricchire il testo con segni non disponibili nei tasti standard
	\end{itemize}
\end{frame}

	
	%------------------------------------------------
	\section{Selezionare e modificare}
	
\begin{frame}{Caratteri non stampabili}
	I \textbf{caratteri non stampabili} non compaiono nel documento stampato, ma aiutano a comprendere la struttura del testo.
	
	\begin{itemize}
		\item Mostrano la presenza degli spazi
		\item Evidenziano i segni di fine paragrafo
		\item Rendono visibili tabulazioni e interruzioni di riga
		\item Sono utili quando il testo appare disordinato e non si individua la causa
	\end{itemize}
\end{frame}
\begin{frame}{Selezionare il testo}
	Prima di modificare il testo è necessario selezionarlo.
	
	\begin{itemize}
		\item Si può usare il mouse trascinando il cursore
		\item Un doppio clic seleziona rapidamente una parola
		\item Con la tastiera si utilizza \textbf{Shift} insieme ai tasti freccia
		\item Senza selezione, i comandi agiscono solo nel punto in cui si trova il cursore
	\end{itemize}
\end{frame}

\begin{frame}{Modificare il contenuto}
	Una volta selezionato, il testo può essere modificato in diversi modi.
	
	\begin{itemize}
		\item Aggiungere nuove parole o frasi
		\item Cancellare parti non corrette
		\item Sostituire porzioni di testo esistenti
		\item Queste operazioni sono fondamentali nella revisione dei documenti
	\end{itemize}
\end{frame}

\begin{frame}{Ricerca e sostituzione}
	Gli strumenti \textbf{Trova} e \textbf{Sostituisci} facilitano la modifica di testi estesi.
	
	\begin{itemize}
		\item Permettono di cercare parole o frasi in tutto il documento
		\item Consentono di sostituire automaticamente un termine con un altro
		\item Sono utili per correggere errori ripetuti
		\item Permettono di aggiornare nomi o termini in pochi secondi
	\end{itemize}
\end{frame}

\begin{frame}{Copiare, spostare e cancellare testo}
	I comandi di modifica permettono di riorganizzare facilmente il contenuto.
	
	\begin{itemize}
		\item \textbf{Copia} e \textbf{Incolla} duplicano parti di testo
		\item \textbf{Taglia} e \textbf{Incolla} spostano il testo in un altro punto o documento
		\item I tasti \textbf{Canc} e \textbf{Backspace} eliminano rapidamente i caratteri non desiderati
	\end{itemize}
\end{frame}

\begin{frame}{Annulla e Ripristina}
	I comandi di correzione consentono di gestire facilmente gli errori.
	
	\begin{itemize}
		\item \textbf{Annulla} torna indietro rispetto all’ultima operazione
		\item \textbf{Ripristina} riapplica un’azione precedentemente annullata
		\item Le scorciatoie \textbf{Ctrl+Z} e \textbf{Ctrl+Y} sono tra le più utilizzate
		\item Permettono di lavorare con maggiore sicurezza durante la modifica
	\end{itemize}
\end{frame}

	
	%------------------------------------------------
	
\end{document}
