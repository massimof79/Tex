\documentclass{beamer}
\usetheme{Madrid}
\usecolortheme{default}

\title{Elaborazione Testi: Competenze di Base}
\subtitle{Programma di videoscrittura}
\author{Prof. Fedeli Massimo - Fabbrica Digitale 4.0}
\date{}

\begin{document}
	
	\frame{\titlepage}
	
	%------------------------------------------------

	
	%------------------------------------------------
	\section{Lavorare con i documenti}
	
	\begin{frame}{Aprire e chiudere il programma}
		Un programma di elaborazione testi può essere avviato in vari modi: dal menu delle applicazioni, da un’icona sul desktop oppure aprendo direttamente un file di testo.
		
		Quando chiudi il programma usando \textbf{File → Esci}, il sistema controlla se ci sono documenti non salvati. In quel caso ti chiede se vuoi salvare: ignorare questo avviso può significare perdere tutto il lavoro fatto.
	\end{frame}
	
	\begin{frame}{Aprire e chiudere documenti}
		È importante distinguere tra \textbf{programma} e \textbf{documento}.  
		Il programma è lo strumento, il documento è il file su cui stai lavorando.
		
		Con \textbf{File → Apri} puoi cercare documenti già esistenti nelle cartelle del computer o nel cloud. Puoi anche chiudere solo il documento, lasciando aperto il programma per lavorare su un altro file.
	\end{frame}
	
	\begin{frame}{Creare un nuovo documento}
		Quando inizi un nuovo lavoro puoi \textbf{creare un documento vuoto} oppure partire da un modello.
		
		I \textbf{modelli} sono documenti già strutturati (per esempio lettere, relazioni, curriculum) che ti aiutano a non partire da zero. Usarli significa lavorare in modo più rapido e con una formattazione già ordinata.
	\end{frame}
	
	\begin{frame}{Salvare un documento}
		Salvare significa memorizzare il lavoro in modo permanente.
		
		La prima volta si usa \textbf{Salva con nome}: devi scegliere la cartella, il nome del file e la posizione (computer o spazio online). Dopo il primo salvataggio, il comando \textbf{Salva} aggiorna semplicemente il file con le ultime modifiche.
	\end{frame}
	
	\begin{frame}{Salvare con un altro nome}
		Il comando \textbf{Salva con nome} non serve solo la prima volta.  
		Puoi usarlo per creare una copia del documento con un nome diverso.
		
		È utile, ad esempio, per creare versioni successive dello stesso lavoro: relazione\_v1, relazione\_v2, versione\_definitiva. In questo modo non rischi di rovinare il file originale.
	\end{frame}
	
	\begin{frame}{Salvare in formati diversi}
		Un documento non deve essere per forza salvato solo nel formato del programma.
		
		Puoi esportarlo in \textbf{PDF} per inviarlo o stamparlo senza che venga modificato, oppure in formato testo semplice (.txt) quando ti serve solo il contenuto senza formattazione. Scegliere il formato giusto dipende dallo scopo del documento.
	\end{frame}
	
	\begin{frame}{Spostarsi tra documenti aperti}
		Spesso si lavora con più documenti aperti contemporaneamente, ad esempio per copiare informazioni da uno all’altro.
		
		Puoi passare da un documento all’altro usando la barra delle applicazioni, il menu del programma oppure scorciatoie da tastiera come \textbf{Alt+Tab}. Saperlo fare velocemente aumenta molto la produttività.
	\end{frame}
	
	%------------------------------------------------
	\section{Migliorare la produttività}
	
	\begin{frame}{Impostazioni di base}
		Ogni programma permette di personalizzare alcune \textbf{impostazioni}.
		
		Puoi inserire il tuo nome come autore dei documenti e scegliere cartelle predefinite per aprire e salvare file. Queste piccole configurazioni fanno risparmiare tempo ogni volta che lavori.
	\end{frame}
	
	\begin{frame}{Usare la Guida}
		La guida integrata è uno strumento spesso sottovalutato.
		
		Permette di cercare funzioni, capire a cosa serve un comando e seguire procedure passo passo. Imparare a usare la guida significa diventare più autonomi, senza dover sempre chiedere aiuto.
	\end{frame}
	
	\begin{frame}{Zoom e ingrandimento}
		Lo zoom modifica solo la visualizzazione sullo schermo, non la dimensione reale del testo.
		
		Puoi ingrandire per leggere meglio o ridurre per avere una visione d’insieme della pagina. È uno strumento utile quando si rivede un documento lungo.
	\end{frame}
	
	\begin{frame}{Barre degli strumenti e barra multifunzione}
		L’interfaccia del programma può essere adattata alle tue esigenze.
		
		Puoi mostrare o nascondere barre degli strumenti e ridurre la barra multifunzione per avere più spazio per il documento. Un ambiente ordinato aiuta a concentrarsi meglio sul contenuto.
	\end{frame}
	
	\begin{frame}{Navigare nel documento}
		Nei documenti lunghi non è efficiente scorrere pagina per pagina.
		
		Si possono usare scorciatoie da tastiera, il riquadro di navigazione o l’elenco dei titoli per spostarsi rapidamente tra le diverse parti del testo. Questo è particolarmente utile nelle relazioni e nelle tesine.
	\end{frame}
	
	\begin{frame}{Lo strumento Vai}
		Il comando \textbf{Vai} permette di raggiungere direttamente una pagina, una sezione o un punto specifico del documento.
		
		Invece di cercare manualmente, si inserisce il numero o il riferimento e il programma porta subito nel punto desiderato.
	\end{frame}
	
	%------------------------------------------------
	\section{Creazione di un documento}
	
	\begin{frame}{Modalità di visualizzazione}
		Un documento può essere visualizzato in modi diversi a seconda del lavoro che stai facendo.
		
		La modalità \textbf{Layout di stampa} mostra il documento come apparirà su carta. La modalità \textbf{Bozza} è più semplice e comoda per concentrarsi solo sul testo. La modalità \textbf{Lettura} è utile per consultare il contenuto.
	\end{frame}
	
	\begin{frame}{Cambiare visualizzazione}
		Dal menu \textbf{Visualizza} puoi passare rapidamente da una modalità all’altra.
		
		Scegliere la visualizzazione più adatta alla fase di lavoro (scrittura, revisione o lettura) rende l’attività più comoda ed efficace.
	\end{frame}
	
	\begin{frame}{Inserire testo}
		Per inserire testo devi posizionare il cursore nel punto desiderato e iniziare a digitare.
		
		Il cursore indica esattamente dove compariranno i nuovi caratteri. Puoi aggiungere testo in qualsiasi parte del documento, non solo alla fine.
	\end{frame}
	
	\begin{frame}{Caratteri speciali e simboli}
		Non tutti i simboli sono presenti sulla tastiera.
		
		Attraverso il menu \textbf{Inserisci → Simbolo} puoi aggiungere caratteri come ©, ® o ™. Questi simboli sono spesso usati in documenti ufficiali o tecnici.
	\end{frame}
	
	%------------------------------------------------
	\section{Selezionare e modificare}
	
	\begin{frame}{Caratteri non stampabili}
		I caratteri non stampabili non compaiono sulla carta, ma aiutano a capire la struttura del testo.
		
		Mostrano spazi, fine paragrafo, tabulazioni e interruzioni di riga. Visualizzarli è molto utile quando un testo “sembra disordinato” e non si capisce il motivo.
	\end{frame}
	
	\begin{frame}{Selezionare il testo}
		Prima di modificare il testo bisogna selezionarlo.
		
		Puoi farlo con il mouse trascinando il cursore, con un doppio clic per selezionare una parola o con la tastiera usando Shift e i tasti freccia. Senza selezione, i comandi agiscono solo nel punto del cursore.
	\end{frame}
	
	\begin{frame}{Modificare il contenuto}
		Una volta selezionato, il testo può essere modificato in diversi modi: aggiungendo nuove parole, cancellando parti non corrette o sostituendo frasi.
		
		Queste operazioni sono alla base della revisione di qualsiasi documento.
	\end{frame}
	
	\begin{frame}{Ricerca e sostituzione}
		Gli strumenti \textbf{Trova} e \textbf{Sostituisci} permettono di cercare parole o frasi in tutto il documento.
		
		Sono molto utili quando devi correggere un termine ripetuto più volte o cambiare un nome in tutto il testo in pochi secondi.
	\end{frame}
	
	\begin{frame}{Copiare, spostare e cancellare testo}
		Con \textbf{Copia} e \textbf{Incolla} puoi duplicare parti di testo.  
		Con \textbf{Taglia} e \textbf{Incolla} puoi spostarle in un altro punto o in un altro documento.
		
		I tasti Canc e Backspace permettono invece di eliminare rapidamente i caratteri non desiderati.
	\end{frame}
	
	\begin{frame}{Annulla e Ripristina}
		Gli errori capitano a tutti. Per questo esistono i comandi \textbf{Annulla} e \textbf{Ripristina}.
		
		Annulla torna indietro rispetto all’ultima operazione, mentre Ripristina riapplica un’azione annullata. Le scorciatoie Ctrl+Z e Ctrl+Y sono tra le più usate in assoluto.
	\end{frame}
	
	%------------------------------------------------
	
\end{document}
