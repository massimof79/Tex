\documentclass[12pt,a4paper]{article}
\usepackage[utf8]{inputenc}
\usepackage[italian]{babel}
\usepackage[margin=2.5cm]{geometry}
\usepackage{tikz}
\usepackage{xcolor}
\usepackage{enumitem}
\usepackage{amsmath}
\usepackage{amssymb}
\usetikzlibrary{shapes,arrows,positioning,shadows}

% Definizione colori
\definecolor{entitycolor}{RGB}{230,230,250}
\definecolor{relationcolor}{RGB}{255,250,205}
\definecolor{highlightcolor}{RGB}{255,255,153}
\definecolor{bluetext}{RGB}{0,102,204}

% Stile per entità
\tikzstyle{entity} = [rectangle, draw, fill=entitycolor, text width=3.5cm, 
                      text centered, minimum height=1.2cm, drop shadow]
% Stile per relazioni
\tikzstyle{relation} = [diamond, draw, fill=relationcolor, text width=2.5cm, 
                        text centered, minimum height=1cm, aspect=2, drop shadow]

\title{\textbf{\LARGE Cardinalità delle relazioni}}
\author{Prof. Fedeli Massimo - ITS 4.0}
\date{}

\begin{document}

\maketitle

\section{Uno-a-uno (1, 1)}

\colorbox{highlightcolor}{\parbox{\dimexpr\textwidth-2\fboxsep}{%
La relazione è \textbf{uno-a-uno} se la cardinalità massima di entrambe le entità è 1.
}}

\vspace{1cm}

\begin{center}
\begin{tikzpicture}[node distance=4cm]
    % Primo schema con cardinalità
    \node[entity] (persona1) {persona};
    \node[relation, right of=persona1] (possiede1) {possiede};
    \node[entity, right of=possiede1] (codice1) {codice\_fiscale};
    
    \draw (persona1) -- node[above] {(1, 1)} (possiede1);
    \draw (possiede1) -- node[above] {(1, 1)} (codice1);
    
    % Freccia rossa
    \node[below of=possiede1, node distance=1.5cm] (arrow) {};
    \draw[->, line width=3pt, red] (possiede1) -- (arrow);
    
    % Secondo schema semplificato
    \node[entity, below of=persona1, node distance=3.5cm] (persona2) {persona};
    \node[relation, right of=persona2] (possiede2) {possiede};
    \node[entity, right of=possiede2] (codice2) {codice\_fiscale};
    
    \draw (persona2) -- (possiede2);
    \draw (possiede2) -- (codice2);
\end{tikzpicture}
\end{center}

\vspace{0.5cm}

Viene indicata come (1, 1) e, generalmente, come prima anticipato, è sottointesa e, quindi, omessa.

\newpage

\section{Uno-a-molti (1, n)}

\colorbox{highlightcolor}{\parbox{\dimexpr\textwidth-2\fboxsep}{%
La relazione è \textbf{uno-a-molti} se la massima cardinalità verso una entità è 1 e la massima cardinalità verso l'altra entità è n.
}}

\vspace{1cm}

\begin{center}
\begin{tikzpicture}[node distance=4.5cm]
    \node[entity] (persona) {persona};
    \node[relation, right of=persona] (risiede) {risiede};
    \node[entity, right of=risiede] (citta) {città};
    
    \draw (persona) -- node[above] {(1, 1)} (risiede);
    \draw (risiede) -- node[above] {(1, n)} (citta);
\end{tikzpicture}
\end{center}

\vspace{1.5cm}

\section{Molti-a-molti (n, n)}

\colorbox{highlightcolor}{\parbox{\dimexpr\textwidth-2\fboxsep}{%
La relazione è \textbf{molti-a-molti} se la massima cardinalità verso entrambe le entità è n.
}}

\vspace{1cm}

\begin{center}
\begin{tikzpicture}[node distance=4.5cm]
    \node[entity] (prodotto) {prodotto};
    \node[relation, right of=prodotto] (venduto) {è venduto con};
    \node[entity, right of=venduto] (fattura) {fattura};
    
    \draw (prodotto) -- node[above] {(1, n)} (venduto);
    \draw (venduto) -- node[above] {(1, n)} (fattura);
\end{tikzpicture}
\end{center}

\newpage

\section{Esempi pratici}

\subsection{Persona-Automobile}

\begin{itemize}[leftmargin=*]
    \item \textcolor{bluetext}{minima cardinalità} \textbf{(persona, proprietario)} = 0: esistono persone che non posseggono alcuna automobile;
    \item \textcolor{bluetext}{massima cardinalità} \textbf{(persona, proprietario)} = n: ogni persona può essere proprietaria di un numero arbitrario di automobili;
    \item \textcolor{bluetext}{minima cardinalità} \textbf{(automobile, proprietario)} = 0: esistono automobili non possedute da alcuna persona;
    \item \textcolor{bluetext}{massima cardinalità} \textbf{(automobile, proprietario)} = 1: ogni automobile può avere al più un proprietario.
\end{itemize}

\vspace{0.5cm}

Quindi lo schema completo di vincoli è il seguente:

\vspace{0.5cm}

\begin{center}
\begin{tikzpicture}[node distance=4.5cm]
    \node[entity] (persona) {persona};
    \node[relation, right of=persona] (proprietario) {proprietario};
    \node[entity, right of=proprietario] (automobile) {automobile};
    
    \draw (persona) -- node[above] {\textcolor{red}{(0, n)}} (proprietario);
    \draw (proprietario) -- node[above] {\textcolor{red}{(0, 1)}} (automobile);
\end{tikzpicture}
\end{center}

\vspace{1cm}

\subsection{Persona-Comune}

\begin{itemize}[leftmargin=*]
    \item a ogni \textcolor{bluetext}{\textbf{persona}} viene associato \textcolor{red}{(1)} un solo \textcolor{red}{(1)} \textcolor{bluetext}{\textbf{comune}} di nascita;
    \item a ogni \textcolor{bluetext}{\textbf{comune}} possono essere associate da \textcolor{red}{0} a \textcolor{red}{n} nascite di \textcolor{bluetext}{\textbf{persone}}.
\end{itemize}

\vspace{0.5cm}

\begin{center}
\begin{tikzpicture}[node distance=4.5cm]
    \node[entity] (persona) {persona};
    \node[relation, right of=persona] (nascita) {nascita};
    \node[entity, right of=nascita] (comune) {comune};
    
    \draw (persona) -- node[above] {(1, 1)} (nascita);
    \draw (nascita) -- node[above] {\textcolor{red}{(0, n)}} (comune);
\end{tikzpicture}
\end{center}

\newpage

\subsection{Turista-Vacanza}

\begin{itemize}[leftmargin=*]
    \item ogni \textcolor{bluetext}{\textbf{turista}} può effettuare \textcolor{red}{nessuna (0)} oppure \textcolor{red}{alcune (n)} prenotazioni di una \textcolor{bluetext}{\textbf{vacanza}}.
    \item a ogni \textcolor{bluetext}{\textbf{vacanza}} possono essere associate \textcolor{red}{nessuna (0)} a \textcolor{red}{tante (n)} prenotazioni di \textcolor{bluetext}{\textbf{turisti}}.
\end{itemize}

\vspace{0.5cm}

\begin{center}
\begin{tikzpicture}[node distance=4.5cm]
    \node[entity] (turista) {turista};
    \node[relation, right of=turista] (prenota) {prenota};
    \node[entity, right of=prenota] (vacanza) {vacanza};
    
    \draw (turista) -- node[above] {\textcolor{red}{(0, n)}} (prenota);
    \draw (prenota) -- node[above] {\textcolor{red}{(0, n)}} (vacanza);
\end{tikzpicture}
\end{center}

\vspace{1cm}

\end{document}
