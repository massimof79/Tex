\documentclass[12pt,a4paper]{article}
\usepackage[utf8]{inputenc}
\usepackage[italian]{babel}
\usepackage[T1]{fontenc}
\usepackage{amsmath}
\usepackage{amsfonts}
\usepackage{amssymb}
\usepackage{graphicx}
\usepackage{geometry}
\usepackage{enumitem}
\usepackage{hyperref}
\usepackage{xcolor}

\geometry{left=2.5cm,right=2.5cm,top=2.5cm,bottom=2.5cm}

\hypersetup{
	colorlinks=true,
	linkcolor=blue,
	filecolor=magenta,      
	urlcolor=cyan,
}

\title{\textbf{Regole di Derivazione dal Modello Concettuale al Modello Logico}}
\author{Prof. Fedeli Massimo\\
	ITS Fabbrica 4.0 Tutti i diritti riservati}
\date{}

\begin{document}
	
	\maketitle
	
	\section*{Introduzione}
	La trasformazione dal modello concettuale (schema Entità-Relazione) al modello logico (schema relazionale) è un processo fondamentale nella progettazione di basi di dati. Questo documento descrive le regole sistematiche per effettuare questa conversione.
	
	\section{Traduzione delle Entità}
	
	\textbf{Regola Base:} Ogni entità del modello E-R diventa una relazione (tabella) nel modello logico.
	
	\subsection{Componenti della traduzione}
	\begin{itemize}
		\item Il \textbf{nome dell'entità} diventa il nome della relazione
		\item Gli \textbf{attributi dell'entità} diventano attributi (colonne) della relazione
		\item Gli \textbf{attributi identificatori} (chiave primaria) dell'entità diventano la chiave primaria della relazione
	\end{itemize}
	
	\subsection{Esempio}
	\textbf{Modello E-R:}
	
	\texttt{STUDENTE (Matricola, Nome, Cognome, DataNascita)}
	\begin{itemize}[label={}]
		\item Chiave: Matricola
	\end{itemize}
	
	\textbf{Modello Logico:}
	
	\texttt{STUDENTE (Matricola, Nome, Cognome, DataNascita)}
	\begin{itemize}[label={}]
		\item \textbf{PK}: Matricola
	\end{itemize}
	
	\section{Traduzione delle Relazioni (Associazioni)}
	
	La traduzione delle relazioni dipende dalla loro cardinalità.
	
	\subsection{Relazioni Uno a Molti (1:N)}
	
	\textbf{Regola:} Si propaga la chiave primaria dell'entità con cardinalità (1,1) o (0,1) nell'entità con cardinalità (0,N) o (1,N) come chiave esterna.
	
	\textbf{Procedura:}
	\begin{enumerate}
		\item Non si crea una nuova tabella per la relazione
		\item Si aggiunge la chiave primaria del lato "uno" come chiave esterna nel lato "molti"
		\item Gli eventuali attributi della relazione vengono aggiunti all'entità sul lato "molti"
	\end{enumerate}
	
	\textbf{Esempio:}
	
	\textbf{Modello E-R:}
	
	\texttt{DIPARTIMENTO (CodDip, Nome)}
	
	\texttt{IMPIEGATO (CodImp, Nome, Stipendio)}
	
	\texttt{APPARTIENE: DIPARTIMENTO (1,1) --- (0,N) IMPIEGATO}
	
	\textbf{Modello Logico:}
	
	\texttt{DIPARTIMENTO (CodDip, Nome)}
	\begin{itemize}[label={}]
		\item PK: CodDip
	\end{itemize}
	
	\texttt{IMPIEGATO (CodImp, Nome, Stipendio, CodDip)}
	\begin{itemize}[label={}]
		\item PK: CodImp
		\item FK: CodDip REFERENCES DIPARTIMENTO
	\end{itemize}
	
	\subsection{Relazioni Molti a Molti (N:M)}
	
	\textbf{Regola:} Si crea una nuova relazione (tabella) che rappresenta l'associazione.
	
	\textbf{Procedura:}
	\begin{enumerate}
		\item Si crea una nuova tabella per la relazione
		\item La chiave primaria è composta dalle chiavi primarie delle due entità coinvolte
		\item Si aggiungono le chiavi esterne verso entrambe le entità
		\item Gli eventuali attributi della relazione diventano attributi della nuova tabella
	\end{enumerate}
	
	\textbf{Esempio:}
	
	\textbf{Modello E-R:}
	
	\texttt{STUDENTE (Matricola, Nome)}
	
	\texttt{CORSO (CodCorso, Titolo)}
	
	\texttt{FREQUENTA: STUDENTE (0,N) --- (0,N) CORSO}
	\begin{itemize}[label={}]
		\item Attributo: Voto
	\end{itemize}
	
	\textbf{Modello Logico:}
	
	\texttt{STUDENTE (Matricola, Nome)}
	\begin{itemize}[label={}]
		\item PK: Matricola
	\end{itemize}
	
	\texttt{CORSO (CodCorso, Titolo)}
	\begin{itemize}[label={}]
		\item PK: CodCorso
	\end{itemize}
	
	\texttt{FREQUENTA (Matricola, CodCorso, Voto)}
	\begin{itemize}[label={}]
		\item PK: (Matricola, CodCorso)
		\item FK: Matricola REFERENCES STUDENTE
		\item FK: CodCorso REFERENCES CORSO
	\end{itemize}
	
	\subsection{Relazioni Uno a Uno (1:1)}
	
	\textbf{Regola:} Esistono tre alternative, da scegliere in base al contesto.
	
	\textbf{Alternativa 1 - Accorpamento:} Unire le due entità in un'unica tabella (se logicamente coerente).
	
	\textbf{Alternativa 2 - Chiave esterna lato opzionale:} Propagare la chiave dell'entità con partecipazione obbligatoria (1,1) nell'entità con partecipazione opzionale (0,1).
	
	\textbf{Alternativa 3 - Tabella separata:} Creare una tabella separata per la relazione (raramente usato).
	
	\textbf{Esempio (Alternativa 2):}
	
	\textbf{Modello E-R:}
	
	\texttt{PERSONA (CF, Nome, Cognome)}
	
	\texttt{PASSAPORTO (NumPassaporto, DataRilascio, Scadenza)}
	
	\texttt{POSSIEDE: PERSONA (0,1) --- (1,1) PASSAPORTO}
	
	\textbf{Modello Logico:}
	
	\texttt{PERSONA (CF, Nome, Cognome)}
	\begin{itemize}[label={}]
		\item PK: CF
	\end{itemize}
	
	\texttt{PASSAPORTO (NumPassaporto, DataRilascio, Scadenza, CF)}
	\begin{itemize}[label={}]
		\item PK: NumPassaporto
		\item FK: CF REFERENCES PERSONA
	\end{itemize}
	
	\subsection{Relazioni Ternarie (e n-arie)}
	
	\textbf{Regola:} Si crea sempre una nuova relazione (tabella) che contiene le chiavi esterne di tutte le entità coinvolte.
	
	\textbf{Procedura:}
	\begin{enumerate}
		\item Si crea una nuova tabella per la relazione ternaria
		\item Si aggiungono le chiavi esterne verso tutte e tre (o più) le entità coinvolte
		\item La chiave primaria è generalmente composta da tutte le chiavi esterne (salvo vincoli specifici di cardinalità)
		\item Gli eventuali attributi della relazione diventano attributi della nuova tabella
	\end{enumerate}
	
	\textbf{Note sulla chiave primaria:}
	\begin{itemize}
		\item Se tutte le cardinalità sono (0,N) o (1,N), la chiave primaria è composta da tutte le chiavi esterne
		\item Se una o più entità hanno cardinalità (0,1) o (1,1), potrebbero non far parte della chiave primaria
		\item La scelta dipende dall'analisi semantica della relazione
	\end{itemize}
	
	\textbf{Esempio:}
	
	\textbf{Modello E-R:}
	
	\texttt{FORNITORE (CodFornitore, Nome, Città)}
	
	\texttt{PRODOTTO (CodProdotto, Descrizione, Prezzo)}
	
	\texttt{PROGETTO (CodProgetto, Titolo, Budget)}
	
	\texttt{FORNITURA: FORNITORE (0,N) --- (0,N) PRODOTTO --- (0,N) PROGETTO}
	\begin{itemize}[label={}]
		\item Attributi: Quantità, DataFornitura
	\end{itemize}
	
	\textit{Nota: Un fornitore può fornire lo stesso prodotto a progetti diversi, e uno stesso prodotto può essere fornito da fornitori diversi allo stesso progetto.}
	
	\textbf{Modello Logico:}
	
	\texttt{FORNITORE (CodFornitore, Nome, Città)}
	\begin{itemize}[label={}]
		\item PK: CodFornitore
	\end{itemize}
	
	\texttt{PRODOTTO (CodProdotto, Descrizione, Prezzo)}
	\begin{itemize}[label={}]
		\item PK: CodProdotto
	\end{itemize}
	
	\texttt{PROGETTO (CodProgetto, Titolo, Budget)}
	\begin{itemize}[label={}]
		\item PK: CodProgetto
	\end{itemize}
	
	\texttt{FORNITURA (CodFornitore, CodProdotto, CodProgetto, Quantità, DataFornitura)}
	\begin{itemize}[label={}]
		\item PK: (CodFornitore, CodProdotto, CodProgetto)
		\item FK: CodFornitore REFERENCES FORNITORE
		\item FK: CodProdotto REFERENCES PRODOTTO
		\item FK: CodProgetto REFERENCES PROGETTO
	\end{itemize}
	
	\textbf{Esempio con cardinalità particolari:}
	
	Se nel modello E-R la cardinalità di PROGETTO fosse (0,1), indicando che ogni combinazione fornitore-prodotto può essere associata al massimo a un progetto, la chiave primaria cambierebbe:
	
	\texttt{FORNITURA (CodFornitore, CodProdotto, CodProgetto, Quantità, DataFornitura)}
	\begin{itemize}[label={}]
		\item PK: (CodFornitore, CodProdotto)
		\item FK: CodFornitore REFERENCES FORNITORE
		\item FK: CodProdotto REFERENCES PRODOTTO
		\item FK: CodProgetto REFERENCES PROGETTO
	\end{itemize}
	
	\section{Traduzione delle Generalizzazioni (Gerarchie IS-A)}
	
	Esistono tre strategie principali per tradurre le gerarchie di generalizzazione.
	
	\subsection{Strategia della Relazione Unica}
	
	\textbf{Descrizione:} Si crea un'unica tabella che contiene tutti gli attributi dell'entità padre e di tutte le entità figlie.
	
	\textbf{Vantaggi:} Semplice, nessun join necessario per recuperare i dati.
	
	\textbf{Svantaggi:} Molti valori NULL se le sottoclassi hanno molti attributi specifici.
	
	\textbf{Quando usarla:} Quando le sottoclassi hanno pochi attributi specifici.
	
	\textbf{Esempio:}
	
	\textbf{Modello E-R:}
	
	\texttt{PERSONA (CF, Nome, Cognome)}
	\begin{itemize}[label={}]
		\item |---STUDENTE (Matricola, CorsoLaurea)
		\item |---DOCENTE (Dipartimento, Ruolo)
	\end{itemize}
	
	\textbf{Modello Logico:}
	
	\texttt{PERSONA (CF, Nome, Cognome, TipoPersona, Matricola, CorsoLaurea, Dipartimento, Ruolo)}
	\begin{itemize}[label={}]
		\item PK: CF
		\item TipoPersona: discriminante per identificare il tipo
	\end{itemize}
	
	\subsection{Strategia delle Relazioni Separate}
	
	\textbf{Descrizione:} Si crea una tabella per l'entità padre e una tabella separata per ogni entità figlia.
	
	\textbf{Vantaggi:} Nessun valore NULL, struttura pulita.
	
	\textbf{Svantaggi:} Richiede join per recuperare informazioni complete.
	
	\textbf{Quando usarla:} Quando le sottoclassi hanno molti attributi specifici.
	
	\textbf{Esempio:}
	
	\textbf{Modello E-R:}
	
	\texttt{PERSONA (CF, Nome, Cognome)}
	\begin{itemize}[label={}]
		\item |--- STUDENTE (Matricola, CorsoLaurea)
		\item |---DOCENTE (Dipartimento, Ruolo)
	\end{itemize}
	
	\textbf{Modello Logico:}
	
	\texttt{PERSONA (CF, Nome, Cognome)}
	\begin{itemize}[label={}]
		\item PK: CF
	\end{itemize}
	
	\texttt{STUDENTE (CF, Matricola, CorsoLaurea)}
	\begin{itemize}[label={}]
		\item PK: CF
		\item FK: CF REFERENCES PERSONA
	\end{itemize}
	
	\texttt{DOCENTE (CF, Dipartimento, Ruolo)}
	\begin{itemize}[label={}]
		\item PK: CF
		\item FK: CF REFERENCES PERSONA
	\end{itemize}
	
	\subsection{Strategia delle Sottoclassi Indipendenti}
	
	\textbf{Descrizione:} Si crea solo una tabella per ogni sottoclasse, duplicando gli attributi del padre.
	
	\textbf{Vantaggi:} Nessun join necessario, nessun NULL.
	
	\textbf{Svantaggi:} Ridondanza degli attributi del padre, problemi di integrità.
	
	\textbf{Quando usarla:} Quando la generalizzazione è \textbf{totale e disgiunta}, e si accede raramente all'entità padre.
	
	\textbf{Esempio:}
	
	\textbf{Modello E-R:}
	
	\texttt{PERSONA (CF, Nome, Cognome)}
	\begin{itemize}[label={}]
		\item |---STUDENTE (Matricola, CorsoLaurea)
		\item |---DOCENTE (Dipartimento, Ruolo)
	\end{itemize}
	
	\textbf{Modello Logico:}
	
	\texttt{STUDENTE (CF, Nome, Cognome, Matricola, CorsoLaurea)}
	\begin{itemize}[label={}]
		\item PK: CF
	\end{itemize}
	
	\texttt{DOCENTE (CF, Nome, Cognome, Dipartimento, Ruolo)}
	\begin{itemize}[label={}]
		\item PK: CF
	\end{itemize}
	
	\section{Traduzione degli Attributi}
	
	\subsection{Attributi Semplici}
	Diventano direttamente attributi della relazione.
	
	\subsection{Attributi Composti}
	Si decompongono nei loro componenti atomici.
	
	\textbf{Esempio:}
	
	Indirizzo (Via, Città, CAP) $\rightarrow$ Via, Città, CAP
	
	\subsection{Attributi Multivalore}
	Si creano tabelle separate con chiave esterna verso l'entità principale.
	
	\textbf{Esempio:}
	
	\textbf{Modello E-R:}
	
	\texttt{PERSONA (CF, Nome, \{Telefono\})}
	
	\textbf{Modello Logico:}
	
	\texttt{PERSONA (CF, Nome)}
	\begin{itemize}[label={}]
		\item PK: CF
	\end{itemize}
	
	\texttt{TELEFONO (CF, NumeroTelefono)}
	\begin{itemize}[label={}]
		\item PK: (CF, NumeroTelefono)
		\item FK: CF REFERENCES PERSONA
	\end{itemize}
	
	\subsection{Attributi Derivati}
	Generalmente non vengono memorizzati (si calcolano quando necessario), a meno che non ci siano esigenze di prestazioni.
	
	\section{Vincoli di Integrità}
	
	Durante la traduzione è importante preservare tutti i vincoli:
	
	\subsection{Vincoli di Chiave}
	\begin{itemize}
		\item Chiave primaria (PRIMARY KEY)
		\item Chiave candidata (UNIQUE)
	\end{itemize}
	
	\subsection{Vincoli di Integrità Referenziale}
	\begin{itemize}
		\item Chiavi esterne (FOREIGN KEY)
		\item Politiche di cancellazione e aggiornamento (CASCADE, SET NULL, RESTRICT)
	\end{itemize}
	
	\subsection{Vincoli di Dominio}
	\begin{itemize}
		\item NOT NULL per partecipazioni obbligatorie
		\item CHECK per vincoli sui valori
	\end{itemize}
	
	\subsection{Vincoli di Cardinalità}
	Alcuni vincoli di cardinalità complessi potrebbero richiedere trigger o asserzioni.
	
	\section*{Conclusione}
	La derivazione dal modello concettuale al modello logico è un processo sistematico che, se seguito correttamente, garantisce uno schema relazionale corretto, efficiente e fedele al modello concettuale di partenza. La conoscenza approfondita di queste regole è fondamentale per ogni progettista di basi di dati.
	
\end{document}