\documentclass{beamer}

% Tema e colori
\usetheme{Madrid}
\usecolortheme{beaver}

% Pacchetti
\usepackage[utf8]{inputenc}
\usepackage[italian]{babel}
\usepackage{listings}
\usepackage{amsmath}
\usepackage{graphicx}

% Configurazione listings per Python
\lstset{
    language=Python,
    basicstyle=\ttfamily\small,
    keywordstyle=\color{blue},
    stringstyle=\color{red},
    commentstyle=\color{green!60!black},
    numbers=left,
    numberstyle=\tiny,
    frame=single,
    breaklines=true
}

% Informazioni documento
\title{Introduzione ai Database}
\subtitle{Modello Relazionale e SQL}
\author{Prof. Esempio}
\institute{IIS Fermi Sacconi Ceci\\Ascoli Piceno}
\date{\today}

% Rimuovi simboli di navigazione
\setbeamertemplate{navigation symbols}{}

\begin{document}

% ===== SLIDE TITOLO =====
\frame{\titlepage}

% ===== INDICE =====
\begin{frame}
\frametitle{Contenuti}
\tableofcontents
\end{frame}

% ===== SEZIONE 1: INTRODUZIONE =====
\section{Introduzione}

\begin{frame}
\frametitle{Cos'è un Database?}

\begin{definition}
Un \textbf{database} è una collezione organizzata e strutturata di dati, gestita attraverso un DBMS (Database Management System).
\end{definition}

\pause

\begin{block}{Caratteristiche principali}
\begin{itemize}
    \item<2-> Persistenza dei dati
    \item<3-> Accesso concorrente
    \item<4-> Integrità e sicurezza
    \item<5-> Efficienza nelle query
\end{itemize}
\end{block}

\end{frame}

\begin{frame}
\frametitle{Tipi di Database}

\begin{columns}
    \column{0.5\textwidth}
    \textbf{Relazionali (SQL)}
    \begin{itemize}
        \item MySQL
        \item PostgreSQL
        \item Oracle
        \item SQLite
    \end{itemize}
    
    \column{0.5\textwidth}
    \textbf{NoSQL}
    \begin{itemize}
        \item MongoDB
        \item Redis
        \item Cassandra
        \item Neo4j
    \end{itemize}
\end{columns}

\vspace{1cm}

\alert{Focus del corso: Database Relazionali}

\end{frame}

% ===== SEZIONE 2: MODELLO RELAZIONALE =====
\section{Modello Relazionale}

\begin{frame}
\frametitle{Concetti Fondamentali}

\begin{exampleblock}{Relazione (Tabella)}
Una tabella composta da righe (tuple) e colonne (attributi)
\end{exampleblock}

\pause

\begin{alertblock}{Chiave Primaria}
Attributo (o insieme di attributi) che identifica univocamente ogni riga
\end{alertblock}

\pause

\begin{block}{Chiave Esterna}
Attributo che referenzia la chiave primaria di un'altra tabella
\end{block}

\end{frame}

\begin{frame}
\frametitle{Esempio: Schema Database Scuola}

\begin{table}
\centering
\begin{tabular}{|l|l|l|}
\hline
\multicolumn{3}{|c|}{\textbf{STUDENTI}} \\
\hline
\textbf{ID} & \textbf{Nome} & \textbf{Classe} \\
\hline
1 & Mario Rossi & 5A \\
2 & Laura Bianchi & 5B \\
3 & Giuseppe Verdi & 5A \\
\hline
\end{tabular}
\end{table}

\vspace{0.5cm}

\begin{table}
\centering
\begin{tabular}{|l|l|l|l|}
\hline
\multicolumn{4}{|c|}{\textbf{VOTI}} \\
\hline
\textbf{ID} & \textbf{ID\_Studente} & \textbf{Materia} & \textbf{Voto} \\
\hline
1 & 1 & Informatica & 8 \\
2 & 1 & Matematica & 7 \\
3 & 2 & Informatica & 9 \\
\hline
\end{tabular}
\end{table}

\end{frame}

% ===== SEZIONE 3: SQL =====
\section{Linguaggio SQL}

\begin{frame}
\frametitle{SQL: Structured Query Language}

SQL è suddiviso in diverse categorie:

\begin{enumerate}
    \item \textbf{DDL} - Data Definition Language
    \begin{itemize}
        \item CREATE, ALTER, DROP
    \end{itemize}
    
    \item \textbf{DML} - Data Manipulation Language
    \begin{itemize}
        \item SELECT, INSERT, UPDATE, DELETE
    \end{itemize}
    
    \item \textbf{DCL} - Data Control Language
    \begin{itemize}
        \item GRANT, REVOKE
    \end{itemize}
\end{enumerate}

\end{frame}

\begin{frame}[fragile]
\frametitle{Creazione Tabella (DDL)}

\begin{lstlisting}[language=SQL]
CREATE TABLE studenti (
    id INTEGER PRIMARY KEY,
    nome VARCHAR(50) NOT NULL,
    cognome VARCHAR(50) NOT NULL,
    classe VARCHAR(10),
    data_nascita DATE
);
\end{lstlisting}

\pause

\begin{lstlisting}[language=SQL]
CREATE TABLE voti (
    id INTEGER PRIMARY KEY,
    id_studente INTEGER,
    materia VARCHAR(30),
    voto DECIMAL(3,1),
    FOREIGN KEY (id_studente) 
        REFERENCES studenti(id)
);
\end{lstlisting}

\end{frame}

\begin{frame}[fragile]
\frametitle{Query SELECT di Base}

\textbf{Selezionare tutti i dati:}
\begin{lstlisting}[language=SQL]
SELECT * FROM studenti;
\end{lstlisting}

\pause

\textbf{Selezionare colonne specifiche:}
\begin{lstlisting}[language=SQL]
SELECT nome, cognome, classe 
FROM studenti;
\end{lstlisting}

\pause

\textbf{Con condizione WHERE:}
\begin{lstlisting}[language=SQL]
SELECT nome, cognome 
FROM studenti 
WHERE classe = '5A';
\end{lstlisting}

\end{frame}

\begin{frame}[fragile]
\frametitle{JOIN: Unire Tabelle}

\begin{lstlisting}[language=SQL]
SELECT 
    s.nome, 
    s.cognome, 
    v.materia, 
    v.voto
FROM studenti s
INNER JOIN voti v 
    ON s.id = v.id_studente
WHERE v.voto >= 7
ORDER BY s.cognome, v.materia;
\end{lstlisting}

\vspace{0.3cm}

\alert{Il JOIN permette di combinare dati da più tabelle correlate}

\end{frame}

% ===== SEZIONE 4: ESERCIZIO =====
\section{Esercizio Pratico}

\begin{frame}[fragile]
\frametitle{Connessione da Python}

\begin{lstlisting}
import sqlite3

# Connessione al database
conn = sqlite3.connect('scuola.db')
cursor = conn.cursor()

# Esecuzione query
cursor.execute('''
    SELECT nome, AVG(voto) as media
    FROM studenti s
    JOIN voti v ON s.id = v.id_studente
    GROUP BY s.id
    HAVING media >= 7
''')

# Recupero risultati
for row in cursor.fetchall():
    print(f"{row[0]}: {row[1]:.2f}")

conn.close()
\end{lstlisting}

\end{frame}

\begin{frame}
\frametitle{Esercizio per Casa}

\begin{alertblock}{Compito}
Progettare uno schema E-R per un database di una biblioteca che gestisca:
\begin{itemize}
    \item Libri (titolo, autore, ISBN, anno)
    \item Utenti (nome, cognome, tessera)
    \item Prestiti (data inizio, data fine)
\end{itemize}
\end{alertblock}

\vspace{0.5cm}

\begin{block}{Deliverable}
\begin{enumerate}
    \item Diagramma E-R
    \item Schema logico (tabelle)
    \item Query SQL per creare le tabelle
\end{enumerate}
\end{block}

\end{frame}

% ===== SEZIONE 5: CONCLUSIONI =====
\section{Conclusioni}

\begin{frame}
\frametitle{Riepilogo}

\textbf{Cosa abbiamo imparato:}

\begin{itemize}
    \item Concetti base dei database relazionali
    \item Struttura del modello relazionale
    \item Sintassi SQL essenziale (DDL e DML)
    \item JOIN tra tabelle
    \item Integrazione con Python
\end{itemize}

\vspace{1cm}

\begin{center}
\Large \alert{Domande?}
\end{center}

\end{frame}

\begin{frame}
\frametitle{Risorse Utili}

\textbf{Documentazione:}
\begin{itemize}
    \item \url{https://www.postgresql.org/docs/}
    \item \url{https://www.sqlite.org/docs.html}
    \item \url{https://www.w3schools.com/sql/}
\end{itemize}

\vspace{0.5cm}

\textbf{Strumenti:}
\begin{itemize}
    \item DBeaver (client universale)
    \item DB Browser for SQLite
    \item phpMyAdmin
\end{itemize}

\vspace{1cm}

\begin{center}
\textbf{Grazie per l'attenzione!}\\
\texttt{email@esempio.it}
\end{center}

\end{frame}

\end{document}
