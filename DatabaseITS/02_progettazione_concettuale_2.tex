\documentclass[aspectratio=169]{beamer}

% Pacchetti per la lingua e la codifica
\usepackage[utf8]{inputenc}
\usepackage[T1]{fontenc}
\usepackage[italian]{babel}
\usepackage{graphicx}
\usepackage{booktabs}
\usepackage{listings}

% Tema della presentazione
\usetheme{Madrid}
\usecolortheme{beaver}

% Informazioni sul documento
\title[Progettazione Concettuale]{Progettazione Concettuale di Basi di Dati}
\subtitle{Strategie, Pattern e Metodologie}
\author{Prof. Fedeli Massimo - Tutti i diritti riservati}
\institute[ITS 4.0]{Fabbrica Digitale}
\date{\today}

\begin{document}
	
	% --- SLIDE 1: Titolo ---
	\begin{frame}
		\titlepage
	\end{frame}
	
	% --- SLIDE 2: Indice ---
	\begin{frame}{Indice}
		\tableofcontents
	\end{frame}
	
	% ==========================================
	% SEZIONE 1: INTRODUZIONE E REQUISITI
	% ==========================================
	\section{Analisi dei Requisiti}
	
	\begin{frame}{Il Ciclo di Vita della Progettazione}
		La progettazione di un database si articola in fasi distinte:
		\begin{itemize}
			\item \textbf{Analisi dei Requisiti} ("COSA"): Raccogliere e analizzare le necessità.
			\item \textbf{Progettazione Concettuale} ("COSA"): Schema Concettuale (indipendente dal DBMS).
			\item \textbf{Progettazione Logica} ("COME"): Schema Logico (es. Relazionale).
			\item \textbf{Progettazione Fisica} ("COME"): Schema Fisico (file, indici).
		\end{itemize}
	\end{frame}
	
	\begin{frame}{Acquisizione dei Requisiti}
		Copre compiti interconnessi:
		\begin{enumerate}
			\item Raccolta dei requisiti (elicitation).
			\item Analisi dei requisiti.
			\item Costruzione del glossario.
			\item Costruzione dello schema concettuale.
		\end{enumerate}
	\end{frame}
	
	\begin{frame}{Fonti dei Requisiti}
		Da dove provengono le informazioni?
		\begin{itemize}
			\item \textbf{Utenti e clienti:} tramite interviste.
			\item \textbf{Documentazione specifica:} documenti ad hoc.
			\item \textbf{Documentazione esistente:} normative, leggi, regolamenti interni, processi aziendali.
			\item \textbf{Soluzioni preesistenti:} software legacy.
			\item \textbf{Modulistica:} form cartacei o digitali in uso.
		\end{itemize}
	\end{frame}
	
	\begin{frame}{Acquisizione tramite Interviste}
		\begin{alertblock}{Attenzione}
			L'acquisizione è un'attività difficile e non standardizzata. I requisiti iniziali spesso necessitano di raffinamenti successivi.
		\end{alertblock}
		\begin{itemize}
			\item Utenti diversi forniscono informazioni diverse.
			\item Livello manageriale: visione ampia ma poco dettagliata.
			\item Livello operativo: visione dettagliata ma limitata.
		\end{itemize}
	\end{frame}
	
	\begin{frame}{Consigli per le Interviste}
		\begin{itemize}
			\item Effettuare frequenti controlli di comprensione e coerenza.
			\item Utilizzare esempi di \textit{Use Case} (casi generici e casi limite).
			\item Chiedere definizioni esplicite e classificazioni.
			\item Chiedere di distinguere tra aspetti essenziali e periferici.
		\end{itemize}
	\end{frame}
	
	\begin{frame}{Analisi della Documentazione Descrittiva}
		Regole empiriche per trattare i testi dei requisiti:
		\begin{itemize}
			\item Scegliere il giusto livello di astrazione.
			\item Mantenere la struttura delle frasi il più standard possibile.
			\item Spezzare le frasi troppo lunghe o complesse.
			\item Distinguere le frasi che descrivono **dati** da quelle che descrivono **funzioni**.
		\end{itemize}
	\end{frame}
	
	\begin{frame}{Costruzione del Glossario}
		Il glossario è fondamentale per uniformare il linguaggio.
		\begin{itemize}
			\item Costruire un glossario per i termini chiave.
			\item Unificare omonimi e sinonimi sotto un unico termine.
			\item Chiarire esplicitamente le relazioni tra i termini.
			\item Ordinare le frasi per concetto.
		\end{itemize}
	\end{frame}
	
	\begin{frame}{Esempio di Glossario}
		\begin{table}
			\centering
			\begin{tabular}{l p{4cm} l l}
				\toprule
				\textbf{Termine} & \textbf{Descrizione} & \textbf{Sinonimo} & \textbf{Relato a} \\
				\midrule
				Partecipante & Chi prende parte ai corsi & Studente & Corso, Società \\
				Docente & Il titolare del corso & Insegnante & Corso \\
				Corso & Corso interno, può avere edizioni & Workshop & Docente \\
				Società & Luogo di lavoro attuale o passato & Luogo & Partecipante \\
				\bottomrule
			\end{tabular}
		\end{table}
	\end{frame}
	
	\begin{frame}{Strutturazione dei Requisiti}
		Organizzare i requisiti in gruppi omogenei di frasi facilita la progettazione.
		\begin{itemize}
			\item Frasi generali (visione d'insieme).
			\item Frasi relative ai Partecipanti.
			\item Frasi relative ai Corsi.
			\item Frasi relative ai Docenti.
		\end{itemize}
		\textit{[Immagine suggerita: Diagramma a blocchi che mostra il raggruppamento delle frasi per argomento]}
	\end{frame}
	
	\begin{frame}{Dai Requisiti allo Schema Concettuale}
		Come mappare i termini nei costruttori ER?
		\begin{itemize}
			\item \textbf{Entità:} se il termine ha proprietà rilevanti e descrive oggetti autonomi.
			\item \textbf{Attributo:} se è un termine semplice senza ulteriori specifiche.
			\item \textbf{Relazione (Associazione):} quando un termine collega altri termini.
			\item \textbf{Generalizzazione:} quando un termine è un caso più generale di un altro.
		\end{itemize}
	\end{frame}
	
	% ==========================================
	% SEZIONE 2: DESIGN PATTERNS
	% ==========================================
	\section{Design Patterns ER}
	
	\begin{frame}{Design Patterns}
		\begin{itemize}
			\item Sono "best practices" comuni nella progettazione software per problemi ricorrenti.
			\item Gli ingegneri del software li usano quotidianamente.
			\item Esistono pattern specifici per la modellazione ER (Entity-Relationship).
		\end{itemize}
	\end{frame}
	
	\begin{frame}{Pattern: Reificazione di Attributi}
		Quando un attributo nasconde concetti complessi, deve diventare un'Entità.
		\vspace{0.5cm}
		
		\textbf{Esempio:} "Azienda" come semplice stringa vs "Azienda" come entità relazionata.
		
		\begin{center}
			\fbox{\begin{minipage}{0.8\textwidth}
					\centering
							\end{minipage}}
		\end{center}
		L'azienda è un concetto diverso dall'impiegato (es. quando le aziende sono coinvolte in altre relazioni).
	\end{frame}
	
	\begin{frame}{Pattern: Part-of (Composizione/Aggregazione)}
		Le relazioni (1,N) possono rappresentare concetti di "parte di".
		\begin{itemize}
			\item \textbf{Composizione:} Un cinema è fatto di sale (senza cinema, la sala non ha senso).
			\item \textbf{Aggregazione:} Un team è composto da esperti (l'esperto esiste anche senza il team).
		\end{itemize}
		
		\begin{center}
			\fbox{\begin{minipage}{0.8\textwidth}
					\centering
				
			\end{minipage}}
		\end{center}
	\end{frame}
	
	\begin{frame}{Pattern: Instance-of}
		A volte servono due entità distinte: una per la rappresentazione astratta e una per l'istanza concreta.
		
		\textbf{Esempio:}
		\begin{itemize}
			\item \textbf{Volo (Flight):} Il concetto astratto (es. Volo AZ123 Roma-Milano delle 8:00).
			\item \textbf{Schedulazione (Scheduled):} L'evento specifico in una data precisa.
		\end{itemize}
		
		\begin{center}
			\fbox{\begin{minipage}{0.6\textwidth}
					\centering

			\end{minipage}}
		\end{center}
	\end{frame}
	
	\begin{frame}{Pattern: Reificazione di Relazioni Binarie (1)}
		Quando una relazione ha attributi complessi o deve essere collegata ad altre entità, può essere trasformata (reificata) in un'entità.
		
		\textbf{Esempio Studente-Esame-Lezione:}
		Un esame collega uno studente e una lezione e ha un voto.
		\begin{center}
			\fbox{\begin{minipage}{0.7\textwidth}
					\centering
							\end{minipage}}
		\end{center}
	\end{frame}
	
	\begin{frame}{Pattern: Reificazione di Relazioni Ricorsive}
		\textbf{Esempio Partita (Match):}
		Una partita potrebbe essere una relazione binaria tra due squadre (Team).
		Tuttavia, poiché due squadre possono giocare contro più volte, è meglio reificare il concetto di "Match".
		
		\begin{center}
			\fbox{\begin{minipage}{0.7\textwidth}
					\centering
					\textit{[Inserire qui diagramma ER slide 30: Team - Home/Visiting - Match]}
			\end{minipage}}
		\end{center}
	\end{frame}
	
	\begin{frame}{Pattern: Reificazione di Relazioni (Strumenti Musicali)}
		Un musicista suona uno strumento in un'orchestra.
		\begin{itemize}
			\item Un musicista suona uno strumento.
			\item Lo stesso strumento (fisico o tipologia) può essere suonato molte volte.
		\end{itemize}
		
		\begin{center}
			\fbox{\begin{minipage}{0.7\textwidth}
					\centering

			\end{minipage}}
		\end{center}
	\end{frame}
	
	\begin{frame}{Pattern: Caso Specifico (Sottoinsiemi)}
		Le generalizzazioni sono usate per definire casi specifici di un'entità.
		
		\textbf{Esempio:}
		Non tutti gli impiegati gestiscono progetti. Solo i "Manager" lo fanno.
		
		\begin{center}
			\fbox{\begin{minipage}{0.6\textwidth}
					\centering

			\end{minipage}}
		\end{center}
	\end{frame}
	
	\begin{frame}{Pattern: Storicizzazione (History)}
		Come mantenere traccia dei cambiamenti nel tempo?
		\begin{itemize}
			\item \textbf{Dati anagrafici:} Indirizzo corrente vs indirizzi passati.
			\item \textbf{Stato software:} Versione corrente vs versioni legacy.
		\end{itemize}
		Si usano generalizzazioni per separare lo stato "Corrente" dallo stato "Storico".
		
		\begin{center}
			\fbox{\begin{minipage}{0.6\textwidth}
					\centering

			\end{minipage}}
		\end{center}
	\end{frame}
	
	\begin{frame}{Pattern: Storicizzazione (Employment)}
		Esempio complesso: Storia lavorativa di un impiegato.
		\begin{itemize}
			\item Un impiegato può avere un impiego corrente e molti passati.
			\item Si può modellare con due relazioni distinte o reificando l'impiego (Employment).
		\end{itemize}
		
		\begin{center}
			\fbox{\begin{minipage}{0.8\textwidth}
					\centering
				\end{minipage}}
		\end{center}
	\end{frame}
	
	\begin{frame}{Pattern: Estensione di un Concetto}
		Le generalizzazioni possono essere usate per estendere implementazioni correnti.
		
		\textbf{Esempio:}
		Un progetto "Accettato" richiede informazioni aggiuntive (data inizio, fondi) rispetto a un progetto generico in fase di proposta.
		
		\begin{center}
			\fbox{\begin{minipage}{0.7\textwidth}
					\centering

			\end{minipage}}
		\end{center}
	\end{frame}
	
	\begin{frame}{Pattern: Relazioni Ternarie}
		Quando tre entità sono coinvolte simultaneamente.
		
		\textbf{Esempio:} Un impiegato lavora su un task in un ufficio specifico.
		\begin{itemize}
			\item Gli impiegati lavorano su vari task in diversi uffici.
			\item Gli uffici ospitano diversi impiegati su vari task.
		\end{itemize}
		
		\begin{center}
			\fbox{\begin{minipage}{0.6\textwidth}
					\centering

			\end{minipage}}
		\end{center}
	\end{frame}
	
	\begin{frame}{Reificazione Relazione Ternaria (1)}
		Spesso le ternarie vengono reificate per maggiore chiarezza o per aggiungere vincoli.
		
		Ogni "Lavoro" (Work) è definito da un Impiegato che lavora in un dato Ufficio per un dato Task.
		
		\begin{center}
			\fbox{\begin{minipage}{0.7\textwidth}
					\centering

			\end{minipage}}
		\end{center}
	\end{frame}
	
	\begin{frame}{Reificazione Relazione Ternaria (2)}
		Se un task può essere svolto da un solo operatore e solo in un ufficio, la reificazione può essere semplificata in una catena di relazioni binarie.
		
		\begin{center}
			\fbox{\begin{minipage}{0.8\textwidth}
					\centering

			\end{minipage}}
		\end{center}
	\end{frame}
	
	% ==========================================
	% SEZIONE 3: STRATEGIE DI PROGETTAZIONE
	% ==========================================
	\section{Strategie di Progettazione}
	
	\begin{frame}{Strategie di Mapping}
		Come trasformare i requisiti in uno schema ER? Esistono tre strategie principali:
		\begin{enumerate}
			\item \textbf{Top-down}
			\item \textbf{Bottom-up}
			\item \textbf{Inside-out}
		\end{enumerate}
	\end{frame}
	
	\begin{frame}{Strategia Top-down}
		Si parte da concetti astratti e macroscopici per scendere nel dettaglio.
		\begin{itemize}
			\item Schema Draft $\rightarrow$ Schema Intermedio $\rightarrow$ Schema Finale.
			\item Si parte con poche entità molto generiche.
		\end{itemize}
		\begin{center}
			\fbox{\begin{minipage}{0.5\textwidth}
					\centering

			\end{minipage}}
		\end{center}
	\end{frame}
	
	\begin{frame}{Esempio Top-down (Raffinamento)}
		\begin{enumerate}
			\item Concetto iniziale: \textbf{Exam}.
			\item Raffinamento: Trasformazione in relazione tra \textbf{Student} e \textbf{Lecture}.
		\end{enumerate}
		\begin{center}
			\fbox{\begin{minipage}{0.6\textwidth}
					\centering

			\end{minipage}}
		\end{center}
	\end{frame}
	
	\begin{frame}{Esempio Top-down (Attributi e Generalizzazione)}
		\begin{enumerate}
			\item Concetto: \textbf{Employee}.
			\item Aggiunta dettagli: Attributi (Surname, Age, Wage).
			\item Oppure specializzazione: \textbf{People} diventa generalizzazione di \textbf{Man} e \textbf{Woman}.
		\end{enumerate}
	\end{frame}
	
	\begin{frame}{Strategia Bottom-up}
		Si parte dai dettagli (attributi, specifiche) per aggregarli in concetti più ampi.
		\begin{itemize}
			\item Requisiti elementari $\rightarrow$ Schemi parziali $\rightarrow$ Integrazione.
		\end{itemize}
		\begin{center}
			\fbox{\begin{minipage}{0.5\textwidth}
					\centering

			\end{minipage}}
		\end{center}
	\end{frame}
	
	\begin{frame}{Esempio Bottom-up}
		\begin{enumerate}
			\item Requisiti su un impiegato $\rightarrow$ Creazione Entità \textbf{Employee}.
			\item Requisiti su esami $\rightarrow$ Creazione relazione \textbf{Student-Exam-Lecture}.
			\item Unione di attributi o entità separate in gerarchie (Man/Woman $\rightarrow$ People).
		\end{enumerate}
	\end{frame}
	
	\begin{frame}{Strategia Inside-out}
		È una variante del bottom-up. Si parte dai concetti più importanti ("nucleus") e ci si espande "a macchia d'olio" seguendo le relazioni.
		\begin{enumerate}
			\item Identificazione Entità principale (es. \textbf{Employee}).
			\item Aggiunta attributi (Code, Surname).
			\item Scoperta relazioni: appartiene a \textbf{Dept}, è supervisionato da...
			\item Espansione: Dept ha un nome, Employee lavora a un \textbf{Project}, ecc.
		\end{enumerate}
	\end{frame}
	
	\begin{frame}{Esempio Inside-out (Espansione)}
		Si naviga attraverso le associazioni:
		\begin{itemize}
			\item Employee $\leftrightarrow$ Dept (Belonging)
			\item Employee $\leftrightarrow$ Project (Enrollment)
			\item Dept $\leftrightarrow$ Office (MadeOf)
		\end{itemize}
		\begin{center}
			\fbox{\begin{minipage}{0.6\textwidth}
					\centering

			\end{minipage}}
		\end{center}
	\end{frame}
	
	\begin{frame}{La Strategia "Mista" (Best Practice)}
		Nella pratica si usa uno stile misto:
		\begin{enumerate}
			\item \textbf{Sketch iniziale:} Creare uno schema di massima con le entità più rilevanti (Top-down/Inside-out).
			\item \textbf{Decomposizione:} Dividere i requisiti in sottogruppi.
			\item \textbf{Raffinamento:} Espandere i dettagli (Bottom-up), integrare le parti.
		\end{enumerate}
	\end{frame}
	
	% ==========================================
	% SEZIONE 4: METODOLOGIA E QUALITÀ
	% ==========================================
	\section{Metodologia e Qualità}
	
	\begin{frame}{Metodologia Best Practice}
		\begin{enumerate}
			\item \textbf{Analisi dei Requisiti:} Glossario, pulizia ambiguità, raggruppamento frasi.
			\item \textbf{Caso Base (Sketch):} Schema bozza con concetti principali.
			\item \textbf{Decomposizione:} Partizionamento dei requisiti per aree di interesse.
			\item \textbf{Iterazione (per ogni area):}
			\begin{itemize}
				\item Raffinare i concetti base usando i requisiti.
				\item Aggiungere concetti mancanti.
			\end{itemize}
			\item \textbf{Integrazione:} Unire i sotto-schemi.
			\item \textbf{Analisi di Qualità:} Verifica finale.
		\end{enumerate}
	\end{frame}
	
	\begin{frame}{Misure di Qualità dello Schema ER}
		Uno schema è buono se soddisfa:
		\begin{itemize}
			\item \textbf{Correttezza (Correctness):} Uso corretto dei costrutti sintattici e semantici.
			\item \textbf{Completezza (Completeness):} Tutti i requisiti sono rappresentati.
			\item \textbf{Chiarezza (Clarity):} Lo schema è leggibile e comprensibile.
			\item \textbf{Minimalità (Minimality):} Evitare ridondanze inutili.
		\end{itemize}
	\end{frame}
	
	% ==========================================
	% SEZIONE 5: CASO DI STUDIO
	% ==========================================
	\section{Caso di Studio: Training Company}
	
	\begin{frame}{Caso di Studio: Training Company}
		Applichiamo la metodologia a un caso reale.
		
		\textbf{Obiettivo:} Un database per una società di formazione.
		\begin{itemize}
			\item Gestione corsi, lezioni e insegnanti.
			\item ~5000 studenti, ~200 corsi, ~300 insegnanti.
		\end{itemize}
	\end{frame}
	
	\begin{frame}{Step 1: Sketch Iniziale}
		Partiamo dalle entità più rilevanti citate esplicitamente o implicitamente.
		\begin{itemize}
			\item Lecture (Lezione/Corso)
			\item Participant (Studente)
			\item Lecturer (Docente)
		\end{itemize}
		Relazioni base: "Presence" (Partecipazione) e "Teaching" (Insegnamento).
		
		\begin{center}
			\fbox{\begin{minipage}{0.5\textwidth}
					\centering
					\textit{[Inserire schema slide 66: Participant - Lecture - Lecturer]}
			\end{minipage}}
		\end{center}
	\end{frame}
	
	\begin{frame}{Step 2: Raffinamento Partecipanti (1)}
		\textbf{Requisiti:} Dati anagrafici (ID, Tax code, name, age...), datore di lavoro, corsi passati e correnti con voto.
		
		Analisi:
		\begin{itemize}
			\item Distinguere corsi attuali e passati (Storicizzazione).
			\item Attributi complessi.
		\end{itemize}
	\end{frame}
	
	\begin{frame}{Step 2: Raffinamento Partecipanti (2)}
		\textbf{Specifiche:}
		\begin{itemize}
			\item Se studente è \textit{freelance}: area di interesse, titolo onorifico.
			\item Se studente è \textit{dipendente}: livello, posizione, struttura aziendale.
		\end{itemize}
		
		$\rightarrow$ Suggerisce una \textbf{Generalizzazione} (Employee vs Freelance).
	\end{frame}
	
	\begin{frame}{Step 2: Raffinamento Partecipanti (3) - Employer}
		\textbf{Requisiti:} Memorizzare datori di lavoro attuali e passati (Nome, indirizzo, telefono).
		
		$\rightarrow$ L'azienda (Employer) diventa un'Entità a sé stante.
		$\rightarrow$ Relazione storicizzata con il Partecipante (CurrentEmpl vs PastEmpl).
		
		\begin{center}
			\fbox{\begin{minipage}{0.6\textwidth}
					\centering

			\end{minipage}}
		\end{center}
	\end{frame}
	
	\begin{frame}{Step 3: Raffinamento Corsi}
		\textbf{Requisiti:}
		\begin{itemize}
			\item Corsi hanno codice, titolo.
			\item Diverse \textbf{edizioni} (data inizio/fine, numero partecipanti).
			\item Ogni edizione ha lezioni in giorni/stanze specifici.
		\end{itemize}
		
		\textbf{Modellazione:}
		Lecture (generico) $\rightarrow$ Edition (specifico nel tempo) $\rightarrow$ Lesson (singola lezione giornaliera). Usiamo pattern Part-of o entità deboli.
		
		\begin{center}
			\fbox{\begin{minipage}{0.6\textwidth}
					\centering

			\end{minipage}}
		\end{center}
	\end{frame}
	
	\begin{frame}{Step 4: Raffinamento Docenti}
		\textbf{Requisiti:}
		\begin{itemize}
			\item Dati anagrafici.
			\item Corsi che possono insegnare (skill) vs corsi che hanno insegnato (storia).
			\item Possono essere dipendenti interni o contractor esterni.
		\end{itemize}
		
		\textbf{Modellazione:}
		Generalizzazione (Independent vs House Employee). Relazioni multiple con i corsi.
		
		\begin{center}
			\fbox{\begin{minipage}{0.6\textwidth}
					\centering

			\end{minipage}}
		\end{center}
	\end{frame}
	
	\begin{frame}{Step 5: Integrazione (Schema Intermedio)}
		Colleghiamo i pezzi.
		\begin{itemize}
			\item Participant si lega a Edition (CurrentPresence) e a Lecture/Edition passate (PastPresence).
			\item Lecturer si lega a Edition (Current teaching) e Lecture (History/Ability).
		\end{itemize}
		
		\begin{center}
			\fbox{\begin{minipage}{0.7\textwidth}
					\centering

			\end{minipage}}
		\end{center}
	\end{frame}
	
	\begin{frame}{Schema Concettuale Finale}
		Il risultato finale unisce:
		\begin{itemize}
			\item Gerarchia dei Partecipanti (Employee/Freelance).
			\item Relazione con Employer.
			\item Struttura del Corso (Lecture-Edition-Lesson).
			\item Gerarchia Docenti.
			\item Relazioni di frequenza e insegnamento.
		\end{itemize}
		
		\begin{center}
			\fbox{\begin{minipage}{0.9\textwidth}
					\centering

			\end{minipage}}
		\end{center}
	\end{frame}
	
	\begin{frame}{Conclusioni}
		\begin{itemize}
			\item La progettazione concettuale è un processo iterativo.
			\item L'uso di pattern semplifica problemi complessi (es. storicizzazione).
			\item Una metodologia mista (Sketch + Raffinamento) è la più efficace.
			\item La qualità si misura in completezza, correttezza e minimalità.
		\end{itemize}
	\end{frame}
	
	\begin{frame}
		\centering
		\Huge \textbf{Grazie per l'attenzione}
	\end{frame}
	
\end{document}