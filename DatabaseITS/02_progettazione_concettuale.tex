\documentclass[aspectratio=169]{beamer}
\usepackage[utf8]{inputenc}
\usepackage[italian]{babel}
\usepackage{tikz}
\usepackage{pgfplots}
\usepackage{graphicx}
\usepackage{amsmath}
\usepackage{amssymb}
\usepackage{booktabs}
\usepackage{array}
\usepackage{colortbl}

\usetikzlibrary{shapes,arrows,positioning,calc,fit,backgrounds,decorations.pathreplacing,shadows,patterns}
\pgfplotsset{compat=1.16}

\usetheme{Madrid}
\usecolortheme{default}
\setbeamertemplate{navigation symbols}{}
\setbeamertemplate{footline}[frame number]

\definecolor{darkblue}{RGB}{0,51,102}
\definecolor{lightblue}{RGB}{102,178,255}
\definecolor{darkgreen}{RGB}{0,102,51}
\definecolor{lightgreen}{RGB}{153,255,153}

\title{La Progettazione delle Basi di Dati}
\subtitle{Progettazione Concettuale - Modello Entità-Relazione}
\author{Prof. Fedeli Massimo - Tutti i diritti riservati}
\institute{ITS 4.0 Fabbrica Digitale}
\date{\today}

\begin{document}

% Slide 1: Titolo
\begin{frame}
\titlepage
\end{frame}

% Slide 2: Indice
\begin{frame}{Sommario}
\tableofcontents
\end{frame}

\section{Introduzione alla Progettazione}

% Slide 3: La progettazione del database
\begin{frame}{La Progettazione del Database}
\begin{block}{I passi principali della progettazione}
\begin{enumerate}
\item Analisi del problema
\item Progettazione concettuale del database $\rightarrow$ \textbf{modello E-R}
\item Progettazione logica del database $\rightarrow$ \textbf{schema logico}
\item Progettazione fisica e implementazione
\item Realizzazione delle applicazioni (Java, VB, PHP, etc.)
\end{enumerate}
\end{block}

\vspace{0.5cm}
\begin{alertblock}{Complessità}
Ognuno di questi passi presenta criticità e implica operazioni complesse
\end{alertblock}
\end{frame}

% Slide 4: Fasi di sviluppo
\begin{frame}{Fasi di Sviluppo}
\begin{center}
\begin{tikzpicture}[scale=0.9]
\node[rectangle, draw, fill=lightblue, minimum width=3cm, minimum height=1cm] (db) at (0,2) {Sviluppo Database};
\node[rectangle, draw, fill=lightgreen, minimum width=3cm, minimum height=1cm] (app) at (0,0) {Sviluppo Applicazione};
\node[rectangle, draw, fill=orange!30, minimum width=3cm, minimum height=1cm] (test) at (0,-2) {Test e Collaudo};

\draw[->, thick] (db) -- (app);
\draw[->, thick] (app) -- (test);
\end{tikzpicture}
\end{center}

\begin{itemize}
\item Fase di sviluppo del database
\item Fase di sviluppo dell'applicazione
\item Test di funzionamento ed collaudo
\end{itemize}
\end{frame}

% Slide 5: Schema delle fasi
\begin{frame}{Le Fasi della Progettazione}
\begin{center}
\begin{tikzpicture}[node distance=2cm, scale=0.8, every node/.style={transform shape}]
\node[rectangle, draw, fill=yellow!30, text width=3cm, align=center] (analisi) {Analisi dei Requisiti};
\node[rectangle, draw, fill=green!30, text width=3cm, align=center, below of=analisi] (concettuale) {Progettazione Concettuale};
\node[rectangle, draw, fill=blue!30, text width=3cm, align=center, below of=concettuale] (logica) {Progettazione Logica};
\node[rectangle, draw, fill=red!30, text width=3cm, align=center, below of=logica] (fisica) {Progettazione Fisica};

\draw[->, very thick] (analisi) -- (concettuale);
\draw[->, very thick] (concettuale) -- (logica);
\draw[->, very thick] (logica) -- (fisica);
\end{tikzpicture}
\end{center}
\end{frame}

% Slide 6: Output progettazione concettuale
\begin{frame}{Output della Progettazione}
\begin{block}{Progettazione Concettuale}
L'output della progettazione concettuale è il \textbf{modello concettuale}
\end{block}

\begin{center}
\begin{tikzpicture}
\node[rectangle, draw, fill=green!40, minimum width=4cm, minimum height=2cm, rounded corners] (input) at (0,0) {
\begin{tabular}{c}
\textbf{Input:} \\
Requisiti \\
Specifiche
\end{tabular}
};

\node[rectangle, draw, fill=blue!40, minimum width=4cm, minimum height=2cm, rounded corners] (output) at (6,0) {
\begin{tabular}{c}
\textbf{Output:} \\
Modello \\
Concettuale E-R
\end{tabular}
};

\draw[->, ultra thick] (input) -- (output) node[midway, above] {Progettazione};
\end{tikzpicture}
\end{center}
\end{frame}

\section{Livelli di Astrazione}

% Slide 7: Livelli di astrazione
\begin{frame}{I Livelli di Astrazione}
\begin{block}{Tre livelli di rappresentazione}
La progettazione avviene a diversi livelli di astrazione:
\end{block}

\begin{enumerate}
\item \textbf{Livello Concettuale} (livello di oggetti)
\begin{itemize}
\item Rappresenta la realtà dei dati e le relazioni
\item Usa uno schema astratto
\end{itemize}

\item \textbf{Livello Logico} (livello di record)
\begin{itemize}
\item Organizzazione dei dati negli archivi
\item Descrive composizione e formato dei dati
\end{itemize}

\item \textbf{Livello Fisico}
\begin{itemize}
\item Installazione fisica su disco
\item Ubicazione nelle memorie di massa
\end{itemize}
\end{enumerate}
\end{frame}

% Slide 8: Visualizzazione livelli
\begin{frame}{Schema dei Livelli di Astrazione}
\begin{center}
\begin{tikzpicture}[scale=0.9]
% Livello Concettuale
\draw[fill=red!30, draw=black, thick] (0,6) rectangle (8,8);
\node at (4,7) {\Large \textbf{Livello Concettuale}};
\node[align=center] at (4,6.5) {Schema E-R, Oggetti, Relazioni};

% Livello Logico
\draw[fill=blue!30, draw=black, thick] (0,3.5) rectangle (8,5.5);
\node at (4,4.7) {\Large \textbf{Livello Logico}};
\node[align=center] at (4,4.1) {Tabelle, Record, Campi};

% Livello Fisico
\draw[fill=green!30, draw=black, thick] (0,1) rectangle (8,3);
\node at (4,2.3) {\Large \textbf{Livello Fisico}};
\node[align=center] at (4,1.7) {File, Blocchi, Indici, Puntatori};

% Frecce
\draw[->, ultra thick] (4,5.9) -- (4,5.5);
\draw[->, ultra thick] (4,3.4) -- (4,3);
\end{tikzpicture}
\end{center}
\end{frame}

% Slide 9: Livello fisico dettaglio
\begin{frame}{Il Livello Fisico}
\begin{block}{Implementazione fisica dei dati}
Il livello fisico è l'implementazione del livello logico sui supporti di memorizzazione fisica
\end{block}

\begin{columns}
\column{0.5\textwidth}
\textbf{Si occupa di:}
\begin{itemize}
\item Partizioni
\item Puntatori
\item Blocchi fisici
\item Cluster
\item Indici
\end{itemize}

\column{0.5\textwidth}
\begin{center}
\begin{tikzpicture}[scale=0.7]
\draw[fill=gray!30] (0,0) circle (2cm);
\node at (0,0) {\Large DISCO};
\draw[fill=blue!20] (-1.5,0.5) rectangle (-0.5,1.5);
\draw[fill=blue!20] (0.5,0.5) rectangle (1.5,1.5);
\draw[fill=blue!20] (-1.5,-1.5) rectangle (-0.5,-0.5);
\draw[fill=blue!20] (0.5,-1.5) rectangle (1.5,-0.5);
\node at (0,-2.5) {Blocchi fisici};
\end{tikzpicture}
\end{center}
\end{columns}
\end{frame}

\section{Analisi Preliminare}

% Slide 10: Analisi preliminare
\begin{frame}{Analisi Preliminare alla Modellazione}
\begin{block}{Obiettivi dell'analisi}
Individuare le esigenze del cliente o il dominio dell'applicazione
\end{block}

\begin{alertblock}{Domande chiave}
\begin{itemize}
\item Quali informazioni devono essere salvate?
\item In che modo queste informazioni verranno manipolate?
\item Chi sono gli utenti finali?
\end{itemize}
\end{alertblock}

\vspace{0.3cm}
\textbf{Approcci:}
\begin{itemize}
\item \textcolor{blue}{Top-down}: dalla visione generale ai dettagli
\item \textcolor{red}{Bottom-up}: dai dati ai concetti (preferibile per DB)
\end{itemize}
\end{frame}

% Slide 11: Top-down vs Bottom-up
\begin{frame}{Approcci di Analisi}
\begin{columns}
\column{0.5\textwidth}
\begin{center}
\textbf{\large Top-Down}
\begin{tikzpicture}[scale=0.6]
\node[circle, draw, fill=red!30] (top) at (0,3) {Sistema};
\node[circle, draw, fill=orange!30] (m1) at (-1.5,1.5) {A};
\node[circle, draw, fill=orange!30] (m2) at (1.5,1.5) {B};
\node[circle, draw, fill=yellow!30] (b1) at (-2,0) {1};
\node[circle, draw, fill=yellow!30] (b2) at (-1,0) {2};
\node[circle, draw, fill=yellow!30] (b3) at (1,0) {3};
\node[circle, draw, fill=yellow!30] (b4) at (2,0) {4};

\draw[->, thick] (top) -- (m1);
\draw[->, thick] (top) -- (m2);
\draw[->, thick] (m1) -- (b1);
\draw[->, thick] (m1) -- (b2);
\draw[->, thick] (m2) -- (b3);
\draw[->, thick] (m2) -- (b4);
\end{tikzpicture}
\end{center}

\column{0.5\textwidth}
\begin{center}
\textbf{\large Bottom-Up}
\begin{tikzpicture}[scale=0.6]
\node[circle, draw, fill=yellow!30] (b1) at (-2,0) {1};
\node[circle, draw, fill=yellow!30] (b2) at (-1,0) {2};
\node[circle, draw, fill=yellow!30] (b3) at (1,0) {3};
\node[circle, draw, fill=yellow!30] (b4) at (2,0) {4};
\node[circle, draw, fill=orange!30] (m1) at (-1.5,1.5) {A};
\node[circle, draw, fill=orange!30] (m2) at (1.5,1.5) {B};
\node[circle, draw, fill=red!30] (top) at (0,3) {Sistema};

\draw[->, thick] (b1) -- (m1);
\draw[->, thick] (b2) -- (m1);
\draw[->, thick] (b3) -- (m2);
\draw[->, thick] (b4) -- (m2);
\draw[->, thick] (m1) -- (top);
\draw[->, thick] (m2) -- (top);
\end{tikzpicture}
\end{center}
\end{columns}
\end{frame}

\section{Modello Entità-Relazione}

% Slide 12: Fase di modellazione
\begin{frame}{La Fase di Modellazione}
\begin{block}{Modelli disponibili}
Al termine dell'analisi, si può ricorrere a:
\end{block}

\begin{enumerate}
\item \textbf{Modello Entità-Relazione (E-R)}
\begin{itemize}
\item Approccio classico e più diffuso
\item Ampiamente utilizzato nell'industria
\item Standard de facto
\end{itemize}

\vspace{0.3cm}

\item \textbf{Modello a Oggetti}
\begin{itemize}
\item Approccio più recente
\item Interessante dal punto di vista accademico
\item Meno diffuso commercialmente
\end{itemize}
\end{enumerate}

\begin{alertblock}{Scelta consigliata}
La stragrande maggioranza delle applicazioni usa l'\textbf{approccio E-R}
\end{alertblock}
\end{frame}

% Slide 13: Vantaggi modello E-R
\begin{frame}{Modello Entità-Relazione: Vantaggi}
\begin{block}{Indipendenza dal linguaggio}
Il modello E-R supera le barriere linguistiche essendo:
\begin{itemize}
\item Assolutamente indipendente dal linguaggio scritto o parlato
\item Comprensibile a tutti gli stakeholder
\item Rappresentazione grafica universale
\end{itemize}
\end{block}

\begin{center}
\begin{tikzpicture}
\node[rectangle, draw, fill=green!30, rounded corners, minimum width=3cm, minimum height=1cm] at (0,0) {Modello E-R};
\node[rectangle, draw, fill=blue!20, text width=2cm, align=center] at (-3,-1.5) {Cliente Italiano};
\node[rectangle, draw, fill=blue!20, text width=2cm, align=center] at (0,-1.5) {Developer USA};
\node[rectangle, draw, fill=blue!20, text width=2cm, align=center] at (3,-1.5) {Team India};

\draw[->, thick] (0,-0.5) -- (-3,-1.2);
\draw[->, thick] (0,-0.5) -- (0,-1.2);
\draw[->, thick] (0,-0.5) -- (3,-1.2);
\end{tikzpicture}
\end{center}
\end{frame}

% Slide 14: Documentazione tecnica
\begin{frame}{Documentazione Associata}
\begin{alertblock}{Importante}
Al modello E-R viene \textbf{sempre} affiancato un documento tecnico
\end{alertblock}

\begin{columns}
\column{0.5\textwidth}
\textbf{Il documento descrive:}
\begin{itemize}
\item Dettagli degli oggetti
\item Vincoli di integrità
\item Regole di business
\item Casi particolari
\item Assunzioni fatte
\end{itemize}

\column{0.5\textwidth}
\begin{center}
\begin{tikzpicture}[scale=0.7]
\draw[fill=white, draw=black, thick] (0,0) rectangle (3,4);
\draw[thick] (0.3,3.5) -- (2.7,3.5);
\draw[thick] (0.3,3) -- (2.7,3);
\draw[thick] (0.3,2.5) -- (2.7,2.5);
\draw[thick] (0.3,2) -- (2.7,2);
\draw[thick] (0.3,1.5) -- (2.7,1.5);
\draw[thick] (0.3,1) -- (2.7,1);
\node at (1.5,0.3) {\small Documentazione};
\end{tikzpicture}
\end{center}
\end{columns}
\end{frame}

% Slide 15: Scopi del modello E-R
\begin{frame}{Progettazione Concettuale: Scopi}
\begin{block}{Primo scopo}
Fornire la \textbf{rappresentazione grafica} di tutti gli oggetti del database
\begin{itemize}
\item Visualizzazione completa della struttura
\item Verifica del flusso delle informazioni
\item Controllo prima dello sviluppo
\end{itemize}
\end{block}

\begin{block}{Secondo scopo}
Base per la \textbf{creazione del database fisico}
\begin{itemize}
\item Gli sviluppatori lo usano come blueprint
\item Guida per creare tabelle e relazioni
\item Riferimento per tutti gli oggetti
\end{itemize}
\end{block}
\end{frame}

% Slide 16: Principi qualità
\begin{frame}{Principi di Qualità del Modello}
Un buon modello concettuale deve rispettare quattro principi:

\begin{enumerate}
\item \textbf{Correttezza}
\begin{itemize}
\item Uso appropriato degli strumenti di modellazione
\end{itemize}

\item \textbf{Completezza}
\begin{itemize}
\item Rappresentazione di tutti gli aspetti rilevanti
\end{itemize}

\item \textbf{Chiarezza}
\begin{itemize}
\item Modello leggibile e comprensibile
\end{itemize}

\item \textbf{Indipendenza}
\begin{itemize}
\item Non dipendente da strumenti specifici
\end{itemize}
\end{enumerate}
\end{frame}

% Slide 17: Correttezza
\begin{frame}{Correttezza del Modello}
\begin{block}{Utilizzo appropriato degli strumenti}
Il modello deve usare gli strumenti secondo la loro finalità specifica
\end{block}

\begin{exampleblock}{Esempio}
Un diagramma E-R serve a rappresentare \textbf{relazioni tra entità}, non flussi di processo
\begin{itemize}
\item Corretto: rappresentare studenti e corsi con la loro relazione
\item Errato: usarlo per descrivere un workflow aziendale
\end{itemize}
\end{exampleblock}

\begin{alertblock}{Coerenza semantica}
Ogni simbolo deve rispettare le convenzioni:
\begin{itemize}
\item Rettangoli per le entità
\item Rombi per le relazioni
\item Ovali per gli attributi
\end{itemize}
\end{alertblock}
\end{frame}

% Slide 18: Completezza
\begin{frame}{Completezza del Modello}
\begin{block}{Modellazione completa}
Rappresentare \textbf{tutti} gli aspetti significativi del dominio applicativo
\end{block}

\begin{itemize}
\item Tutte le entità rilevanti
\item Tutte le relazioni importanti
\item Tutti gli attributi necessari
\item Tutti i vincoli di integrità
\end{itemize}

\vspace{0.5cm}

\begin{center}
\begin{tikzpicture}[scale=0.8]
\draw[fill=red!20] (0,0) circle (2cm);
\node at (0,0.5) {\textbf{Dominio}};
\node at (0,-0.5) {\textbf{Applicativo}};

\draw[fill=blue!30, opacity=0.7] (0,0) circle (1.8cm);
\node at (0,0) {\small Modello};
\node at (0,-0.3) {\small E-R};
\end{tikzpicture}
\end{center}
\end{frame}

% Slide 19: Chiarezza
\begin{frame}{Chiarezza del Modello}
\begin{block}{Leggibilità e comprensibilità}
Il modello deve essere facilmente leggibile e le informazioni comprensibili
\end{block}

\textbf{Come ottenere chiarezza:}
\begin{itemize}
\item Nomi significativi per entità e attributi
\item Layout ordinato e ben organizzato
\item Evitare sovrapposizioni di linee
\item Raggruppare concetti correlati
\item Usare colori o evidenziazioni (quando possibile)
\item Limitare la complessità di ogni diagramma
\end{itemize}

\begin{alertblock}{Regola pratica}
Se un diagramma non è comprensibile a prima vista, va semplificato!
\end{alertblock}
\end{frame}

% Slide 20: Indipendenza
\begin{frame}{Indipendenza del Modello}
\begin{block}{Astrazione tecnologica}
Il modello concettuale deve essere \textbf{astratto e neutrale}
\end{block}

\begin{columns}
\column{0.5\textwidth}
\textbf{NON dipendente da:}
\begin{itemize}
\item DBMS specifici
\item Linguaggi di programmazione
\item Tecnologie particolari
\item Piattaforme hardware
\end{itemize}

\column{0.5\textwidth}
\textbf{Vantaggi:}
\begin{itemize}
\item Longevità del modello
\item Flessibilità implementativa
\item Portabilità
\item Manutenibilità
\end{itemize}
\end{columns}

\vspace{0.5cm}

\begin{exampleblock}{Cosa descrivere}
\textbf{Cosa} fare, non \textbf{come} implementarlo
\end{exampleblock}
\end{frame}

% Slide 21: Esempio indipendenza
\begin{frame}{Indipendenza: Esempio Pratico}
\begin{columns}
\column{0.5\textwidth}
\begin{alertblock}{Dipendente}
``Usare PostgreSQL con trigger per gestire cancellazioni''
\begin{itemize}
\item Riferimento a tecnologia specifica
\item Dettagli implementativi
\item Non portabile
\end{itemize}
\end{alertblock}

\column{0.5\textwidth}
\begin{exampleblock}{Indipendente}
``Una prenotazione cancellata deve aggiornare la disponibilità della risorsa''
\begin{itemize}
\item Regola di business
\item Tecnologicamente neutrale
\item Implementabile ovunque
\end{itemize}
\end{exampleblock}
\end{columns}
\end{frame}

% Slide 22: Fasi progettazione concettuale
\begin{frame}{Fasi della Progettazione Concettuale}
\begin{center}
\begin{tikzpicture}[node distance=1.5cm, scale=0.75, every node/.style={transform shape}]
\node[rectangle, draw, fill=yellow!30, text width=3.5cm, align=center] (req) {Raccolta Requisiti};
\node[rectangle, draw, fill=orange!30, text width=3.5cm, align=center, below of=req] (analisi) {Analisi Requisiti};
\node[rectangle, draw, fill=green!30, text width=3.5cm, align=center, below of=analisi] (entita) {Identificazione Entità};
\node[rectangle, draw, fill=blue!30, text width=3.5cm, align=center, below of=entita] (relazioni) {Identificazione Relazioni};
\node[rectangle, draw, fill=purple!30, text width=3.5cm, align=center, below of=relazioni] (attributi) {Definizione Attributi};
\node[rectangle, draw, fill=red!30, text width=3.5cm, align=center, below of=attributi] (schema) {Schema E-R Completo};

\draw[->, very thick] (req) -- (analisi);
\draw[->, very thick] (analisi) -- (entita);
\draw[->, very thick] (entita) -- (relazioni);
\draw[->, very thick] (relazioni) -- (attributi);
\draw[->, very thick] (attributi) -- (schema);
\end{tikzpicture}
\end{center}
\end{frame}

% Slide 23: Caratteristiche progettazione
\begin{frame}{Caratteristiche della Progettazione Concettuale}
\begin{block}{Rigorosa}
Per non lasciare dubbi sulle caratteristiche della base di dati
\begin{itemize}
\item Precisione nelle definizioni
\item Specifiche non ambigue
\item Documentazione dettagliata
\end{itemize}
\end{block}

\begin{block}{Semplice nei formalismi}
Per consentire lettura e comprensione anche agli utenti non tecnici
\begin{itemize}
\item Notazione grafica intuitiva
\item Simboli standard
\item Documentazione in linguaggio naturale
\end{itemize}
\end{block}

\begin{alertblock}{Importante}
Gli utenti devono essere certi che i progettisti abbiano compreso le loro esigenze
\end{alertblock}
\end{frame}

% Slide 24: Modello concettuale
\begin{frame}{Il Modello Concettuale}
\begin{block}{Aspetti rilevanti}
Concetti e formalismi utilizzati nella costruzione del modello entità/associazioni
\end{block}

\textbf{Caratteristiche:}
\begin{itemize}
\item Molto utilizzato nella progettazione
\item \alert{Non ha rappresentazione standardizzata}
\item Composto da: entità, associazioni e attributi
\item Diversi modi di rappresentazione
\end{itemize}

\vspace{0.3cm}

\textbf{Notazioni principali:}
\begin{enumerate}
\item Notazione classica (Chen)
\item Notazione standard UML
\end{enumerate}
\end{frame}

% Slide 25: Database designer
\begin{frame}{Chi Realizza il Modello? Il Database Designer}
\begin{block}{Ruolo del Database Designer}
Responsabile dell'astrazione dei dati dal mondo reale
\end{block}

\begin{center}
\begin{tikzpicture}[scale=0.8]
\node[circle, draw, fill=blue!30, minimum size=2cm] (designer) at (0,0) {
\begin{tabular}{c}
Database \\
Designer
\end{tabular}
};

\node[rectangle, draw, fill=yellow!30, text width=2cm, align=center] (req) at (-3,2) {Analisi Requisiti};
\node[rectangle, draw, fill=green!30, text width=2cm, align=center] (conc) at (3,2) {Schema Concettuale};
\node[rectangle, draw, fill=red!30, text width=2cm, align=center] (log) at (3,-2) {Schema Logico};

\draw[->, thick] (req) -- (designer);
\draw[->, thick] (designer) -- (conc);
\draw[->, thick] (designer) -- (log);
\end{tikzpicture}
\end{center}
\end{frame}

% Slide 26: Schema logico
\begin{frame}{Schema Logico}
\begin{center}
\begin{tikzpicture}[scale=0.9]
% Modello E-R
\node[rectangle, draw, fill=green!30, text width=3cm, align=center, minimum height=2cm] (er) at (0,2) {
\textbf{Schema E-R} \\[0.3cm]
Entità \\
Relazioni \\
Attributi
};

% Freccia trasformazione
\draw[->, ultra thick] (er) -- (0,-0.5) node[midway, right] {Trasformazione};

% Schema Logico
\node[rectangle, draw, fill=blue!30, text width=4cm, align=center, minimum height=2.5cm] (log) at (0,-3) {
\textbf{Schema Logico} \\[0.3cm]
Tabelle \\
Chiavi Primarie/Esterne \\
Vincoli di Integrità \\
Indici
};
\end{tikzpicture}
\end{center}
\end{frame}

% Slide 27: Compiti database designer
\begin{frame}{I Compiti del Database Designer}
\begin{block}{Primo compito}
Analizzare le informazioni raccolte durante l'analisi dei requisiti
\end{block}

\begin{block}{Obiettivo principale}
Costruire il modello di base, da raffinare fino al completamento
\end{block}

\vspace{0.3cm}

\textbf{Prime operazioni:}
\begin{enumerate}
\item Classificare gli oggetti come \textbf{entità} o \textbf{attributi}
\item Partire dalla documentazione del progetto
\begin{itemize}
\item Raccolta e analisi della documentazione
\item Definizione del glossario dei termini
\end{itemize}
\end{enumerate}
\end{frame}

% Slide 28: Analisi documentazione
\begin{frame}{Analisi della Documentazione}
\begin{block}{Tipi di documentazione}
\end{block}

\textbf{Documentazione specifica del progetto:}
\begin{itemize}
\item Note delle riunioni tecniche
\item Richieste del cliente
\item Appunti dalle interviste agli utenti
\item Documentazione scritta ad hoc
\end{itemize}

\textbf{Documentazione esistente:}
\begin{itemize}
\item Normative generali e del settore
\item Regolamenti interni
\item Procedure aziendali
\end{itemize}

\textbf{Sistema esistente:}
\begin{itemize}
\item Sistema da rimpiazzare
\item Specifiche di integrazione con sistemi esistenti
\end{itemize}
\end{frame}

\section{Entità e Attributi}

% Slide 29: Identificare entità
\begin{frame}{Identificare le Entità}
\begin{block}{Che cos'è un'entità?}
Un'entità rappresenta un \textbf{oggetto} (concreto o astratto) del mondo reale che ha un'esistenza autonoma
\end{block}

\begin{center}
\begin{tikzpicture}[scale=0.8]
% Entità
\node[rectangle, draw, fill=yellow!30, minimum width=2.5cm, minimum height=1cm] (stud) at (0,2) {\textbf{STUDENTE}};
\node[rectangle, draw, fill=blue!30, minimum width=2.5cm, minimum height=1cm] (corso) at (4,2) {\textbf{CORSO}};
\node[rectangle, draw, fill=green!30, minimum width=2.5cm, minimum height=1cm] (prof) at (0,-0.5) {\textbf{PROFESSORE}};
\node[rectangle, draw, fill=red!30, minimum width=2.5cm, minimum height=1cm] (aula) at (4,-0.5) {\textbf{AULA}};

% Esempi
\node[below of=stud, node distance=1.2cm, font=\small] {Mario Rossi};
\node[below of=corso, node distance=1.2cm, font=\small] {Basi di Dati};
\node[below of=prof, node distance=1.2cm, font=\small] {Prof. Bianchi};
\node[below of=aula, node distance=1.2cm, font=\small] {Aula 101};
\end{tikzpicture}
\end{center}
\end{frame}

% Slide 30: Entità - esempi
\begin{frame}{Esempi di Entità}
\begin{columns}
\column{0.5\textwidth}
\textbf{Entità Concrete:}
\begin{itemize}
\item Persona
\item Automobile
\item Edificio
\item Prodotto
\item Computer
\end{itemize}

\column{0.5\textwidth}
\textbf{Entità Astratte:}
\begin{itemize}
\item Corso universitario
\item Prenotazione
\item Transazione
\item Progetto
\item Evento
\end{itemize}
\end{columns}

\vspace{0.5cm}

\begin{alertblock}{Caratteristica comune}
Tutte hanno proprietà (attributi) che le descrivono e un'identità univoca
\end{alertblock}
\end{frame}

% Slide 31: Identificare attributi
\begin{frame}{Identificare gli Attributi}
\begin{block}{Che cos'è un attributo?}
Gli attributi \textbf{descrivono} un'entità: corrispondono ai campi dei record
\end{block}

\textbf{Regole per individuare gli attributi:}
\begin{enumerate}
\item Gli attributi devono essere \textbf{atomici}
\begin{itemize}
\item Non ulteriormente scomponibili
\end{itemize}

\item Gli attributi \textbf{derivati} non dovrebbero essere memorizzati
\begin{itemize}
\item Esempio: età (derivabile dalla data di nascita)
\end{itemize}

\item Utilizzare \textbf{codici} per classificare gli attributi
\begin{itemize}
\item Quando si presenta l'opportunità
\end{itemize}
\end{enumerate}
\end{frame}

% Slide 32: Esempio attributi
\begin{frame}{Esempio di Attributi}
\begin{center}
\begin{tikzpicture}[scale=0.85]
% Entità centrale
\node[rectangle, draw, fill=yellow!30, minimum width=2.5cm, minimum height=1.5cm] (stud) at (0,0) {\Large \textbf{STUDENTE}};

% Attributi
\node[ellipse, draw, fill=blue!20] (matr) at (-3,2) {Matricola};
\node[ellipse, draw, fill=blue!20] (nome) at (-1,2.5) {Nome};
\node[ellipse, draw, fill=blue!20] (cogn) at (1,2.5) {Cognome};
\node[ellipse, draw, fill=blue!20] (data) at (3,2) {DataNascita};
\node[ellipse, draw, fill=blue!20] (email) at (-3,-2) {Email};
\node[ellipse, draw, fill=blue!20] (tel) at (3,-2) {Telefono};

% Collegamenti
\draw[thick] (stud) -- (matr);
\draw[thick] (stud) -- (nome);
\draw[thick] (stud) -- (cogn);
\draw[thick] (stud) -- (data);
\draw[thick] (stud) -- (email);
\draw[thick] (stud) -- (tel);

% Sottolineatura chiave
\draw[thick] (matr.south west) -- (matr.south east);
\end{tikzpicture}
\end{center}
\end{frame}

% Slide 33: Atomicità attributi
\begin{frame}{Atomicità degli Attributi}
\begin{exampleblock}{Esempio CORRETTO}
\textbf{PERSONA:} Nome, Cognome, Via, Città, CAP
\end{exampleblock}

\begin{alertblock}{Esempio ERRATO}
\textbf{PERSONA:} NomeCompleto, Indirizzo
\begin{itemize}
\item NomeCompleto non è atomico (contiene nome e cognome)
\item Indirizzo non è atomico (contiene via, città, CAP)
\end{itemize}
\end{alertblock}

\vspace{0.3cm}

\textbf{Perché l'atomicità è importante:}
\begin{itemize}
\item Facilita le ricerche specifiche
\item Migliora l'ordinamento
\item Evita ridondanza
\item Semplifica aggiornamenti
\end{itemize}
\end{frame}

% Slide 34: Scegliere nomi
\begin{frame}{Scegliere Correttamente i Nomi}
\begin{block}{Regole per i nomi}
\end{block}

\textbf{I nomi degli oggetti devono essere unici}
\begin{itemize}
\item Evitare ambiguità
\item Nessuna duplicazione nel modello
\end{itemize}

\textbf{Devono avere significato per l'utente finale}
\begin{itemize}
\item Usare terminologia del dominio
\item Comprensibili senza spiegazioni
\end{itemize}

\textbf{Contenere il minimo numero di parole necessarie}
\begin{itemize}
\item Descrizione univoca e accurata
\item Bilanciare brevità e chiarezza
\end{itemize}
\end{frame}

% Slide 35: Esempi nomi
\begin{frame}{Esempi di Denominazione}
\begin{columns}
\column{0.5\textwidth}
\begin{exampleblock}{Nomi CORRETTI}
\begin{itemize}
\item Studente
\item DataIscrizione
\item CorsoDiLaurea
\item NumeroMatricola
\item IndirizzoEmail
\end{itemize}
\end{exampleblock}

\column{0.5\textwidth}
\begin{alertblock}{Nomi da EVITARE}
\begin{itemize}
\item S (troppo generico)
\item Data1 (non significativo)
\item CorsoDiLaureaTriennaleInInformatica (troppo lungo)
\item Num (abbreviazione oscura)
\item x\_email (non user-friendly)
\end{itemize}
\end{alertblock}
\end{columns}
\end{frame}

\section{Chiavi}

% Slide 36: Concetto di chiave
\begin{frame}{Il Concetto di Chiave}
\begin{block}{Definizione}
Una \textbf{chiave} è un attributo (o insieme di attributi) che identifica univocamente un'istanza di un'entità
\end{block}

\begin{center}
\begin{tikzpicture}[scale=0.8]
\node[rectangle, draw, fill=yellow!30, minimum width=3cm, minimum height=1.2cm] (ent) at (0,0) {\textbf{STUDENTE}};

\node[ellipse, draw, fill=green!40] (key) at (-3,0) {
\begin{tabular}{c}
\underline{Matricola} \\
\small (CHIAVE)
\end{tabular}
};

\node[ellipse, draw, fill=blue!20] (attr1) at (0,2) {Nome};
\node[ellipse, draw, fill=blue!20] (attr2) at (3,0) {Cognome};

\draw[ultra thick] (key) -- (ent);
\draw (attr1) -- (ent);
\draw (attr2) -- (ent);
\end{tikzpicture}
\end{center}

\textbf{Proprietà della chiave:}
\begin{itemize}
\item \textcolor{red}{Univocità}: nessun duplicato
\item \textcolor{blue}{Minimalità}: minimo numero di attributi necessari
\end{itemize}
\end{frame}

% Slide 37: Tipi di chiavi
\begin{frame}{Tipi di Chiavi}
\begin{enumerate}
\item \textbf{Chiave Primaria (Primary Key - PK)}
\begin{itemize}
\item Identifica univocamente ogni istanza
\item Non può contenere valori NULL
\item Unica per ogni entità
\end{itemize}

\vspace{0.3cm}

\item \textbf{Chiave Candidata}
\begin{itemize}
\item Potenziale chiave primaria
\item Soddisfa requisiti di univocità
\end{itemize}

\vspace{0.3cm}

\item \textbf{Chiave Esterna (Foreign Key - FK)}
\begin{itemize}
\item Riferisce la chiave primaria di un'altra entità
\item Stabilisce relazioni tra entità
\end{itemize}
\end{enumerate}
\end{frame}

% Slide 38: Chiave primaria esempio
\begin{frame}{Esempio di Chiave Primaria}
\begin{center}
\begin{tikzpicture}[scale=0.9]
% Tabella concettuale
\node[rectangle, draw, fill=yellow!30, minimum width=8cm, minimum height=0.8cm] at (0,2) {
\textbf{AUTOMOBILE}
};

\node[rectangle, draw, minimum width=8cm, minimum height=2.5cm] (tab) at (0,0) {};

\node[align=left] at (0,0.8) {
\textcolor{red}{\underline{Targa}} $\leftarrow$ \textbf{Chiave Primaria} \\[0.2cm]
Marca \\[0.2cm]
Modello \\[0.2cm]
Anno \\[0.2cm]
Colore
};

\draw[thick, red] (-3.5,1.2) -- (3.5,1.2);
\end{tikzpicture}
\end{center}

La \textbf{Targa} identifica univocamente ogni automobile
\end{frame}

% Slide 39: Chiavi composte
\begin{frame}{Chiavi Composte}
\begin{block}{Definizione}
Una chiave composta è formata da \textbf{più attributi} che insieme identificano univocamente un'istanza
\end{block}

\begin{exampleblock}{Esempio}
Entità \textbf{ESAME}:
\begin{itemize}
\item Matricola (da sola non univoca)
\item CodiceCorso (da solo non univoco)
\item \textcolor{red}{\textbf{(Matricola, CodiceCorso)}} $\rightarrow$ Chiave composta univoca
\end{itemize}
\end{exampleblock}

\vspace{0.3cm}

\begin{center}
\begin{tikzpicture}[scale=0.7]
\node[rectangle, draw, fill=blue!30] (esame) at (0,0) {\textbf{ESAME}};
\node[ellipse, draw, fill=red!30] (matr) at (-2,1.5) {\underline{Matricola}};
\node[ellipse, draw, fill=red!30] (cod) at (2,1.5) {\underline{CodiceCorso}};
\node[ellipse, draw, fill=green!20] (voto) at (0,-1.5) {Voto};

\draw[thick] (esame) -- (matr);
\draw[thick] (esame) -- (cod);
\draw (esame) -- (voto);
\end{tikzpicture}
\end{center}
\end{frame}

% Slide 40: Entità forti e deboli
\begin{frame}{Entità Forti e Deboli}
\begin{block}{Entità Forti}
Hanno una chiave primaria propria e esistenza autonoma
\end{block}

\begin{block}{Entità Deboli}
Non hanno chiave primaria propria, dipendono da altre entità
\end{block}

\begin{center}
\begin{tikzpicture}[scale=0.8]
% Entità forte
\node[rectangle, draw, thick, fill=green!30, minimum width=2.5cm, minimum height=1cm] (forte) at (-2,0) {
\begin{tabular}{c}
\textbf{CLIENTE} \\
\underline{CodiceFiscale}
\end{tabular}
};

% Relazione
\node[diamond, draw, thick, fill=yellow!30, minimum width=1cm] (rel) at (1.5,0) {effettua};

% Entità debole
\node[rectangle, draw, thick, double, fill=red!30, minimum width=2.5cm, minimum height=1cm] (debole) at (5,0) {
\begin{tabular}{c}
\textbf{ORDINE} \\
NumeroOrdine
\end{tabular}
};

\draw[thick] (forte) -- (rel);
\draw[thick] (rel) -- (debole);
\end{tikzpicture}
\end{center}

Nota: il doppio bordo indica un'entità debole
\end{frame}

% Slide 41: Esempio entità debole
\begin{frame}{Esempio: Conto Corrente}
\begin{center}
\begin{tikzpicture}[scale=0.8]
% Entità forte
\node[rectangle, draw, thick, fill=green!30, minimum width=2.5cm, minimum height=1.2cm] (conto) at (0,2) {
\begin{tabular}{c}
\textbf{CONTO} \\
\underline{NumeroConto}
\end{tabular}
};

% Relazione
\node[diamond, draw, thick, fill=yellow!30] (ha) at (0,0) {ha};

% Entità debole
\node[rectangle, draw, thick, double, fill=red!30, minimum width=2.5cm, minimum height=1.5cm] (mov) at (0,-2.5) {
\begin{tabular}{c}
\textbf{MOVIMENTO} \\
Data \\
Causale \\
Importo
\end{tabular}
};

\draw[thick] (conto) -- (ha);
\draw[thick] (ha) -- (mov);

\node[right of=mov, node distance=4cm, text width=3cm, font=\small] {
Movimento non ha chiave primaria propria: dipende da Conto
};
\end{tikzpicture}
\end{center}
\end{frame}

% Slide 42: Soluzione entità debole
\begin{frame}{Soluzione: Aggiunta di Contatore}
\begin{alertblock}{Problema}
Entità deboli possono creare complessità nella gestione
\end{alertblock}

\begin{exampleblock}{Soluzione}
Aggiungere un attributo \textbf{numero progressivo} auto-generato
\end{exampleblock}

\begin{center}
\begin{tikzpicture}[scale=0.75]
\node[rectangle, draw, thick, fill=green!30, minimum width=2.5cm, minimum height=1.8cm] (mov) at (0,0) {
\begin{tabular}{c}
\textbf{MOVIMENTO} \\
\textcolor{red}{\underline{IdMovimento}} \\
Data \\
Causale \\
Importo
\end{tabular}
};

\node[right of=mov, node distance=5cm, text width=4cm, align=left] {
\textbf{Vantaggi:}
\begin{itemize}
\item Entità diventa forte
\item Ordinamento cronologico
\item Identificazione univoca
\end{itemize}
};
\end{tikzpicture}
\end{center}
\end{frame}

% Slide 43: Notazioni a confronto
\begin{frame}{Chiavi: Notazione Classica vs UML}
\begin{columns}
\column{0.5\textwidth}
\begin{center}
\textbf{Notazione Classica}
\begin{tikzpicture}[scale=0.65]
\node[rectangle, draw, fill=yellow!30, minimum width=2.5cm, minimum height=1cm] (ent) at (0,1) {\textbf{STUDENTE}};
\node[ellipse, draw, fill=green!30] (key) at (0,-0.5) {\underline{Matricola}};
\draw[thick] (ent) -- (key);
\end{tikzpicture}

Chiave sottolineata nell'ovale
\end{center}

\column{0.5\textwidth}
\begin{center}
\textbf{Notazione UML}
\begin{tikzpicture}[scale=0.65]
\node[rectangle, draw, fill=yellow!30, minimum width=2.5cm, minimum height=1.8cm] (ent) at (0,0) {
\begin{tabular}{l}
\textbf{STUDENTE} \\
\hline
\textcolor{red}{\textbf{PK}} \underline{Matricola} \\
Nome \\
Cognome
\end{tabular}
};
\end{tikzpicture}

Chiave con marcatore PK
\end{center}
\end{columns}

\vspace{0.5cm}
\begin{alertblock}{Nota}
Entrambe le notazioni sono valide e ampiamente usate
\end{alertblock}
\end{frame}

% Slide 44: Tool online
\begin{frame}{Tool per Diagrammi E-R}
\begin{block}{Designer Online}
Disponibile gratuitamente:
\begin{center}
\large
\textcolor{blue}{\texttt{https://designer-basic.polito.it/}}
\end{center}
\end{block}

\begin{center}
\begin{tikzpicture}
\draw[fill=white, draw=black, thick] (0,0) rectangle (8,4);
\node at (4,3.5) {\Large Designer E-R Online};
\draw[fill=blue!20] (0.5,0.5) rectangle (3.5,2.5);
\node at (2,1.5) {Entità};
\draw[fill=green!20] (4.5,0.5) rectangle (7.5,2.5);
\node at (6,1.5) {Relazioni};
\draw[->, ultra thick] (3.5,1.5) -- (4.5,1.5);
\end{tikzpicture}
\end{center}
\end{frame}

\section{Relazioni}

% Slide 45: Molteplicità relazioni
\begin{frame}{Molteplicità delle Relazioni}
\begin{block}{Definizione}
La molteplicità indica il numero di possibili istanze di un'entità associate ad un'istanza dell'altra entità
\end{block}

\textbf{Rappresentazione:} coppia di valori \texttt{min..max}
\begin{itemize}
\item \texttt{1..1} - esattamente uno
\item \texttt{0..1} - zero o uno (opzionale)
\item \texttt{1..N} - uno o molti
\item \texttt{0..N} - zero o molti
\end{itemize}

\vspace{0.3cm}

\begin{exampleblock}{Concetti importanti}
\begin{itemize}
\item \textbf{Valore minimo} $\rightarrow$ Obbligatorietà (0=facoltativo, 1=obbligatorio)
\item \textbf{Valore massimo} $\rightarrow$ Cardinalità (1=uno, N=molti)
\end{itemize}
\end{exampleblock}
\end{frame}

% Slide 46: Cardinalità notazione classica
\begin{frame}{Cardinalità - Notazione Classica}
\begin{center}
\begin{tikzpicture}[scale=0.9]
% Esempio 1:1
\node[rectangle, draw, fill=blue!30, minimum width=2cm] (e1) at (0,3) {Entità A};
\node[diamond, draw, fill=yellow!30] (r1) at (3,3) {R};
\node[rectangle, draw, fill=blue!30, minimum width=2cm] (e2) at (6,3) {Entità B};
\draw[thick] (e1) -- (r1) node[midway, above] {\small 1};
\draw[thick] (r1) -- (e2) node[midway, above] {\small 1};
\node at (3,2.3) {Relazione 1:1};

% Esempio 1:N
\node[rectangle, draw, fill=green!30, minimum width=2cm] (e3) at (0,0) {Entità C};
\node[diamond, draw, fill=yellow!30] (r2) at (3,0) {S};
\node[rectangle, draw, fill=green!30, minimum width=2cm] (e4) at (6,0) {Entità D};
\draw[thick] (e3) -- (r2) node[midway, above] {\small 1};
\draw[thick] (r2) -- (e4) node[midway, above] {\small N};
\node at (3,-0.7) {Relazione 1:N};

% Esempio N:M
\node[rectangle, draw, fill=red!30, minimum width=2cm] (e5) at (0,-3) {Entità E};
\node[diamond, draw, fill=yellow!30] (r3) at (3,-3) {T};
\node[rectangle, draw, fill=red!30, minimum width=2cm] (e6) at (6,-3) {Entità F};
\draw[thick] (e5) -- (r3) node[midway, above] {\small N};
\draw[thick] (r3) -- (e6) node[midway, above] {\small M};
\node at (3,-3.7) {Relazione N:M};
\end{tikzpicture}
\end{center}
\end{frame}

% Slide 47: Relazione 1:1
\begin{frame}{Le Relazioni di Tipo 1:1}
\begin{block}{Definizione}
Ad ogni istanza della prima entità corrisponde \textbf{al più} un'istanza della seconda, e viceversa
\end{block}

\begin{center}
\begin{tikzpicture}[scale=0.9]
\node[rectangle, draw, fill=yellow!30, minimum width=2.5cm, minimum height=1cm] (naz) at (0,0) {\textbf{NAZIONE}};
\node[diamond, draw, fill=blue!30, minimum width=1.5cm] (rel) at (4,0) {ha capitale};
\node[rectangle, draw, fill=green!30, minimum width=2.5cm, minimum height=1cm] (cit) at (8,0) {\textbf{CITTÀ}};

\draw[thick] (naz) -- (rel) node[midway, above] {1};
\draw[thick] (rel) -- (cit) node[midway, above] {1};
\end{tikzpicture}
\end{center}

\begin{exampleblock}{Esempio}
\begin{itemize}
\item Ogni nazione ha \textbf{una} capitale (1 $\rightarrow$ 1)
\item Ogni capitale appartiene a \textbf{una} nazione (1 $\leftarrow$ 1)
\end{itemize}
\end{exampleblock}
\end{frame}

% Slide 48: Esempio relazione 1:1
\begin{frame}{Relazione 1:1 - Altro Esempio}
\begin{center}
\begin{tikzpicture}[scale=0.85]
\node[rectangle, draw, fill=blue!30, minimum width=2.5cm, minimum height=1cm] (pers) at (0,0) {\textbf{PERSONA}};
\node[diamond, draw, fill=yellow!30, minimum width=1.5cm] (rel) at (4,0) {possiede};
\node[rectangle, draw, fill=green!30, minimum width=2.5cm, minimum height=1cm] (cf) at (8,0) {\textbf{CODICE\_FISCALE}};

\draw[thick] (pers) -- (rel) node[midway, above] {1};
\draw[thick] (rel) -- (cf) node[midway, above] {1};

\node[below of=pers, node distance=1.5cm, text width=3cm, font=\small, align=center] {
Partecipazione obbligatoria \\
Molteplicità 1
};

\node[below of=cf, node distance=1.5cm, text width=3cm, font=\small, align=center] {
Partecipazione obbligatoria \\
Molteplicità 1
};
\end{tikzpicture}
\end{center}
\end{frame}

% Slide 49: Relazione 1:N
\begin{frame}{Relazioni Uno a Molti (1:N)}
\begin{block}{Definizione}
Ad un'istanza della prima entità corrispondono \textbf{più istanze} della seconda, ma ad ogni istanza della seconda corrisponde \textbf{al più una} istanza della prima
\end{block}

\begin{center}
\begin{tikzpicture}[scale=0.85]
\node[rectangle, draw, fill=yellow!30, minimum width=2.5cm, minimum height=1cm] (citta) at (0,0) {\textbf{CITTÀ}};
\node[diamond, draw, fill=blue!30, minimum width=1.5cm] (rel) at (4,0) {risiede};
\node[rectangle, draw, fill=green!30, minimum width=2.5cm, minimum height=1cm] (pers) at (8,0) {\textbf{PERSONA}};

\draw[thick] (citta) -- (rel) node[midway, above] {1};
\draw[thick] (rel) -- (pers) node[midway, above] {N};
\end{tikzpicture}
\end{center}

\begin{exampleblock}{Interpretazione}
\begin{itemize}
\item Una città ha \textbf{molte} persone residenti
\item Una persona risiede in \textbf{una sola} città
\end{itemize}
\end{exampleblock}
\end{frame}

% Slide 50: Relazione N:M
\begin{frame}{Relazioni Molti a Molti (N:M)}
\begin{block}{Definizione}
Ad un'istanza della prima entità corrispondono \textbf{più istanze} della seconda, e viceversa
\end{block}

\begin{center}
\begin{tikzpicture}[scale=0.85]
\node[rectangle, draw, fill=yellow!30, minimum width=2.5cm, minimum height=1cm] (stud) at (0,0) {\textbf{STUDENTE}};
\node[diamond, draw, fill=blue!30, minimum width=1.5cm] (rel) at (4,0) {frequenta};
\node[rectangle, draw, fill=green!30, minimum width=2.5cm, minimum height=1cm] (corso) at (8,0) {\textbf{CORSO}};

\draw[thick] (stud) -- (rel) node[midway, above] {N};
\draw[thick] (rel) -- (corso) node[midway, above] {M};
\end{tikzpicture}
\end{center}

\begin{exampleblock}{Interpretazione}
\begin{itemize}
\item Uno studente frequenta \textbf{più corsi}
\item Un corso è frequentato da \textbf{più studenti}
\end{itemize}
\end{exampleblock}
\end{frame}

% Slide 51: Esistenza obbligatoria
\begin{frame}{Esistenza Obbligatoria}
\begin{block}{Definizione}
Un'istanza di un'entità \textbf{deve necessariamente} partecipare alla relazione
\end{block}

\begin{center}
\begin{tikzpicture}[scale=0.85]
\node[rectangle, draw, fill=yellow!30, minimum width=2.5cm, minimum height=1cm] (ord) at (0,0) {\textbf{ORDINE}};
\node[diamond, draw, fill=blue!30, minimum width=1.5cm] (rel) at (4,0) {effettuato da};
\node[rectangle, draw, fill=green!30, minimum width=2.5cm, minimum height=1cm] (cli) at (8,0) {\textbf{CLIENTE}};

\draw[ultra thick] (ord) -- (rel) node[midway, above] {(1,1)};
\draw[thick] (rel) -- (cli) node[midway, above] {(0,N)};

\node[below of=ord, node distance=1.5cm, font=\small, text width=3cm, align=center] {
\textcolor{red}{Partecipazione OBBLIGATORIA} \\
(minimo = 1)
};
\end{tikzpicture}
\end{center}

Un ordine DEVE avere un cliente
\end{frame}

% Slide 52: Esistenza opzionale
\begin{frame}{Esistenza Opzionale}
\begin{block}{Definizione}
Un'istanza di un'entità \textbf{può} partecipare facoltativamente alla relazione
\end{block}

\begin{center}
\begin{tikzpicture}[scale=0.85]
\node[rectangle, draw, fill=yellow!30, minimum width=2.5cm, minimum height=1cm] (utente) at (0,0) {\textbf{UTENTE}};
\node[diamond, draw, fill=blue!30, minimum width=1.5cm] (rel) at (4,0) {prende prestito};
\node[rectangle, draw, fill=green!30, minimum width=2.5cm, minimum height=1cm] (libro) at (8,0) {\textbf{LIBRO}};

\draw[thick, dashed] (utente) -- (rel) node[midway, above] {(0,1)};
\draw[thick] (rel) -- (libro) node[midway, above] {(1,N)};

\node[below of=utente, node distance=1.5cm, font=\small, text width=3cm, align=center] {
\textcolor{blue}{Partecipazione OPZIONALE} \\
(minimo = 0)
};
\end{tikzpicture}
\end{center}

Un utente PUÒ prendere in prestito un libro (o anche nessuno)
\end{frame}

% Slide 53: Molteplicità - sintesi
\begin{frame}{Molteplicità - Sintesi}
\begin{center}
\begin{tikzpicture}[scale=0.7]
\node[rectangle, draw, fill=blue!30, minimum width=2.5cm, minimum height=1cm] (e1) at (0,4) {\textbf{Entità A}};
\node[diamond, draw, fill=yellow!30] (r1) at (4,4) {R1};
\node[rectangle, draw, fill=blue!30, minimum width=2.5cm, minimum height=1cm] (e2) at (8,4) {\textbf{Entità B}};
\draw[thick] (e1) -- (r1) node[midway, above] {(1,1)};
\draw[thick] (r1) -- (e2) node[midway, above] {(1,1)};
\node at (4,3) {\small Obbligatorio-Scalare};

\node[rectangle, draw, fill=green!30, minimum width=2.5cm, minimum height=1cm] (e3) at (0,1.5) {\textbf{Entità C}};
\node[diamond, draw, fill=yellow!30] (r2) at (4,1.5) {R2};
\node[rectangle, draw, fill=green!30, minimum width=2.5cm, minimum height=1cm] (e4) at (8,1.5) {\textbf{Entità D}};
\draw[thick] (e3) -- (r2) node[midway, above] {(0,1)};
\draw[thick] (r2) -- (e4) node[midway, above] {(1,N)};
\node at (4,0.5) {\small Opzionale-Multiplo};

\node[rectangle, draw, fill=red!30, minimum width=2.5cm, minimum height=1cm] (e5) at (0,-1) {\textbf{Entità E}};
\node[diamond, draw, fill=yellow!30] (r3) at (4,-1) {R3};
\node[rectangle, draw, fill=red!30, minimum width=2.5cm, minimum height=1cm] (e6) at (8,-1) {\textbf{Entità F}};
\draw[thick] (e5) -- (r3) node[midway, above] {(1,N)};
\draw[thick] (r3) -- (e6) node[midway, above] {(0,N)};
\node at (4,-2) {\small Obbligatorio-Opzionale Multiplo};
\end{tikzpicture}
\end{center}
\end{frame}

% Slide 54: Vincoli cardinalità esempio
\begin{frame}{Esempi di Vincoli di Cardinalità}
\begin{exampleblock}{Esempio 1: Persona - Codice Fiscale}
\begin{itemize}
\item Una persona possiede \textbf{almeno 1 e massimo 1} codice fiscale ($\rightarrow$)
\item Un codice fiscale appartiene a \textbf{almeno 1 e massimo 1} persona ($\leftarrow$)
\end{itemize}
\end{exampleblock}

\begin{exampleblock}{Esempio 2: Persona - Città}
\begin{itemize}
\item Una persona risiede in \textbf{almeno 1 e al massimo 1} città ($\rightarrow$)
\item In una città risiedono \textbf{almeno 1 e al massimo N} persone ($\leftarrow$)
\end{itemize}
\end{exampleblock}
\end{frame}

% Slide 55: Direzione relazione
\begin{frame}{Direzione della Relazione}
\begin{block}{Concetto di "padre"}
Nella relazione uno-a-molti, l'entità con cardinalità 1 è detta \textbf{padre}
\end{block}

\begin{center}
\begin{tikzpicture}[scale=0.9]
\node[rectangle, draw, fill=green!40, minimum width=2.5cm, minimum height=1cm] (citta) at (0,0) {
\begin{tabular}{c}
\textbf{CITTÀ} \\
\small (PADRE)
\end{tabular}
};
\node[diamond, draw, fill=yellow!30] (rel) at (4,0) {risiede};
\node[rectangle, draw, fill=blue!30, minimum width=2.5cm, minimum height=1cm] (stud) at (8,0) {
\begin{tabular}{c}
\textbf{STUDENTE} \\
\small (FIGLIO)
\end{tabular}
};

\draw[->, ultra thick, red] (citta) -- (rel) node[midway, above] {1};
\draw[thick] (rel) -- (stud) node[midway, above] {N};

\node[below of=rel, node distance=1.5cm, text width=4cm, align=center] {
La direzione va da CITTÀ a STUDENTE \\
\small (dal padre al figlio)
};
\end{tikzpicture}
\end{center}
\end{frame}

% Slide 56: Regole lettura
\begin{frame}{Le Regole di Lettura}
\begin{center}
\begin{tikzpicture}[scale=0.85]
\node[rectangle, draw, fill=yellow!30, minimum width=2.5cm, minimum height=1cm] (ut) at (0,0) {\textbf{UTENTE}};
\node[diamond, draw, fill=blue!30, text width=2cm, align=center] (rel) at (4,0) {prende in prestito};
\node[rectangle, draw, fill=green!30, minimum width=2.5cm, minimum height=1cm] (lib) at (8,0) {\textbf{LIBRO}};

\draw[thick] (ut) -- (rel) node[midway, above] {(0,1)};
\draw[thick] (rel) -- (lib) node[midway, above] {(1,N)};
\end{tikzpicture}
\end{center}

\textbf{Lettura da sinistra (UTENTE):}
\begin{itemize}
\item \textcolor{red}{(0,} $\rightarrow$ può prendere in prestito
\item \textcolor{red}{,1)} $\rightarrow$ un solo libro
\end{itemize}

\textbf{Lettura da destra (LIBRO):}
\begin{itemize}
\item \textcolor{blue}{(1,} $\rightarrow$ un libro deve essere prestato
\item \textcolor{blue}{,N)} $\rightarrow$ a uno o più utenti (nel tempo)
\end{itemize}
\end{frame}

\section{Relazioni Gerarchiche}

% Slide 57: Gerarchia tra entità
\begin{frame}{Relazione Gerarchica tra Entità}
\begin{block}{Definizione}
Situazioni in cui tra le entità può essere stabilita una gerarchia, simile alle classi nella programmazione OOP
\end{block}

\begin{center}
\begin{tikzpicture}[scale=0.8]
\node[rectangle, draw, fill=blue!30, minimum width=2.5cm, minimum height=1cm] (alfa) at (0,2) {\textbf{ALFA}};
\node[rectangle, draw, fill=green!30, minimum width=2cm, minimum height=0.8cm] (beta1) at (-2,0) {\textbf{BETA1}};
\node[rectangle, draw, fill=green!30, minimum width=2cm, minimum height=0.8cm] (beta2) at (2,0) {\textbf{BETA2}};

\draw[->, ultra thick] (beta1) -- (alfa);
\draw[->, ultra thick] (beta2) -- (alfa);

\node[left of=alfa, node distance=3.5cm, text width=3cm, font=\small] {
\textbf{Superclasse} \\
(Generalizzazione)
};

\node[left of=beta1, node distance=3cm, text width=2.5cm, font=\small] {
\textbf{Sottoclassi} \\
(Specializzazioni)
};
\end{tikzpicture}
\end{center}
\end{frame}

% Slide 58: Vincoli gerarchia
\begin{frame}{Vincoli nelle Gerarchie}
\begin{block}{Due vincoli fondamentali:}
\end{block}

\textbf{1. Vincolo di Struttura (Ereditarietà)}
\begin{itemize}
\item La sottoclasse eredita tutti gli attributi della superclasse
\item La sottoclasse partecipa a tutte le associazioni della superclasse
\item La sottoclasse può avere attributi e associazioni aggiuntive
\end{itemize}

\vspace{0.3cm}

\textbf{2. Vincolo di Insieme (Subset)}
\begin{itemize}
\item Ogni oggetto della sottoclasse è anche un oggetto della superclasse
\item La sottoclasse è un \textbf{sottoinsieme} della superclasse
\end{itemize}
\end{frame}

% Slide 59: Esempio gerarchia veicoli
\begin{frame}{Esempio: Gerarchia Veicoli}
\begin{center}
\begin{tikzpicture}[scale=0.75]
\node[rectangle, draw, fill=blue!30, minimum width=2.5cm, minimum height=1.2cm] (veic) at (0,3) {
\begin{tabular}{c}
\textbf{VEICOLO} \\
\small Targa, Marca
\end{tabular}
};

\node[rectangle, draw, fill=green!30, minimum width=2cm, minimum height=1cm] (auto) at (-3,0.5) {
\begin{tabular}{c}
\textbf{AUTO} \\
\small NumPorte
\end{tabular}
};

\node[rectangle, draw, fill=green!30, minimum width=2cm, minimum height=1cm] (moto) at (0,0.5) {
\begin{tabular}{c}
\textbf{MOTO} \\
\small Cilindrata
\end{tabular}
};

\node[rectangle, draw, fill=green!30, minimum width=2cm, minimum height=1cm] (bici) at (3,0.5) {
\begin{tabular}{c}
\textbf{BICICLETTA} \\
\small NumMarce
\end{tabular}
};

\draw[->, thick] (auto) -- (veic);
\draw[->, thick] (moto) -- (veic);
\draw[->, thick] (bici) -- (veic);

\node[below of=veic, node distance=1cm, font=\small] {$\triangle$ Generalizzazione};
\end{tikzpicture}
\end{center}

Auto, Moto e Bicicletta \textbf{specializzano} Veicolo
\end{frame}

% Slide 60: Copertura generalizzazioni
\begin{frame}{Copertura delle Generalizzazioni}
\begin{block}{Concetto di Copertura}
Le sottoentità non sempre contemplano tutti gli elementi della classe padre
\end{block}

\textbf{Due dimensioni indipendenti:}

\begin{enumerate}
\item \textbf{Confronto con l'unione:}
\begin{itemize}
\item \textcolor{red}{Totale}: la superclasse è l'unione delle sottoclassi
\item \textcolor{blue}{Parziale}: la superclasse contiene l'unione delle sottoclassi
\end{itemize}

\vspace{0.3cm}

\item \textbf{Confronto tra sottoclassi:}
\begin{itemize}
\item \textcolor{green}{Esclusiva}: le sottoclassi sono disgiunte
\item \textcolor{orange}{Sovrapposta}: può esistere intersezione tra sottoclassi
\end{itemize}
\end{enumerate}
\end{frame}

% Slide 61: Grazie
\begin{frame}[plain]
\begin{center}
\Huge \textbf{Grazie per l'attenzione!}

\vspace{1cm}

\Large Domande?

\vspace{1.5cm}

\normalsize
Prof. Fedeli Massimo \\
ITS Academy - Fabbrica Digitale

\vspace{0.5cm}

\begin{tikzpicture}
\draw[fill=blue!20, draw=blue, thick] (0,0) circle (1.5cm);
\node[font=\Large] at (0,0.3) {E-R};
\node[font=\small] at (0,-0.3) {Modeling};
\end{tikzpicture}
\end{center}
\end{frame}

\end{document}
