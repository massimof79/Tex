\documentclass[a4paper,12pt]{article}
\usepackage[utf8]{inputenc}
\usepackage[italian]{babel}
\usepackage{geometry}
\usepackage{graphicx}
\usepackage{enumitem}
\usepackage{xcolor}
\usepackage{tcolorbox}
\usepackage{tikz}
\usetikzlibrary{er,positioning,shapes}

\geometry{margin=2.5cm}

\title{\textbf{Tipologie di Attributi nel Modello Entità-Relazione}}
\author{Prof. Fedeli Massimo - Tutti i diritti riservati}


\definecolor{primarycolor}{RGB}{41,128,185}
\definecolor{secondarycolor}{RGB}{52,152,219}
\definecolor{examplecolor}{RGB}{241,196,15}

\begin{document}
	
	\maketitle
	
	\section{Introduzione}
	
	Nel modello Entità-Relazione (E-R), gli attributi rappresentano le proprietà che caratterizzano un'entità o una relazione. La corretta classificazione degli attributi è fondamentale per una progettazione efficace del database.
	
	In questo documento esamineremo le diverse tipologie di attributi:
	\begin{itemize}
		\item Attributi semplici vs. composti
		\item Attributi monovalore vs. multivalore
		\item Attributi calcolati (derivati)
	\end{itemize}
	
	\section{Attributi Semplici}
	
	\subsection{Definizione}
	Un \textbf{attributo semplice} (o atomico) è un attributo che non può essere suddiviso in componenti più elementari. Rappresenta un valore indivisibile all'interno del contesto del dominio applicativo.
	
	\subsection{Caratteristiche}
	\begin{itemize}
		\item Non possono essere ulteriormente scomposti
		\item Rappresentano informazioni elementari
		\item Sono gli attributi più comuni nei database
	\end{itemize}
	
	\subsection{Esempi Pratici}
	
	\begin{tcolorbox}[colback=examplecolor!10,colframe=examplecolor!80,title=Esempio 1: Entità STUDENTE]
		\textbf{Attributi semplici:}
		\begin{itemize}
			\item \texttt{Matricola}: "2024001"
			\item \texttt{Nome}: "Mario"
			\item \texttt{Cognome}: "Rossi"
			\item \texttt{DataNascita}: "15/03/2005"
			\item \texttt{Email}: "mario.rossi@studenti.school.it"
		\end{itemize}
	\end{tcolorbox}
	
	\begin{tcolorbox}[colback=examplecolor!10,colframe=examplecolor!80,title=Esempio 2: Entità PRODOTTO]
		\textbf{Attributi semplici:}
		\begin{itemize}
			\item \texttt{CodiceProdotto}: "P-12345"
			\item \texttt{NomeProdotto}: "Laptop Dell XPS 15"
			\item \texttt{Prezzo}: 1299.99
			\item \texttt{Peso}: 1.8 (kg)
			\item \texttt{Colore}: "Argento"
		\end{itemize}
	\end{tcolorbox}
	
	\section{Attributi Composti}
	
	\subsection{Definizione}
	Un \textbf{attributo composto} è un attributo che può essere suddiviso in sotto-attributi più elementari, ciascuno con un significato indipendente. Rappresenta un'informazione strutturata che può essere vista sia come unità sia come insieme di parti.
	
	\subsection{Caratteristiche}
	\begin{itemize}
		\item Possono essere scomposti in componenti più semplici
		\item Ogni componente ha un significato autonomo
		\item Offrono flessibilità nell'accesso ai dati
		\item Possono avere più livelli di composizione
	\end{itemize}
	
	\subsection{Esempi Pratici}
	
	\begin{tcolorbox}[colback=secondarycolor!10,colframe=secondarycolor!80,title=Esempio 1: Attributo INDIRIZZO]
		\textbf{Attributo composto:} \texttt{Indirizzo}
		
		\textbf{Sotto-attributi:}
		\begin{itemize}
			\item \texttt{Via}: "Via Roma"
			\item \texttt{NumeroCivico}: "123"
			\item \texttt{CAP}: "63100"
			\item \texttt{Città}: "Ascoli Piceno"
			\item \texttt{Provincia}: "AP"
		\end{itemize}
		
		\textbf{Valore completo:} "Via Roma 123, 63100 Ascoli Piceno (AP)"
	\end{tcolorbox}
	
	\begin{tcolorbox}[colback=secondarycolor!10,colframe=secondarycolor!80,title=Esempio 2: Attributo NOME\_COMPLETO]
		\textbf{Attributo composto:} \texttt{NomeCompleto}
		
		\textbf{Sotto-attributi:}
		\begin{itemize}
			\item \texttt{Nome}: "Giovanni"
			\item \texttt{SecondoNome}: "Paolo"
			\item \texttt{Cognome}: "Bianchi"
		\end{itemize}
		
		\textbf{Valore completo:} "Giovanni Paolo Bianchi"
	\end{tcolorbox}
	
	\begin{tcolorbox}[colback=secondarycolor!10,colframe=secondarycolor!80,title=Esempio 3: Attributo DATA\_ORA (composizione a più livelli)]
		\textbf{Attributo composto:} \texttt{DataOra}
		
		\textbf{Sotto-attributi:}
		\begin{itemize}
			\item \texttt{Data}:
			\begin{itemize}
				\item \texttt{Giorno}: 15
				\item \texttt{Mese}: 11
				\item \texttt{Anno}: 2024
			\end{itemize}
			\item \texttt{Ora}:
			\begin{itemize}
				\item \texttt{Ore}: 14
				\item \texttt{Minuti}: 30
				\item \texttt{Secondi}: 00
			\end{itemize}
		\end{itemize}
	\end{tcolorbox}
	
	\subsection{Quando utilizzare attributi composti}
	
	\textbf{Utilizzare attributi composti quando:}
	\begin{itemize}
		\item Si necessita di accedere sia all'informazione completa che alle singole componenti
		\item Le componenti hanno un significato autonomo nell'applicazione
		\item Si vuole mantenere flessibilità per query e ricerche sui singoli elementi
	\end{itemize}
	
	\section{Attributi Monovalore}
	
	\subsection{Definizione}
	Un \textbf{attributo monovalore} (o singolo) è un attributo che può assumere un solo valore per ciascuna istanza dell'entità. È la tipologia più comune di attributo.
	
	\subsection{Caratteristiche}
	\begin{itemize}
		\item Hanno cardinalità (1,1)
		\item Un'istanza dell'entità ha esattamente un valore per l'attributo
		\item Rappresentano proprietà univoche
	\end{itemize}
	
	\subsection{Esempi Pratici}
	
	\begin{tcolorbox}[colback=primarycolor!10,colframe=primarycolor!80,title=Esempio: Entità DIPENDENTE]
		\textbf{Attributi monovalore:}
		\begin{itemize}
			\item \texttt{CodiceFiscale}: "RSSMRA85M15F839K" \\
			\textit{Ogni dipendente ha un solo codice fiscale}
			
			\item \texttt{DataAssunzione}: "01/09/2020" \\
			\textit{Ogni dipendente ha una sola data di assunzione}
			
			\item \texttt{Stipendio}: 2500.00 \\
			\textit{Ogni dipendente ha un solo stipendio corrente}
			
			\item \texttt{Reparto}: "Amministrazione" \\
			\textit{Ogni dipendente appartiene a un solo reparto}
		\end{itemize}
	\end{tcolorbox}
	
	\section{Attributi Multivalore}
	
	\subsection{Definizione}
	Un \textbf{attributo multivalore} è un attributo che può assumere più valori contemporaneamente per la stessa istanza dell'entità. Rappresenta un insieme di valori per una singola proprietà.
	
	\subsection{Caratteristiche}
	\begin{itemize}
		\item Possono avere zero, uno o più valori
		\item Hanno cardinalità (0,n) o (1,n)
		\item Richiedono attenzione particolare nella normalizzazione
		\item Nel diagramma E-R sono rappresentati con doppia ellisse
	\end{itemize}
	
	\subsection{Esempi Pratici}
	
	\begin{tcolorbox}[colback=primarycolor!10,colframe=primarycolor!80,title=Esempio 1: Entità PERSONA]
		\textbf{Attributo multivalore:} \texttt{NumeriTelefono}
		
		\textbf{Valori possibili per Mario Rossi:}
		\begin{itemize}
			\item "0736-123456" (telefono fisso)
			\item "333-1234567" (cellulare personale)
			\item "348-7654321" (cellulare lavoro)
		\end{itemize}
		
		\textit{Una persona può avere più numeri di telefono}
	\end{tcolorbox}
	
	\begin{tcolorbox}[colback=primarycolor!10,colframe=primarycolor!80,title=Esempio 2: Entità LIBRO]
		\textbf{Attributo multivalore:} \texttt{Autori}
		
		\textbf{Valori per "Fondamenti di Database":}
		\begin{itemize}
			\item "Elmasri, Ramez"
			\item "Navathe, Shamkant"
		\end{itemize}
		
		\textit{Un libro può avere più autori}
	\end{tcolorbox}
	
	\begin{tcolorbox}[colback=primarycolor!10,colframe=primarycolor!80,title=Esempio 3: Entità STUDENTE]
		\textbf{Attributo multivalore:} \texttt{LingueStraniere}
		
		\textbf{Valori per uno studente:}
		\begin{itemize}
			\item "Inglese"
			\item "Francese"
			\item "Spagnolo"
		\end{itemize}
		
		\textit{Uno studente può conoscere più lingue straniere}
	\end{tcolorbox}
	
	\subsection{Considerazioni sulla Normalizzazione}
	
	Gli attributi multivalore violano la Prima Forma Normale (1NF). Durante la trasformazione in modello relazionale, vanno trattati creando:
	\begin{enumerate}
		\item Una tabella separata per l'attributo multivalore
		\item Una relazione uno-a-molti tra l'entità principale e la nuova tabella
	\end{enumerate}
	
	\textbf{Esempio di trasformazione:}
	
	\textit{Modello E-R:}
	\begin{verbatim}
		PERSONA(CodiceFiscale, Nome, Cognome, {NumeriTelefono})
	\end{verbatim}
	
	\textit{Modello Relazionale normalizzato:}
	\begin{verbatim}
		PERSONA(CodiceFiscale, Nome, Cognome)
		TELEFONO(CodiceFiscale*, NumeroTelefono)
	\end{verbatim}
	
	\section{Attributi Calcolati (Derivati)}
	
	\subsection{Definizione}
	Un \textbf{attributo calcolato} (o derivato) è un attributo il cui valore può essere ricavato da altri attributi presenti nel database, mediante un'operazione o calcolo. Il valore non viene memorizzato fisicamente ma calcolato quando necessario.
	
	\subsection{Caratteristiche}
	\begin{itemize}
		\item Il valore è derivato da altri attributi
		\item Non dovrebbero essere memorizzati nel database (ridondanza)
		\item Sono indicati con ellisse tratteggiata nel diagramma E-R
		\item Possono essere calcolati mediante query o viste
	\end{itemize}
	
	\subsection{Esempi Pratici}
	
	\begin{tcolorbox}[colback=green!10,colframe=green!60,title=Esempio 1: PERSONA - Età]
		\textbf{Attributo calcolato:} \texttt{Età}
		
		\textbf{Attributo base:} \texttt{DataNascita} = "15/03/2005"
		
		\textbf{Calcolo:}
		\begin{verbatim}
			Età = AnnoCorrente - AnnoNascita
			Età = 2024 - 2005 = 19 anni
		\end{verbatim}
		
		\textit{L'età cambia automaticamente ogni anno e dipende dalla data di nascita}
	\end{tcolorbox}
	
	\begin{tcolorbox}[colback=green!10,colframe=green!60,title=Esempio 2: ORDINE - Totale]
		\textbf{Attributo calcolato:} \texttt{TotaleOrdine}
		
		\textbf{Attributi base:}
		\begin{itemize}
			\item Prodotti nell'ordine con Quantità e PrezzoUnitario
		\end{itemize}
		
		\textbf{Calcolo:}
		\begin{verbatim}
			TotaleOrdine = Somma(Quantità × PrezzoUnitario)
			
			Esempio:
			- Prodotto A: 2 × 10.00 = 20.00
			- Prodotto B: 1 × 35.00 = 35.00
			- Prodotto C: 3 × 8.50 = 25.50
			TotaleOrdine = 80.50 €
		\end{verbatim}
	\end{tcolorbox}
	
	\begin{tcolorbox}[colback=green!10,colframe=green!60,title=Esempio 3: STUDENTE - Media Voti]
		\textbf{Attributo calcolato:} \texttt{MediaVoti}
		
		\textbf{Attributi base:} Voti degli esami sostenuti
		
		\textbf{Calcolo:}
		\begin{verbatim}
			MediaVoti = Somma(Voti) / NumeroEsami
			
			Esempio:
			Voti: 28, 30, 27, 25, 29, 30
			MediaVoti = (28+30+27+25+29+30) / 6 = 28.17
		\end{verbatim}
	\end{tcolorbox}
	
	\begin{tcolorbox}[colback=green!10,colframe=green!60,title=Esempio 4: DIPENDENTE - Anzianità]
		\textbf{Attributo calcolato:} \texttt{AnniServizio}
		
		\textbf{Attributo base:} \texttt{DataAssunzione} = "01/09/2015"
		
		\textbf{Calcolo:}
		\begin{verbatim}
			AnniServizio = DataCorrente - DataAssunzione
			AnniServizio = 2024 - 2015 = 9 anni
		\end{verbatim}
	\end{tcolorbox}
	
	\begin{tcolorbox}[colback=green!10,colframe=green!60,title=Esempio 5: RETTANGOLO - Area]
		\textbf{Attributo calcolato:} \texttt{Area}
		
		\textbf{Attributi base:}
		\begin{itemize}
			\item \texttt{Base} = 10 cm
			\item \texttt{Altezza} = 5 cm
		\end{itemize}
		
		\textbf{Calcolo:}
		\begin{verbatim}
			Area = Base × Altezza
			Area = 10 × 5 = 50 cm²
		\end{verbatim}
	\end{tcolorbox}
	
	\subsection{Vantaggi e Svantaggi}
	
	\textbf{Vantaggi del non memorizzare:}
	\begin{itemize}
		\item Evita ridondanza dei dati
		\item Garantisce sempre valori aggiornati
		\item Riduce lo spazio di archiviazione
		\item Elimina problemi di inconsistenza
	\end{itemize}
	
	\textbf{Quando considerare la memorizzazione:}
	\begin{itemize}
		\item Calcoli molto complessi o costosi
		\item Valori storici da preservare (es: età al momento dell'iscrizione)
		\item Necessità di elevate performance
		\item Dati che raramente cambiano
	\end{itemize}
	
	\section{Tabella Riepilogativa}
	
	\begin{table}[h]
		\centering
		\begin{tabular}{|l|p{10cm}|}
			\hline
			\textbf{Tipo} & \textbf{Caratteristiche Principali} \\
			\hline
			\textbf{Semplice} & Valore atomico, indivisibile (es: Nome, Prezzo) \\
			\hline
			\textbf{Composto} & Suddivisibile in sotto-attributi (es: Indirizzo, NomeCompleto) \\
			\hline
			\textbf{Monovalore} & Un solo valore per istanza (es: CodiceFiscale, DataNascita) \\
			\hline
			\textbf{Multivalore} & Più valori per istanza (es: NumeriTelefono, Lingue) \\
			\hline
			\textbf{Calcolato} & Derivato da altri attributi (es: Età, TotaleOrdine) \\
			\hline
		\end{tabular}
		\caption{Confronto tra tipologie di attributi}
	\end{table}
	
	\section{Esempi Combinati}
	
	Un attributo può appartenere a più categorie contemporaneamente:
	
	\begin{tcolorbox}[colback=yellow!10,colframe=orange!80,title=Esempio Complesso: Entità DIPENDENTE]
		\textbf{Attributi con classificazione multipla:}
		
		\begin{itemize}
			\item \texttt{CodiceFiscale}: \textit{Semplice + Monovalore}
			\item \texttt{Indirizzo}: \textit{Composto + Monovalore}
			\item \texttt{NumeriTelefono}: \textit{Semplice + Multivalore}
			\item \texttt{NomeCompleto}: \textit{Composto + Monovalore}
			\item \texttt{Età}: \textit{Semplice + Monovalore + Calcolato}
			\item \texttt{Competenze}: \textit{Semplice + Multivalore}
		\end{itemize}
	\end{tcolorbox}
	
	\section{Esercizi}
	
	\subsection{Esercizio 1: Classificazione}
	Classifica i seguenti attributi dell'entità \texttt{VEICOLO}:
	\begin{enumerate}
		\item Targa
		\item Proprietario (Nome, Cognome, CodiceFiscale)
		\item AnnoImmatricolazione
		\item EtàVeicolo
		\item Colori (un veicolo può essere bicolore)
	\end{enumerate}
	
	\subsection{Esercizio 2: Progettazione}
	Progetta gli attributi per un'entità \texttt{CORSO\_UNIVERSITARIO}, identificando:
	\begin{itemize}
		\item Almeno 2 attributi semplici
		\item Almeno 1 attributo composto
		\item Almeno 1 attributo multivalore
		\item Almeno 1 attributo calcolato
	\end{itemize}
	
	\subsection{Esercizio 3: Normalizzazione}
	Dato lo schema:
	\begin{verbatim}
		CLIENTE(ID, Nome, Cognome, {Email}, Indirizzo, NumeroOrdini)
	\end{verbatim}
	
	Identifica gli attributi problematici e proponi una normalizzazione.
	
	\section{Conclusioni}
	
	La corretta identificazione e classificazione degli attributi è fondamentale per:
	\begin{itemize}
		\item Una progettazione database efficiente
		\item La normalizzazione corretta delle tabelle
		\item L'ottimizzazione delle performance
		\item La manutenibilità del sistema
	\end{itemize}
	
	Ricorda sempre di analizzare il dominio applicativo per scegliere la tipologia di attributo più appropriata per ogni informazione da rappresentare.
	
\end{document}