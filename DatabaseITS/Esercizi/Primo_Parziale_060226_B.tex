\documentclass[a4paper,12pt]{article}
\usepackage[italian]{babel}
\usepackage[T1]{fontenc}
\usepackage[utf8]{inputenc}
\usepackage{geometry}
\geometry{margin=2.5cm}

\title{Esame di basi di dati - Primo parziale}
\author{ITS Academy - Fabbrica Digitale 4.0}
\date{Prof. Fedeli Massimo}

\begin{document}
	
	\maketitle
	
	Una palestra desidera realizzare un sistema informativo per gestire i propri iscritti, i corsi offerti e gli istruttori.
	
	La palestra gestisce un insieme di clienti che sottoscrivono un abbonamento per poter accedere alle attività. Per ogni cliente si vogliono memorizzare il codice cliente, nome, cognome, data di nascita, recapito telefonico ed email. Ogni cliente può avere al più un abbonamento attivo alla volta, caratterizzato da un tipo (mensile, trimestrale, annuale), una data di inizio e una data di fine.
	
	La palestra organizza diversi corsi, come ad esempio yoga, spinning e pilates. Per ogni corso sono registrati un codice corso, il nome, la descrizione, il numero massimo di partecipanti e il livello di difficoltà.
	
	I corsi sono tenuti da istruttori. Per ogni istruttore si vogliono memorizzare il codice istruttore, nome, cognome, specializzazione principale e anni di esperienza. Ogni corso è affidato a un solo istruttore, mentre un istruttore può tenere più corsi.
	
	I clienti possono iscriversi ai corsi. Per ogni iscrizione a un corso si desidera conoscere la data di iscrizione e l’eventuale stato (attiva, annullata, completata). Un cliente può iscriversi a più corsi, e ogni corso può avere più clienti iscritti, nel rispetto del numero massimo di partecipanti.
	
	\bigskip
	
	\textbf{Richieste}
	
	Sulla base della descrizione fornita si richiede di:
	
	\begin{enumerate}
		\item Individuare le entità principali del dominio, i loro attributi  e le chiavi primarie.
		\item Definire le relazioni tra le entità, specificandone cardinalità e partecipazione (obbligatoria o facoltativa).
		\item Disegnare lo schema E-R.
	\end{enumerate}
Consegna: inviare al docente tramite email lo schema e-r  con nome file "cognome nome.json"
	e lo schema logico all'interno di un file pdf  denominato "logicocognomenome.docx"



	
\end{document}
