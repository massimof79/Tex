\documentclass[a4paper,12pt]{article}
\usepackage[italian]{babel}
\usepackage[T1]{fontenc}
\usepackage[utf8]{inputenc}
\usepackage{geometry}
\geometry{margin=2.5cm}

\title{Esame di basi di dati - Primo parziale}
\author{ITS Academy - Fabbrica Digitale 4.0}
\date{Prof. Fedeli Massimo}

\begin{document}
	
	\maketitle
	
	Un Istituto di Istruzione Superiore intende realizzare un sistema informativo per la gestione delle attività di \textbf{formazione scuola--lavoro}. Il sistema deve supportare l’organizzazione delle esperienze formative svolte dagli studenti presso aziende del territorio e la gestione dei tutor coinvolti.
	
	L’istituto gestisce più indirizzi di studio e, per ciascun anno accademico, un elenco di studenti iscritti. Per ogni studente devono essere memorizzati almeno: codice identificativo, nome, cognome, data di nascita, luogo di nascita, codice fiscale, indirizzo di residenza, email istituzionale, classe frequentata e indirizzo di studi. Uno studente può partecipare, nel corso degli anni, a più esperienze di formazione scuola--lavoro.
	
	Le attività coinvolgono due tipologie di tutor. I tutor scolastici sono docenti dell’istituto. Per ciascun tutor scolastico si vogliono memorizzare: codice docente, nome, cognome, email, dipartimento o area disciplinare di appartenenza e numero massimo di studenti che può seguire contemporaneamente. Ogni esperienza deve avere un tutor scolastico responsabile, mentre un tutor può seguire più esperienze anche nello stesso periodo, purché non superi il limite massimo di studenti assegnati.
	
	Le aziende del territorio possono rendersi disponibili ad ospitare studenti. Per ogni azienda devono essere registrati: ragione sociale, partita IVA, sede legale, sede operativa (se diversa), settore di attività prevalente, referente aziendale, email e telefono del referente. Un’azienda può offrire nel tempo più proposte di esperienza formativa.
	
	Ogni proposta aziendale rappresenta una disponibilità concreta ad accogliere uno o più studenti. Per ciascuna disponibilità devono essere memorizzati: codice identificativo, azienda proponente, periodo previsto (data inizio e data fine), numero massimo di studenti ospitabili, descrizione delle attività previste, competenze richieste o preferenziali ed eventuale indirizzo di studi consigliato. Una stessa azienda può avere più disponibilità anche nello stesso anno scolastico.
	
	L’assegnazione degli studenti alle disponibilità avviene a cura della scuola. Quando uno studente viene assegnato a una disponibilità, si genera un’esperienza di formazione scuola--lavoro. L’esperienza rappresenta l’attività effettivamente svolta dallo studente presso l’azienda.
	
	Per ogni esperienza devono essere memorizzati: studente assegnato, azienda ospitante, tutor scolastico assegnato, tutor aziendale (nome, cognome, ruolo in azienda, email), periodo effettivo (che può differire da quello previsto nella disponibilità), numero di ore previste, numero di ore effettivamente svolte, valutazione finale ed esito (positivo, negativo, interrotto). Ogni esperienza è associata a una e una sola disponibilità originaria, mentre una disponibilità può generare più esperienze, fino al numero massimo di posti previsti.
	
	Il sistema deve permettere di conoscere, per ogni studente, lo storico completo delle esperienze svolte; per ogni tutor scolastico, l’elenco degli studenti seguiti e delle esperienze supervisionate; per ogni azienda, il numero di studenti ospitati nel tempo e le esperienze attivate a partire dalle proprie disponibilità.
	
	\bigskip
	
	\textbf{Richieste}
	
	Sulla base della descrizione fornita si richiede di:
	
	\begin{enumerate}
		\item Individuare le entità principali del dominio, i loro attributi  e le chiavi primarie.
		\item Definire le relazioni tra le entità, specificandone cardinalità e partecipazione (obbligatoria o facoltativa).
		\item Disegnare lo schema E-R.
		\item Tradurre lo schema E-R nel corrispondente schema logico. 
	\end{enumerate}
Consegna: inviare al docente tramite email lo schema e-r  con nome file "cognome nome.json"
	e lo schema logico all'interno di un file pdf  denominato "logicocognomenome.docx"



	
\end{document}
