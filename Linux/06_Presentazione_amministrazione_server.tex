\documentclass[aspectratio=169]{beamer}
\usepackage[utf8]{inputenc}
\usepackage[italian]{babel}
\usepackage[T1]{fontenc}
\usepackage{listings}
\usepackage{xcolor}
\usepackage{tikz}
\usepackage{graphicx}
\usepackage{booktabs}
\usepackage{colortbl}

% Tema Beamer
\usetheme{Madrid}
\usecolortheme{default}

% Colori personalizzati
\definecolor{primaryblue}{RGB}{0,102,204}
\definecolor{secondaryblue}{RGB}{51,153,255}
\definecolor{accentorange}{RGB}{255,102,0}
\definecolor{darkgray}{RGB}{64,64,64}
\definecolor{codebackground}{RGB}{245,245,245}

\setbeamercolor{structure}{fg=primaryblue}
\setbeamercolor{frametitle}{bg=primaryblue,fg=white}
\setbeamercolor{block title}{bg=secondaryblue,fg=white}
\setbeamercolor{block body}{bg=secondaryblue!10}
\setbeamercolor{block title alerted}{bg=accentorange,fg=white}
\setbeamercolor{block body alerted}{bg=accentorange!10}

% Configurazione listings
\lstset{
    basicstyle=\ttfamily\footnotesize,
    keywordstyle=\color{primaryblue}\bfseries,
    commentstyle=\color{green!60!black}\itshape,
    stringstyle=\color{red!80!black},
    backgroundcolor=\color{codebackground},
    showstringspaces=false,
    breaklines=true,
    frame=single,
    rulecolor=\color{black!30},
    numbers=none,
    xleftmargin=5pt,
    xrightmargin=5pt,
    language=bash
}

% Informazioni del documento
\title{Amministrazione Server Linux}
\subtitle{Network e Condivisione Risorse}
\author{Prof. Fedeli Massimo}
\institute{Tutti i diritti riservati}
\date{\today}

% Logo nell'angolo
\logo{\includegraphics[height=0.6cm]{example-image}}

\begin{document}

% Slide titolo
\begin{frame}
\titlepage
\end{frame}

% Indice
\begin{frame}{Sommario}
\tableofcontents
\end{frame}

%=============================================================================
\section{Introduzione}
%=============================================================================

\begin{frame}{Amministrazione Server Linux}
\begin{block}{Definizione}
Un \textbf{server Linux} è un sistema operativo configurato per fornire servizi specifici ad altri computer attraverso una rete, operando 24/7/365.
\end{block}

\vspace{0.5cm}

\begin{columns}
\column{0.5\textwidth}
\textbf{Caratteristiche principali:}
\begin{itemize}
    \item Disponibilità continua
    \item Gestione remota
    \item Sicurezza rafforzata
    \item Monitoraggio automatico
    \item Scalabilità
\end{itemize}

\column{0.5\textwidth}
\begin{center}
\begin{tikzpicture}[scale=0.8]
    % Server centrale
    \draw[fill=primaryblue!30] (0,0) rectangle (2,1.5);
    \node at (1,0.75) {\small Server};
    
    % Clients
    \foreach \x/\y in {-2/2, 2/2, -2/-1, 2/-1} {
        \draw[fill=secondaryblue!30] (\x,\y) circle (0.3);
        \draw[->,thick] (\x,\y) -- (1,0.75);
    }
\end{tikzpicture}
\end{center}
\end{columns}
\end{frame}

\begin{frame}{Porte e Servizi Standard}
\begin{table}
\centering
\begin{tabular}{lcc}
\toprule
\textbf{Servizio} & \textbf{Porta} & \textbf{Protocollo} \\
\midrule
SSH & 22 & TCP \\
HTTP & 80 & TCP \\
HTTPS & 443 & TCP \\
FTP & 21 & TCP \\
DNS & 53 & TCP/UDP \\
SMTP & 25 & TCP \\
MySQL & 3306 & TCP \\
PostgreSQL & 5432 & TCP \\
\bottomrule
\end{tabular}
\end{table}

\begin{alertblock}{Importante}
La gestione corretta delle porte è cruciale per sicurezza e funzionalità!
\end{alertblock}
\end{frame}

%=============================================================================
\section{Procedura di Setup}
%=============================================================================

\begin{frame}{Le 5 Fasi di Setup}
\begin{center}
\begin{tikzpicture}[
    box/.style={rectangle, draw, fill=primaryblue!20, minimum width=3cm, minimum height=1cm, text centered},
    arrow/.style={->, thick}
]
    \node[box] (install) at (0,4) {\textbf{1. Installazione}};
    \node[box] (config) at (0,3) {\textbf{2. Configurazione}};
    \node[box] (start) at (0,2) {\textbf{3. Avvio}};
    \node[box] (secure) at (0,1) {\textbf{4. Sicurezza}};
    \node[box] (monitor) at (0,0) {\textbf{5. Monitoraggio}};
    
    \draw[arrow] (install) -- (config);
    \draw[arrow] (config) -- (start);
    \draw[arrow] (start) -- (secure);
    \draw[arrow] (secure) -- (monitor);
    \draw[arrow, bend right=45] (monitor.west) to node[left] {\tiny Continuo} (install.west);
\end{tikzpicture}
\end{center}
\end{frame}

\begin{frame}[fragile]{Fase 1: Installazione - Distribuzioni}
\begin{columns}
\column{0.5\textwidth}
\textbf{Enterprise:}
\begin{itemize}
    \item Red Hat Enterprise Linux (RHEL)
    \item Ubuntu Server
    \item SUSE Linux Enterprise
\end{itemize}

\vspace{0.3cm}
\textbf{Community:}
\begin{itemize}
    \item Fedora
    \item Debian
    \item Rocky Linux / AlmaLinux
\end{itemize}

\column{0.5\textwidth}
\begin{block}{Installazione Pacchetti}
\begin{lstlisting}
# dnf grouplist
# dnf groupinstall "Web Server"
# dnf install httpd mod_ssl
\end{lstlisting}
\end{block}
\end{columns}
\end{frame}

\begin{frame}[fragile]{Principali Tipi di Server}
\begin{columns}[t]
\column{0.33\textwidth}
\textbf{Web \& File}
\begin{itemize}
    \item Apache/Nginx
    \item Samba (SMB/CIFS)
    \item NFS
    \item FTP (vsftpd)
\end{itemize}

\column{0.33\textwidth}
\textbf{Database \& Mail}
\begin{itemize}
    \item MariaDB/MySQL
    \item PostgreSQL
    \item Postfix
    \item Dovecot
\end{itemize}

\column{0.33\textwidth}
\textbf{Infrastruttura}
\begin{itemize}
    \item DNS (BIND)
    \item DHCP
    \item LDAP
    \item NTP (chrony)
\end{itemize}
\end{columns}

\vspace{0.5cm}

\begin{exampleblock}{Esempio: Installazione Apache}
\begin{lstlisting}
# dnf install httpd mod_ssl
# systemctl enable httpd
# systemctl start httpd
\end{lstlisting}
\end{exampleblock}
\end{frame}

\begin{frame}[fragile]{Fase 2: File di Configurazione}
\textbf{Struttura tipica in /etc:}

\begin{lstlisting}
/etc/httpd/
├── conf/
│   └── httpd.conf          # Config principale
├── conf.d/
│   ├── ssl.conf            # Moduli aggiuntivi
│   └── php.conf
└── conf.modules.d/
\end{lstlisting}

\begin{block}{Best Practice}
\begin{itemize}
    \item Usa \texttt{vim} invece di \texttt{vi} (syntax highlighting)
    \item Backup prima di modificare: \texttt{cp file file.bak}
    \item Test configurazione prima di riavviare servizio
    \item Documenta le modifiche
\end{itemize}
\end{block}
\end{frame}

\begin{frame}[fragile]{Fase 3: Gestione Servizi con systemd}
\begin{columns}
\column{0.5\textwidth}
\textbf{Comandi Base:}
\begin{lstlisting}
# Stato servizio
systemctl status httpd

# Start/Stop
systemctl start httpd
systemctl stop httpd
systemctl restart httpd

# Ricarica config
systemctl reload httpd
\end{lstlisting}

\column{0.5\textwidth}
\textbf{Avvio Automatico:}
\begin{lstlisting}
# Abilita all'avvio
systemctl enable httpd

# Disabilita
systemctl disable httpd

# Verifica
systemctl is-enabled httpd
\end{lstlisting}
\end{columns}

\vspace{0.3cm}

\begin{alertblock}{Attenzione}
Verificare sempre lo stato dopo modifiche con \texttt{systemctl status}!
\end{alertblock}
\end{frame}

%=============================================================================
\section{Sicurezza}
%=============================================================================

\begin{frame}{Livelli di Sicurezza}
\begin{center}
\begin{tikzpicture}[scale=0.9]
    % Layer architecture
    \draw[fill=red!20] (0,0) rectangle (10,1);
    \node at (5,0.5) {\textbf{Password \& Autenticazione}};
    
    \draw[fill=orange!20] (0,1) rectangle (10,2);
    \node at (5,1.5) {\textbf{Firewall (iptables/firewalld)}};
    
    \draw[fill=yellow!20] (0,2) rectangle (10,3);
    \node at (5,2.5) {\textbf{SELinux}};
    
    \draw[fill=green!20] (0,3) rectangle (10,4);
    \node at (5,3.5) {\textbf{Configurazioni Specifiche Servizi}};
    
    \draw[fill=blue!20] (0,4) rectangle (10,5);
    \node at (5,4.5) {\textbf{Monitoraggio e Logging}};
\end{tikzpicture}
\end{center}

\textbf{Principio:} Difesa in profondità (Defense in Depth)
\end{frame}

\begin{frame}[fragile]{Autenticazione: Password vs Chiavi SSH}
\begin{columns}
\column{0.5\textwidth}
\textbf{Password}
\begin{itemize}
    \item[$-$] Vulnerabile a brute force
    \item[$-$] Può essere intercettata
    \item[$-$] Deve essere ricordata
    \item[$+$] Semplice da configurare
\end{itemize}

\vspace{0.3cm}

\begin{lstlisting}
# Disabilita login root
# /etc/ssh/sshd_config
PermitRootLogin no
\end{lstlisting}

\column{0.5\textwidth}
\textbf{Chiavi SSH}
\begin{itemize}
    \item[$+$] Impossibile brute force
    \item[$+$] Crittograficamente sicura
    \item[$+$] Automazione possibile
    \item[$+$] No password da ricordare
\end{itemize}

\vspace{0.3cm}

\begin{lstlisting}
# Genera chiave
ssh-keygen -t rsa -b 4096

# Copia su server
ssh-copy-id user@server
\end{lstlisting}
\end{columns}
\end{frame}

\begin{frame}[fragile]{Firewall: firewalld}
\begin{columns}
\column{0.5\textwidth}
\textbf{Concetti Base:}
\begin{itemize}
    \item \textbf{Zone}: Livelli di fiducia
    \item \textbf{Servizi}: Porte predefinite
    \item \textbf{Runtime}: Temporaneo
    \item \textbf{Permanent}: Persistente
\end{itemize}

\column{0.5\textwidth}
\begin{lstlisting}
# Stato firewall
firewall-cmd --state

# Zone attive
firewall-cmd --get-active-zones

# Permetti HTTP
firewall-cmd --permanent \
  --add-service=http
firewall-cmd --reload
\end{lstlisting}
\end{columns}

\vspace{0.3cm}

\begin{exampleblock}{Esempio: Apertura Porta Custom}
\begin{lstlisting}
firewall-cmd --permanent --add-port=8080/tcp
firewall-cmd --reload
firewall-cmd --list-ports
\end{lstlisting}
\end{exampleblock}
\end{frame}

\begin{frame}{SELinux: Security Enhanced Linux}
\begin{block}{Modalità Operative}
\begin{description}
    \item[Enforcing] Blocca azioni non autorizzate (PRODUZIONE)
    \item[Permissive] Registra violazioni ma non blocca (DEBUG)
    \item[Disabled] SELinux disattivato (NON RACCOMANDATO)
\end{description}
\end{block}

\vspace{0.3cm}

\begin{columns}[t]
\column{0.33\textwidth}
\textbf{Context}
\begin{itemize}
    \item User
    \item Role
    \item Type
    \item Level
\end{itemize}

\column{0.33\textwidth}
\textbf{Boolean}
\begin{itemize}
    \item On/Off switches
    \item Modifica policy
    \item Runtime/Permanent
\end{itemize}

\column{0.33\textwidth}
\textbf{Porte}
\begin{itemize}
    \item Port types
    \item Servizi associati
    \item Permessi custom
\end{itemize}
\end{columns}
\end{frame}

\begin{frame}[fragile]{SELinux: Comandi Essenziali}
\begin{lstlisting}
# Verifica stato
getenforce
sestatus

# Cambia modalità (temporaneo)
setenforce 0  # Permissive
setenforce 1  # Enforcing

# Context file
ls -Z /var/www/html/
restorecon -Rv /var/www/html/

# Boolean
getsebool httpd_can_network_connect
setsebool -P httpd_can_network_connect on

# Porte
semanage port -l | grep http
semanage port -a -t http_port_t -p tcp 8080

# Troubleshooting
sealert -a /var/log/audit/audit.log
\end{lstlisting}
\end{frame}

%=============================================================================
\section{Monitoraggio}
%=============================================================================

\begin{frame}{Sistema di Logging: rsyslog}
\begin{block}{Architettura}
\textbf{Facility.Priority} $\rightarrow$ \textbf{Destination}
\end{block}

\begin{columns}[t]
\column{0.45\textwidth}
\textbf{Facility:}
\begin{itemize}
    \item \texttt{kern} - Kernel
    \item \texttt{mail} - Email
    \item \texttt{authpriv} - Auth
    \item \texttt{cron} - Scheduler
    \item \texttt{daemon} - Servizi
\end{itemize}

\column{0.55\textwidth}
\textbf{Priority (crescente):}
\begin{itemize}
    \item \texttt{debug} $<$ \texttt{info} $<$ \texttt{notice}
    \item \texttt{warning} $<$ \texttt{err} $<$ \texttt{crit}
    \item \texttt{alert} $<$ \texttt{emerg}
\end{itemize}
\end{columns}

\vspace{0.5cm}

\begin{exampleblock}{File: /var/log/}
messages, secure, maillog, cron, httpd/, audit/
\end{exampleblock}
\end{frame}

\begin{frame}[fragile]{Logging Centralizzato}
\begin{center}
\begin{tikzpicture}[scale=0.8]
    % Loghost
    \node[draw, rectangle, fill=green!20, minimum width=2cm, minimum height=1.5cm] (loghost) at (0,0) {Loghost};
    \node[below=0.1cm of loghost] {\tiny :514 UDP/TCP};
    
    % Servers
    \foreach \x/\name in {-4/Web, -2/DB, 2/Mail, 4/DNS} {
        \node[draw, circle, fill=blue!20] (s\x) at (\x,3) {\name};
        \draw[->, thick] (s\x) -- (loghost);
    }
    
    \node[above=1cm of loghost] {\textbf{Logging Centralizzato}};
\end{tikzpicture}
\end{center}

\begin{columns}
\column{0.5\textwidth}
\textbf{Client:}
\begin{lstlisting}[basicstyle=\ttfamily\tiny]
# /etc/rsyslog.conf
*.* @loghost  # UDP
*.* @@loghost # TCP
\end{lstlisting}

\column{0.5\textwidth}
\textbf{Server:}
\begin{lstlisting}[basicstyle=\ttfamily\tiny]
module(load="imudp")
input(type="imudp" port="514")
\end{lstlisting}
\end{columns}
\end{frame}

\begin{frame}{System Activity Reporter (sar)}
\begin{block}{Cosa Monitora?}
\begin{itemize}
    \item \textbf{CPU}: Utilizzo, idle, I/O wait
    \item \textbf{Memoria}: RAM, swap, paging
    \item \textbf{Disco}: Throughput, latenza, IOPS
    \item \textbf{Rete}: Pacchetti, bandwidth, errori
\end{itemize}
\end{block}

\vspace{0.3cm}

\begin{columns}
\column{0.5\textwidth}
\textbf{Raccolta Dati:}
\begin{itemize}
    \item Ogni 10 minuti
    \item Stored in /var/log/sa/
    \item 1 mese di history
\end{itemize}

\column{0.5\textwidth}
\textbf{Visualizzazione:}
\begin{itemize}
    \item Report storici
    \item Live monitoring
    \item Export per analisi
\end{itemize}
\end{columns}
\end{frame}

\begin{frame}[fragile]{sar: Esempi Pratici}
\begin{lstlisting}
# CPU usage da mezzanotte
sar -u

# Disco I/O
sar -d

# Network traffic
sar -n DEV

# Memoria
sar -r

# Live: campiona ogni 2 sec, 10 volte
sar -u 2 10

# Dati giorno specifico
sar -u -f /var/log/sa/sa15
\end{lstlisting}

\begin{alertblock}{Pro Tip}
Combina con \texttt{grep}, \texttt{awk} per analisi mirate!
\end{alertblock}
\end{frame}

\begin{frame}{Cockpit: Monitoring Web-Based}
\begin{center}
\includegraphics[width=0.7\textwidth]{example-image}
\end{center}

\textbf{Funzionalità:}
\begin{columns}[t]
\column{0.5\textwidth}
\begin{itemize}
    \item Grafici real-time
    \item Gestione servizi
    \item Log viewer
    \item Storage management
\end{itemize}

\column{0.5\textwidth}
\begin{itemize}
    \item Network config
    \item User accounts
    \item Terminal integrato
    \item Updates management
\end{itemize}
\end{columns}

\vspace{0.3cm}
\textbf{Accesso:} \texttt{https://server:9090}
\end{frame}

%=============================================================================
\section{SSH}
%=============================================================================

\begin{frame}{Secure Shell (SSH)}
\begin{block}{Perché SSH?}
Sostituisce protocolli insicuri (telnet, rlogin, rsh, rcp) con comunicazione \textbf{crittografata end-to-end}.
\end{block}

\vspace{0.3cm}

\begin{columns}
\column{0.5\textwidth}
\textbf{Client Tools:}
\begin{itemize}
    \item \texttt{ssh} - Remote login
    \item \texttt{scp} - Secure copy
    \item \texttt{sftp} - Secure FTP
    \item \texttt{rsync} - Sync incrementale
\end{itemize}

\column{0.5\textwidth}
\textbf{Funzionalità:}
\begin{itemize}
    \item Login remoto
    \item Esecuzione comandi
    \item Trasferimento file
    \item Port forwarding
    \item X11 forwarding
    \item Tunnel VPN
\end{itemize}
\end{columns}
\end{frame}

\begin{frame}[fragile]{SSH: Configurazione Server}
\begin{lstlisting}
# /etc/ssh/sshd_config

Port 22                          # Porta (considera 2222)
PermitRootLogin no              # NO LOGIN ROOT!
PasswordAuthentication yes      # Si/No password
PubkeyAuthentication yes        # Chiavi SSH
X11Forwarding yes               # GUI remoto
ClientAliveInterval 300         # Keep-alive
AllowUsers user1 user2          # Whitelist utenti
DenyUsers baduser               # Blacklist utenti
\end{lstlisting}

\begin{alertblock}{Sicurezza Critica}
\texttt{PermitRootLogin no} è \textbf{OBBLIGATORIO} in produzione!
\end{alertblock}
\end{frame}

\begin{frame}[fragile]{SSH: Autenticazione con Chiavi}
\textbf{Setup (una volta):}

\begin{lstlisting}
# 1. Genera chiave (client)
ssh-keygen -t rsa -b 4096 -C "mio-laptop"

# 2. Copia su server
ssh-copy-id user@server.com

# 3. Test
ssh user@server.com  # No password!
\end{lstlisting}

\vspace{0.3cm}

\begin{block}{Vantaggi}
\begin{itemize}
    \item[$+$] Sicurezza massima
    \item[$+$] Automazione (script, backup)
    \item[$+$] Una chiave, N server
    \item[$+$] Revoca facile (rimuovi chiave pubblica)
\end{itemize}
\end{block}
\end{frame}

\begin{frame}[fragile]{Trasferimento File}
\begin{columns}
\column{0.5\textwidth}
\textbf{scp - Copia Singola}
\begin{lstlisting}[basicstyle=\ttfamily\tiny]
# Locale -> Remoto
scp file.txt user@host:/path/

# Remoto -> Locale
scp user@host:/file.txt ./

# Ricorsivo
scp -r dir/ user@host:/path/
\end{lstlisting}

\column{0.5\textwidth}
\textbf{rsync - Sincronizzazione}
\begin{lstlisting}[basicstyle=\ttfamily\tiny]
# Sync con delete
rsync -avz --delete \
  /local/ user@host:/remote/

# Bandwidth limit
rsync -avz --bwlimit=1000 \
  /src/ user@host:/dst/
\end{lstlisting}
\end{columns}

\vspace{0.5cm}

\begin{exampleblock}{rsync vs scp}
rsync trasferisce solo \textbf{differenze} = molto più efficiente!
\end{exampleblock}
\end{frame}

\begin{frame}[fragile]{sftp: FTP Sicuro}
\begin{lstlisting}
$ sftp user@server
sftp> ls              # Lista remota
sftp> lls             # Lista locale
sftp> get file.txt    # Download
sftp> put file.txt    # Upload
sftp> get -r dir/     # Download ricorsivo
sftp> put -r dir/     # Upload ricorsivo
sftp> mget *.log      # Download multipli
sftp> mkdir newdir    # Crea directory
sftp> rm file.txt     # Elimina file
sftp> bye             # Esci
\end{lstlisting}

\begin{block}{Quando usare sftp?}
Sessioni \textbf{interattive} di esplorazione e trasferimento file.
\end{block}
\end{frame}

%=============================================================================
\section{Gestione Disco}
%=============================================================================

\begin{frame}[fragile]{Monitoraggio Spazio: df e du}
\begin{columns}
\column{0.5\textwidth}
\textbf{df - Filesystem}
\begin{lstlisting}[basicstyle=\ttfamily\tiny]
# Human-readable
df -h

# Exclude tmpfs
df -h -x tmpfs -x devtmpfs

# Inodes
df -i

# Tipo specifico
df -t xfs
\end{lstlisting}

\column{0.5\textwidth}
\textbf{du - Directory}
\begin{lstlisting}[basicstyle=\ttfamily\tiny]
# Directory usage
du -h /var

# Solo totale
du -sh /var

# Top 10 largest
du -h /var | sort -hr | head

# Max depth
du -h --max-depth=2 /var
\end{lstlisting}
\end{columns}

\vspace{0.3cm}

\begin{exampleblock}{Pro Tip}
\texttt{df} per filesystem totali, \texttt{du} per drill-down dettagliato
\end{exampleblock}
\end{frame}

\begin{frame}[fragile]{find: Ricerca Avanzata}
\begin{lstlisting}
# File > 100MB
find / -xdev -size +100M -ls

# File utente specifico, ordinati
find / -xdev -user john | xargs ls -lhS > /tmp/john.txt

# Modificati ultimi 7 giorni > 50MB
find /var/log -mtime -7 -size +50M

# Non acceduti da 1 anno
find /home -atime +365 -size +10M

# Pulizia file temporanei vecchi
find /tmp -type f -mtime +30 -delete

# Top 20 directory
find / -xdev -type d -exec du -sh {} \; | sort -hr | head -20
\end{lstlisting}
\end{frame}

\begin{frame}{logrotate: Rotazione Automatica}
\begin{block}{Configurazione}
File principale: \texttt{/etc/logrotate.conf}\\
Configs servizi: \texttt{/etc/logrotate.d/*}
\end{block}

\begin{columns}[t]
\column{0.5\textwidth}
\textbf{Opzioni Comuni:}
\begin{itemize}
    \item \texttt{daily/weekly/monthly}
    \item \texttt{rotate N} - Copie
    \item \texttt{compress} - Gzip
    \item \texttt{delaycompress}
    \item \texttt{missingok}
    \item \texttt{notifempty}
\end{itemize}

\column{0.5\textwidth}
\textbf{Scripts:}
\begin{itemize}
    \item \texttt{prerotate}
    \item \texttt{postrotate}
    \item \texttt{sharedscripts}
\end{itemize}

\vspace{0.3cm}

\textbf{Esecuzione:}\\
Via cron: \texttt{/etc/cron.daily/}
\end{columns}
\end{frame}

%=============================================================================
\section{Enterprise}
%=============================================================================

\begin{frame}{Gestione Server Enterprise}
\begin{center}
\textbf{Da Gestione Manuale ad Automazione Scalabile}
\end{center}

\begin{columns}
\column{0.5\textwidth}
\textbf{Tradizionale (NON scalabile):}
\begin{itemize}
    \item[$\times$] Installazione manuale
    \item[$\times$] Configurazione host-by-host
    \item[$\times$] SSH ad ogni server
    \item[$\times$] Updates individuali
    \item[$\times$] Inconsistenze
\end{itemize}

\column{0.5\textwidth}
\textbf{Enterprise (Scalabile):}
\begin{itemize}
    \item[$\checkmark$] PXE boot
    \item[$\checkmark$] Config management
    \item[$\checkmark$] Orchestrazione
    \item[$\checkmark$] Automazione
    \item[$\checkmark$] Consistenza
\end{itemize}
\end{columns}

\vspace{0.5cm}

\begin{alertblock}{Regola d'oro}
Se devi fare la stessa cosa su $>3$ server $\rightarrow$ \textbf{AUTOMATIZZA}!
\end{alertblock}
\end{frame}

\begin{frame}{PXE Boot: Installazione di Massa}
\begin{center}
\begin{tikzpicture}[scale=0.7]
    % PXE Server
    \node[draw, rectangle, fill=green!30, minimum width=2.5cm, minimum height=1.5cm] (pxe) at (0,3) {PXE Server};
    \node[below=0.1cm of pxe] {\tiny DHCP+TFTP};
    
    % Clients
    \foreach \x in {-3, 0, 3} {
        \node[draw, rectangle, fill=blue!20] (c\x) at (\x,0) {Client};
        \draw[->, thick] (c\x) -- (pxe);
    }
    
    % Steps
    \node[right=0.5cm of pxe, text width=3cm, font=\tiny] {
        1. DHCP Request\\
        2. Get IP + Boot file\\
        3. TFTP download\\
        4. Boot \& Install\\
        5. Reboot to disk
    };
\end{tikzpicture}
\end{center}

\textbf{Vantaggi:}
\begin{itemize}
    \item Installa 100 server simultaneamente
    \item Configurazione standardizzata
    \item Zero intervento umano
\end{itemize}
\end{frame}

\begin{frame}{Configuration Management}
\begin{block}{Tool Popolari}
Ansible, Puppet, Chef, Salt
\end{block}

\textbf{Principi:}
\begin{enumerate}
    \item \textbf{Infrastructure as Code} (IaC)
    \item \textbf{Idempotenza}: Esecuzione multipla = stesso risultato
    \item \textbf{Declarativo}: "Cosa" non "Come"
    \item \textbf{Versionamento}: Git per tracking
\end{enumerate}

\vspace{0.3cm}

\begin{exampleblock}{Esempio: 1 comando, 100 server}
\texttt{ansible webservers -m yum -a "name=httpd state=latest"}
\end{exampleblock}
\end{frame}

\begin{frame}{Architettura Management/Worker}
\begin{center}
\begin{tikzpicture}[scale=0.8]
    % Management Plane
    \node[draw, rectangle, fill=red!20, minimum width=4cm, minimum height=1.2cm] (mgmt) at (0,3) {Management Nodes};
    \node[below=0.05cm of mgmt, font=\tiny] {API, Scheduler, Database};
    
    % Worker Plane
    \foreach \x in {-3, -1, 1, 3} {
        \node[draw, rectangle, fill=green!20, minimum width=1.5cm] (w\x) at (\x,0) {Worker};
    }
    
    % Connections
    \foreach \x in {-3, -1, 1, 3} {
        \draw[->, thick] (mgmt) -- (w\x);
    }
    
    \node[above=0.3cm of mgmt] {\textbf{Control Plane}};
    \node[below=0.3cm of w1] {\textbf{Data Plane}};
\end{tikzpicture}
\end{center}

\textbf{Esempi:} Kubernetes, OpenShift, OpenStack
\end{frame}

%=============================================================================
\section{Best Practices}
%=============================================================================

\begin{frame}{Best Practices: Sicurezza}
\begin{enumerate}
    \item \textbf{Principio del Minimo Privilegio}
    \begin{itemize}
        \item Utenti: solo permessi necessari
        \item Servizi: utenti dedicati non-root
        \item sudo: comandi specifici
    \end{itemize}
    
    \item \textbf{Difesa in Profondità}
    \begin{itemize}
        \item Firewall + SELinux + App security
        \item Multi-factor authentication
        \item Encryption (data at rest \& in transit)
    \end{itemize}
    
    \item \textbf{Updates \& Patching}
    \begin{itemize}
        \item Security patches ASAP
        \item Test in staging first
        \item Finestre manutenzione pianificate
    \end{itemize}
\end{enumerate}
\end{frame}

\begin{frame}{Best Practices: Monitoraggio}
\begin{block}{Metriche Essenziali (Golden Signals)}
\begin{itemize}
    \item \textbf{Latency}: Response time
    \item \textbf{Traffic}: Request rate
    \item \textbf{Errors}: Error rate
    \item \textbf{Saturation}: Resource utilization
\end{itemize}
\end{block}

\textbf{Alert Intelligenti:}
\begin{itemize}
    \item Threshold basati su baseline
    \item Escalation policy chiara
    \item Evita alert fatigue (troppi falsi positivi)
    \item Runbook documentati
\end{itemize}

\textbf{Proattività:}
\begin{itemize}
    \item Trend analysis
    \item Capacity planning
    \item Predictive maintenance
\end{itemize}
\end{frame}

\begin{frame}{Best Practices: Backup}
\begin{block}{Strategia 3-2-1}
\begin{itemize}
    \item \textbf{3} copie dei dati
    \item Su \textbf{2} media diversi (disco + tape/cloud)
    \item \textbf{1} copia off-site (disaster recovery)
\end{itemize}
\end{block}

\vspace{0.3cm}

\textbf{Testing Regolare:}
\begin{itemize}
    \item Restore test mensili
    \item DR drill trimestrali
    \item Documenta RTO/RPO (Recovery Time/Point Objective)
\end{itemize}

\vspace{0.3cm}

\begin{alertblock}{Remember}
Backup non testato = Backup non esistente!
\end{alertblock}
\end{frame}

\begin{frame}{Best Practices: Documentazione}
\begin{columns}
\column{0.5\textwidth}
\textbf{Architecture Docs:}
\begin{itemize}
    \item Network diagrams
    \item Data flow
    \item Dependencies map
    \item Infrastructure inventory
\end{itemize}

\vspace{0.3cm}

\textbf{Operational Docs:}
\begin{itemize}
    \item Runbooks
    \item Troubleshooting guides
    \item Emergency procedures
    \item On-call playbooks
\end{itemize}

\column{0.5\textwidth}
\textbf{Change Docs:}
\begin{itemize}
    \item Change log
    \item Approval workflow
    \item Rollback plans
    \item Post-mortem reports
\end{itemize}

\vspace{0.3cm}

\textbf{Tools:}
\begin{itemize}
    \item Wiki (Confluence)
    \item Git (docs as code)
    \item Diagrams (draw.io)
    \item CMDB
\end{itemize}
\end{columns}

\vspace{0.3cm}

\begin{center}
\textit{"Documentazione obsoleta > Nessuna documentazione"}
\end{center}
\end{frame}

%=============================================================================
\section{Conclusioni}
%=============================================================================

\begin{frame}{Riepilogo: 5 Pilastri}
\begin{enumerate}
    \item \textbf{Setup Sistematico}
    \begin{itemize}
        \item Installazione $\rightarrow$ Config $\rightarrow$ Start $\rightarrow$ Secure $\rightarrow$ Monitor
    \end{itemize}
    
    \item \textbf{Sicurezza Multi-Layer}
    \begin{itemize}
        \item Password/Keys + Firewall + SELinux + App Config
    \end{itemize}
    
    \item \textbf{Monitoraggio Continuo}
    \begin{itemize}
        \item rsyslog + sar + Cockpit + logwatch
    \end{itemize}
    
    \item \textbf{Gestione Remota SSH}
    \begin{itemize}
        \item ssh + scp + rsync + sftp
    \end{itemize}
    
    \item \textbf{Automazione Enterprise}
    \begin{itemize}
        \item PXE + Config Management + Orchestration
    \end{itemize}
\end{enumerate}
\end{frame}

\begin{frame}{Roadmap di Apprendimento}
\begin{center}
\begin{tikzpicture}[
    level/.style={rectangle, draw, fill=blue!20, minimum width=2.5cm, minimum height=0.8cm, text centered},
    arrow/.style={->, thick}
]
    \node[level, fill=green!30] (base) at (0,4) {Base: Linux CLI};
    \node[level] (admin) at (0,3) {Sys Admin};
    \node[level] (security) at (0,2) {Security};
    \node[level] (auto) at (0,1) {Automation};
    \node[level, fill=orange!30] (expert) at (0,0) {Expert: DevOps};
    
    \draw[arrow] (base) -- (admin);
    \draw[arrow] (admin) -- (security);
    \draw[arrow] (security) -- (auto);
    \draw[arrow] (auto) -- (expert);
    
    \node[right=0.5cm of base, text width=3cm, font=\tiny] {Comandi, filesystem, shell};
    \node[right=0.5cm of admin, text width=3cm, font=\tiny] {Setup, config, servizi};
    \node[right=0.5cm of security, text width=3cm, font=\tiny] {Firewall, SELinux, SSH};
    \node[right=0.5cm of auto, text width=3cm, font=\tiny] {Ansible, scripting};
    \node[right=0.5cm of expert, text width=3cm, font=\tiny] {K8s, CI/CD, IaC};
\end{tikzpicture}
\end{center}
\end{frame}

\begin{frame}{Risorse per Approfondire}
\begin{columns}
\column{0.5\textwidth}
\textbf{Documentazione:}
\begin{itemize}
    \item Man pages: \texttt{man comando}
    \item RHEL Docs
    \item Arch Wiki
    \item Gentoo Handbook
\end{itemize}

\vspace{0.3cm}

\textbf{Certificazioni:}
\begin{itemize}
    \item RHCSA
    \item RHCE
    \item LFCS
    \item CompTIA Linux+
\end{itemize}

\column{0.5\textwidth}
\textbf{Community:}
\begin{itemize}
    \item Stack Overflow
    \item Reddit: r/linux, r/linuxadmin
    \item IRC/Discord channels
    \item Local LUG
\end{itemize}

\vspace{0.3cm}

\textbf{Libri:}
\begin{itemize}
    \item UNIX \& Linux Sys Admin Handbook
    \item Linux Command Line (Shotts)
    \item How Linux Works (Ward)
\end{itemize}
\end{columns}
\end{frame}

\begin{frame}{Takeaway Messages}
\begin{alertblock}{3 Principi Fondamentali}
\begin{enumerate}
    \item \textbf{Automazione}: Se ripeti $>2$ volte, scrivi script
    \item \textbf{Sicurezza}: Defense in depth, mai singolo punto di fallimento
    \item \textbf{Documentazione}: Future-you ti ringrazierà
\end{enumerate}
\end{alertblock}

\vspace{0.5cm}

\begin{block}{L'Admin Ideale}
\begin{itemize}
    \item \textbf{Lazy}: Automatizza tutto il possibile
    \item \textbf{Paranoico}: Assume sempre il peggio (security)
    \item \textbf{Curioso}: Continua ad imparare
    \item \textbf{Metodico}: Processo > Improvvisazione
\end{itemize}
\end{block}
\end{frame}

\begin{frame}[standout]
\Huge Grazie per l'Attenzione!

\vspace{1cm}

\Large Domande?

\vspace{1cm}

\normalsize
Prof. Fedeli Massimo\\
Tutti i diritti riservati
\end{frame}

\end{document}
