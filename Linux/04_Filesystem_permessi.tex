\documentclass[aspectratio=169,10pt]{beamer}

\usepackage[utf8]{inputenc}
\usepackage[italian]{babel}
\usepackage{graphicx}
\usepackage{tikz}
\usepackage{listings}
\usepackage{xcolor}
\usepackage{adjustbox}
\usepackage{multirow}
\usepackage{booktabs}
\usepackage{amssymb}
\usepackage{fontawesome} % per \faLock
\usepackage{makecell}

\usetikzlibrary{
	shapes,
	arrows,
	positioning,
	trees,
	shadows,
	fit,
	backgrounds,
	decorations.pathreplacing, % necessario per brace
	calc
}

% Tema e colori
\usetheme{Madrid}
\usecolortheme{default}

% Definizione colori personalizzati
\definecolor{darkblue}{RGB}{0,51,102}
\definecolor{lightblue}{RGB}{102,178,255}
\definecolor{codecolor}{RGB}{40,40,40}
\definecolor{commentcolor}{RGB}{0,128,0}
\definecolor{stringcolor}{RGB}{163,21,21}
\definecolor{terminalgreen}{RGB}{0,200,0}

% Configurazione colori tema
\setbeamercolor{palette primary}{bg=darkblue,fg=white}
\setbeamercolor{palette secondary}{bg=lightblue,fg=white}
\setbeamercolor{palette tertiary}{bg=darkblue,fg=white}
\setbeamercolor{structure}{fg=darkblue}
\setbeamercolor{frametitle}{bg=darkblue,fg=white}
\setbeamercolor{block title}{bg=lightblue,fg=white}
\setbeamercolor{block body}{bg=lightblue!10}

% Configurazione listing per codice
\lstset{
	basicstyle=\ttfamily\small,
	keywordstyle=\color{blue}\bfseries,
	commentstyle=\color{commentcolor}\itshape,
	stringstyle=\color{stringcolor},
	showstringspaces=false,
	breaklines=true,
	frame=single,
	backgroundcolor=\color{codecolor!5},
	numbers=none,
	xleftmargin=10pt,
	xrightmargin=10pt,
	framexleftmargin=5pt,
	framexrightmargin=5pt,
	language=bash
}

% Informazioni titolo
\title[Filesystem e Permessi Linux]{Il Filesystem Linux e la Gestione dei Permessi}
\subtitle{Struttura, Organizzazione e Sicurezza}
\author{IIS Fermi Sacconi Ceci}
\institute{Corso di Informatica - Sistemi Operativi}
\date{\today}

\begin{document}
	
	% Slide 1: Titolo
	\begin{frame}
		\titlepage
	\end{frame}
	
	% Slide 2: Indice
	\begin{frame}{Indice}
		\tableofcontents
	\end{frame}
	
	% SEZIONE 1: INTRODUZIONE AL FILESYSTEM
	\section{Introduzione al Filesystem Linux}
	
	% Slide 3: Cos'è un Filesystem
	\begin{frame}{Cos'è un Filesystem?}
		\begin{columns}
			\column{0.5\textwidth}
			\begin{block}{Definizione}
				Un \textbf{filesystem} è il metodo con cui un sistema operativo organizza e gestisce i file su un dispositivo di memorizzazione.
			\end{block}
			
			\vspace{0.3cm}
			\textbf{Funzioni principali:}
			\begin{itemize}
				\item Organizzazione gerarchica dei dati
				\item Gestione dello spazio su disco
				\item Controllo degli accessi
				\item Metadati dei file
			\end{itemize}
			
			\column{0.5\textwidth}
			\begin{center}
				\begin{tikzpicture}[scale=0.8]
					\node[draw, rectangle, fill=lightblue!30, minimum width=4cm, minimum height=1cm] (disk) at (0,0) {\textbf{Dispositivo di Storage}};
					\node[draw, rectangle, fill=yellow!30, minimum width=3.5cm, minimum height=0.8cm] (fs) at (0,1.5) {\textbf{Filesystem}};
					\node[draw, rectangle, fill=green!30, minimum width=3cm, minimum height=0.7cm] (files) at (0,3) {\textbf{File e Directory}};
					\draw[->,thick] (disk) -- (fs);
					\draw[->,thick] (fs) -- (files);
				\end{tikzpicture}
			\end{center}
		\end{columns}
	\end{frame}
	
	% Slide 4: Struttura gerarchica
	\begin{frame}{Struttura Gerarchica del Filesystem Linux}
		\begin{columns}
			\column{0.45\textwidth}
			\textbf{Caratteristiche:}
			\begin{itemize}
				\item \textbf{Root directory} (/) alla base
				\item Struttura ad \textbf{albero invertito}
				\item Tutti i dispositivi integrati nella gerarchia
				\item Nessuna lettera di unità (C:, D:, ecc.)
			\end{itemize}
			
			\vspace{0.3cm}
			\begin{alertblock}{Importante}
				A differenza di Windows, Linux usa \texttt{/} (forward slash) come separatore di directory, non \texttt{\textbackslash} (backslash).
			\end{alertblock}
			
			\column{0.55\textwidth}
			\begin{center}
				\begin{tikzpicture}[
					level distance=1.2cm,
					level 1/.style={sibling distance=2.5cm},
					level 2/.style={sibling distance=1.5cm},
					every node/.style={draw,rectangle,rounded corners,fill=lightblue!20,minimum width=1.2cm,font=\small}
					]
					\node {\textbf{/}}
					child {node {bin}}
					child {node {etc}}
					child {node {home}
						child {node {joe}}
						child {node {mary}}
					}
					child {node {usr}}
					child {node {var}};
				\end{tikzpicture}
			\end{center}
		\end{columns}
	\end{frame}
	
	% Slide 5-7: Directory principali
	\begin{frame}[fragile]{Directory Principali del Sistema (1/3)}
		\begin{table}
			\footnotesize
			\begin{tabular}{@{}p{1.5cm}p{9cm}@{}}
				\toprule
				\textbf{Directory} & \textbf{Descrizione} \\
				\midrule
				\texttt{\textbf{/}} & \textbf{Root} - Directory radice del sistema, punto di partenza della gerarchia \\[0.2cm]
				\texttt{/bin} & Comandi binari essenziali del sistema (\texttt{ls}, \texttt{cp}, \texttt{mv}, \texttt{cat}, ecc.) \\[0.2cm]
				\texttt{/boot} & File di avvio del sistema: kernel Linux, RAM disk iniziale, bootloader (GRUB) \\[0.2cm]
				\texttt{/dev} & File di dispositivo (device files): hard disk (\texttt{sda}, \texttt{sdb}), terminali (\texttt{tty}), CD-ROM \\[0.2cm]
				\texttt{/etc} & File di configurazione del sistema (file testuali editabili) \\
				\bottomrule
			\end{tabular}
		\end{table}
		
		\vspace{0.3cm}
		\begin{exampleblock}{Esempio}
			\texttt{/etc/passwd} contiene informazioni sugli utenti del sistema
		\end{exampleblock}
	\end{frame}
	
	\begin{frame}[fragile]{Directory Principali del Sistema (2/3)}
		\begin{table}
			\footnotesize
			\begin{tabular}{@{}p{1.5cm}p{9cm}@{}}
				\toprule
				\textbf{Directory} & \textbf{Descrizione} \\
				\midrule
				\texttt{/home} & Directory home degli utenti normali (\texttt{/home/joe}, \texttt{/home/mary}) \\[0.2cm]
				\texttt{/lib} & Librerie condivise necessarie per l'avvio del sistema \\[0.2cm]
				\texttt{/media} & Punto di mount per dispositivi removibili (USB, CD-ROM) con automount \\[0.2cm]
				\texttt{/mnt} & Punto di mount temporaneo per filesystem montati manualmente \\[0.2cm]
				\texttt{/opt} & Pacchetti software opzionali e applicazioni di terze parti \\
				\bottomrule
			\end{tabular}
		\end{table}
		
		\vspace{0.3cm}
		\begin{exampleblock}{Esempio di montaggio}
			USB drive con nome \texttt{myusb} viene montato su \texttt{/media/myusb}
		\end{exampleblock}
	\end{frame}
	
	\begin{frame}[fragile]{Directory Principali del Sistema (3/3)}
		\begin{table}
			\footnotesize
			\begin{tabular}{@{}p{1.5cm}p{9cm}@{}}
				\toprule
				\textbf{Directory} & \textbf{Descrizione} \\
				\midrule
				\texttt{/proc} & Filesystem virtuale con informazioni sui processi e risorse di sistema \\[0.2cm]
				\texttt{/root} & Directory home dell'utente \textbf{root} (amministratore) - separata per sicurezza \\[0.2cm]
				\texttt{/sbin} & Comandi di sistema per l'amministrazione (usati da root) \\[0.2cm]
				\texttt{/tmp} & File temporanei (eliminati al riavvio) \\[0.2cm]
				\texttt{/usr} & Programmi e dati utente (applicazioni, documentazione, librerie) \\[0.2cm]
				\texttt{/var} & Dati variabili: log di sistema, spool di stampa, cache, database \\
				\bottomrule
			\end{tabular}
		\end{table}
	\end{frame}
	
	% Slide 8: Differenze Linux vs Windows
	\begin{frame}{Filesystem Linux vs Windows}
		\begin{columns}
			\column{0.5\textwidth}
			\textbf{\color{darkblue}Linux}
			\begin{itemize}
				\item[$\checkmark$] Unica gerarchia partendo da \texttt{/}
				\item[$\checkmark$] Usa \texttt{/} (forward slash)
				\item[$\checkmark$] Case-sensitive: \texttt{File.txt} $\neq$ \texttt{file.txt}
				\item[$\checkmark$] Estensioni facoltative
				\item[$\checkmark$] Permessi integrati nel filesystem
				\item[$\checkmark$] Dispositivi integrati nella gerarchia
			\end{itemize}
			
			\column{0.5\textwidth}
			\textbf{\color{darkblue}Windows}
			\begin{itemize}
				\item[$\times$] Lettere di unità separate (C:, D:)
				\item[$\times$] Usa \texttt{\textbackslash} (backslash)
				\item[$\times$] Case-insensitive
				\item[$\times$] Estensioni obbligatorie (.exe, .bat)
				\item[$\times$] Permessi aggiunti successivamente
				\item[$\times$] Dispositivi con lettere separate
			\end{itemize}
		\end{columns}
		
		\vspace{0.4cm}
		\begin{block}{Esempio di path}
			\textbf{Linux:} \texttt{/home/joe/documenti/relazione.txt}\\
			\textbf{Windows:} \texttt{C:\textbackslash Users\textbackslash joe\textbackslash Documents\textbackslash relazione.txt}
		\end{block}
	\end{frame}
	
	% SEZIONE 2: NAVIGAZIONE
	\section{Navigazione nel Filesystem}
	
	% ... (le slide successive rimangono quasi identiche, con piccoli aggiustamenti minori)
	
	% Esempio di correzione importante: Slide 15 (lock icon)
	\begin{frame}{Sistema dei Permessi Linux}
		\begin{columns}
			\column{0.5\textwidth}
			\begin{block}{Perché i permessi?}
				\begin{itemize}
					\item \textbf{Sicurezza}: protezione dei file
					\item \textbf{Privacy}: isolamento tra utenti
					\item \textbf{Integrità}: protezione file di sistema
					\item \textbf{Multiutenza}: sistema condiviso
				\end{itemize}
			\end{block}
			
			\vspace{0.3cm}
			\begin{alertblock}{Importante}
				Ogni file e directory ha:
				\begin{enumerate}
					\item Proprietario (owner)
					\item Gruppo (group)
					\item Permessi di accesso
				\end{enumerate}
			\end{alertblock}
			
			\column{0.5\textwidth}
			\begin{center}
				\begin{tikzpicture}[scale=0.9]
					\node[draw, rectangle, fill=yellow!20, minimum width=4cm, minimum height=2.5cm] (file) at (0,0) {};
					\node[font=\small\bfseries] at (0,0.8) {File / Directory};
					\node[font=\small, text width=3.5cm, align=left] at (0,0.1) {
						\textbf{Owner:} joe\\
						\textbf{Group:} developers\\
						\textbf{Permessi:} rwxr-xr--
					};
					\node[font=\Huge] at (0,-0.7) {\faLock};
				\end{tikzpicture}
			\end{center}
		\end{columns}
	\end{frame}
	
	% Slide 16: Struttura permessi con brace corretti
	\begin{frame}[fragile]{Struttura dei 9 Bit di Permesso}
		\begin{center}
			\begin{tikzpicture}[scale=1.1]
				\foreach \x/\letter/\color in {0/r/red!30, 1/w/green!30, 2/x/blue!30, 3/r/red!20, 4/w/green!20, 5/x/blue!20, 6/r/red!10, 7/w/green!10, 8/x/blue!10} {
					\node[draw, rectangle, minimum width=0.8cm, minimum height=0.8cm, fill=\color] at (\x*0.9,0) {\texttt{\letter}};
				}
				
				\draw[decorate,decoration={brace,amplitude=5pt,mirror}] (-0.2,-0.8) -- (2.3,-0.8) node[midway,below,yshift=-0.3cm] {\small Owner};
				\draw[decorate,decoration={brace,amplitude=5pt,mirror}] (2.6,-0.8) -- (5.0,-0.8) node[midway,below,yshift=-0.3cm] {\small Group};
				\draw[decorate,decoration={brace,amplitude=5pt,mirror}] (5.3,-0.8) -- (7.7,-0.8) node[midway,below,yshift=-0.3cm] {\small Others};
				
				\node[above=0.4cm of {$(0*0.9,0)$}] {\small read};
				\node[above=0.4cm of {$(1*0.9,0)$}] {\small write};
				\node[above=0.4cm of {$(2*0.9,0)$}] {\small execute};
			\end{tikzpicture}
		\end{center}
		
		% ... resto del contenuto identico
	\end{frame}
	
	% Slide finale corretta
	\begin{frame}[plain]
		\begin{center}
			{\huge \textbf{Grazie per l'attenzione!}}
			
			\vspace{1cm}
			{\large Domande?}
			
			\vspace{1.5cm}
			\begin{tikzpicture}
				\node[draw, rectangle, rounded corners, fill=darkblue, text=white, minimum width=8cm, minimum height=1.5cm, font=\large] {
					\textbf{IIS Fermi Sacconi Ceci}
				};
			\end{tikzpicture}
			
			\vspace{0.5cm}
			{\small Corso di Informatica - Sistemi Operativi Linux}
		\end{center}
	\end{frame}
	
\end{document}