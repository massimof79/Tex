\documentclass[aspectratio=169]{beamer}
\usetheme{Madrid}
\usecolortheme{default}

\usepackage[utf8]{inputenc}
\usepackage[italian]{babel}
\usepackage{graphicx}
\usepackage{tikz}
\usepackage{listings}
\usepackage{xcolor}

\usetikzlibrary{shapes.geometric, arrows, positioning, calc}

% Configurazione listing per codice
\lstset{
	basicstyle=\ttfamily\small,
	keywordstyle=\color{blue},
	commentstyle=\color{green!60!black},
	stringstyle=\color{red},
	breaklines=true,
	frame=single
}

\title{Sistemi Operativi: Concetti Fondamentali}
\subtitle{Kernel, Shell e Distribuzioni Linux}
\author{Prof. Massimo Fedeli}
\institute{ITS Academy - Fabbrica Digitale}
\date{\today}

\begin{document}
	
	% Slide 1: Titolo
	\frame{\titlepage}
	
	% Slide 2: Indice
	\begin{frame}{Indice}
		\tableofcontents
	\end{frame}
	
	\section{Introduzione ai Sistemi Operativi}
	
	% Slide 3: Cos'è un Sistema Operativo
	\begin{frame}{Cos'è un Sistema Operativo?}
		\begin{columns}
			\column{0.6\textwidth}
			\textbf{Definizione:} Un sistema operativo (SO) è un software di base che:
			\begin{itemize}
				\item Gestisce le risorse hardware del computer
				\item Fornisce servizi ai programmi applicativi
				\item Fa da intermediario tra utente e hardware
				\item Coordina l'esecuzione dei programmi
			\end{itemize}
			\column{0.4\textwidth}
			\begin{tikzpicture}[scale=0.8]
				\node[draw, rectangle, fill=blue!20, minimum width=3cm, minimum height=0.8cm] (hw) at (0,0) {Hardware};
				\node[draw, rectangle, fill=green!20, minimum width=3cm, minimum height=0.8cm] (so) at (0,1.2) {Sistema Operativo};
				\node[draw, rectangle, fill=yellow!20, minimum width=3cm, minimum height=0.8cm] (app) at (0,2.4) {Applicazioni};
				\node[draw, rectangle, fill=red!20, minimum width=3cm, minimum height=0.8cm] (user) at (0,3.6) {Utente};
				
				\draw[->, thick] (hw) -- (so);
				\draw[->, thick] (so) -- (app);
				\draw[->, thick] (app) -- (user);
			\end{tikzpicture}
		\end{columns}
	\end{frame}
	
	% Slide 4: Funzioni principali del SO
	\begin{frame}{Funzioni Principali del Sistema Operativo}
		\begin{enumerate}
			\item \textbf{Gestione dei processi}
			\begin{itemize}
				\item Creazione, scheduling, terminazione
				\item Sincronizzazione e comunicazione tra processi
			\end{itemize}
			
			\item \textbf{Gestione della memoria}
			\begin{itemize}
				\item Allocazione e deallocazione
				\item Memoria virtuale
			\end{itemize}
			
			\item \textbf{Gestione del file system}
			\begin{itemize}
				\item Organizzazione dei file
				\item Controllo degli accessi
			\end{itemize}
			
			\item \textbf{Gestione I/O}
			\begin{itemize}
				\item Driver dei dispositivi
				\item Buffering e caching
			\end{itemize}
		\end{enumerate}
	\end{frame}
	
	% Slide 5: Tipi di Sistemi Operativi
	\begin{frame}{Tipologie di Sistemi Operativi}
		\begin{columns}
			\column{0.5\textwidth}
			\textbf{Per struttura:}
			\begin{itemize}
				\item Monolitici
				\item A microkernel
				\item Ibridi
				\item A strati
			\end{itemize}
			
			\textbf{Per utilizzo:}
			\begin{itemize}
				\item Desktop (Windows, macOS, Linux)
				\item Server
				\item Mobile (Android, iOS)
				\item Embedded
				\item Real-time
			\end{itemize}
			
			\column{0.5\textwidth}
			\begin{tikzpicture}[scale=0.7]
				% SO Monolitico
				\node[draw, circle, fill=blue!30, minimum size=2.5cm] at (0,2) {\begin{tabular}{c}Kernel\\Monolitico\end{tabular}};
				\node at (0,0.5) {\small Tutto nel kernel};
				
				% SO Microkernel
				\node[draw, circle, fill=green!30, minimum size=1cm] at (4,2.5) {\tiny Kernel};
				\node[draw, rectangle, fill=yellow!20] at (3.2,1.2) {\tiny Servizio};
				\node[draw, rectangle, fill=yellow!20] at (4.8,1.2) {\tiny Servizio};
				\node at (4,0.5) {\small Microkernel};
			\end{tikzpicture}
		\end{columns}
	\end{frame}
	
	\section{Il Kernel}
	
	% Slide 6: Il Kernel - Introduzione
	\begin{frame}{Il Kernel: Cuore del Sistema Operativo}
		\begin{block}{Definizione}
			Il \textbf{kernel} è il nucleo del sistema operativo, il componente che ha controllo completo su tutto il sistema.
		\end{block}
		
		\vspace{0.3cm}
		
		\textbf{Responsabilità principali:}
		\begin{itemize}
			\item Gestione della CPU e scheduling dei processi
			\item Gestione della memoria RAM
			\item Gestione dei dispositivi hardware (driver)
			\item Gestione delle chiamate di sistema (system calls)
			\item Gestione della sicurezza e dei permessi
		\end{itemize}
	\end{frame}
	
	% Slide 7: Modalità di esecuzione
	\begin{frame}{Modalità di Esecuzione: User Mode vs Kernel Mode}
		\begin{columns}
			\column{0.5\textwidth}
			\textbf{User Mode (Modalità utente):}
			\begin{itemize}
				\item Accesso limitato alle risorse
				\item Esecuzione applicazioni utente
				\item Protezione del sistema
				\item Necessita system calls per operazioni privilegiate
			\end{itemize}
			
			\vspace{0.3cm}
			
			\textbf{Kernel Mode (Modalità kernel):}
			\begin{itemize}
				\item Accesso completo all'hardware
				\item Esecuzione codice kernel
				\item Nessuna restrizione
				\item Operazioni critiche del sistema
			\end{itemize}
			
			\column{0.5\textwidth}
			\begin{tikzpicture}[scale=0.9]
				% User Space
				\draw[fill=blue!10] (0,3) rectangle (4,5);
				\node at (2,4.5) {User Space};
				\node[draw, fill=white] at (1,3.7) {\tiny App 1};
				\node[draw, fill=white] at (2,3.7) {\tiny App 2};
				\node[draw, fill=white] at (3,3.7) {\tiny App 3};
				
				% Kernel Space
				\draw[fill=red!10] (0,0) rectangle (4,2.8);
				\node at (2,2.3) {Kernel Space};
				\node at (2,1.5) {\small Kernel};
				\node at (2,0.8) {\tiny Hardware};
				
				% Freccia system call
				\draw[->, very thick, blue] (2,3) -- (2,2.8);
				\node[right] at (2.1,2.9) {\tiny syscall};
			\end{tikzpicture}
		\end{columns}
	\end{frame}
	
	% Slide 8: Architetture del Kernel
	\begin{frame}{Architetture del Kernel}
		\begin{columns}
			\column{0.5\textwidth}
			\textbf{1. Kernel Monolitico}
			\begin{itemize}
				\item Tutto il codice in un unico blocco
				\item Alte prestazioni
				\item Più difficile da manutenere
				\item Esempio: Linux, Unix tradizionale
			\end{itemize}
			
			\vspace{0.3cm}
			
			\textbf{2. Microkernel}
			\begin{itemize}
				\item Kernel minimale
				\item Servizi in user space
				\item Maggiore stabilità
				\item Esempio: Minix, QNX
			\end{itemize}
			
			\column{0.5\textwidth}
			\textbf{3. Kernel Ibrido}
			\begin{itemize}
				\item Compromesso tra i due
				\item Alcuni servizi nel kernel
				\item Altri in user space
				\item Esempio: Windows NT, macOS
			\end{itemize}
			
			\vspace{0.3cm}
			
			\begin{tikzpicture}[scale=0.6]
				\node[draw, rectangle, fill=blue!30, minimum width=2cm, minimum height=1.5cm] at (0,0) {\begin{tabular}{c}\tiny FS\\\tiny Driver\\\tiny Net\end{tabular}};
				\node at (0,-1.2) {\tiny Monolitico};
				
				\node[draw, rectangle, fill=green!30, minimum width=1cm, minimum height=0.5cm] at (3,0) {\tiny Core};
				\node[draw, rectangle, fill=yellow!20, minimum width=0.4cm, minimum height=0.3cm] at (2.5,0.7) {\tiny};
				\node[draw, rectangle, fill=yellow!20, minimum width=0.4cm, minimum height=0.3cm] at (3.5,0.7) {\tiny};
				\node at (3,-1.2) {\tiny Micro};
			\end{tikzpicture}
		\end{columns}
	\end{frame}
	
	% Slide 9: Il Kernel Linux
	\begin{frame}{Il Kernel Linux}
		\begin{columns}
			\column{0.6\textwidth}
			\textbf{Caratteristiche:}
			\begin{itemize}
				\item Architettura monolitica modulare
				\item Open Source (GPL v2)
				\item Supporto multipiattaforma
				\item Altamente configurabile
				\item Sviluppo collaborativo
			\end{itemize}
			
			\vspace{0.3cm}
			
			\textbf{Componenti principali:}
			\begin{itemize}
				\item Process scheduler
				\item Memory manager
				\item Virtual File System (VFS)
				\item Network stack
				\item Device drivers
			\end{itemize}
			
			\column{0.4\textwidth}
			\begin{block}{Storia}
				\begin{itemize}
					\item Creato da Linus Torvalds nel 1991
					\item Versione 0.01: 10.000 righe
					\item Oggi: oltre 27 milioni di righe
					\item Migliaia di contribuitori
				\end{itemize}
			\end{block}
			
			\vspace{0.2cm}
			
			\begin{tikzpicture}[scale=0.7]
				\node[draw, ellipse, fill=orange!30, minimum width=2.5cm, minimum height=1.5cm] at (0,0) {Kernel Linux};
				\node[draw, rectangle, fill=blue!20] at (-1.5,1.2) {\tiny Moduli};
				\node[draw, rectangle, fill=blue!20] at (1.5,1.2) {\tiny Driver};
				\node[draw, rectangle, fill=blue!20] at (-1.5,-1.2) {\tiny FS};
				\node[draw, rectangle, fill=blue!20] at (1.5,-1.2) {\tiny Network};
			\end{tikzpicture}
		\end{columns}
	\end{frame}
	
	% Slide 10: Versioning del Kernel Linux
	\begin{frame}{Versioning del Kernel Linux}
		\textbf{Schema di numerazione:}
		\begin{center}
			\Large X.Y.Z
		\end{center}
		
		\begin{itemize}
			\item \textbf{X} = Versione principale (major)
			\item \textbf{Y} = Versione minore (minor)
			\item \textbf{Z} = Revisione (patch level)
		\end{itemize}
		
		\vspace{0.3cm}
		
		\textbf{Esempi:}
		\begin{itemize}
			\item 5.15.0 → Versione 5, release 15, nessuna patch
			\item 6.1.38 → Versione 6, release 1, patch 38
		\end{itemize}
		
		\vspace{0.3cm}
		
		\textbf{Tipi di rilascio:}
		\begin{itemize}
			\item \textbf{Mainline:} versione in sviluppo attivo
			\item \textbf{Stable:} versioni stabili per produzione
			\item \textbf{Longterm (LTS):} supporto esteso (2-6 anni)
		\end{itemize}
	\end{frame}
	
	\section{La Shell}
	
	% Slide 11: La Shell - Introduzione
	\begin{frame}{La Shell: Interfaccia a Riga di Comando}
		\begin{block}{Definizione}
			La \textbf{shell} è un interprete di comandi che fornisce un'interfaccia tra l'utente e il kernel del sistema operativo.
		\end{block}
		
		\vspace{0.3cm}
		
		\textbf{Funzioni principali:}
		\begin{itemize}
			\item Interpreta ed esegue comandi
			\item Gestisce input/output
			\item Supporta scripting e automazione
			\item Gestisce variabili d'ambiente
			\item Controllo dei processi
			\item Espansione di caratteri speciali (wildcards)
		\end{itemize}
	\end{frame}
	
	% Slide 12: Tipi di Shell
	\begin{frame}[fragile]{Tipi di Shell in Linux}
		\begin{columns}
			\column{0.5\textwidth}
			\textbf{Shell principali:}
			
			\begin{enumerate}
				\item \textbf{sh (Bourne Shell)}
				\begin{itemize}
					\item Shell originale Unix
					\item Standard POSIX
				\end{itemize}
				
				\item \textbf{bash (Bourne Again Shell)}
				\begin{itemize}
					\item Shell predefinita in molte distro
					\item Compatibile con sh
					\item Funzionalità avanzate
				\end{itemize}
				
				\item \textbf{zsh (Z Shell)}
				\begin{itemize}
					\item Altamente personalizzabile
					\item Auto-completamento avanzato
				\end{itemize}
				
				\item \textbf{fish (Friendly Interactive Shell)}
				\begin{itemize}
					\item Sintassi user-friendly
					\item Suggerimenti automatici
				\end{itemize}
			\end{enumerate}
			
			\column{0.5\textwidth}
			\textbf{Altre shell:}
			\begin{itemize}
				\item csh (C Shell)
				\item tcsh (TENEX C Shell)
				\item ksh (Korn Shell)
				\item dash (Debian Almquist Shell)
			\end{itemize}
			
			\vspace{0.3cm}
			
			\textbf{Controllare la shell corrente:}
			\begin{lstlisting}[language=bash]
				echo $SHELL
				# Output: /bin/bash
				
				ps -p $$
			\end{lstlisting}
		\end{columns}
	\end{frame}
	
	% Slide 13: Componenti della Shell
	\begin{frame}{Componenti e Funzionalità della Shell}
		\begin{enumerate}
			\item \textbf{Prompt dei comandi}
			\begin{itemize}
				\item Indica che la shell è pronta
				\item Personalizzabile (PS1)
				\item Mostra info utili (utente, path, ecc.)
			\end{itemize}
			
			\item \textbf{Interprete di comandi}
			\begin{itemize}
				\item Parsing della linea di comando
				\item Esecuzione programmi
			\end{itemize}
			
			\item \textbf{Espansione e sostituzione}
			\begin{itemize}
				\item Wildcards: *, ?, [ ]
				\item Variabili: \$VAR
				\item Command substitution: \$(comando)
			\end{itemize}
			
			\item \textbf{Redirezione I/O}
			\begin{itemize}
				\item Input: <
				\item Output: >, >>
				\item Pipe: |
			\end{itemize}
		\end{enumerate}
	\end{frame}
	
	% Slide 14: Comandi Shell fondamentali
	\begin{frame}[fragile]{Comandi Shell Fondamentali}
		\begin{columns}
			\column{0.5\textwidth}
			\textbf{Navigazione file system:}
			\begin{lstlisting}[language=bash]
				pwd      # stampa directory corrente
				ls       # elenca file
				cd /path # cambia directory
				mkdir    # crea directory
				rmdir    # rimuove directory
			\end{lstlisting}
			
			\textbf{Gestione file:}
			\begin{lstlisting}[language=bash]
				cp       # copia
				mv       # sposta/rinomina
				rm       # rimuove
				touch    # crea file vuoto
				cat      # mostra contenuto
			\end{lstlisting}
			
			\column{0.5\textwidth}
			\textbf{Informazioni di sistema:}
			\begin{lstlisting}[language=bash]
				uname -a # info sistema
				whoami   # utente corrente
				date     # data e ora
				df -h    # spazio disco
				free -h  # memoria
			\end{lstlisting}
			
			\textbf{Gestione processi:}
			\begin{lstlisting}[language=bash]
				ps       # processi attivi
				top      # monitor processi
				kill PID # termina processo
				bg/fg    # background/foreground
			\end{lstlisting}
		\end{columns}
	\end{frame}
	
	% Slide 15: Scripting Shell
	\begin{frame}[fragile]{Shell Scripting}
		\textbf{Uno script shell è un file di testo con una sequenza di comandi.}
		
		\begin{lstlisting}[language=bash]
			#!/bin/bash
			# Primo script di esempio
			
			# Variabili
			NOME="Mario"
			echo "Ciao $NOME!"
			
			# Controllo condizionale
			if [ -f /etc/passwd ]; then
			echo "Il file esiste"
			fi
			
			# Ciclo
			for i in {1..5}; do
			echo "Iterazione $i"
			done
		\end{lstlisting}
		
		\textbf{Esecuzione:}
		\begin{lstlisting}[language=bash]
			chmod +x script.sh
			./script.sh
		\end{lstlisting}
	\end{frame}
	
	% Slide 16: Variabili d'ambiente
	\begin{frame}[fragile]{Variabili d'Ambiente}
		\textbf{Variabili che influenzano il comportamento del sistema e dei programmi.}
		
		\begin{columns}
			\column{0.5\textwidth}
			\textbf{Variabili comuni:}
			\begin{itemize}
				\item \texttt{PATH} - percorsi eseguibili
				\item \texttt{HOME} - directory home
				\item \texttt{USER} - nome utente
				\item \texttt{SHELL} - shell corrente
				\item \texttt{LANG} - lingua sistema
				\item \texttt{PWD} - directory corrente
			\end{itemize}
			
			\column{0.5\textwidth}
			\textbf{Gestione variabili:}
			\begin{lstlisting}[language=bash]
				# Visualizzare
				echo $PATH
				env | grep PATH
				
				# Impostare (temporanea)
				export MYVAR="valore"
				
				# Impostare (permanente)
				# Aggiungere in ~/.bashrc
				export EDITOR=vim
				
				# Rimuovere
				unset MYVAR
			\end{lstlisting}
		\end{columns}
	\end{frame}
	
	\section{GNU/Linux}
	
	% Slide 17: Storia di GNU/Linux
	\begin{frame}{Storia di GNU/Linux}
		\begin{columns}
			\column{0.5\textwidth}
			\textbf{Progetto GNU (1983)}
			\begin{itemize}
				\item Fondato da Richard Stallman
				\item Sistema operativo libero
				\item Filosofia Free Software
				\item GNU's Not Unix
				\item Strumenti: gcc, glibc, bash, ecc.
			\end{itemize}
			
			\vspace{0.3cm}
			
			\textbf{Linux Kernel (1991)}
			\begin{itemize}
				\item Creato da Linus Torvalds
				\item Inizialmente hobby personale
				\item Rilasciato con licenza GPL
				\item Completa il sistema GNU
			\end{itemize}
			
			\column{0.5\textwidth}
			\textbf{Timeline:}
			\begin{itemize}
				\item 1983 - Progetto GNU inizia
				\item 1991 - Linux 0.01 rilasciato
				\item 1992 - Licenza GPL per Linux
				\item 1993 - Slackware (prima distro)
				\item 1993 - Debian fondata
				\item 1994 - Red Hat fondata
				\item 2004 - Ubuntu rilasciata
				\item 2011 - Linux 3.0
				\item 2022 - Linux 6.0
			\end{itemize}
			
			\vspace{0.2cm}
			
			\begin{block}{GNU/Linux}
				Sistema operativo completo = Kernel Linux + Strumenti GNU + Altro software
			\end{block}
		\end{columns}
	\end{frame}
	
	% Slide 18: Filosofia Free Software
	\begin{frame}{Filosofia del Software Libero}
		\textbf{Le Quattro Libertà Fondamentali:}
		
		\begin{enumerate}
			\item[0.] \textbf{Libertà di eseguire} il programma per qualsiasi scopo
			
			\item[1.] \textbf{Libertà di studiare} come funziona il programma e modificarlo
			
			\item[2.] \textbf{Libertà di redistribuire} copie del programma
			
			\item[3.] \textbf{Libertà di migliorare} il programma e distribuire i miglioramenti
		\end{enumerate}
		
		\vspace{0.5cm}
		
		\begin{block}{Open Source vs Free Software}
			\begin{itemize}
				\item \textbf{Free Software:} enfasi sulle libertà etiche
				\item \textbf{Open Source:} enfasi sui vantaggi pratici
				\item Spesso sovrapposti ma filosofie diverse
			\end{itemize}
		\end{block}
	\end{frame}
	
	% Slide 19: Licenze Open Source
	\begin{frame}{Principali Licenze Open Source}
		\begin{columns}
			\column{0.5\textwidth}
			\textbf{Licenze Copyleft:}
			\begin{itemize}
				\item \textbf{GPL (GNU General Public License)}
				\begin{itemize}
					\item Versioni: GPLv2, GPLv3
					\item Derivati devono essere GPL
					\item Usata da Linux
				\end{itemize}
				
				\item \textbf{LGPL (Lesser GPL)}
				\begin{itemize}
					\item Permette linking con software proprietario
					\item Usata per librerie
				\end{itemize}
				
				\item \textbf{AGPL}
				\begin{itemize}
					\item Come GPL ma per software web
				\end{itemize}
			\end{itemize}
			
			\column{0.5\textwidth}
			\textbf{Licenze Permissive:}
			\begin{itemize}
				\item \textbf{MIT License}
				\begin{itemize}
					\item Molto permissiva
					\item Breve e semplice
				\end{itemize}
				
				\item \textbf{BSD License}
				\begin{itemize}
					\item 2-clause, 3-clause
					\item Simile a MIT
				\end{itemize}
				
				\item \textbf{Apache License 2.0}
				\begin{itemize}
					\item Include protezione brevetti
					\item Usata da Apache, Android
				\end{itemize}
			\end{itemize}
		\end{columns}
		
		\vspace{0.3cm}
		\textbf{Differenza chiave:} Copyleft richiede che i derivati mantengano la stessa licenza; permissive permettono uso anche in software proprietario.
	\end{frame}
	
	\section{Distribuzioni Linux}
	
	% Slide 20: Cosa sono le Distribuzioni
	\begin{frame}{Distribuzioni Linux: Concetto}
		\begin{block}{Definizione}
			Una \textbf{distribuzione Linux} (distro) è un sistema operativo completo basato sul kernel Linux, che include:
		\end{block}
		
		\begin{itemize}
			\item Kernel Linux
			\item Strumenti GNU e utilità di sistema
			\item Package manager
			\item Desktop environment (opzionale)
			\item Applicazioni preinstallate
			\item Strumenti di configurazione
			\item Sistema di init (systemd, OpenRC, ecc.)
		\end{itemize}
		
		\vspace{0.3cm}
		
		\textbf{Perché esistono tante distribuzioni?}
		\begin{itemize}
			\item Scopi diversi (desktop, server, embedded)
			\item Filosofie diverse
			\item Target di utenza diverso
			\item Cicli di rilascio diversi
		\end{itemize}
	\end{frame}
	
	% Slide 21: Famiglie di Distribuzioni
	\begin{frame}{Famiglie di Distribuzioni Linux}
		\begin{columns}
			\column{0.5\textwidth}
			\textbf{1. Famiglia Debian}
			\begin{itemize}
				\item Debian
				\item Ubuntu (e derivate)
				\item Linux Mint
				\item Pop!\_OS
				\item Package: .deb (apt)
			\end{itemize}
			
			\vspace{0.3cm}
			
			\textbf{2. Famiglia Red Hat}
			\begin{itemize}
				\item Red Hat Enterprise Linux (RHEL)
				\item Fedora
				\item CentOS / Rocky Linux / AlmaLinux
				\item Package: .rpm (dnf/yum)
			\end{itemize}
			
			\column{0.5\textwidth}
			\textbf{3. Famiglia Arch}
			\begin{itemize}
				\item Arch Linux
				\item Manjaro
				\item EndeavourOS
				\item Package: pacman
			\end{itemize}
			
			\vspace{0.3cm}
			
			\textbf{4. Altre famiglie}
			\begin{itemize}
				\item SUSE (openSUSE, SLES)
				\item Gentoo (source-based)
				\item Slackware
				\item Alpine Linux
				\item Distribuzioni indipendenti
			\end{itemize}
		\end{columns}
	\end{frame}
	
	% Slide 22: Ubuntu
	\begin{frame}{Ubuntu: La Distribuzione più Popolare}
		\begin{columns}
			\column{0.6\textwidth}
			\textbf{Caratteristiche:}
			\begin{itemize}
				\item Basata su Debian
				\item Sviluppata da Canonical
				\item Focus su usabilità
				\item Rilasci ogni 6 mesi
				\item LTS ogni 2 anni (5 anni supporto)
				\item Desktop environment: GNOME (default)
			\end{itemize}
			
			\vspace{0.3cm}
			
			\textbf{Varianti ufficiali:}
			\begin{itemize}
				\item Kubuntu (KDE)
				\item Xubuntu (Xfce)
				\item Lubuntu (LXQt)
				\item Ubuntu MATE
				\item Ubuntu Server
			\end{itemize}
			
			\column{0.4\textwidth}
			\textbf{Versioni LTS:}
			\begin{itemize}
				\item 18.04 (Bionic Beaver)
				\item 20.04 (Focal Fossa)
				\item 22.04 (Jammy Jellyfish)
				\item 24.04 (Noble Numbat)
			\end{itemize}
			
			\vspace{0.3cm}
			
			\textbf{Punti di forza:}
			\begin{itemize}
				\item Documentazione eccellente
				\item Comunità molto attiva
				\item Ampio supporto hardware
				\item Repository ricco
				\item Installazione semplice
			\end{itemize}
		\end{columns}
	\end{frame}
	
	% Slide 23: Fedora
	\begin{frame}{Fedora: Innovazione e Cutting Edge}
		\begin{columns}
			\column{0.6\textwidth}
			\textbf{Caratteristiche:}
			\begin{itemize}
				\item Sponsorizzata da Red Hat
				\item Tecnologie all'avanguardia
				\item Rilascio ogni ~6 mesi
				\item Base per RHEL
				\item Desktop environment: GNOME (Workstation)
				\item Package manager: DNF
			\end{itemize}
			
			\vspace{0.3cm}
			
			\textbf{Edizioni:}
			\begin{itemize}
				\item Fedora Workstation (desktop)
				\item Fedora Server
				\item Fedora IoT
				\item Fedora CoreOS
			\end{itemize}
			
			\column{0.4\textwidth}
			\textbf{Spin ufficiali:}
			\begin{itemize}
				\item KDE Plasma
				\item Xfce
				\item LXQt
				\item MATE-Compiz
				\item Cinnamon
				\item LXDE
				\item i3 (tiling WM)
			\end{itemize}
			
			\vspace{0.3cm}
			
			\textbf{Filosofia:}
			\begin{itemize}
				\item Software libero prioritario
				\item Innovazione
				\item "First" - prime implementazioni
				\item SELinux di default
			\end{itemize}
		\end{columns}
	\end{frame}
	
	% Slide 24: Debian
	\begin{frame}{Debian: Stabilità e Affidabilità}
		\begin{columns}
			\column{0.6\textwidth}
			\textbf{Caratteristiche:}
			\begin{itemize}
				\item Progetto comunitario (no azienda)
				\item Rilasci "when ready"
				\item Contratto sociale
				\item Supporto multi-architettura
				\item Repository enormi (59.000+ pacchetti)
			\end{itemize}
			
			\vspace{0.3cm}
			
			\textbf{Rami (branches):}
			\begin{itemize}
				\item \textbf{Stable} - produzione
				\item \textbf{Testing} - test per prossima stable
				\item \textbf{Unstable (Sid)} - sviluppo
				\item \textbf{Experimental} - sperimentale
			\end{itemize}
			
			\column{0.4\textwidth}
			\textbf{Versioni recenti:}
			\begin{itemize}
				\item Debian 10 (Buster)
				\item Debian 11 (Bullseye)
				\item Debian 12 (Bookworm)
			\end{itemize}
			
			\vspace{0.3cm}
			
			\textbf{Punti di forza:}
			\begin{itemize}
				\item Estrema stabilità
				\item Vasta scelta di software
				\item Base per molte distro
				\item Governata democraticamente
				\item Supporto lungo
			\end{itemize}
		\end{columns}
	\end{frame}
	
	% Slide 25: Arch Linux
	\begin{frame}{Arch Linux: Semplicità e Controllo}
		\begin{columns}
			\column{0.6\textwidth}
			\textbf{Caratteristiche:}
			\begin{itemize}
				\item Rolling release
				\item Configurazione manuale
				\item Filosofia KISS (Keep It Simple, Stupid)
				\item Sempre aggiornata
				\item Pacman come package manager
				\item AUR (Arch User Repository)
			\end{itemize}
			
			\vspace{0.3cm}
			
			\textbf{Filosofia:}
			\begin{itemize}
				\item L'utente controlla tutto
				\item Documentazione eccellente (ArchWiki)
				\item Software vanilla (non modificato)
				\item Installazione da zero
			\end{itemize}
			
			\column{0.4\textwidth}
			\textbf{Vantaggi:}
			\begin{itemize}
				\item Software sempre recente
				\item Massima personalizzazione
				\item Leggera e veloce
				\item Imparare Linux a fondo
			\end{itemize}
			
			\vspace{0.3cm}
			
			\textbf{Derivate user-friendly:}
			\begin{itemize}
				\item Manjaro
				\item EndeavourOS
				\item Garuda Linux
				\item ArcoLinux
			\end{itemize}
			
			\vspace{0.3cm}
			
			\textbf{Target:} Utenti esperti
		\end{columns}
	\end{frame}
	
	% Slide 26: Package Management
	\begin{frame}[fragile]{Gestione dei Pacchetti}
		\textbf{I package manager gestiscono l'installazione, aggiornamento e rimozione del software.}
		
		\begin{columns}
			\column{0.5\textwidth}
			\textbf{APT (Debian/Ubuntu):}
			\begin{lstlisting}[language=bash]
				# Aggiornare repository
				sudo apt update
				
				# Aggiornare sistema
				sudo apt upgrade
				
				# Installare pacchetto
				sudo apt install nome-pacchetto
				
				# Rimuovere pacchetto
				sudo apt remove nome-pacchetto
				
				# Cercare pacchetto
				apt search termine
			\end{lstlisting}
			
			\column{0.5\textwidth}
			\textbf{DNF (Fedora/RHEL):}
			\begin{lstlisting}[language=bash]
				# Aggiornare sistema
				sudo dnf upgrade
				
				# Installare pacchetto
				sudo dnf install nome-pacchetto
				
				# Rimuovere pacchetto
				sudo dnf remove nome-pacchetto
				
				# Cercare pacchetto
				dnf search termine
				
				# Info pacchetto
				dnf info nome-pacchetto
			\end{lstlisting}
		\end{columns}
	\end{frame}
	
	% Slide 27: Desktop Environments
	\begin{frame}{Desktop Environment}
		\textbf{Ambiente grafico che fornisce l'interfaccia utente.}
		
		\begin{columns}
			\column{0.5\textwidth}
			\textbf{GNOME}
			\begin{itemize}
				\item Moderno e minimalista
				\item Workflow basato su attività
				\item Richiede risorse medie-alte
				\item Default: Ubuntu, Fedora
			\end{itemize}
			
			\vspace{0.2cm}
			
			\textbf{KDE Plasma}
			\begin{itemize}
				\item Altamente personalizzabile
				\item Look simile a Windows
				\item Molte funzionalità
				\item Default: Kubuntu
			\end{itemize}
			
			\vspace{0.2cm}
			
			\textbf{Xfce}
			\begin{itemize}
				\item Leggero e veloce
				\item Tradizionale
				\item Poche risorse
				\item Default: Xubuntu
			\end{itemize}
			
			\column{0.5\textwidth}
			\textbf{MATE}
			\begin{itemize}
				\item Fork di GNOME 2
				\item Tradizionale
				\item Medio-leggero
				\item Default: Ubuntu MATE
			\end{itemize}
			
			\vspace{0.2cm}
			
			\textbf{Cinnamon}
			\begin{itemize}
				\item Da Linux Mint
				\item Simile a Windows
				\item Elegante
				\item Default: Linux Mint
			\end{itemize}
			
			\vspace{0.2cm}
			
			\textbf{LXQt / LXDE}
			\begin{itemize}
				\item Leggerissimi
				\item Per hardware datato
				\item Essenziali
				\item Default: Lubuntu
			\end{itemize}
		\end{columns}
	\end{frame}
	
	% Slide 28: File System Hierarchy
	\begin{frame}{File System Hierarchy Standard (FHS)}
		\textbf{Struttura standard delle directory in Linux:}
		
		\begin{columns}
			\column{0.5\textwidth}
			\begin{itemize}
				\item \textbf{/} - root, radice del filesystem
				\item \textbf{/bin} - comandi essenziali
				\item \textbf{/boot} - file boot loader
				\item \textbf{/dev} - file dispositivi
				\item \textbf{/etc} - configurazioni sistema
				\item \textbf{/home} - directory utenti
				\item \textbf{/lib} - librerie condivise
				\item \textbf{/media} - mount point removibili
			\end{itemize}
			
			\column{0.5\textwidth}
			\begin{itemize}
				\item \textbf{/mnt} - mount point temporanei
				\item \textbf{/opt} - software opzionale
				\item \textbf{/proc} - filesystem virtuale
				\item \textbf{/root} - home dell'utente root
				\item \textbf{/sbin} - comandi amministrazione
				\item \textbf{/tmp} - file temporanei
				\item \textbf{/usr} - programmi utente
				\item \textbf{/var} - dati variabili
			\end{itemize}
		\end{columns}
		
		\vspace{0.3cm}
		\textbf{Tutto è un file} in Linux: dispositivi, processi, configurazioni...
	\end{frame}
	
	% Slide 29: Systemd
	\begin{frame}{Systemd: Sistema di Init Moderno}
		\begin{columns}
			\column{0.6\textwidth}
			\textbf{Cos'è systemd?}
			\begin{itemize}
				\item Sistema di init e service manager
				\item Sostituisce SysV init
				\item Avvio parallelo dei servizi
				\item Gestione dipendenze
				\item Logging centralizzato (journald)
			\end{itemize}
			
			\vspace{0.3cm}
			
			\textbf{Componenti principali:}
			\begin{itemize}
				\item systemd - processo init (PID 1)
				\item systemctl - gestione servizi
				\item journalctl - visualizza log
				\item Unit files - configurazione servizi
			\end{itemize}
			
			\column{0.4\textwidth}
			\textbf{Comandi comuni:}
			\begin{itemize}
				\item \texttt{systemctl start servizio}
				\item \texttt{systemctl stop servizio}
				\item \texttt{systemctl restart servizio}
				\item \texttt{systemctl status servizio}
				\item \texttt{systemctl enable servizio}
				\item \texttt{systemctl disable servizio}
			\end{itemize}
			
			\vspace{0.3cm}
			
			\textbf{Visualizzare log:}
			\begin{itemize}
				\item \texttt{journalctl -xe}
				\item \texttt{journalctl -u servizio}
				\item \texttt{journalctl -f}
			\end{itemize}
		\end{columns}
	\end{frame}
	
	% Slide 30: Permessi e Utenti
	\begin{frame}[fragile]{Sistema di Permessi in Linux}
		\textbf{Linux è un sistema multiutente con controllo degli accessi.}
		
		\begin{columns}
			\column{0.5\textwidth}
			\textbf{Tipi di permessi:}
			\begin{itemize}
				\item \textbf{r} (read) - lettura
				\item \textbf{w} (write) - scrittura
				\item \textbf{x} (execute) - esecuzione
			\end{itemize}
			
			\vspace{0.2cm}
			
			\textbf{Categorie:}
			\begin{itemize}
				\item \textbf{u} (user) - proprietario
				\item \textbf{g} (group) - gruppo
				\item \textbf{o} (others) - altri
			\end{itemize}
			
			\vspace{0.2cm}
			
			\textbf{Esempio:}
			\begin{lstlisting}
				-rwxr-xr-x 1 user group
			\end{lstlisting}
			\begin{itemize}
				\item user: rwx (7)
				\item group: r-x (5)
				\item others: r-x (5)
			\end{itemize}
			
			\column{0.5\textwidth}
			\textbf{Modificare permessi:}
			\begin{lstlisting}[language=bash]
				# Simbolico
				chmod u+x file
				chmod go-w file
				chmod a+r file
				
				# Numerico (ottale)
				chmod 755 file
				chmod 644 file
				chmod 600 file
			\end{lstlisting}
			
			\textbf{Cambiare proprietà:}
			\begin{lstlisting}[language=bash]
				chown user:group file
				chown user file
				chgrp group file
			\end{lstlisting}
			
			\textbf{sudo:}
			\begin{lstlisting}[language=bash]
				sudo comando
				# Esegue comando come root
			\end{lstlisting}
		\end{columns}
	\end{frame}
	
	% Slide 31: Vantaggi di Linux
	\begin{frame}{Vantaggi di GNU/Linux}
		\begin{columns}
			\column{0.5\textwidth}
			\textbf{Libertà e Apertura:}
			\begin{itemize}
				\item Codice sorgente disponibile
				\item Nessun costo di licenza
				\item Personalizzazione totale
				\item Privacy e controllo
			\end{itemize}
			
			\vspace{0.3cm}
			
			\textbf{Stabilità e Sicurezza:}
			\begin{itemize}
				\item Meno malware
				\item Sistema di permessi robusto
				\item Aggiornamenti frequenti
				\item Server affidabili
			\end{itemize}
			
			\column{0.5\textwidth}
			\textbf{Prestazioni:}
			\begin{itemize}
				\item Efficiente nell'uso risorse
				\item Ottimizzabile
				\item Lunga durata hardware
				\item Veloce anche su PC datati
			\end{itemize}
			
			\vspace{0.3cm}
			
			\textbf{Flessibilità:}
			\begin{itemize}
				\item Molteplici distribuzioni
				\item Desktop environments vari
				\item Adatto a ogni utilizzo
				\item Da embedded a supercomputer
			\end{itemize}
		\end{columns}
		
		\vspace{0.3cm}
		\begin{center}
			\textbf{Linux domina:} Server (96\% top500 supercomputer), Cloud, Android, IoT
		\end{center}
	\end{frame}
	
	% Slide 32: Casi d'uso di Linux
	\begin{frame}{Casi d'Uso di Linux}
		\begin{enumerate}
			\item \textbf{Server e Data Center}
			\begin{itemize}
				\item Web server (Apache, Nginx)
				\item Database server (MySQL, PostgreSQL)
				\item Cloud computing (AWS, Azure, Google Cloud)
			\end{itemize}
			
			\item \textbf{Sviluppo Software}
			\begin{itemize}
				\item Ambienti di sviluppo
				\item DevOps e CI/CD
				\item Container (Docker, Kubernetes)
			\end{itemize}
			
			\item \textbf{Sistemi Embedded}
			\begin{itemize}
				\item Router e dispositivi di rete
				\item Smart TV
				\item Dispositivi IoT
			\end{itemize}
			
			\item \textbf{Desktop e Workstation}
			\begin{itemize}
				\item Programmazione
				\item Grafica e multimedia
				\item Uso quotidiano
			\end{itemize}
			
			\item \textbf{Educazione e Ricerca}
			\begin{itemize}
				\item Laboratori didattici
				\item Calcolo scientifico
				\item Supercomputer
			\end{itemize}
		\end{enumerate}
	\end{frame}
	
	% Slide 33: Risorse e Community
	\begin{frame}{Risorse per Imparare Linux}
		\textbf{Documentazione ufficiale:}
		\begin{itemize}
			\item Man pages (comando \texttt{man})
			\item Info pages (\texttt{info})
			\item \texttt{/usr/share/doc}
			\item Wiki delle distribuzioni
		\end{itemize}
		
		\vspace{0.3cm}
		
		\textbf{Risorse online:}
		\begin{itemize}
			\item ArchWiki - documentazione eccellente
			\item Linux Documentation Project
			\item Ubuntu Documentation
			\item Stack Overflow / Unix StackExchange
		\end{itemize}
		
		\vspace{0.3cm}
		
		\textbf{Community:}
		\begin{itemize}
			\item Forum delle distribuzioni
			\item Reddit (r/linux, r/linuxquestions)
			\item IRC e Discord
			\item Linux User Groups (LUG) locali
		\end{itemize}
	\end{frame}
	
	% Slide 34: Conclusioni
	\begin{frame}{Conclusioni}
		\textbf{Punti chiave:}
		\begin{itemize}
			\item Il \textbf{sistema operativo} è il software fondamentale che gestisce l'hardware
			\item Il \textbf{kernel} è il cuore del SO, con accesso diretto all'hardware
			\item La \textbf{shell} fornisce l'interfaccia a riga di comando per interagire con il sistema
			\item \textbf{GNU/Linux} combina il kernel Linux con gli strumenti GNU
			\item Le \textbf{distribuzioni} offrono sistemi completi per diverse esigenze
			\item Linux è \textbf{libero, sicuro, stabile e flessibile}
		\end{itemize}
		
		\vspace{0.5cm}
		
		\begin{center}
			\Large{Linux non è solo un sistema operativo,\\è una filosofia di libertà e condivisione!}
		\end{center}
	\end{frame}
	
	% Slide 35: Bibliografia
	\begin{frame}{Bibliografia}
		\small
		\textbf{Libri:}
		\begin{itemize}
			\item Silberschatz, A., Galvin, P. B., \& Gagne, G. (2018). \textit{Operating System Concepts} (10th ed.). Wiley.
			\item Tanenbaum, A. S., \& Bos, H. (2014). \textit{Modern Operating Systems} (4th ed.). Pearson.
			\item Love, R. (2010). \textit{Linux Kernel Development} (3rd ed.). Addison-Wesley.
			\item Shotts, W. (2019). \textit{The Linux Command Line} (2nd ed.). No Starch Press.
			\item Ward, B. (2021). \textit{How Linux Works} (3rd ed.). No Starch Press.
		\end{itemize}
		
		\textbf{Risorse Online:}
		\begin{itemize}
			\item Linux Kernel Archives: \texttt{https://www.kernel.org/}
			\item The Linux Documentation Project: \texttt{https://tldp.org/}
			\item ArchWiki: \texttt{https://wiki.archlinux.org/}
			\item GNU Project: \texttt{https://www.gnu.org/}
			\item Free Software Foundation: \texttt{https://www.fsf.org/}
		\end{itemize}
		
		\textbf{Standard:}
		\begin{itemize}
			\item POSIX Standard: IEEE Std 1003.1
			\item Filesystem Hierarchy Standard (FHS): \texttt{https://refspecs.linuxfoundation.org/fhs.shtml}
		\end{itemize}
	\end{frame}
	
\end{document}