\documentclass[aspectratio=169]{beamer}
\usepackage[utf8]{inputenc}
\usepackage[italian]{babel}
\usepackage{listings}
\usepackage{xcolor}
\usepackage{graphicx}
\usepackage{tikz}
\usetikzlibrary{shapes.geometric, arrows, positioning}

% Tema
\usetheme{Madrid}
\usecolortheme{default}

% Configurazione listing per codice
\lstset{
    basicstyle=\ttfamily\small,
    keywordstyle=\color{blue},
    commentstyle=\color{gray},
    stringstyle=\color{red},
    showstringspaces=false,
    breaklines=true,
    frame=single,
    backgroundcolor=\color{lightgray!20}
}

% Informazioni documento
\title[Comandi Linux]{I Comandi Fondamentali di Linux}
\subtitle{Una guida pratica basata sull'approccio 80/20}
\author{Basato su "The Linux Commands Handbook" di Flavio Copes}
\institute{IIS Fermi Sacconi Ceci}
\date{\today}

\begin{document}

% Slide 1: Titolo
\frame{\titlepage}

% Slide 2: Indice
\begin{frame}{Indice dei Contenuti}
\tableofcontents
\end{frame}

% Slide 3: Introduzione
\section{Introduzione a Linux}

\begin{frame}{Cos'è Linux?}
\begin{itemize}
    \item \textbf{Sistema Operativo} open source e libero
    \item Nato nel 1991 in Finlandia da Linus Torvalds
    \item Kernel del sistema GNU/Linux
    \item Alimenta la maggior parte dei server di Internet
    \item Base di Android
    \item Disponibile in diverse \textbf{distribuzioni}: Debian, Ubuntu, Red Hat, Fedora, ecc.
\end{itemize}

\begin{center}
\begin{tikzpicture}[scale=0.6]
    \draw[fill=blue!20] (0,0) circle (2cm);
    \node at (0,0) {\Large\textbf{Linux}};
    \node at (0,-0.7) {Kernel};
\end{tikzpicture}
\end{center}
\end{frame}

% Slide 4: Filosofia Linux
\begin{frame}{La Filosofia di Linux}
\begin{block}{Libertà}
\begin{itemize}
    \item Libertà di fare qualsiasi cosa con il tuo computer
    \item Nessuna azienda può dettare cosa puoi o non puoi fare
    \item Codice sorgente completamente accessibile
\end{itemize}
\end{block}

\begin{block}{Regola 80/20}
Questo corso segue la regola 80/20:
\begin{itemize}
    \item Impara il 20\% dei comandi
    \item Che userai l'80\% del tempo
    \item Focus sui comandi essenziali
\end{itemize}
\end{block}
\end{frame}

% Slide 5: Shell
\begin{frame}{Cos'è una Shell?}
\begin{columns}
\column{0.6\textwidth}
\begin{itemize}
    \item \textbf{Interprete di comandi} testuale
    \item Interfaccia tra utente e sistema operativo
    \item Permette di eseguire operazioni tramite comandi
    \item Consente di creare \textbf{script} automatizzati
    \item Più potente e veloce di una GUI
\end{itemize}

\column{0.4\textwidth}
\begin{center}
\begin{tikzpicture}
    \draw[fill=green!20, rounded corners] (0,0) rectangle (3,2);
    \node at (1.5,1.5) {\textbf{Shell}};
    \node at (1.5,1) {\small Bash/Zsh/Fish};
    \draw[->, thick] (1.5,0) -- (1.5,-0.5);
    \draw[fill=blue!20, rounded corners] (0,-1.5) rectangle (3,-0.5);
    \node at (1.5,-1) {\textbf{Kernel}};
\end{tikzpicture}
\end{center}
\end{columns}
\end{frame}

% Slide 6: Tipi di Shell
\begin{frame}{Principali Shell UNIX}
\begin{itemize}
    \item \textbf{sh} (Bourne Shell) - La shell originale di Steve Bourne
    \item \textbf{Bash} (Bourne Again Shell) - Shell predefinita su molti sistemi Linux
    \item \textbf{Zsh} (Z Shell) - Shell predefinita su macOS (da Catalina)
    \item \textbf{Fish} (Friendly Interactive Shell) - Shell moderna e user-friendly
    \item \textbf{Csh/Tcsh} - C Shell e TC Shell
\end{itemize}

\vspace{0.5cm}
\begin{center}
\fbox{\texttt{\$ echo \$SHELL}}
\end{center}
\end{frame}

% Slide 7: man
\section{Comandi di Base e Aiuto}

\begin{frame}[fragile]{Il Comando \texttt{man}}
\begin{block}{Manual Pages}
Il primo comando da conoscere per imparare tutti gli altri!
\end{block}

\begin{lstlisting}[language=bash]
$ man <comando>
\end{lstlisting}

\textbf{Esempio:}
\begin{lstlisting}[language=bash]
$ man ls
\end{lstlisting}

\begin{alertblock}{Alternativa Moderna: tldr}
\texttt{tldr} (Too Long; Didn't Read) fornisce esempi pratici e concisi:
\begin{lstlisting}[language=bash]
$ tldr ls
\end{lstlisting}
\end{alertblock}
\end{frame}

% Slide 8: Navigazione - ls
\section{Navigazione nel Filesystem}

\begin{frame}[fragile]{Comando \texttt{ls} - Listare File}
\textbf{Lista i file in una directory}

\begin{lstlisting}[language=bash]
$ ls                    # lista file directory corrente
$ ls /bin              # lista file in /bin
$ ls -l                # formato lungo (dettagliato)
$ ls -a                # mostra anche file nascosti
$ ls -al               # combinazione opzioni
$ ls -lh               # dimensioni human-readable
\end{lstlisting}

\begin{block}{Output \texttt{ls -l}}
\texttt{-rw-r--r--  1  user  group  1234  Dec 7 10:30  file.txt}
\begin{itemize}
    \item Permessi, link, proprietario, gruppo, dimensione, data, nome
\end{itemize}
\end{block}
\end{frame}

% Slide 9: cd
\begin{frame}[fragile]{Comando \texttt{cd} - Cambiare Directory}
\textbf{Change Directory - Naviga tra le cartelle}

\begin{lstlisting}[language=bash]
$ cd cartella          # entra nella cartella
$ cd /etc              # percorso assoluto
$ cd ..                # torna alla directory padre
$ cd ~                 # vai alla home directory
$ cd -                 # torna alla directory precedente
\end{lstlisting}

\begin{block}{Percorsi Speciali}
\begin{itemize}
    \item \texttt{.} = directory corrente
    \item \texttt{..} = directory padre
    \item \texttt{\textasciitilde} = home directory dell'utente
    \item \texttt{/} = directory radice (root)
\end{itemize}
\end{block}
\end{frame}

% Slide 10: pwd
\begin{frame}[fragile]{Comando \texttt{pwd} - Posizione Corrente}
\textbf{Print Working Directory - Dove mi trovo?}

\begin{lstlisting}[language=bash]
$ pwd
/home/username/documents
\end{lstlisting}

\begin{center}
\begin{tikzpicture}[scale=0.8]
    \node[draw, circle] (root) at (0,0) {/};
    \node[draw, circle] (home) at (-2,-1.5) {home};
    \node[draw, circle] (etc) at (2,-1.5) {etc};
    \node[draw, circle, fill=yellow!30] (user) at (-2,-3) {user};
    
    \draw[-] (root) -- (home);
    \draw[-] (root) -- (etc);
    \draw[-] (home) -- (user);
    
    \node[below] at (user) {\small\texttt{pwd}: /home/user};
\end{tikzpicture}
\end{center}
\end{frame}

% Slide 11: Gestione Directory
\section{Gestione File e Directory}

\begin{frame}[fragile]{Comando \texttt{mkdir} - Creare Directory}
\textbf{Make Directory - Crea nuove cartelle}

\begin{lstlisting}[language=bash]
$ mkdir cartella       # crea una cartella
$ mkdir doc img src    # crea più cartelle
$ mkdir -p path/to/dir # crea percorso completo
\end{lstlisting}

\begin{exampleblock}{Esempio con \texttt{-p}}
\begin{lstlisting}[language=bash]
$ mkdir -p progetti/web/frontend
# Crea tutte le directory intermedie se non esistono
\end{lstlisting}
\end{exampleblock}
\end{frame}

% Slide 12: rmdir
\begin{frame}[fragile]{Comando \texttt{rmdir} - Eliminare Directory}
\textbf{Remove Directory - Elimina cartelle vuote}

\begin{lstlisting}[language=bash]
$ rmdir cartella       # elimina cartella vuota
$ rmdir dir1 dir2      # elimina più cartelle
\end{lstlisting}

\begin{alertblock}{Attenzione!}
\texttt{rmdir} funziona solo con directory vuote. Per eliminare directory con contenuto:
\begin{lstlisting}[language=bash]
$ rm -rf cartella      # -r ricorsivo, -f forza
\end{lstlisting}
\textbf{Usare con cautela!} Nessuna conferma richiesta.
\end{alertblock}
\end{frame}

% Slide 13: cp
\begin{frame}[fragile]{Comando \texttt{cp} - Copiare File}
\textbf{Copy - Copia file e directory}

\begin{lstlisting}[language=bash]
$ cp file1 file2           # copia file1 in file2
$ cp file /path/to/dest    # copia in altra directory
$ cp -r dir1 dir2          # copia directory ricorsivamente
$ cp -i file dest          # chiede conferma prima di sovrascrivere
$ cp -v file dest          # verbose, mostra cosa fa
\end{lstlisting}

\begin{block}{Opzioni Comuni}
\begin{itemize}
    \item \texttt{-r} o \texttt{-R}: ricorsivo (per directory)
    \item \texttt{-i}: interattivo (chiede conferma)
    \item \texttt{-v}: verbose (mostra dettagli)
    \item \texttt{-p}: preserva attributi (permessi, timestamp)
\end{itemize}
\end{block}
\end{frame}

% Slide 14: mv
\begin{frame}[fragile]{Comando \texttt{mv} - Spostare/Rinominare}
\textbf{Move - Sposta o rinomina file e directory}

\begin{lstlisting}[language=bash]
$ mv file1 file2           # rinomina file1 in file2
$ mv file /path/to/dest    # sposta file in altra directory
$ mv -i file dest          # chiede conferma
$ mv dir1 dir2             # rinomina/sposta directory
\end{lstlisting}

\begin{exampleblock}{Esempi}
\begin{lstlisting}[language=bash]
$ mv documento.txt backup/documento.txt
$ mv vecchionome.txt nuovonome.txt
$ mv *.txt documenti/      # sposta tutti i file .txt
\end{lstlisting}
\end{exampleblock}
\end{frame}

% Slide 15: touch
\begin{frame}[fragile]{Comando \texttt{touch} - Creare File}
\textbf{Touch - Crea file vuoti o aggiorna timestamp}

\begin{lstlisting}[language=bash]
$ touch file.txt           # crea file vuoto
$ touch file1 file2 file3  # crea più file
$ touch existing.txt       # aggiorna data modifica
\end{lstlisting}

\begin{block}{Usi Comuni}
\begin{itemize}
    \item Creare file vuoti rapidamente
    \item Aggiornare timestamp di file esistenti
    \item Creare file placeholder per test
\end{itemize}
\end{block}
\end{frame}

% Slide 16: find
\begin{frame}[fragile]{Comando \texttt{find} - Cercare File}
\textbf{Find - Potente strumento di ricerca}

\begin{lstlisting}[language=bash]
$ find . -name "*.txt"         # cerca file .txt
$ find /home -name documento   # cerca nella directory
$ find . -type f               # solo file
$ find . -type d               # solo directory
$ find . -size +1M             # file > 1 MB
$ find . -mtime -7             # modificati ultimi 7 giorni
\end{lstlisting}

\begin{exampleblock}{Esempio Complesso}
\begin{lstlisting}[language=bash]
$ find . -name "*.log" -type f -mtime +30 -delete
# Trova ed elimina file .log più vecchi di 30 giorni
\end{lstlisting}
\end{exampleblock}
\end{frame}

% Slide 17: ln
\begin{frame}[fragile]{Comando \texttt{ln} - Link}
\textbf{Link - Crea collegamenti tra file}

\begin{lstlisting}[language=bash]
$ ln file link             # hard link
$ ln -s file link          # symbolic link (soft link)
$ ln -s /path/to/file link # link simbolico
\end{lstlisting}

\begin{columns}
\column{0.5\textwidth}
\textbf{Hard Link}
\begin{itemize}
    \item Stesso inode del file
    \item Indistinguibile dall'originale
    \item Rimane anche se l'originale è eliminato
\end{itemize}

\column{0.5\textwidth}
\textbf{Symbolic Link}
\begin{itemize}
    \item Puntatore al file
    \item Come un "collegamento"
    \item Si rompe se l'originale è eliminato
\end{itemize}
\end{columns}
\end{frame}

% Slide 18: open
\begin{frame}[fragile]{Comando \texttt{open} - Aprire File}
\textbf{Open - Apre file con applicazione predefinita (macOS)}

\begin{lstlisting}[language=bash]
$ open file.txt            # apre con editor predefinito
$ open image.jpg           # apre con visualizzatore immagini
$ open -a "TextEdit" file  # apre con app specifica
$ open .                   # apre directory corrente nel Finder
\end{lstlisting}

\begin{alertblock}{Linux}
Su Linux, usare:
\begin{lstlisting}[language=bash]
$ xdg-open file.txt
\end{lstlisting}
\end{alertblock}
\end{frame}

% Slide 19: Compressione
\section{Compressione e Archiviazione}

\begin{frame}[fragile]{Comando \texttt{gzip} - Compressione}
\textbf{Gzip - Comprime file}

\begin{lstlisting}[language=bash]
$ gzip file.txt            # crea file.txt.gz
$ gzip -k file.txt         # mantiene file originale
$ gzip -v file.txt         # verbose
$ gzip -9 file.txt         # massima compressione
\end{lstlisting}

\vspace{0.3cm}
\textbf{Decompressione:}
\begin{lstlisting}[language=bash]
$ gunzip file.txt.gz       # decomprime
$ gzip -d file.txt.gz      # equivalente
\end{lstlisting}
\end{frame}

% Slide 20: tar
\begin{frame}[fragile]{Comando \texttt{tar} - Archiviazione}
\textbf{Tape Archive - Crea e gestisce archivi}

\begin{lstlisting}[language=bash]
$ tar -cvf archive.tar dir/     # crea archivio
$ tar -xvf archive.tar          # estrae archivio
$ tar -czvf archive.tar.gz dir/ # comprimi con gzip
$ tar -xzvf archive.tar.gz      # estrai .tar.gz
$ tar -tf archive.tar           # lista contenuto
\end{lstlisting}

\begin{block}{Opzioni Principali}
\begin{itemize}
    \item \texttt{c}: create (crea)
    \item \texttt{x}: extract (estrai)
    \item \texttt{v}: verbose (dettagliato)
    \item \texttt{f}: file (specifica nome file)
    \item \texttt{z}: gzip (compressione)
\end{itemize}
\end{block}
\end{frame}

% Slide 21: alias
\section{Personalizzazione e Utilità}

\begin{frame}[fragile]{Comando \texttt{alias} - Scorciatoie}
\textbf{Alias - Crea abbreviazioni per comandi}

\begin{lstlisting}[language=bash]
$ alias ll='ls -alh'           # crea alias
$ alias gs='git status'
$ alias ..='cd ..'
$ alias                        # mostra tutti gli alias
$ unalias ll                   # rimuove alias
\end{lstlisting}

\begin{exampleblock}{Alias Permanenti}
Aggiungi nel file \texttt{\textasciitilde/.bashrc} o \texttt{\textasciitilde/.zshrc}:
\begin{lstlisting}[language=bash]
alias update='sudo apt update && sudo apt upgrade'
alias cls='clear'
\end{lstlisting}
\end{exampleblock}
\end{frame}

% Slide 22: Visualizzazione Testo
\section{Visualizzazione e Manipolazione Testo}

\begin{frame}[fragile]{Comando \texttt{cat} - Visualizzare File}
\textbf{Concatenate - Visualizza e concatena file}

\begin{lstlisting}[language=bash]
$ cat file.txt             # visualizza contenuto
$ cat file1 file2          # concatena file
$ cat file1 file2 > nuovo  # concatena in nuovo file
$ cat -n file.txt          # con numeri di riga
\end{lstlisting}

\begin{exampleblock}{Creare File al Volo}
\begin{lstlisting}[language=bash]
$ cat > newfile.txt
Digita il testo...
(premi Ctrl+D per terminare)
\end{lstlisting}
\end{exampleblock}
\end{frame}

% Slide 23: less
\begin{frame}[fragile]{Comando \texttt{less} - Visualizzatore Paginato}
\textbf{Less - Visualizza file grandi con navigazione}

\begin{lstlisting}[language=bash]
$ less file.txt            # apre visualizzatore
$ less -N file.txt         # con numeri di riga
\end{lstlisting}

\begin{block}{Navigazione in less}
\begin{itemize}
    \item \texttt{Spazio}: pagina successiva
    \item \texttt{b}: pagina precedente
    \item \texttt{/parola}: cerca parola
    \item \texttt{n}: prossima occorrenza
    \item \texttt{q}: esci
    \item \texttt{G}: vai alla fine
    \item \texttt{g}: vai all'inizio
\end{itemize}
\end{block}
\end{frame}

% Slide 24: tail
\begin{frame}[fragile]{Comando \texttt{tail} - Fine File}
\textbf{Tail - Mostra le ultime righe di un file}

\begin{lstlisting}[language=bash]
$ tail file.txt            # ultime 10 righe
$ tail -n 20 file.txt      # ultime 20 righe
$ tail -f logfile.log      # segue il file in tempo reale
$ tail -f -n 50 log.txt    # segue con 50 righe iniziali
\end{lstlisting}

\begin{alertblock}{Monitoraggio Log}
\texttt{tail -f} è essenziale per monitorare file di log in tempo reale:
\begin{lstlisting}[language=bash]
$ tail -f /var/log/syslog
\end{lstlisting}
\end{alertblock}
\end{frame}

% Slide 25: head
\begin{frame}[fragile]{Comando \texttt{head} - Inizio File}
\textbf{Head - Mostra le prime righe di un file}

\begin{lstlisting}[language=bash]
$ head file.txt            # prime 10 righe
$ head -n 20 file.txt      # prime 20 righe
$ head -n 5 *.txt          # prime 5 righe di più file
\end{lstlisting}

\begin{exampleblock}{Combinazione head e tail}
\begin{lstlisting}[language=bash]
$ head -n 100 file.txt | tail -n 10
# Mostra le righe dalla 91 alla 100
\end{lstlisting}
\end{exampleblock}
\end{frame}

% Slide 26: wc
\begin{frame}[fragile]{Comando \texttt{wc} - Contare}
\textbf{Word Count - Conta righe, parole e caratteri}

\begin{lstlisting}[language=bash]
$ wc file.txt              # righe parole caratteri
$ wc -l file.txt           # solo righe
$ wc -w file.txt           # solo parole
$ wc -c file.txt           # solo byte
$ wc -m file.txt           # solo caratteri
\end{lstlisting}

\begin{exampleblock}{Output}
\begin{lstlisting}[language=bash]
$ wc document.txt
  142  1024  7891  document.txt
  |     |     |
righe parole byte
\end{lstlisting}
\end{exampleblock}
\end{frame}

% Slide 27: grep
\begin{frame}[fragile]{Comando \texttt{grep} - Cercare Testo}
\textbf{Global Regular Expression Print - Cerca pattern}

\begin{lstlisting}[language=bash]
$ grep "parola" file.txt       # cerca parola
$ grep -i "parola" file.txt    # case insensitive
$ grep -r "pattern" dir/       # ricorsivo
$ grep -n "text" file.txt      # con numeri riga
$ grep -v "pattern" file.txt   # righe NON corrispondenti
$ grep -c "word" file.txt      # conta occorrenze
\end{lstlisting}

\begin{exampleblock}{Con Pipe}
\begin{lstlisting}[language=bash]
$ ps aux | grep firefox
$ ls -l | grep "\.txt$"
\end{lstlisting}
\end{exampleblock}
\end{frame}

% Slide 28: sort
\begin{frame}[fragile]{Comando \texttt{sort} - Ordinare}
\textbf{Sort - Ordina righe di testo}

\begin{lstlisting}[language=bash]
$ sort file.txt            # ordina alfabeticamente
$ sort -r file.txt         # ordine inverso
$ sort -n file.txt         # ordine numerico
$ sort -u file.txt         # rimuove duplicati
$ sort -k 2 file.txt       # ordina per colonna 2
\end{lstlisting}

\begin{exampleblock}{Esempio Pratico}
\begin{lstlisting}[language=bash]
$ du -h * | sort -h        # ordina per dimensione
$ ls -l | sort -k 5 -n     # ordina per dimensione file
\end{lstlisting}
\end{exampleblock}
\end{frame}

% Slide 29: uniq
\begin{frame}[fragile]{Comando \texttt{uniq} - Righe Uniche}
\textbf{Unique - Rimuove o conta righe duplicate}

\begin{lstlisting}[language=bash]
$ uniq file.txt            # rimuove duplicati adiacenti
$ uniq -c file.txt         # conta occorrenze
$ uniq -d file.txt         # mostra solo duplicati
$ uniq -u file.txt         # mostra solo righe uniche
\end{lstlisting}

\begin{alertblock}{Nota Importante}
\texttt{uniq} funziona solo su righe adiacenti! Usare con \texttt{sort}:
\begin{lstlisting}[language=bash]
$ sort file.txt | uniq
$ sort file.txt | uniq -c | sort -nr  # più frequenti
\end{lstlisting}
\end{alertblock}
\end{frame}

% Slide 30: diff
\begin{frame}[fragile]{Comando \texttt{diff} - Differenze}
\textbf{Diff - Confronta file}

\begin{lstlisting}[language=bash]
$ diff file1.txt file2.txt     # mostra differenze
$ diff -u file1 file2          # formato unified (Git)
$ diff -y file1 file2          # side by side
$ diff -r dir1/ dir2/          # ricorsivo per directory
$ diff -q dir1/ dir2/          # solo nomi file diversi
\end{lstlisting}

\begin{block}{Simboli Output}
\begin{itemize}
    \item \texttt{<}: riga presente solo nel primo file
    \item \texttt{>}: riga presente solo nel secondo file
    \item \texttt{a}: aggiunta
    \item \texttt{c}: cambiamento
    \item \texttt{d}: cancellazione
\end{itemize}
\end{block}
\end{frame}

% Slide 31: echo
\begin{frame}[fragile]{Comando \texttt{echo} - Stampare Testo}
\textbf{Echo - Stampa argomenti sull'output}

\begin{lstlisting}[language=bash]
$ echo "Hello World"           # stampa testo
$ echo "text" > file.txt       # scrive in file (sovrascrive)
$ echo "text" >> file.txt      # aggiunge a file
$ echo $HOME                   # stampa variabile ambiente
$ echo $(pwd)                  # esegue comando
$ echo {1..10}                 # espansione range
\end{lstlisting}

\begin{exampleblock}{Utilizzi Creativi}
\begin{lstlisting}[language=bash]
$ echo *.txt                   # espansione wildcard
$ echo "PATH is $PATH"         # interpolazione variabili
\end{lstlisting}
\end{exampleblock}
\end{frame}

% Slide 32: Permessi - chown
\section{Permessi e Proprietà}

\begin{frame}[fragile]{Comando \texttt{chown} - Cambiare Proprietario}
\textbf{Change Owner - Cambia proprietario di file/directory}

\begin{lstlisting}[language=bash]
$ chown user file.txt          # cambia proprietario
$ chown user:group file.txt    # cambia owner e group
$ chown -R user dir/           # ricorsivo
$ sudo chown root file.txt     # richiede permessi
\end{lstlisting}

\begin{block}{Uso Comune}
Spesso necessario dopo aver copiato file come root o da altri utenti.
\begin{lstlisting}[language=bash]
$ sudo chown $USER:$USER file.txt
\end{lstlisting}
\end{block}
\end{frame}

% Slide 33: chmod
\begin{frame}[fragile]{Comando \texttt{chmod} - Permessi File (1/2)}
\textbf{Change Mode - Modifica permessi di accesso}

\begin{block}{Permessi Unix}
Tre tipi di permessi per tre categorie:
\begin{itemize}
    \item \textbf{r} (read): lettura (4)
    \item \textbf{w} (write): scrittura (2)
    \item \textbf{x} (execute): esecuzione (1)
\end{itemize}

Tre categorie:
\begin{itemize}
    \item \textbf{u} (user): proprietario
    \item \textbf{g} (group): gruppo
    \item \textbf{o} (others): altri
    \item \textbf{a} (all): tutti
\end{itemize}
\end{block}
\end{frame}

% Slide 34: chmod parte 2
\begin{frame}[fragile]{Comando \texttt{chmod} - Permessi File (2/2)}

\textbf{Modalità Simbolica:}
\begin{lstlisting}[language=bash]
$ chmod u+x script.sh          # aggiungi esecuzione owner
$ chmod g-w file.txt           # rimuovi scrittura group
$ chmod a+r file.txt           # aggiungi lettura tutti
$ chmod o-rwx file.txt         # rimuovi tutti per others
\end{lstlisting}

\textbf{Modalità Numerica:}
\begin{lstlisting}[language=bash]
$ chmod 755 script.sh          # rwxr-xr-x
$ chmod 644 file.txt           # rw-r--r--
$ chmod 777 file.txt           # rwxrwxrwx (sconsigliato!)
\end{lstlisting}

\begin{center}
\small
\texttt{755} = \texttt{7} (rwx) + \texttt{5} (r-x) + \texttt{5} (r-x)
\end{center}
\end{frame}

% Slide 35: umask
\begin{frame}[fragile]{Comando \texttt{umask} - Permessi Predefiniti}
\textbf{Umask - Imposta permessi predefiniti}

\begin{lstlisting}[language=bash]
$ umask                    # mostra umask corrente
0022
$ umask -S                 # formato simbolico
u=rwx,g=rx,o=rx
$ umask 002                # imposta nuova umask
\end{lstlisting}

\begin{block}{Valori Umask Comuni}
\begin{itemize}
    \item \texttt{0022}: file creati con rw-r--r-- (644)
    \item \texttt{0002}: file creati con rw-rw-r-- (664)
    \item \texttt{0077}: file creati con rw------- (600)
\end{itemize}
\end{block}
\end{frame}

% Slide 36: Spazio Disco
\section{Gestione Spazio Disco}

\begin{frame}[fragile]{Comando \texttt{du} - Uso Disco}
\textbf{Disk Usage - Calcola spazio occupato}

\begin{lstlisting}[language=bash]
$ du                       # spazio directory corrente
$ du -h                    # human readable (KB, MB, GB)
$ du -sh *                 # sommario di ogni elemento
$ du -sh dir/              # spazio totale directory
$ du -ah                   # include file singoli
$ du -h | sort -h          # ordina per dimensione
\end{lstlisting}

\begin{exampleblock}{Top 10 Directory Più Grandi}
\begin{lstlisting}[language=bash]
$ du -h /home | sort -hr | head -10
\end{lstlisting}
\end{exampleblock}
\end{frame}

% Slide 37: df
\begin{frame}[fragile]{Comando \texttt{df} - Spazio Filesystem}
\textbf{Disk Free - Mostra spazio disponibile su filesystem}

\begin{lstlisting}[language=bash]
$ df                       # mostra tutti i filesystem
$ df -h                    # formato human readable
$ df -h /home              # spazio su filesystem specifico
$ df -T                    # mostra tipo filesystem
\end{lstlisting}

\begin{block}{Output Tipico}
\begin{center}
\small
\begin{tabular}{lrrrrr}
Filesystem & Size & Used & Avail & Use\% & Mounted \\
/dev/sda1 & 50G & 35G & 13G & 73\% & / \\
/dev/sdb1 & 200G & 120G & 72G & 63\% & /home \\
\end{tabular}
\end{center}
\end{block}
\end{frame}

% Slide 38: Utilità Path
\section{Utilità di Path}

\begin{frame}[fragile]{Comandi \texttt{basename} e \texttt{dirname}}

\textbf{basename - Estrae nome file da path:}
\begin{lstlisting}[language=bash]
$ basename /usr/local/bin/script.sh
script.sh
$ basename /home/user/
user
\end{lstlisting}

\vspace{0.5cm}
\textbf{dirname - Estrae directory da path:}
\begin{lstlisting}[language=bash]
$ dirname /usr/local/bin/script.sh
/usr/local/bin
$ dirname /home/user/file.txt
/home/user
\end{lstlisting}
\end{frame}

% Slide 39: Gestione Processi
\section{Gestione dei Processi}

\begin{frame}[fragile]{Comando \texttt{ps} - Processi in Esecuzione}
\textbf{Process Status - Mostra processi attivi}

\begin{lstlisting}[language=bash]
$ ps                       # processi utente corrente
$ ps aux                   # tutti i processi
$ ps axww                  # con comandi completi
$ ps aux | grep firefox    # cerca processo specifico
$ ps -u username           # processi di un utente
\end{lstlisting}

\begin{block}{Colonne Importanti}
\begin{itemize}
    \item \textbf{PID}: Process ID (identificatore)
    \item \textbf{USER}: utente proprietario
    \item \textbf{\%CPU}: utilizzo CPU
    \item \textbf{\%MEM}: utilizzo memoria
    \item \textbf{STAT}: stato (R=running, S=sleeping, Z=zombie)
    \item \textbf{COMMAND}: comando eseguito
\end{itemize}
\end{block}
\end{frame}

% Slide 40: top
\begin{frame}[fragile]{Comando \texttt{top} - Monitor Processi}
\textbf{Top - Monitor interattivo processi in tempo reale}

\begin{lstlisting}[language=bash]
$ top                      # avvia monitor
$ top -o mem               # ordina per memoria
$ top -n 1                 # un refresh e poi esci
\end{lstlisting}

\begin{block}{Comandi Interattivi in top}
\begin{itemize}
    \item \texttt{q}: esci
    \item \texttt{k}: kill processo (chiede PID)
    \item \texttt{M}: ordina per memoria
    \item \texttt{P}: ordina per CPU
    \item \texttt{1}: mostra ogni CPU
    \item \texttt{h}: help
\end{itemize}
\end{block}
\end{frame}

% Slide 41: kill
\begin{frame}[fragile]{Comando \texttt{kill} - Terminare Processi}
\textbf{Kill - Invia segnali ai processi}

\begin{lstlisting}[language=bash]
$ kill PID                 # termina processo (SIGTERM)
$ kill -9 PID              # termina forzatamente (SIGKILL)
$ kill -15 PID             # SIGTERM (default)
$ kill -l                  # lista tutti i segnali
\end{lstlisting}

\begin{block}{Segnali Principali}
\begin{itemize}
    \item \textbf{SIGTERM (15)}: terminazione gentile
    \item \textbf{SIGKILL (9)}: terminazione immediata
    \item \textbf{SIGHUP (1)}: ricarica configurazione
    \item \textbf{SIGINT (2)}: interruzione (come Ctrl+C)
    \item \textbf{SIGSTOP}: pausa processo
    \item \textbf{SIGCONT}: riprendi processo
\end{itemize}
\end{block}
\end{frame}

% Slide 42: killall
\begin{frame}[fragile]{Comando \texttt{killall} - Kill per Nome}
\textbf{Killall - Termina processi per nome}

\begin{lstlisting}[language=bash]
$ killall firefox          # termina tutti i Firefox
$ killall -9 chrome        # termina forzatamente Chrome
$ killall -u username      # termina processi di un utente
$ killall -i process       # chiede conferma
\end{lstlisting}

\begin{alertblock}{Attenzione!}
\texttt{killall} termina TUTTI i processi con quel nome. Usare con cautela!
\end{alertblock}
\end{frame}

% Slide 43: jobs bg fg
\begin{frame}[fragile]{Comandi \texttt{jobs}, \texttt{bg}, \texttt{fg}}
\textbf{Gestione Job in Background e Foreground}

\begin{lstlisting}[language=bash]
$ command &                # esegui in background
$ jobs                     # lista job attivi
$ fg                       # porta ultimo job in foreground
$ fg %1                    # porta job 1 in foreground
$ bg                       # riprendi job in background
$ bg %2                    # riprendi job 2 in background
\end{lstlisting}

\begin{block}{Control Keys}
\begin{itemize}
    \item \texttt{Ctrl+Z}: sospendi job corrente
    \item \texttt{Ctrl+C}: termina job corrente
\end{itemize}
\end{block}
\end{frame}

% Slide 44: type which
\begin{frame}[fragile]{Comandi \texttt{type} e \texttt{which}}

\textbf{type - Tipo di comando:}
\begin{lstlisting}[language=bash]
$ type ls
ls is aliased to `ls --color=auto'
$ type cd
cd is a shell builtin
$ type python
python is /usr/bin/python
\end{lstlisting}

\textbf{which - Percorso eseguibile:}
\begin{lstlisting}[language=bash]
$ which python
/usr/bin/python
$ which -a python          # mostra tutti i path
\end{lstlisting}
\end{frame}

% Slide 45: nohup xargs
\begin{frame}[fragile]{Comandi \texttt{nohup} e \texttt{xargs}}

\textbf{nohup - Esegui comando ignorando hangup:}
\begin{lstlisting}[language=bash]
$ nohup command &          # continua anche dopo logout
$ nohup ./script.sh &
# output in nohup.out
\end{lstlisting}

\textbf{xargs - Costruisci ed esegui comandi:}
\begin{lstlisting}[language=bash]
$ find . -name "*.tmp" | xargs rm
$ echo "file1 file2" | xargs cat
$ ls | xargs -I {} echo "File: {}"
\end{lstlisting}
\end{frame}

% Slide 46: Editor di Testo
\section{Editor di Testo}

\begin{frame}[fragile]{Editor di Testo in Linux}

\begin{columns}
\column{0.33\textwidth}
\textbf{vim}
\begin{lstlisting}[language=bash]
$ vim file.txt
\end{lstlisting}
\begin{itemize}
    \item Molto potente
    \item Curva apprendimento ripida
    \item Modalità comando/inserimento
\end{itemize}

\column{0.33\textwidth}
\textbf{nano}
\begin{lstlisting}[language=bash]
$ nano file.txt
\end{lstlisting}
\begin{itemize}
    \item User-friendly
    \item Comandi visibili
    \item Ideale per principianti
\end{itemize}

\column{0.33\textwidth}
\textbf{emacs}
\begin{lstlisting}[language=bash]
$ emacs file.txt
\end{lstlisting}
\begin{itemize}
    \item Estremamente potente
    \item Personalizzabile
    \item Curva apprendimento
\end{itemize}
\end{columns}
\end{frame}

% Slide 47: vim basics
\begin{frame}[fragile]{Vim - Comandi Essenziali}

\begin{block}{Modalità}
\begin{itemize}
    \item \textbf{Normale}: navigazione e comandi
    \item \textbf{Inserimento}: digitare testo (tasto \texttt{i})
    \item \textbf{Visuale}: selezione testo (tasto \texttt{v})
\end{itemize}
\end{block}

\begin{columns}
\column{0.5\textwidth}
\textbf{Comandi Base:}
\begin{itemize}
    \item \texttt{i}: modalità inserimento
    \item \texttt{ESC}: modalità normale
    \item \texttt{:w}: salva
    \item \texttt{:q}: esci
    \item \texttt{:wq}: salva ed esci
    \item \texttt{:q!}: esci senza salvare
\end{itemize}

\column{0.5\textwidth}
\textbf{Navigazione:}
\begin{itemize}
    \item \texttt{h,j,k,l}: sinistra, giù, su, destra
    \item \texttt{dd}: cancella riga
    \item \texttt{yy}: copia riga
    \item \texttt{p}: incolla
    \item \texttt{u}: undo
    \item \texttt{/text}: cerca
\end{itemize}
\end{columns}
\end{frame}

% Slide 48: nano
\begin{frame}[fragile]{Nano - Editor Semplice}

\begin{lstlisting}[language=bash]
$ nano filename.txt
\end{lstlisting}

\begin{block}{Comandi Principali (^ = Ctrl)}
\begin{itemize}
    \item \texttt{Ctrl+O}: salva file (WriteOut)
    \item \texttt{Ctrl+X}: esci dall'editor
    \item \texttt{Ctrl+K}: taglia riga
    \item \texttt{Ctrl+U}: incolla
    \item \texttt{Ctrl+W}: cerca testo
    \item \texttt{Ctrl+G}: mostra help
    \item \texttt{Alt+/}: vai alla fine del file
\end{itemize}
\end{block}

I comandi sono sempre visibili in basso allo schermo!
\end{frame}

% Slide 49: Utenti
\section{Gestione Utenti}

\begin{frame}[fragile]{Comandi Utente - Base}

\textbf{whoami - Chi sono?}
\begin{lstlisting}[language=bash]
$ whoami
username
\end{lstlisting}

\textbf{who - Chi è connesso?}
\begin{lstlisting}[language=bash]
$ who
user1    pts/0    2024-12-07 10:30
user2    pts/1    2024-12-07 11:15
\end{lstlisting}

\textbf{id - Informazioni utente}
\begin{lstlisting}[language=bash]
$ id
uid=1000(user) gid=1000(user) groups=1000(user),27(sudo)
\end{lstlisting}
\end{frame}

% Slide 50: su sudo
\begin{frame}[fragile]{Comandi \texttt{su} e \texttt{sudo}}

\textbf{su - Switch User:}
\begin{lstlisting}[language=bash]
$ su                       # diventa root (chiede password root)
$ su username              # diventa altro utente
$ su -                     # diventa root con ambiente
\end{lstlisting}

\textbf{sudo - Execute as SuperUser:}
\begin{lstlisting}[language=bash]
$ sudo command             # esegui comando come root
$ sudo -i                  # shell root interattiva
$ sudo -u user command     # esegui come altro utente
$ sudo !!                  # riesegui ultimo comando come root
\end{lstlisting}

\begin{alertblock}{Sicurezza}
Usa sempre \texttt{sudo} invece di \texttt{su -} quando possibile!
\end{alertblock}
\end{frame}

% Slide 51: passwd
\begin{frame}[fragile]{Comando \texttt{passwd} - Password}
\textbf{Password - Cambia password utente}

\begin{lstlisting}[language=bash]
$ passwd                   # cambia la tua password
$ sudo passwd username     # cambia password altro utente
$ sudo passwd -l username  # blocca account
$ sudo passwd -u username  # sblocca account
$ sudo passwd -e username  # forza cambio al prossimo login
\end{lstlisting}

\begin{block}{Buone Pratiche}
\begin{itemize}
    \item Password lunghe (min 12 caratteri)
    \item Mix di maiuscole, minuscole, numeri, simboli
    \item Non riutilizzare password
    \item Cambiare periodicamente
\end{itemize}
\end{block}
\end{frame}

% Slide 52: Networking
\section{Comandi di Rete}

\begin{frame}[fragile]{Comando \texttt{ping} - Test Connettività}
\textbf{Ping - Verifica connessione di rete}

\begin{lstlisting}[language=bash]
$ ping google.com          # ping continuo
$ ping -c 4 google.com     # 4 pacchetti e stop
$ ping -i 2 host           # intervallo 2 secondi
$ ping -W 1 host           # timeout 1 secondo
\end{lstlisting}

\begin{exampleblock}{Output Tipico}
\begin{lstlisting}
64 bytes from google.com: icmp_seq=1 ttl=117 time=12.4 ms
64 bytes from google.com: icmp_seq=2 ttl=117 time=11.8 ms
--- google.com ping statistics ---
2 packets transmitted, 2 received, 0% packet loss
round-trip min/avg/max = 11.8/12.1/12.4 ms
\end{lstlisting}
\end{exampleblock}
\end{frame}

% Slide 53: traceroute
\begin{frame}[fragile]{Comando \texttt{traceroute} - Percorso Rete}
\textbf{Traceroute - Traccia percorso pacchetti}

\begin{lstlisting}[language=bash]
$ traceroute google.com    # mostra tutti gli hop
$ traceroute -m 15 host    # max 15 hop
$ traceroute -n host       # non risolvere nomi
\end{lstlisting}

\begin{block}{Utilizzo}
\begin{itemize}
    \item Diagnostica problemi di rete
    \item Identifica dove avviene il ritardo
    \item Mostra il percorso attraverso router
    \item Utile per troubleshooting
\end{itemize}
\end{block}

Su alcuni sistemi Linux: \texttt{tracepath}
\end{frame}

% Slide 54: Altri comandi rete
\begin{frame}[fragile]{Altri Comandi di Rete}

\textbf{ifconfig/ip - Configurazione interfacce:}
\begin{lstlisting}[language=bash]
$ ifconfig                 # mostra interfacce (deprecato)
$ ip addr show             # comando moderno
$ ip link show             # mostra solo link
\end{lstlisting}

\textbf{netstat/ss - Connessioni:}
\begin{lstlisting}[language=bash]
$ netstat -tuln            # connessioni TCP/UDP
$ ss -tuln                 # comando moderno
$ ss -t                    # solo TCP
\end{lstlisting}

\textbf{wget/curl - Download:}
\begin{lstlisting}[language=bash]
$ wget http://example.com/file.zip
$ curl -O http://example.com/file.zip
\end{lstlisting}
\end{frame}

% Slide 55: Variabili Ambiente
\section{Variabili d'Ambiente}

\begin{frame}[fragile]{Comando \texttt{export} - Variabili Ambiente}
\textbf{Export - Esporta variabili all'ambiente}

\begin{lstlisting}[language=bash]
$ export VAR="value"       # crea/esporta variabile
$ export PATH=$PATH:/new/path  # aggiunge a PATH
$ export                   # lista tutte le variabili
$ export -n VAR            # rimuove export
\end{lstlisting}

\begin{block}{Variabili Comuni}
\begin{itemize}
    \item \texttt{PATH}: percorsi eseguibili
    \item \texttt{HOME}: directory home utente
    \item \texttt{USER}: nome utente
    \item \texttt{SHELL}: shell corrente
    \item \texttt{PWD}: directory corrente
    \item \texttt{EDITOR}: editor predefinito
\end{itemize}
\end{block}
\end{frame}

% Slide 56: env printenv
\begin{frame}[fragile]{Comandi \texttt{env} e \texttt{printenv}}

\textbf{env - Esegui con ambiente specifico:}
\begin{lstlisting}[language=bash]
$ env                      # mostra tutte le variabili
$ env VAR=value command    # esegui con variabile
$ env -i command           # esegui con ambiente pulito
$ env -u VAR command       # rimuovi variabile
\end{lstlisting}

\textbf{printenv - Stampa variabili:}
\begin{lstlisting}[language=bash]
$ printenv                 # tutte le variabili
$ printenv PATH            # solo PATH
$ printenv HOME USER       # più variabili
\end{lstlisting}
\end{frame}

% Slide 57: Variabili - Esempi
\begin{frame}[fragile]{Variabili d'Ambiente - Esempi}

\textbf{Impostare Editor Predefinito:}
\begin{lstlisting}[language=bash]
$ export EDITOR=nano
$ export VISUAL=nano
\end{lstlisting}

\textbf{Modificare PATH:}
\begin{lstlisting}[language=bash]
$ export PATH=$PATH:$HOME/bin
$ export PATH="/usr/local/bin:$PATH"
\end{lstlisting}

\textbf{Variabili Personalizzate:}
\begin{lstlisting}[language=bash]
$ export API_KEY="abc123"
$ export DB_HOST="localhost"
$ echo $API_KEY
\end{lstlisting}
\end{frame}

% Slide 58: File Configurazione
\begin{frame}[fragile]{File di Configurazione Shell}

\begin{block}{Bash}
\begin{itemize}
    \item \texttt{\textasciitilde/.bashrc}: configurazione shell interattiva
    \item \texttt{\textasciitilde/.bash\_profile}: configurazione login
    \item \texttt{\textasciitilde/.bash\_aliases}: alias personalizzati
    \item \texttt{/etc/bash.bashrc}: configurazione sistema
\end{itemize}
\end{block}

\begin{block}{Zsh}
\begin{itemize}
    \item \texttt{\textasciitilde/.zshrc}: configurazione principale
    \item \texttt{\textasciitilde/.zshenv}: variabili d'ambiente
    \item \texttt{\textasciitilde/.zprofile}: configurazione login
\end{itemize}
\end{block}

\textbf{Ricaricare configurazione:}
\begin{lstlisting}[language=bash]
$ source ~/.bashrc
\end{lstlisting}
\end{frame}

% Slide 59: crontab
\section{Automazione}

\begin{frame}[fragile]{Comando \texttt{crontab} - Job Schedulati}
\textbf{Crontab - Pianifica esecuzione automatica comandi}

\begin{lstlisting}[language=bash]
$ crontab -l               # lista cron job
$ crontab -e               # modifica cron job
$ crontab -r               # rimuovi tutti i job
$ crontab -u user -e       # edita per altro utente
\end{lstlisting}

\begin{block}{Sintassi Crontab}
\begin{center}
\texttt{minuto ora giorno mese giorno\_sett comando}
\end{center}
\begin{itemize}
    \item \texttt{*}: ogni valore
    \item \texttt{*/5}: ogni 5 unità
    \item \texttt{1,5,9}: valori specifici
    \item \texttt{1-5}: range di valori
\end{itemize}
\end{block}
\end{frame}

% Slide 60: crontab esempi
\begin{frame}[fragile]{Crontab - Esempi Pratici}

\begin{lstlisting}[language=bash]
# Ogni giorno alle 2:30 AM
30 2 * * * /path/to/backup.sh

# Ogni ora
0 * * * * /path/to/script.sh

# Ogni 15 minuti
*/15 * * * * /path/to/check.sh

# Ogni lunedì alle 9:00
0 9 * * 1 /path/to/weekly.sh

# Primo giorno del mese alle 6:00
0 6 1 * * /path/to/monthly.sh

# Ogni 12 ore
0 */12 * * * /path/to/script.sh
\end{lstlisting}

Usa: \url{https://crontab-generator.org/}
\end{frame}

% Slide 61: Utilità Sistema
\section{Comandi di Sistema}

\begin{frame}[fragile]{Comando \texttt{uname} - Info Sistema}
\textbf{Unix Name - Informazioni sul sistema}

\begin{lstlisting}[language=bash]
$ uname                    # nome kernel
$ uname -a                 # tutte le info
$ uname -s                 # nome sistema operativo
$ uname -r                 # release kernel
$ uname -m                 # architettura hardware
$ uname -p                 # tipo processore
$ uname -n                 # nome host
\end{lstlisting}

\begin{exampleblock}{Output Esempio}
\begin{lstlisting}
$ uname -a
Linux hostname 5.15.0-56 x86_64 GNU/Linux
\end{lstlisting}
\end{exampleblock}
\end{frame}

% Slide 62: history
\begin{frame}[fragile]{Comando \texttt{history} - Cronologia}
\textbf{History - Cronologia comandi eseguiti}

\begin{lstlisting}[language=bash]
$ history                  # mostra cronologia
$ history 20               # ultimi 20 comandi
$ !123                     # esegui comando numero 123
$ !!                       # ripeti ultimo comando
$ !ping                    # ultimo comando che inizia con ping
$ history -c               # pulisci cronologia
\end{lstlisting}

\begin{block}{Ricerca Incrementale}
\begin{itemize}
    \item \texttt{Ctrl+R}: ricerca nella cronologia
    \item Digita parte del comando
    \item \texttt{Ctrl+R} ancora per precedente
    \item \texttt{Enter} per eseguire
\end{itemize}
\end{block}
\end{frame}

% Slide 63: clear
\begin{frame}[fragile]{Comando \texttt{clear} - Pulire Schermo}
\textbf{Clear - Pulisce il terminale}

\begin{lstlisting}[language=bash]
$ clear
\end{lstlisting}

\textbf{Scorciatoia da tastiera:}
\begin{center}
\Large
\texttt{Ctrl + L}
\end{center}

\begin{block}{Altri Modi per "Pulire"}
\begin{lstlisting}[language=bash]
$ reset                    # reset completo terminale
$ tput clear               # alternativa a clear
\end{lstlisting}
\end{block}
\end{frame}

% Slide 64: Redirezione
\section{Redirezione e Pipe}

\begin{frame}[fragile]{Redirezione Input/Output}

\textbf{Output Redirection:}
\begin{lstlisting}[language=bash]
$ command > file.txt       # sovrascrive file
$ command >> file.txt      # aggiunge a file
$ command 2> error.log     # solo errori
$ command &> all.log       # output ed errori
$ command > /dev/null      # scarta output
\end{lstlisting}

\textbf{Input Redirection:}
\begin{lstlisting}[language=bash]
$ command < input.txt      # legge da file
$ command << EOF           # heredoc
Testo multiplo
EOF
\end{lstlisting}
\end{frame}

% Slide 65: Pipe
\begin{frame}[fragile]{Pipe - Concatenare Comandi}
\textbf{Pipe (|) - Output di un comando come input di un altro}

\begin{lstlisting}[language=bash]
$ ls -l | grep ".txt"          # filtra output
$ cat file | grep "error" | wc -l
$ ps aux | grep firefox | awk '{print $2}'
$ history | tail -20 | head -10
\end{lstlisting}

\begin{exampleblock}{Esempi Complessi}
\begin{lstlisting}[language=bash]
# Top 10 processi per memoria
$ ps aux | sort -nrk 4 | head -10

# Conta file per tipo
$ find . -type f | sed 's/.*\.//' | sort | uniq -c

# Log errors nelle ultime 100 righe
$ tail -100 app.log | grep ERROR | wc -l
\end{lstlisting}
\end{exampleblock}
\end{frame}

% Slide 66: Wildcards
\section{Wildcards e Pattern}

\begin{frame}[fragile]{Wildcards - Pattern Matching}

\begin{block}{Wildcards Comuni}
\begin{itemize}
    \item \texttt{*}: qualsiasi sequenza di caratteri
    \item \texttt{?}: un singolo carattere
    \item \texttt{[abc]}: uno tra a, b, o c
    \item \texttt{[a-z]}: range di caratteri
    \item \texttt{[!abc]}: qualsiasi tranne a, b, c
\end{itemize}
\end{block}

\begin{lstlisting}[language=bash]
$ ls *.txt                 # tutti i file .txt
$ ls file?.txt             # file1.txt, fileA.txt, etc.
$ ls [abc]*.txt            # iniziano con a, b, o c
$ rm temp*                 # rimuove tutti temp...
$ cp project[1-5].doc backup/
\end{lstlisting}
\end{frame}

% Slide 67: Espansioni
\begin{frame}[fragile]{Espansioni della Shell}

\textbf{Brace Expansion:}
\begin{lstlisting}[language=bash]
$ echo {1..10}             # 1 2 3 ... 10
$ echo file{1,2,3}.txt     # file1.txt file2.txt file3.txt
$ mkdir -p dir/{sub1,sub2,sub3}
$ echo {A..Z}              # A B C ... Z
\end{lstlisting}

\textbf{Command Substitution:}
\begin{lstlisting}[language=bash]
$ echo "Oggi è $(date)"
$ files=$(ls *.txt)
$ current_user=`whoami`    # vecchia sintassi
\end{lstlisting}

\textbf{Tilde Expansion:}
\begin{lstlisting}[language=bash]
$ cd ~/documents           # $HOME/documents
\end{lstlisting}
\end{frame}

% Slide 68: Operatori Logici
\begin{frame}[fragile]{Operatori Logici}

\begin{lstlisting}[language=bash]
# AND - esegui secondo solo se primo ha successo
$ mkdir newdir && cd newdir

# OR - esegui secondo solo se primo fallisce
$ command1 || command2

# Sequential - esegui entrambi indipendentemente
$ command1 ; command2

# Background
$ long_task &
\end{lstlisting}

\begin{exampleblock}{Esempi Pratici}
\begin{lstlisting}[language=bash]
$ make && make install
$ ping -c 1 google.com && echo "Online" || echo "Offline"
$ cd /tmp && rm -f tempfile ; cd -
\end{lstlisting}
\end{exampleblock}
\end{frame}

% Slide 69: Script Bash
\section{Script Bash Basics}

\begin{frame}[fragile]{Creare Script Bash}

\begin{lstlisting}[language=bash]
#!/bin/bash
# Questo è un commento

echo "Hello World!"
echo "Utente: $USER"
echo "Home: $HOME"

# Variabili
NOME="Mario"
echo "Ciao $NOME"

# Parametri
echo "Primo argomento: $1"
echo "Tutti gli argomenti: $@"
echo "Numero argomenti: $#"
\end{lstlisting}

\textbf{Rendere eseguibile:}
\begin{lstlisting}[language=bash]
$ chmod +x script.sh
$ ./script.sh arg1 arg2
\end{lstlisting}
\end{frame}

% Slide 70: Script - Condizioni
\begin{frame}[fragile]{Script Bash - Condizioni e Loop}

\textbf{If Statement:}
\begin{lstlisting}[language=bash]
if [ -f "file.txt" ]; then
    echo "File esiste"
else
    echo "File non trovato"
fi
\end{lstlisting}

\textbf{For Loop:}
\begin{lstlisting}[language=bash]
for i in {1..5}; do
    echo "Iterazione $i"
done

for file in *.txt; do
    echo "Processing $file"
done
\end{lstlisting}
\end{frame}

% Slide 71: Best Practices
\section{Best Practices}

\begin{frame}{Best Practices - Sicurezza}

\begin{alertblock}{Non Usare mai:}
\begin{itemize}
    \item \texttt{rm -rf /} (cancella tutto!)
    \item \texttt{chmod 777} su file sensibili
    \item Eseguire script da fonti non fidate
    \item Usare \texttt{sudo} senza capire il comando
\end{itemize}
\end{alertblock}

\begin{block}{Raccomandazioni}
\begin{itemize}
    \item Leggere \texttt{man} prima di usare comandi nuovi
    \item Fare backup prima di operazioni distruttive
    \item Usare \texttt{-i} (interactive) con \texttt{rm}, \texttt{mv}, \texttt{cp}
    \item Verificare con \texttt{ls} prima di \texttt{rm *}
    \item Non loggarsi come root se non necessario
\end{itemize}
\end{block}
\end{frame}

% Slide 72: Best Practices - Produttività
\begin{frame}[fragile]{Best Practices - Produttività}

\begin{block}{Alias Utili}
\begin{lstlisting}[language=bash]
alias ll='ls -alh'
alias ..='cd ..'
alias ...='cd ../..'
alias grep='grep --color=auto'
alias df='df -h'
alias du='du -h'
\end{lstlisting}
\end{block}

\begin{block}{Consigli}
\begin{itemize}
    \item Usa \texttt{Tab} per autocompletamento
    \item Usa \texttt{Ctrl+R} per cercare nella cronologia
    \item Impara le scorciatoie da tastiera
    \item Crea script per task ripetitivi
    \item Documenta i tuoi script
\end{itemize}
\end{block}
\end{frame}

% Slide 73: Scorciatoie Tastiera
\begin{frame}{Scorciatoie da Tastiera Essenziali}

\begin{columns}
\column{0.5\textwidth}
\textbf{Controllo Processi:}
\begin{itemize}
    \item \texttt{Ctrl+C}: interrompi processo
    \item \texttt{Ctrl+Z}: sospendi processo
    \item \texttt{Ctrl+D}: EOF / logout
\end{itemize}

\textbf{Navigazione:}
\begin{itemize}
    \item \texttt{Ctrl+A}: inizio riga
    \item \texttt{Ctrl+E}: fine riga
    \item \texttt{Ctrl+U}: cancella fino a inizio
    \item \texttt{Ctrl+K}: cancella fino a fine
\end{itemize}

\column{0.5\textwidth}
\textbf{Editing:}
\begin{itemize}
    \item \texttt{Ctrl+W}: cancella parola
    \item \texttt{Ctrl+L}: clear screen
    \item \texttt{Ctrl+R}: ricerca cronologia
    \item \texttt{Tab}: autocompletamento
\end{itemize}

\textbf{Altro:}
\begin{itemize}
    \item \texttt{!!}: ripeti ultimo comando
    \item \texttt{!}: ultimo argomento
    \item \texttt{!*}: tutti gli argomenti
\end{itemize}
\end{columns}
\end{frame}

% Slide 74: Risorse
\begin{frame}{Risorse per Approfondire}

\begin{block}{Documentazione}
\begin{itemize}
    \item \texttt{man} pages: documentazione locale
    \item \texttt{info} pages: documentazione GNU
    \item \texttt{help comando}: help builtin shell
    \item \texttt{comando --help}: opzioni comando
\end{itemize}
\end{block}

\begin{block}{Online}
\begin{itemize}
    \item \url{https://tldr.sh/} - Esempi pratici comandi
    \item \url{https://explainshell.com/} - Spiega comandi
    \item \url{https://www.gnu.org/software/bash/manual/} - Manuale Bash
    \item \url{https://stackoverflow.com/} - Q\&A community
\end{itemize}
\end{block}
\end{frame}

% Slide 75: Riepilogo Comandi
\begin{frame}{Riepilogo - Comandi Essenziali}

\begin{columns}
\column{0.5\textwidth}
\textbf{Navigazione:}
\begin{itemize}
    \item \texttt{ls, cd, pwd}
    \item \texttt{mkdir, rmdir}
    \item \texttt{find}
\end{itemize}

\textbf{File:}
\begin{itemize}
    \item \texttt{cp, mv, rm}
    \item \texttt{touch, cat, less}
    \item \texttt{grep, diff}
\end{itemize}

\textbf{Permessi:}
\begin{itemize}
    \item \texttt{chmod, chown}
    \item \texttt{sudo, su}
\end{itemize}

\column{0.5\textwidth}
\textbf{Processi:}
\begin{itemize}
    \item \texttt{ps, top, kill}
    \item \texttt{jobs, bg, fg}
\end{itemize}

\textbf{Sistema:}
\begin{itemize}
    \item \texttt{df, du}
    \item \texttt{uname, history}
\end{itemize}

\textbf{Rete:}
\begin{itemize}
    \item \texttt{ping, traceroute}
\end{itemize}

\textbf{Utility:}
\begin{itemize}
    \item \texttt{tar, gzip}
    \item \texttt{export, env}
\end{itemize}
\end{columns}
\end{frame}

% Slide 76: Conclusioni
\begin{frame}{Conclusioni}

\begin{block}{Punti Chiave}
\begin{itemize}
    \item Linux è potente, flessibile e libero
    \item La shell è più efficiente di una GUI per molte operazioni
    \item Regola 80/20: pochi comandi coprono la maggior parte dei casi d'uso
    \item \texttt{man} è il tuo migliore amico
    \item Pratica costante è la chiave dell'apprendimento
\end{itemize}
\end{block}

\begin{center}
\Large
\textbf{La linea di comando è uno strumento\\
che richiede tempo per essere padroneggiato,\\
ma ne vale assolutamente la pena!}
\end{center}
\end{frame}

% Slide 77: Grazie
\begin{frame}[plain]
\begin{center}
{\Huge Grazie per l'attenzione!}

\vspace{1cm}

{\Large Domande?}

\vspace{1.5cm}

\begin{tikzpicture}
    \draw[fill=blue!20, rounded corners] (-4,-1.5) rectangle (4,1.5);
    \node at (0,0.5) {\Large\textbf{Basato su:}};
    \node at (0,-0.3) {"The Linux Commands Handbook"};
    \node at (0,-0.8) {di Flavio Copes};
\end{tikzpicture}

\vspace{1cm}

{\small Presentazione creata per IIS Fermi Sacconi Ceci}
\end{center}
\end{frame}

\end{document}
