\documentclass[aspectratio=169]{beamer}
\usepackage[utf8]{inputenc}
\usepackage[italian]{babel}
\usepackage{tikz}
\usepackage{amsmath}
\usepackage{listings}
\usepackage{xcolor}

\usetikzlibrary{shapes.geometric, arrows, positioning, calc}

\usetheme{Madrid}
\usecolortheme{default}

\title{Triple DES (3-DES)}
\subtitle{Cifratura a Blocchi e Sicurezza delle Informazioni}
\author{Prof. Fedeli Massimo - Tutti i diritti riservati}
\institute{IIS Fermi Sacconi Cpia - Ascoli Piceno}
\date{\today}

% Definizione stili TikZ
\tikzstyle{block} = [rectangle, draw, fill=blue!20, text width=6em, text centered, rounded corners, minimum height=3em]
\tikzstyle{arrow} = [thick,->,>=stealth]
\tikzstyle{key} = [rectangle, draw, fill=red!20, text width=4em, text centered, rounded corners, minimum height=2em]

\begin{document}

% Slide 1: Titolo
\frame{\titlepage}

% Slide 2: Indice
\begin{frame}{Indice}
\tableofcontents
\end{frame}

% Sezione 1: Introduzione
\section{Introduzione alla Crittografia}

\begin{frame}{Cos'è la Crittografia?}
\begin{block}{Definizione}
La \textbf{crittografia} è la scienza che protegge le informazioni trasformandole in modo che solo chi possiede la "chiave" corretta possa leggerle.
\end{block}

\vspace{0.5cm}

\begin{columns}
\column{0.5\textwidth}
\textbf{Testo in chiaro:}
\begin{center}
\Large CIAO MONDO
\end{center}

\column{0.5\textwidth}
\textbf{Testo cifrato:}
\begin{center}
\Large X9K2 QW8RT5
\end{center}
\end{columns}

\vspace{0.5cm}
\begin{alertblock}{Perché è importante?}
\begin{itemize}
\item Protezione dei dati personali
\item Sicurezza nelle transazioni online
\item Protezione della privacy
\end{itemize}
\end{alertblock}
\end{frame}

\begin{frame}{Cifratura Simmetrica vs Asimmetrica}
\begin{columns}
\column{0.5\textwidth}
\begin{block}{Cifratura Simmetrica}
\begin{itemize}
\item \textbf{Una sola chiave} per cifrare e decifrare
\item Più veloce
\item Chiave segreta condivisa
\item Es: DES, AES, 3-DES
\end{itemize}
\end{block}

\column{0.5\textwidth}
\begin{block}{Cifratura Asimmetrica}
\begin{itemize}
\item \textbf{Due chiavi}: pubblica e privata
\item Più lenta
\item Chiave pubblica distribuita
\item Es: RSA, ECC
\end{itemize}
\end{block}
\end{columns}

\vspace{0.5cm}
\begin{center}
\textbf{3-DES è un algoritmo di cifratura SIMMETRICA}
\end{center}
\end{frame}

% Sezione 2: Storia di DES
\section{Da DES a 3-DES}

\begin{frame}{La Storia di DES}
\begin{block}{Data Encryption Standard (DES)}
\begin{itemize}
\item Sviluppato da IBM negli anni '70
\item Adottato come standard dal governo USA nel 1977
\item Chiave di \textbf{56 bit} (effettivi, 64 bit totali con parità)
\item Blocchi di \textbf{64 bit}
\end{itemize}
\end{block}

\vspace{0.3cm}

\begin{alertblock}{Il Problema di DES}
Con l'aumento della potenza di calcolo dei computer, la chiave a 56 bit è diventata troppo \textbf{corta} e vulnerabile agli attacchi \textit{brute force}.
\end{alertblock}

\vspace{0.3cm}

\begin{exampleblock}{Soluzione}
Nel 1998 è stato dimostrato che DES poteva essere violato in meno di 3 giorni! \\
\textbf{Necessaria una soluzione più sicura} $\rightarrow$ 3-DES
\end{exampleblock}
\end{frame}

\begin{frame}{Perché Triple DES?}
\begin{columns}
\column{0.5\textwidth}
\textbf{Problema:}
\begin{itemize}
\item DES troppo debole (56 bit)
\item Moltissimi sistemi già usavano DES
\item Costoso cambiare tutto l'hardware
\end{itemize}

\vspace{0.5cm}

\textbf{Soluzione Intelligente:}
\begin{itemize}
\item Applicare DES \textbf{tre volte}
\item Usare chiavi diverse
\item Compatibilità con hardware esistente
\end{itemize}

\column{0.5\textwidth}
\begin{center}
\begin{tikzpicture}[scale=0.8]
\node[block] (des1) at (0,0) {DES};
\node[block] (des2) at (0,-2) {DES};
\node[block] (des3) at (0,-4) {DES};

\node[key] (k1) at (-2.5,-0) {K1};
\node[key] (k2) at (-2.5,-2) {K2};
\node[key] (k3) at (-2.5,-4) {K3};

\draw[arrow] (0,1) -- node[right] {Testo chiaro} (des1);
\draw[arrow] (des1) -- (des2);
\draw[arrow] (des2) -- (des3);
\draw[arrow] (des3) -- node[right] {Cifrato} (0,-5);

\draw[arrow] (k1) -- (des1);
\draw[arrow] (k2) -- (des2);
\draw[arrow] (k3) -- (des3);
\end{tikzpicture}
\end{center}
\end{columns}
\end{frame}

% Sezione 3: Come Funziona 3-DES
\section{Funzionamento di 3-DES}

\begin{frame}{Schema EDE: Encrypt-Decrypt-Encrypt}
\begin{block}{Il Metodo Standard di 3-DES}
3-DES utilizza lo schema \textbf{EDE} (Encrypt-Decrypt-Encrypt):
\begin{enumerate}
\item \textbf{Encrypt} con chiave K1
\item \textbf{Decrypt} con chiave K2
\item \textbf{Encrypt} con chiave K3
\end{enumerate}
\end{block}

\vspace{0.3cm}

\begin{center}
\begin{tikzpicture}[node distance=2.5cm]
\node[block] (input) {Testo in Chiaro};
\node[block, right of=input] (e1) {Cifratura DES};
\node[block, right of=e1] (d2) {Decifratura DES};
\node[block, right of=d2] (e3) {Cifratura DES};
\node[block, right of=e3] (output) {Testo Cifrato};

\draw[arrow] (input) -- (e1);
\draw[arrow] (e1) -- (d2);
\draw[arrow] (d2) -- (e3);
\draw[arrow] (e3) -- (output);

\node[key, above of=e1, node distance=1.2cm] (k1) {K1};
\node[key, above of=d2, node distance=1.2cm] (k2) {K2};
\node[key, above of=e3, node distance=1.2cm] (k3) {K3};

\draw[arrow] (k1) -- (e1);
\draw[arrow] (k2) -- (d2);
\draw[arrow] (k3) -- (e3);
\end{tikzpicture}
\end{center}

\vspace{0.3cm}
\begin{alertblock}{Perché Decrypt al centro?}
Permette la \textbf{retrocompatibilità} con DES: se K1 = K2 = K3, otteniamo il DES originale!
\end{alertblock}
\end{frame}

\begin{frame}{Le Tre Opzioni di Chiavi}
3-DES può essere configurato in tre modi diversi:

\vspace{0.5cm}

\begin{block}{Opzione 1: Tre chiavi distinte (3TDES - Keying Option 1)}
\textbf{K1 $\neq$ K2 $\neq$ K3} \\
Sicurezza: 168 bit (3 × 56) \\
\textcolor{green!60!black}{\textbf{Più sicuro}} ma meno usato
\end{block}

\begin{block}{Opzione 2: Due chiavi distinte (3TDES - Keying Option 2)}
\textbf{K1 = K3 $\neq$ K2} \\
Sicurezza: 112 bit (2 × 56) \\
\textcolor{blue}{\textbf{Più comune}} - buon compromesso sicurezza/efficienza
\end{block}

\begin{block}{Opzione 3: Una chiave (retrocompatibilità)}
\textbf{K1 = K2 = K3} \\
Sicurezza: 56 bit \\
\textcolor{red}{\textbf{Equivalente a DES}} - solo per compatibilità legacy
\end{block}
\end{frame}

\begin{frame}{Esempio Pratico: Opzione 2}
\begin{exampleblock}{Scenario Reale (Opzione 2: K1 = K3)}
\begin{enumerate}
\item \textbf{Chiavi:}
\begin{itemize}
\item K1 = K3 = \texttt{0x133457799BBCDFF1}
\item K2 = \texttt{0x0E329232EA6D0D73}
\end{itemize}

\vspace{0.3cm}

\item \textbf{Testo in chiaro:}
\begin{center}
\texttt{"HELLO\_\_\_"} (64 bit = 8 caratteri)
\end{center}

\vspace{0.3cm}

\item \textbf{Processo:}
\begin{itemize}
\item Cifratura DES con K1 $\rightarrow$ \texttt{X1}
\item Decifratura DES con K2 su X1 $\rightarrow$ \texttt{X2}
\item Cifratura DES con K3 su X2 $\rightarrow$ \textbf{Testo Cifrato}
\end{itemize}
\end{enumerate}
\end{exampleblock}

\begin{center}
\textit{Il processo inverso permette di recuperare il testo originale}
\end{center}
\end{frame}

\begin{frame}{Decifratura con 3-DES}
\begin{block}{Il Processo Inverso}
Per decifrare, si inverte l'ordine e si scambiano cifratura con decifratura:
\begin{enumerate}
\item \textbf{Decrypt} con chiave K3
\item \textbf{Encrypt} con chiave K2
\item \textbf{Decrypt} con chiave K1
\end{enumerate}
\end{block}

\vspace{0.3cm}

\begin{center}
\begin{tikzpicture}[node distance=2.5cm]
\node[block] (input) {Testo Cifrato};
\node[block, right of=input] (d3) {Decifratura DES};
\node[block, right of=d3] (e2) {Cifratura DES};
\node[block, right of=e2] (d1) {Decifratura DES};
\node[block, right of=d1] (output) {Testo in Chiaro};

\draw[arrow] (input) -- (d3);
\draw[arrow] (d3) -- (e2);
\draw[arrow] (e2) -- (d1);
\draw[arrow] (d1) -- (output);

\node[key, above of=d3, node distance=1.2cm] (k3) {K3};
\node[key, above of=e2, node distance=1.2cm] (k2) {K2};
\node[key, above of=d1, node distance=1.2cm] (k1) {K1};

\draw[arrow] (k3) -- (d3);
\draw[arrow] (k2) -- (e2);
\draw[arrow] (k1) -- (d1);
\end{tikzpicture}
\end{center}
\end{frame}

% Sezione 4: Sicurezza
\section{Sicurezza e Vulnerabilità}

\begin{frame}{Quanto è Sicuro 3-DES?}
\begin{columns}
\column{0.5\textwidth}
\begin{block}{Punti di Forza}
\begin{itemize}
\item Chiavi lunghe: fino a 168 bit
\item Testato per decenni
\item Nessun attacco pratico conosciuto
\item Ampiamente certificato
\end{itemize}
\end{block}

\begin{alertblock}{Attacco Meet-in-the-Middle}
Riduce la sicurezza effettiva:
\begin{itemize}
\item 3 chiavi: da 168 a \textbf{112 bit}
\item 2 chiavi: da 112 a \textbf{80 bit}
\end{itemize}
\end{alertblock}

\column{0.5\textwidth}
\begin{block}{Limitazioni}
\begin{itemize}
\item Più lento di AES (3x operazioni)
\item Blocchi piccoli (64 bit)
\item Birthday attack dopo $2^{32}$ blocchi
\item Deprecato dal NIST (2023)
\end{itemize}
\end{block}

\begin{exampleblock}{Tempo per Violazione}
Con tecnologia attuale:
\begin{itemize}
\item 112 bit: \textbf{milioni di anni}
\item 80 bit: ancora \textbf{impraticabile}
\end{itemize}
\end{exampleblock}
\end{columns}
\end{frame}

\begin{frame}{Confronto: 3-DES vs AES}
\begin{table}
\centering
\begin{tabular}{|l|c|c|}
\hline
\textbf{Caratteristica} & \textbf{3-DES} & \textbf{AES} \\
\hline
\hline
Anno di sviluppo & 1978/1998 & 2001 \\
\hline
Dimensione chiave & 112-168 bit & 128/192/256 bit \\
\hline
Dimensione blocco & 64 bit & 128 bit \\
\hline
Velocità & Lenta (3× DES) & Veloce \\
\hline
Sicurezza & Buona ma limitata & Eccellente \\
\hline
Hardware richiesto & Maggiore & Minore \\
\hline
Status & \textcolor{red}{Deprecato (2023)} & \textcolor{green!60!black}{Standard attuale} \\
\hline
Uso consigliato & Solo legacy & \textcolor{green!60!black}{Raccomandato} \\
\hline
\end{tabular}
\end{table}

\vspace{0.5cm}
\begin{alertblock}{Raccomandazione}
Per nuovi sistemi: usare \textbf{AES}! \\
3-DES solo per mantenere compatibilità con sistemi esistenti.
\end{alertblock}
\end{frame}

% Sezione 5: Applicazioni
\section{Applicazioni Pratiche}

\begin{frame}{Dove si Usa 3-DES?}
\begin{columns}
\column{0.5\textwidth}
\begin{block}{Sistemi Finanziari}
\begin{itemize}
\item \textbf{Carte di credito/debito}
\item Bancomat (ATM)
\item Transazioni POS
\item Standard EMV (chip)
\item Bonifici bancari
\end{itemize}
\end{block}

\begin{block}{Telecomunicazioni}
\begin{itemize}
\item Protocolli VPN legacy
\item Crittografia voce
\item Reti private
\end{itemize}
\end{block}

\column{0.5\textwidth}
\begin{block}{Compatibilità Legacy}
\begin{itemize}
\item Sistemi governativi vecchi
\item Hardware industriale
\item Protocolli di sicurezza esistenti
\item Database cifrati
\end{itemize}
\end{block}

\begin{alertblock}{Transizione in Corso}
Molti sistemi stanno migrando da 3-DES ad AES per:
\begin{itemize}
\item Maggiore velocità
\item Migliore sicurezza
\item Minor consumo energetico
\end{itemize}
\end{alertblock}
\end{columns}
\end{frame}

\begin{frame}{Esempio: Pagamento con Carta}
\begin{exampleblock}{Il Processo di una Transazione}
\begin{enumerate}
\item Inserisci la carta nel POS
\item Il chip genera un codice crittografico con \textbf{3-DES}
\item Il codice viene inviato alla banca
\item La banca verifica usando la stessa chiave
\item Transazione autorizzata o rifiutata
\end{enumerate}
\end{exampleblock}

\vspace{0.3cm}

\begin{center}
\begin{tikzpicture}[node distance=2cm]
\node[block, fill=yellow!30] (carta) {Carta Chip};
\node[block, fill=blue!30, right of=carta, node distance=3cm] (pos) {POS};
\node[block, fill=green!30, right of=pos, node distance=3cm] (banca) {Banca};

\draw[arrow] (carta) -- node[above] {Dati + 3-DES} (pos);
\draw[arrow] (pos) -- node[above] {Cifrato} (banca);
\draw[arrow] (banca) -- node[below] {OK/NO} (pos);

\node[key, below of=carta] (k) {Chiave Segreta};
\node[key, below of=banca] (k2) {Stessa Chiave};

\draw[dashed] (k) -- (carta);
\draw[dashed] (k2) -- (banca);
\end{tikzpicture}
\end{center}

\vspace{0.3cm}
\small{\textit{La chiave segreta non viaggia mai sulla rete!}}
\end{frame}

% Sezione 6: Il Futuro
\section{Il Futuro della Crittografia}

\begin{frame}{La Fine di 3-DES}
\begin{block}{Deprecazione Ufficiale}
\begin{itemize}
\item \textbf{2017}: NIST annuncia la dismissione
\item \textbf{2023}: Fine del supporto ufficiale
\item \textbf{2024-2030}: Periodo di transizione
\end{itemize}
\end{block}

\vspace{0.3cm}

\begin{columns}
\column{0.5\textwidth}
\begin{alertblock}{Perché?}
\begin{itemize}
\item Blocchi troppo piccoli (64 bit)
\item Vulnerabilità Sweet32
\item Prestazioni inadeguate
\item Alternative migliori (AES)
\end{itemize}
\end{alertblock}

\column{0.5\textwidth}
\begin{exampleblock}{Migrazione}
\textbf{Passaggio ad AES:}
\begin{itemize}
\item AES-128: standard
\item AES-256: alta sicurezza
\item Hardware moderno
\item Supporto software
\end{itemize}
\end{exampleblock}
\end{columns}

\vspace{0.3cm}
\begin{center}
\Large{\textbf{3-DES: 45 anni di servizio onorevole!}}
\end{center}
\end{frame}

\begin{frame}{Conclusioni}
\begin{block}{Cosa Abbiamo Imparato}
\begin{itemize}
\item 3-DES è una \textbf{evoluzione} di DES per aumentare la sicurezza
\item Usa tre applicazioni di DES con chiavi diverse (schema EDE)
\item Offre sicurezza tra 80 e 112 bit (effettivi)
\item È stato fondamentale per la sicurezza delle transazioni finanziarie
\item Oggi è in fase di dismissione a favore di AES
\end{itemize}
\end{block}

\vspace{0.3cm}

\begin{exampleblock}{Lezione Importante}
La crittografia è un campo in \textbf{continua evoluzione}:
\begin{itemize}
\item Ciò che è sicuro oggi potrebbe non esserlo domani
\item È necessario aggiornare costantemente i sistemi
\item La sicurezza richiede attenzione e manutenzione
\end{itemize}
\end{exampleblock}
\end{frame}

\begin{frame}{Domande per Riflettere}
\begin{enumerate}
\item Perché non si è semplicemente raddoppiata la lunghezza della chiave di DES invece di applicarlo tre volte?

\vspace{0.5cm}

\item Se 3-DES è più sicuro, perché si preferisce AES per i nuovi sistemi?

\vspace{0.5cm}

\item Cosa succederebbe se qualcuno scoprisse la tua chiave 3-DES?

\vspace{0.5cm}

\item Come pensi che la crittografia dovrà evolversi con l'arrivo dei computer quantistici?
\end{enumerate}

\vspace{0.5cm}
\begin{center}
\Large{\textbf{Grazie per l'attenzione!}}
\end{center}
\end{frame}

% Bibliografia e Risorse
\begin{frame}{Risorse per Approfondire}
\begin{block}{Standard e Documentazione}
\begin{itemize}
\item NIST Special Publication 800-67: Recommendation for Triple DES
\item ISO/IEC 18033-3: Encryption algorithms
\end{itemize}
\end{block}

\begin{block}{Libri Consigliati}
\begin{itemize}
\item "Applied Cryptography" - Bruce Schneier
\item "Cryptography and Network Security" - William Stallings
\end{itemize}
\end{block}

\begin{block}{Risorse Online}
\begin{itemize}
\item Khan Academy - Cryptography Course
\item Coursera - Cryptography (Stanford University)
\item CryptoHack - Esercizi pratici interattivi
\end{itemize}
\end{block}
\end{frame}

\end{document}
