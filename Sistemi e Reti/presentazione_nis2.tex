\documentclass[aspectratio=169]{beamer}
\usepackage[utf8]{inputenc}
\usepackage[italian]{babel}
\usepackage{graphicx}
\usepackage{tikz}
\usepackage{pgfplots}
\usepackage{multicol}
\usepackage{xcolor}

\usetikzlibrary{shapes,arrows,positioning,shadows,calc}

\usetheme{Madrid}
\usecolortheme{default}

% Definizione colori personalizzati
\definecolor{darkblue}{RGB}{0,51,102}
\definecolor{lightblue}{RGB}{51,153,255}
\definecolor{alertred}{RGB}{204,0,0}
\definecolor{successgreen}{RGB}{0,153,51}

\setbeamercolor{structure}{fg=darkblue}
\setbeamercolor{palette primary}{bg=darkblue,fg=white}
\setbeamercolor{palette secondary}{bg=lightblue,fg=white}
\setbeamercolor{palette tertiary}{bg=darkblue,fg=white}

\title[Cybersecurity e NIS 2]{Sicurezza Informatica e Direttiva NIS 2}
\subtitle{Minacce Cyber e Nuova Normativa Europea}
\author{Prof. Fedeli Massimo - IIS Fermi Sacconi Cpia}
\institute{Ascoli Piceno}
\date{\today}

\begin{document}

% Slide 1 - Titolo
\begin{frame}
\titlepage
\end{frame}

% Slide 2 - Indice
\begin{frame}{Indice}
\tableofcontents
\end{frame}

\section{Il Panorama delle Minacce Informatiche}

% Slide 3 - Introduzione alle minacce
\begin{frame}{Il Panorama delle Minacce Informatiche}
\begin{columns}[T]
\column{0.5\textwidth}
\textbf{Scenario Attuale:}
\begin{itemize}
\item Crescita esponenziale degli attacchi
\item Maggiore complessità e sofisticazione
\item Impatto su aziende, PA e infrastrutture critiche
\item Evoluzione continua delle tecniche di attacco
\end{itemize}

\column{0.5\textwidth}
\begin{tikzpicture}[scale=0.8]
\begin{axis}[
    ybar,
    width=6cm,
    height=5cm,
    ylabel={Numero attacchi},
    symbolic x coords={2020,2021,2022,2023,2024},
    xtick=data,
    ymin=0,
    enlarge x limits=0.2,
    bar width=15pt,
    nodes near coords,
]
\addplot[fill=alertred] coordinates {(2020,100) (2021,145) (2022,198) (2023,265) (2024,340)};
\end{axis}
\end{tikzpicture}
\end{columns}
\end{frame}

% Slide 4 - Principali tipologie di attacchi
\begin{frame}{Principali Tipologie di Attacchi}
\centering

\begin{tikzpicture}[scale=0.9]
% Nodo centrale
\node[circle, draw=darkblue, fill=lightblue!30, minimum size=2cm, font=\Large\bfseries] (center) at (0,0) {Minacce\\Cyber};

% Nodi attacchi
\node[rectangle, draw=alertred, fill=alertred!20, minimum width=2.5cm, minimum height=1cm, rounded corners] (ransomware) at (-4,2) {\textbf{Ransomware}};
\node[rectangle, draw=alertred, fill=alertred!20, minimum width=2.5cm, minimum height=1cm, rounded corners] (phishing) at (4,2) {\textbf{Phishing}};
\node[rectangle, draw=alertred, fill=alertred!20, minimum width=2.5cm, minimum height=1cm, rounded corners] (ddos) at (-4,-2) {\textbf{DDoS}};
\node[rectangle, draw=alertred, fill=alertred!20, minimum width=2.5cm, minimum height=1cm, rounded corners] (supply) at (4,-2) {\textbf{Supply Chain}};

% Collegamenti
\draw[->, thick, darkblue] (center) -- (ransomware);
\draw[->, thick, darkblue] (center) -- (phishing);
\draw[->, thick, darkblue] (center) -- (ddos);
\draw[->, thick, darkblue] (center) -- (supply);
\end{tikzpicture}
\end{frame}

\section{Ransomware}

% Slide 5 - Ransomware
\begin{frame}{Ransomware: Il Ricatto Digitale}
\begin{columns}[T]
\column{0.5\textwidth}
\textbf{Cos'è:}
\begin{itemize}
\item Malware che cifra i dati
\item Rende inaccessibili file e sistemi
\item Richiesta di riscatto in criptovaluta
\item Minaccia di pubblicazione dati
\end{itemize}

\vspace{0.5cm}
\textbf{Caso emblematico:}\\
\textcolor{alertred}{Sistema Sanitario Irlandese (2021)}
\begin{itemize}
\item Ospedali e ambulatori bloccati
\item Sospensione servizi per giorni
\item Impatto su migliaia di pazienti
\end{itemize}

\column{0.5\textwidth}
\begin{tikzpicture}[scale=0.85]
% Computer
\node[rectangle, draw=black, fill=gray!30, minimum width=3cm, minimum height=2cm] (pc) at (0,0) {};
\node at (0,0.5) {$\bigstar$};
\node[font=\small] at (0,-0.3) {File Cifrati};

% Lucchetto
\node[font=\Huge, text=alertred] at (0,-1.5) {$\blacksquare$};

% Hacker
\node[rectangle, draw=alertred, fill=alertred!20, minimum width=2.5cm, minimum height=1.5cm, rounded corners] (hacker) at (0,-4) {
\begin{minipage}{2.2cm}
\centering

$\clubsuit$\\
\small Richiesta\\
\small Riscatto
\end{minipage}
};

% Freccia
\draw[->, very thick, alertred] (0,-2.2) -- (0,-3.2);
\end{tikzpicture}
\end{columns}
\end{frame}

% Slide 6 - Meccanismo Ransomware
\begin{frame}{Come Funziona un Attacco Ransomware}
\centering

\begin{tikzpicture}[
    node distance=1.5cm,
    every node/.style={font=\small},
    block/.style={rectangle, draw=darkblue, fill=lightblue!30, text width=3cm, text centered, rounded corners, minimum height=1cm},
    alert/.style={rectangle, draw=alertred, fill=alertred!20, text width=3cm, text centered, rounded corners, minimum height=1cm}
]
\node[block] (1) {\textbf{1. Infezione}\\Email, download, vulnerabilità};
\node[block, right=of 1] (2) {\textbf{2. Propagazione}\\Diffusione nella rete};
\node[alert, right=of 2] (3) {\textbf{3. Cifratura}\\Dati inaccessibili};
\node[alert, below=of 2] (4) {\textbf{4. Richiesta}\\Pagamento riscatto};
\node[block, left=of 4] (5) {\textbf{5. Decisione}\\Pagare o ripristinare};

\draw[->, very thick] (1) -- (2);
\draw[->, very thick] (2) -- (3);
\draw[->, very thick] (3) -- (4);
\draw[->, very thick] (4) -- (5);
\end{tikzpicture}

\vspace{0.5cm}
\begin{alertblock}{Nota Importante}
Non esiste garanzia che pagando si ottengano i dati. È fondamentale avere backup offline aggiornati!
\end{alertblock}
\end{frame}

\section{Phishing}

% Slide 7 - Phishing
\begin{frame}{Phishing: L'Inganno via Email}
\begin{columns}[T]
\column{0.5\textwidth}
\textbf{Definizione:}\\
Email fraudolente che imitano comunicazioni ufficiali per rubare credenziali e dati sensibili.

\vspace{0.5cm}
\textbf{Obiettivi:}
\begin{itemize}
\item Password e credenziali
\item Numeri carte di credito
\item Dati personali sensibili
\item Accesso a sistemi aziendali
\end{itemize}

\vspace{0.5cm}
\textbf{Caratteristiche comuni:}
\begin{itemize}
\item Senso di urgenza
\item Mittente apparentemente legittimo
\item Link a siti contraffatti
\item Richiesta dati riservati
\end{itemize}

\column{0.5\textwidth}
\begin{tikzpicture}[scale=0.8]
% Email fraudolenta
\node[rectangle, draw=alertred, fill=white, minimum width=4cm, minimum height=5cm, line width=2pt] (email) at (0,0) {};
\node[font=\small] at (0,2) {$\triangleright$~\textbf{Aggiorna Password}};
\node[font=\tiny, text width=3.5cm, align=left] at (0,1) {
Da: it-support@azienda.com\\
\vspace{0.2cm}
Gentile utente,\\
\vspace{0.1cm}
La sua password scadrà tra 24 ore. Clicchi qui per aggiornarla...
};

\node[rectangle, draw=alertred, fill=alertred!30, minimum width=2cm, minimum height=0.6cm, rounded corners] at (0,-0.8) {\textbf{CLICCA QUI}};

% Freccia verso sito fake
\draw[->, very thick, alertred] (0,-1.5) -- (0,-2.5);
\node[rectangle, draw=black, fill=gray!20, minimum width=3.5cm, minimum height=1cm] at (0,-3.3) {
\begin{minipage}{3cm}
\centering

\tiny Sito fake\\
\tiny Furto credenziali
\end{minipage}
};
\end{tikzpicture}
\end{columns}
\end{frame}

% Slide 8 - Come riconoscere il Phishing
\begin{frame}{Come Riconoscere un Tentativo di Phishing}
\begin{columns}[T]
\column{0.5\textwidth}
\textbf{\textcolor{alertred}{Segnali di allarme:}}
\begin{itemize}
\item[$\blacktriangledown$] Errori grammaticali o ortografici
\item[$\blacktriangledown$] Indirizzo email sospetto
\item[$\blacktriangledown$] Richieste urgenti
\item[$\blacktriangledown$] Link con URL strani
\item[$\blacktriangledown$] Allegati inaspettati
\item[$\blacktriangledown$] Richieste di dati sensibili
\end{itemize}

\column{0.5\textwidth}
\textbf{\textcolor{successgreen}{Buone pratiche:}}
\begin{itemize}
\item[$\checkmark$] Verificare sempre il mittente
\item[$\checkmark$] Controllare l'URL prima di cliccare
\item[$\checkmark$] Non fornire mai password via email
\item[$\checkmark$] Usare autenticazione a due fattori
\item[$\checkmark$] Contattare direttamente l'ente
\item[$\checkmark$] Segnalare email sospette
\end{itemize}
\end{columns}

\vspace{0.5cm}
\begin{block}{Esempio URL fraudolento}
\texttt{www.b\textcolor{alertred}{an}ca-intesa.com} invece di \texttt{www.intesa.it}\\
\small Notare la differenza: caratteri aggiunti, domini simili ma diversi
\end{block}
\end{frame}

\section{Attacchi DDoS}

% Slide 9 - Attacchi DDoS
\begin{frame}{Attacchi DDoS: Il Sovraccarico dei Sistemi}
\begin{columns}[T]
\column{0.5\textwidth}
\textbf{Distributed Denial of Service}

\vspace{0.3cm}
\textbf{Obiettivo:}\\
Rendere un servizio inaccessibile sovraccaricandolo con richieste massive.

\vspace{0.3cm}
\textbf{Come funziona:}
\begin{itemize}
\item Utilizzo di botnet (migliaia di dispositivi compromessi)
\item Invio simultaneo di richieste
\item Saturazione banda e risorse
\item Impossibilità di servire utenti legittimi
\end{itemize}

\vspace{0.3cm}
\textbf{Impatti:}
\begin{itemize}
\item Blocco servizi online
\item Perdite economiche
\item Danni reputazionali
\end{itemize}

\column{0.5\textwidth}
\begin{tikzpicture}[scale=0.75]
% Server target
\node[rectangle, draw=alertred, fill=alertred!30, minimum width=2cm, minimum height=1.5cm, line width=2pt] (server) at (0,0) {
\begin{minipage}{1.8cm}
\centering

$\bigstar$\\
\small Server\\
\small Sovraccarico
\end{minipage}
};

% Botnet
\foreach \x in {1,...,8} {
    \pgfmathsetmacro{\angle}{360/8*\x}
    \pgfmathsetmacro{\radius}{4}
    \node[circle, draw=darkblue, fill=lightblue!30, minimum size=0.7cm] (bot\x) at (\angle:\radius) {\tiny$\bigstar$};
    \draw[->, thick, alertred] (bot\x) -- (server);
}

% Etichetta botnet
\node[font=\small\bfseries, text=darkblue] at (0,-3) {Botnet: migliaia di dispositivi};

% Frecce richieste multiple
\node[font=\tiny, text=alertred] at (2.5,2) {Richieste};
\node[font=\tiny, text=alertred] at (-2.5,2) {Richieste};
\node[font=\tiny, text=alertred] at (2.5,-2) {Richieste};
\end{tikzpicture}
\end{columns}
\end{frame}

% Slide 10 - Tipologie DDoS
\begin{frame}{Tipologie di Attacchi DDoS}
\centering

\begin{tikzpicture}[
    node distance=2cm,
    block/.style={rectangle, draw=darkblue, fill=lightblue!20, text width=4.5cm, text centered, rounded corners, minimum height=1.2cm, font=\small}
]
\node[block] (volumetric) at (0,3) {\textbf{Attacchi Volumetrici}\\Saturazione della banda di rete};
\node[block] (protocol) at (0,1) {\textbf{Attacchi a Livello Protocollo}\\Esaurimento risorse server/firewall};
\node[block] (application) at (0,-1) {\textbf{Attacchi Applicativi}\\Target: applicazioni web specifiche};

% Esempi
\node[text width=5cm, font=\tiny, align=left] at (6,3) {
Es: UDP flood, ICMP flood\\
Volume: 100+ Gbps
};
\node[text width=5cm, font=\tiny, align=left] at (6,1) {
Es: SYN flood, Ping of Death\\
Esaurimento connessioni
};
\node[text width=5cm, font=\tiny, align=left] at (6,-1) {
Es: HTTP flood, Slowloris\\
Richieste HTTP apparentemente legittime
};

\draw[->, thick] (volumetric.east) -- ++(1,0);
\draw[->, thick] (protocol.east) -- ++(1,0);
\draw[->, thick] (application.east) -- ++(1,0);
\end{tikzpicture}

\vspace{0.3cm}
\begin{alertblock}{Caso Pratico}
Blocco di piattaforme bancarie online che impedisce ai clienti di accedere ai propri conti per ore o giorni, causando disservizi massivi.
\end{alertblock}
\end{frame}

\section{Supply Chain Attack}

% Slide 11 - Supply Chain Attack
\begin{frame}{Supply Chain Attack: L'Attacco alla Catena di Fornitura}
\begin{columns}[T]
\column{0.5\textwidth}
\textbf{Strategia dell'attacco:}
\begin{enumerate}
\item Target: fornitore o partner meno protetto
\item Compromissione del fornitore
\item Propagazione all'organizzazione principale
\item Infezione multipla attraverso la catena
\end{enumerate}

\vspace{0.5cm}
\textbf{Perché è pericoloso:}
\begin{itemize}
\item Difficile da rilevare
\item Fiducia nei fornitori
\item Effetto a cascata
\item Scala d'impatto enorme
\end{itemize}

\column{0.5\textwidth}
\begin{tikzpicture}[scale=0.75]
% Fornitore (compromesso)
\node[rectangle, draw=alertred, fill=alertred!30, minimum width=2.5cm, minimum height=1.2cm, line width=2pt, rounded corners] (supplier) at (0,3) {
\begin{minipage}{2.3cm}
\centering

$\diamond$\\
\small Fornitore\\
\scriptsize (COMPROMESSO)
\end{minipage}
};

% Clienti
\node[rectangle, draw=darkblue, fill=lightblue!30, minimum width=2cm, minimum height=1cm, rounded corners] (client1) at (-3,0) {\small Azienda A};
\node[rectangle, draw=darkblue, fill=lightblue!30, minimum width=2cm, minimum height=1cm, rounded corners] (client2) at (0,0) {\small Azienda B};
\node[rectangle, draw=darkblue, fill=lightblue!30, minimum width=2cm, minimum height=1cm, rounded corners] (client3) at (3,0) {\small Azienda C};

% Frecce di propagazione
\draw[->, very thick, alertred] (supplier) -- (client1) node[midway, left, font=\tiny] {Malware};
\draw[->, very thick, alertred] (supplier) -- (client2) node[midway, right, font=\tiny] {Malware};
\draw[->, very thick, alertred] (supplier) -- (client3) node[midway, right, font=\tiny] {Malware};

% Utenti finali
\foreach \x in {-3,0,3} {
    \node[circle, draw=darkblue, fill=white, minimum size=0.6cm] at (\x,-1.5) {\tiny$\bullet$};
    \draw[->, thick, alertred] (\x,-0.6) -- (\x,-1.2);
}
\node[font=\small] at (0,-2.5) {Migliaia di utenti finali infettati};
\end{tikzpicture}
\end{columns}
\end{frame}

% Slide 12 - Caso SolarWinds
\begin{frame}{Caso SolarWinds (2020): Un Attacco Emblematico}
\textbf{Dinamica dell'attacco:}
\begin{itemize}
\item \textbf{Target primario:} SolarWinds (software di gestione IT)
\item \textbf{Metodo:} Modifica del software Orion con malware nascosto
\item \textbf{Distribuzione:} Update automatici inviati ai clienti
\item \textbf{Scala:} \textcolor{alertred}{18.000+ organizzazioni infettate}
\end{itemize}

\vspace{0.5cm}
\begin{columns}[T]
\column{0.5\textwidth}
\textbf{Vittime:}
\begin{itemize}
\item Agenzie governative USA
\item Dipartimento della Difesa
\item Multinazionali Fortune 500
\item Istituzioni finanziarie
\end{itemize}

\column{0.5\textwidth}
\textbf{Conseguenze:}
\begin{itemize}
\item Accesso prolungato ai sistemi (mesi)
\item Furto di dati sensibili
\item Costi di remediation miliardari
\item Perdita di fiducia nel software
\end{itemize}
\end{columns}

\vspace{0.5cm}
\begin{alertblock}{Lezione appresa}
Necessità di verificare e monitorare costantemente anche i fornitori di fiducia. La sicurezza della supply chain è fondamentale.
\end{alertblock}
\end{frame}

\section{Intelligenza Artificiale nelle Minacce Cyber}

% Slide 13 - IA e Cybersecurity
\begin{frame}{Il Ruolo dell'Intelligenza Artificiale negli Attacchi}
\begin{columns}[T]
\column{0.5\textwidth}
\textbf{L'IA come arma degli attaccanti:}

\vspace{0.3cm}
\textcolor{alertred}{\textbf{AI}~Applicazioni offensive:}
\begin{itemize}
\item Phishing personalizzato e convincente
\item Imitazione stile di scrittura
\item Analisi vulnerabilità in tempo reale
\item Identificazione target redditizi
\item Evasione sistemi di rilevamento
\item Botnet intelligenti e adattive
\end{itemize}

\vspace{0.3cm}
\textbf{Ransomware avanzati:}
\begin{itemize}
\item Selezione automatica vittime
\item Occultamento attività malevole
\item Adattamento comportamentale
\end{itemize}

\column{0.5\textwidth}
\begin{tikzpicture}[scale=0.8]
% Cervello IA
\node[circle, draw=darkblue, fill=lightblue!30, minimum size=2.5cm, line width=2pt] (ai) at (0,2) {
\begin{minipage}{2cm}
\centering

\Large\textbf{AI}\\
\small IA\\
\small Offensiva
\end{minipage}
};

% Capacità
\node[rectangle, draw=alertred, fill=alertred!20, text width=2.2cm, align=center, rounded corners, font=\tiny] (phish) at (-3,0) {Phishing\\Automatizzato};
\node[rectangle, draw=alertred, fill=alertred!20, text width=2.2cm, align=center, rounded corners, font=\tiny] (vuln) at (3,0) {Scansione\\Vulnerabilità};
\node[rectangle, draw=alertred, fill=alertred!20, text width=2.2cm, align=center, rounded corners, font=\tiny] (evasion) at (-3,-2) {Evasione\\Rilevamento};
\node[rectangle, draw=alertred, fill=alertred!20, text width=2.2cm, align=center, rounded corners, font=\tiny] (adapt) at (3,-2) {Adattamento\\Dinamico};

% Collegamenti
\draw[->, thick] (ai) -- (phish);
\draw[->, thick] (ai) -- (vuln);
\draw[->, thick] (ai) -- (evasion);
\draw[->, thick] (ai) -- (adapt);
\end{tikzpicture}

\vspace{0.3cm}
\begin{alertblock}{Sfida crescente}
La velocità e la scala degli attacchi potenziati dall'IA richiedono contromisure altrettanto avanzate.
\end{alertblock}
\end{columns}
\end{frame}

% Slide 14 - DDoS intelligenti
\begin{frame}{DDoS Intelligenti e IA}
\textbf{Botnet potenziate dall'IA:}

\vspace{0.5cm}
\begin{columns}[T]
\column{0.5\textwidth}
\textbf{Capacità tradizionali:}
\begin{itemize}
\item Pattern di attacco fissi
\item Rilevamento più semplice
\item Comportamento prevedibile
\item Difese statiche efficaci
\end{itemize}

\vspace{0.5cm}
\textcolor{successgreen}{$\odot$~Vulnerabilità:}\\
Le difese tradizionali possono identificare pattern ripetuti.

\column{0.5\textwidth}
\textbf{Con IA:}
\begin{itemize}
\item[\textbf{AI}] Pattern dinamici e variabili
\item[\textbf{AI}] Elusione dei sistemi di rilevamento
\item[\textbf{AI}] Adattamento in tempo reale
\item[\textbf{AI}] Necessarie difese dinamiche
\end{itemize}

\vspace{0.5cm}
\textcolor{alertred}{$\blacktriangledown$~Minaccia:}\\
Difficile distinguere traffico legittimo da attacco.
\end{columns}

\vspace{0.5cm}
\begin{block}{Esempio concreto}
Una botnet con IA può alternare tipologie di attacco (volumetrico, protocollo, applicativo) in base alle reazioni del sistema di difesa, rendendo molto più difficile la mitigazione.
\end{block}
\end{frame}

\section{La Direttiva NIS 2}

% Slide 15 - Introduzione NIS 2
\begin{frame}{La Direttiva NIS 2: Risposta Europea}
\begin{columns}[T]
\column{0.5\textwidth}
\textbf{Direttiva (UE) 2022/2555}

\vspace{0.3cm}
\textbf{Data}~\textbf{Timeline:}
\begin{itemize}
\item Approvata: 14 dicembre 2022
\item Entrata in vigore: 17 gennaio 2023
\item Recepimento Stati UE: 17 ottobre 2024
\end{itemize}

\vspace{0.5cm}
\textbf{Obiettivo:}\\
Garantire un livello \textcolor{successgreen}{\textbf{elevato e omogeneo}} di sicurezza informatica in tutta l'UE.

\vspace{0.5cm}
\textbf{Principi cardine:}
\begin{itemize}
\item Resilienza dei sistemi
\item Risposta rapida agli incidenti
\item Cultura condivisa della cybersecurity
\item Cooperazione pubblico-privato
\end{itemize}

\column{0.5\textwidth}
\begin{tikzpicture}[scale=0.75]
% Mappa Europa stilizzata
\node[draw=darkblue, fill=lightblue!30, ellipse, minimum width=4cm, minimum height=5cm, line width=2pt] (eu) at (0,0) {};
\node[font=\Large\bfseries, text=darkblue] at (0,1.5) {UNIONE};
\node[font=\Large\bfseries, text=darkblue] at (0,0.8) {EUROPEA};

% Simbolo sicurezza
\node[circle, draw=successgreen, fill=successgreen!20, minimum size=2cm, line width=2pt] at (0,-0.5) {\Huge\textbf{Shield}};

% Stati membri
\node[font=\small] at (0,-2.3) {27 Stati Membri};
\node[font=\small, text=successgreen] at (0,-2.9) {Un unico standard di sicurezza};

% Stelle simboliche
\foreach \angle in {0,72,144,216,288} {
    \node[star, star points=5, star point ratio=2.5, draw=yellow!80!orange, fill=yellow!60, minimum size=0.3cm] at ({1.8*cos(\angle)},{1.8*sin(\angle)+0.5}) {};
}
\end{tikzpicture}
\end{columns}
\end{frame}

% Slide 16 - Limiti NIS 1
\begin{frame}{Perché NIS 2? I Limiti della Prima Direttiva}
\textbf{Direttiva NIS 1 (2016) - Problematiche riscontrate:}

\vspace{0.5cm}
\begin{columns}[T]
\column{0.5\textwidth}
\textcolor{alertred}{$\times$~Criticità:}
\begin{enumerate}
\item \textbf{Applicazione disomogenea}
\begin{itemize}
\item Discrezionalità degli Stati
\item Frammentazione normativa
\item Differenze nei livelli di sicurezza
\end{itemize}

\item \textbf{Ambito troppo ristretto}
\begin{itemize}
\item Solo 4 settori coperti
\item Realtà digitali escluse
\item Copertura insufficiente
\end{itemize}
\end{enumerate}

\column{0.5\textwidth}
\begin{enumerate}
\setcounter{enumi}{2}
\item \textbf{Requisiti poco specifici}
\begin{itemize}
\item Misure generiche
\item Mancanza di linee guida chiare
\item Implementazione incerta
\end{itemize}

\item \textbf{Monitoraggio inefficace}
\begin{itemize}
\item Sistema di controllo debole
\item Sanzioni inadeguate
\item Scarsa conformità
\end{itemize}
\end{enumerate}
\end{columns}

\vspace{0.5cm}
\begin{alertblock}{Fattori acceleranti}
\textbf{Pandemia COVID-19} (lavoro remoto), \textbf{Guerra in Ucraina} (cyber warfare), \textbf{Attacchi supply chain} (interconnessione vulnerabilità)
\end{alertblock}
\end{frame}

% Slide 17 - Settori NIS 2
\begin{frame}{Ampliamento dei Settori Coinvolti}
\textbf{Da 4 a oltre 15 settori:}

\vspace{0.3cm}
\begin{columns}[T]
\column{0.5\textwidth}
\textcolor{darkblue}{\textbf{NIS 1 (2016):}}
\begin{itemize}
\item[$\checkmark$] Energia
\item[$\checkmark$] Trasporti
\item[$\checkmark$] Banche
\item[$\checkmark$] Sanità
\end{itemize}

\vspace{0.5cm}
\textcolor{successgreen}{\textbf{Nuovi settori NIS 2:}}
\begin{itemize}
\item[$+$] Pubblica Amministrazione
\item[$+$] Acque reflue e rifiuti
\item[$+$] Spazio e satelliti
\item[$+$] Servizi postali
\item[$+$] Produzione microchip e farmaci
\item[$+$] Gestione servizi TIC
\item[$+$] Servizi digitali
\item[$+$] Ricerca
\end{itemize}

\column{0.5\textwidth}
\begin{tikzpicture}[scale=0.7]
% Grafico comparativo
\begin{axis}[
    ybar,
    width=6.5cm,
    height=7cm,
    ylabel={Numero settori coperti},
    symbolic x coords={NIS 1,NIS 2},
    xtick=data,
    ymin=0,
    ymax=18,
    bar width=40pt,
    nodes near coords,
    nodes near coords style={font=\Large\bfseries},
    ylabel style={font=\small},
]
\addplot[fill=darkblue] coordinates {(NIS 1,4)};
\addplot[fill=successgreen] coordinates {(NIS 2,15)};
\end{axis}

% Etichetta incremento
\node[font=\Large\bfseries, text=successgreen] at (3.5,0) {+275\%};
\end{tikzpicture}

\vspace{0.3cm}
\begin{block}{Impatto}
Maggiore copertura = maggiore protezione dell'ecosistema digitale europeo
\end{block}
\end{columns}
\end{frame}

% Slide 18 - Obblighi NIS 2
\begin{frame}{Obblighi di Sicurezza Informatica}
\textbf{Misure concrete e documentate:}

\vspace{0.3cm}
\begin{columns}[T]
\column{0.5\textwidth}
\textbf{Shield}~\textbf{Prevenzione:}
\begin{itemize}
\item Analisi dei rischi
\item Valutazione vulnerabilità
\item Gestione incidenti
\item Piani di continuità operativa
\end{itemize}

\vspace{0.3cm}
$\blacksquare$~\textbf{Protezione:}
\begin{itemize}
\item Crittografia dati
\item Controllo accessi
\item Sicurezza supply chain
\item Backup e ripristino
\end{itemize}

\column{0.5\textwidth}
$\bullet$s~\textbf{Persone:}
\begin{itemize}
\item Formazione continua
\item Consapevolezza rischi
\item Responsabilità dirigenziali
\item Cultura della sicurezza
\end{itemize}

\vspace{0.3cm}
\textbf{Check}~\textbf{Conformità:}
\begin{itemize}
\item Documentazione completa
\item Audit regolari
\item Report periodici
\item Verifiche di sicurezza
\end{itemize}
\end{columns}

\vspace{0.5cm}
\begin{exampleblock}{Esempio pratico}
Un'azienda energetica deve dimostrare di avere un piano per reagire rapidamente a un attacco ransomware, garantendo comunque la fornitura minima di servizio.
\end{exampleblock}
\end{frame}

% Slide 19 - Notifica incidenti e sanzioni
\begin{frame}{Notifica degli Incidenti e Sanzioni}
\begin{columns}[T]
\column{0.5\textwidth}
\textbf{Timeline di notifica:}

\begin{tikzpicture}[scale=0.85]
% Freccia temporale
\draw[->, very thick, darkblue] (0,0) -- (6,0);

% 24 ore
\node[circle, draw=alertred, fill=alertred!30, minimum size=1cm, font=\tiny\bfseries] at (1,0) {24h};
\node[text width=2cm, font=\tiny, align=center] at (1,-1) {Prima\\segnalazione};

% 72 ore
\node[circle, draw=orange, fill=orange!30, minimum size=1cm, font=\tiny\bfseries] at (3.5,0) {72h};
\node[text width=2cm, font=\tiny, align=center] at (3.5,-1) {Relazione\\tecnica};

% 1 mese
\node[circle, draw=successgreen, fill=successgreen!30, minimum size=1cm, font=\tiny\bfseries] at (6,0) {1 mese};
\node[text width=2cm, font=\tiny, align=center] at (6,-1) {Relazione\\finale};

% Asse temporale
\node[font=\small] at (0,-0.3) {0};
\node[font=\small] at (7,-0.3) {Tempo};
\end{tikzpicture}

\vspace{0.5cm}
\textbf{Esempio:}\\
Un ospedale colpito da malware deve informare immediatamente l'ACN (Agenzia per la Cybersicurezza Nazionale).

\column{0.5\textwidth}
\textbf{Sanzioni per inadempienza:}

\vspace{0.3cm}
\textcolor{alertred}{$\blacktriangledown$~Soggetti essenziali:}
\begin{itemize}
\item Fino a \textbf{10 milioni €}
\item Oppure \textbf{2\% fatturato globale}
\end{itemize}

\vspace{0.3cm}
\textcolor{orange}{$\blacktriangledown$~Altri soggetti:}
\begin{itemize}
\item Fino a \textbf{7 milioni €}
\item Oppure \textbf{1,4\% fatturato globale}
\end{itemize}

\vspace{0.3cm}
\begin{alertblock}{Responsabilità dirigenti}
I dirigenti possono essere sanzionati personalmente per negligenza, con possibili interdizioni.
\end{alertblock}
\end{columns}
\end{frame}

% Slide 20 - Applicazione in Italia
\begin{frame}{Applicazione in Italia}
\textbf{Decreto Legislativo n. 138 del 4 settembre 2024}

\vspace{0.5cm}
\begin{columns}[T]
\column{0.5\textwidth}
\textbf{Loc}~\textbf{Autorità competente:}\\
\textcolor{darkblue}{\textbf{Agenzia per la Cybersicurezza Nazionale (ACN)}}

\vspace{0.3cm}
\textbf{Istituita:} 2021\\
\textbf{Dipende:} Presidenza del Consiglio

\vspace{0.5cm}
\textbf{Compiti principali:}
\begin{itemize}
\item[\textbf{Config}] Strategia nazionale cybersecurity
\item[\textbf{Support}] Coordinamento risposta incidenti
\item[\textbf{Help}] Supporto tecnico PA e privati
\item[$\triangleright$] Gestione notifiche NIS 2
\item[\textbf{Edu}] Formazione e sensibilizzazione
\item[\textbf{Law}] Vigilanza e sanzioni
\end{itemize}

\column{0.5\textwidth}
\begin{tikzpicture}[scale=0.75]
% Logo ACN simbolico
\node[rectangle, draw=darkblue, fill=lightblue!30, minimum width=4cm, minimum height=2cm, line width=2pt, rounded corners] (acn) at (0,3) {
\begin{minipage}{3.5cm}
\centering

\Large\textbf{Shield}\\
\vspace{0.2cm}
\textbf{ACN}\\
\small Agenzia Cybersicurezza\\
\small Nazionale
\end{minipage}
};

% Funzioni
\node[rectangle, draw=darkblue, fill=white, text width=1.5cm, align=center, rounded corners, font=\tiny] at (-2,0.5) {Prevenzione};
\node[rectangle, draw=darkblue, fill=white, text width=1.5cm, align=center, rounded corners, font=\tiny] at (2,0.5) {Monitoraggio};
\node[rectangle, draw=darkblue, fill=white, text width=1.5cm, align=center, rounded corners, font=\tiny] at (-2,-1) {Risposta};
\node[rectangle, draw=darkblue, fill=white, text width=1.5cm, align=center, rounded corners, font=\tiny] at (2,-1) {Recupero};

% Collegamenti
\draw[->, thick] (acn.south) -- (-2,1.2);
\draw[->, thick] (acn.south) -- (2,1.2);
\draw[->, thick] (acn.south) -- (-2,-0.3);
\draw[->, thick] (acn.south) -- (2,-0.3);
\end{tikzpicture}

\vspace{0.3cm}
\begin{block}{Scadenza registrazione}
\textbf{28 febbraio 2025:} termine per registrarsi sulla piattaforma ACN
\end{block}
\end{columns}
\end{frame}

% Slide 21 - Benefici per i cittadini
\begin{frame}{Benefici Concreti per i Cittadini}
\textbf{Impatti positivi della NIS 2:}

\vspace{0.5cm}
\begin{columns}[T]
\column{0.5\textwidth}
\textcolor{successgreen}{$\checkmark$~Servizi più affidabili:}
\begin{itemize}
\item Continuità ospedali e sanità
\item Stabilità servizi pubblici
\item Energia e acqua garantite
\item Trasporti funzionanti
\end{itemize}

\vspace{0.3cm}
\textcolor{successgreen}{$\blacksquare$~Dati protetti:}
\begin{itemize}
\item Protezione dati personali
\item Riduzione furti identità
\item Sicurezza fascicoli sanitari
\item Privacy rispettata
\end{itemize}

\column{0.5\textwidth}
\textcolor{successgreen}{$\smile$~Maggiore fiducia:}
\begin{itemize}
\item Servizi online sicuri
\item Home banking protetto
\item E-government affidabile
\item Transazioni sicure
\end{itemize}

\vspace{0.3cm}
\textcolor{successgreen}{\textbf{Shield}~Prevenzione truffe:}
\begin{itemize}
\item Meno phishing efficace
\item Consapevolezza rischi
\item Educazione digitale
\item Difesa attiva
\end{itemize}
\end{columns}

\vspace{0.5cm}
\begin{exampleblock}{In sintesi}
NIS 2 non è solo una normativa per aziende ed enti: protegge anche la vita digitale quotidiana dei cittadini, riducendo disagi e aumentando la sicurezza.
\end{exampleblock}
\end{frame}

% Slide 22 - Conclusioni
\begin{frame}{Conclusioni}
\textbf{Verso un'Europa digitalmente sicura}

\vspace{0.5cm}
\begin{columns}[T]
\column{0.5\textwidth}
\textcolor{darkblue}{\textbf{Chart}~Il contesto:}
\begin{itemize}
\item Minacce in continua evoluzione
\item IA che potenzia gli attacchi
\item Interconnessione crescente
\item Necessità di protezione coordinata
\end{itemize}

\vspace{0.5cm}
\textcolor{successgreen}{\textbf{Shield}~La risposta:}
\begin{itemize}
\item NIS 2 come framework comune
\item Obblighi chiari e misurabili
\item Sanzioni deterrenti
\item Cultura della cybersecurity
\end{itemize}

\column{0.5\textwidth}
\begin{tikzpicture}[scale=0.75]
% Piramide della sicurezza
\draw[fill=successgreen!30, draw=darkblue, line width=1.5pt] (0,0) -- (3,0) -- (1.5,4) -- cycle;
\draw[fill=successgreen!20, draw=darkblue, line width=1.5pt] (3,0) -- (4.5,0) -- (1.5,4) -- cycle;

% Livelli
\node[font=\tiny, text width=2cm, align=center] at (1.5,0.5) {\textbf{Tecnologia}\\Misure tecniche};
\node[font=\tiny, text width=2cm, align=center] at (1.5,1.8) {\textbf{Processi}\\Procedure e policy};
\node[font=\tiny, text width=2cm, align=center] at (1.5,3.2) {\textbf{Persone}\\Formazione};

% Obiettivo
\node[font=\small\bfseries, text=darkblue] at (2.5,-0.7) {Ecosistema resiliente};
\end{tikzpicture}

\vspace{0.3cm}
\begin{block}{Responsabilità condivisa}
Pubblico e privato devono collaborare per costruire un ambiente digitale sicuro.
\end{block}
\end{columns}

\vspace{0.5cm}
\centering

\large\textcolor{darkblue}{\textbf{La sicurezza informatica è responsabilità di tutti}}
\end{frame}

% Slide 23 - Riferimenti
\begin{frame}{Riferimenti Bibliografici}
\small
\begin{itemize}
\item Direttiva (UE) 2022/2555 del Parlamento Europeo e del Consiglio
\item \url{https://digital-strategy.ec.europa.eu/it/policies/nis2-directive}
\item \url{https://direttivanis2.eu}
\item \url{https://www.akamai.com/it/glossary/what-is-nis2}
\item Decreto Legislativo 4 settembre 2024, n. 138
\item \url{https://www.acn.gov.it} - Agenzia per la Cybersicurezza Nazionale
\end{itemize}

\vspace{1cm}
\centering

\Large\textcolor{darkblue}{\textbf{Grazie per l'attenzione!}}

\vspace{0.5cm}
\normalsize
\textbf{?}~Domande?
\end{frame}

\end{document}