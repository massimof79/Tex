\documentclass[10pt]{beamer}
\usepackage[utf8]{inputenc}
\usepackage[T1]{fontenc}
\usepackage[italian]{babel}
\usepackage{amsmath}
\usepackage{graphicx}
\usetheme{Madrid}
\usecolortheme{seahorse}

\title{Confusione, Diffusione e Teorema di Shannon}
\subtitle{Concetti fondamentali della crittografia moderna}
\author{ Prof. Fedeli Massimo - IIS Fermi Sacconi Cpia}
\date{ }

\begin{document}
	
	\begin{frame}
		\titlepage
	\end{frame}
	

\begin{frame}{Il modello di un cifrario}
	Un sistema di cifratura classico può essere visto come una funzione:
	
	\[
	C = E_K(P)
	\]
	
	Dove:
	\begin{itemize}
		\item $P$ è il testo in chiaro (plaintext),
		\item $K$ è la chiave segreta,
		\item $C$ è il testo cifrato (ciphertext).
	\end{itemize}
	
	L'obiettivo è rendere difficile risalire a $P$ o a $K$ partendo da $C$.
\end{frame}

	
	\begin{frame}{Il problema dell'analisi crittografica}
		Un attaccante cerca di sfruttare:
		\begin{itemize}
			\item regolarità del linguaggio,
			\item schemi ripetitivi,
			\item relazioni semplici tra testo in chiaro e testo cifrato.
		\end{itemize}
		
		La crittografia moderna mira a eliminare o mascherare queste debolezze.
	\end{frame}
	
	\begin{frame}{Esempio di analisi crittografica}
		Supponiamo di intercettare il seguwhile testo cifrato:
		
		\[
		XQZZQ XQZZQ XQZZQ
		\]
		
		Un attaccante può ipotizzare che:
		\begin{itemize}
			\item il messaggio contenga parole ripetute,
			\item il cifrario non elimini le ripetizioni,
			\item lettere uguali nel testo in chiaro producano lettere uguali nel testo cifrato.
		\end{itemize}
		
		Questo tipo di osservazione permette di formulare ipotesi sul testo originale o sulla chiave.
		
		La crittografia moderna evita questi attacchi tramite confusione e diffusione.
	\end{frame}
	
	
	\begin{frame}{Confusione: idea intuitiva}
		La \textbf{confusione} ha lo scopo di:
		
		\begin{quote}
			Rendere complessa e non intuitiva la relazione tra la chiave segreta e il testo cifrato.
		\end{quote}
		
		In presenza di confusione, anche una piccola variazione della chiave produce un risultato apparentemente imprevedibile.
	\end{frame}
	
	\begin{frame}{Esempio di confusione}
		Supponiamo di cifrare una lettera usando una sostituzione dipendente dalla chiave:
		
		\begin{itemize}
			\item Chiave = 3 \rightarrow Cifrario di Cesare
			\item A \rightarrow D
			\item B \rightarrow E
		\end{itemize}
		
		Se la chiave cambia (ad esempio 4 invece di 3), l'intero schema di sostituzione cambia.
		
		La confusione aumenta quando la sostituzione è complessa e non lineare.
	\end{frame}
	
	\begin{frame}{Diffusione: idea intuitiva}
		La \textbf{diffusione} ha lo scopo di:
		
		\begin{quote}
			Distribuire l'informazione del testo in chiaro su molte parti del testo cifrato.
		\end{quote}
		
		In questo modo, una piccola modifica del messaggio originale produce molte modifiche nel testo cifrato.
	\end{frame}
	
	\begin{frame}{Esempio di diffusione}
		Consideriamo un messaggio binario:
		
		[ P = 10100010 ]
		
		Dopo una buona diffusione:
		\begin{itemize}
			\item ogni bit di $P$ influenza molti bit di $C$,
			\item non è possibile individuare direttamente la posizione dei bit originali.
		\end{itemize}
		
		Questo principio è noto come \emph{effetto valanga}.
	\end{frame}
	
	\begin{frame}{Confusione e diffusione insieme}
		Un cifrario sicuro combina:
		\begin{itemize}
			\item confusione, per nascondere il ruolo della chiave,
			\item diffusione, per eliminare strutture e regolarità.
		\end{itemize}
		
		I moderni algoritmi a blocchi applicano questi principi in più cicli (round).
	\end{frame}
	
	\begin{frame}{Claude Shannon e la crittografia}
		Claude Shannon è considerato il padre della teoria dell'informazione.
		
		Nel 1949 formulò i principi fondamentali della crittografia moderna, introducendo i concetti di confusione e diffusione.
	\end{frame}
	
	\begin{frame}{Il teorema di Shannon (idea di base)}
		Il teorema di Shannon afferma che:
		
		\begin{quote}
			La sicurezza di un sistema crittografico dipende esclusivamente dalla segretezza della chiave, non dall'algoritmo.
		\end{quote}
		
		Questo principio è noto come \emph{Kerckhoffs-Shannon}.
	\end{frame}
	
	\begin{frame}{Conseguenze del teorema di Shannon}
		Da questo principio derivano alcune conseguenze importanti:
		\begin{itemize}
			\item l'algoritmo può essere pubblico,
			\item la chiave deve essere segreta e sufficientemente lunga,
			\item la sicurezza non deve basarsi sull'"oscurità" del metodo.
		\end{itemize}
	\end{frame}
	
	\begin{frame}{Sicurezza perfetta}
		Shannon dimostrò che esiste una condizione di sicurezza perfetta:
		
		\begin{quote}
			Il testo cifrato non fornisce alcuna informazione sul testo in chiaro.
		\end{quote}
		
		Un esempio teorico è il \emph{one-time pad}, in cui:
		\begin{itemize}
			\item la chiave è lunga quanto il messaggio,
			\item la chiave è casuale,
			\item la chiave è usata una sola volta.
		\end{itemize}
	\end{frame}
	
	\begin{frame}{Limiti pratici}
		Sebbene la sicurezza perfetta sia teoricamente possibile, nella pratica:
		\begin{itemize}
			\item la gestione delle chiavi è complessa,
			\item si preferiscono sistemi computazionalmente sicuri,
			\item confusione e diffusione diventano fondamentali.
		\end{itemize}
	\end{frame}
	
	\begin{frame}{Riepilogo finale}
		\begin{itemize}
			\item La confusione nasconde il ruolo della chiave.
			\item La diffusione elimina le strutture del messaggio.
			\item Il teorema di Shannon stabilisce le basi teoriche della crittografia moderna.
		\end{itemize}
		
		Questi concetti sono alla base di tutti i sistemi crittografici utilizzati oggi.
	\end{frame}
	
\end{document}
