\documentclass{beamer}
\usepackage[utf8]{inputenc}
\usepackage[italian]{babel}
\usepackage{graphicx}
\usepackage{tikz}
\usepackage{listings}
\usepackage{xcolor}

% Tema
\usetheme{Madrid}
\usecolortheme{seahorse}

% Colori personalizzati
\definecolor{darkblue}{RGB}{0,82,155}
\definecolor{lightblue}{RGB}{100,149,237}
\definecolor{red}{RGB}{220,20,60}

% Titolo
\title[Cybersecurity e NIS2]{La Cybersecurity nell'era dell'Intelligenza Artificiale:\\La Direttiva NIS 2}
\subtitle{Una nuova frontiera per la protezione europea}
\author{Prof. Fedeli Massimo}
\institute{IIS Fermi Sacconi Cpia}
\date{\today}

\begin{document}
	
	% Slide 1 - Titolo
	\begin{frame}
		\titlepage
		\begin{center}
		\end{center}
	\end{frame}
	
	% Slide 2 - Agenda
	\begin{frame}{Agenda}
		\tableofcontents
	\end{frame}
	
	% Slide 3 - Introduzione
	\section{Introduzione}
	\begin{frame}{Il Panorama delle Minacce Informatiche}
		\begin{columns}
			\column{0.6\textwidth}
			\begin{itemize}
				\item Crescita esponenziale degli attacchi
				\item Aumento di complessità e impatto
				\item Target: aziende, PA, infrastrutture critiche
				\item Nuove tecnologie = nuove vulnerabilità
			\end{itemize}
			\column{0.4\textwidth}
			\begin{tikzpicture}
			\includegraphics[width=0.4\textwidth]{threats.png}
			\end{tikzpicture}
		\end{columns}
	\end{frame}
	
	% Slide 4 - Ransomware
	\section{Tipologie di Attacchi}
	\begin{frame}{Ransomware}
		\begin{block}{Definizione}				
			\draw[->, thick, red] (0,2) -- (1,3);
			\draw[->, thick, red] (0.5,1.5) -- (1.5,2.5);
			\draw[->, thick, red] (1,1) -- (2,2);
			Malware che cifra i dati richiedendo un riscatto per la decrittazione
		\end{block}
		\begin{columns}
			\column{0.5\textwidth}
			\textbf{Caratteristiche:}
			\begin{itemize}
				\item Richiesta in criptovaluta
				\item Blocco totale dei dati
				\item Tempi di recupero lunghi
			\end{itemize}
			\column{0.5\textwidth}
			\textbf{Caso Irlanda 2021:}
			\begin{itemize}
				\item Sistema sanitario nazionale
				\item Ospedali bloccati per giorni
				\item Interruzione servizi essenziali
			\end{itemize}
		\end{columns}
		\begin{center}
			\includegraphics[width=0.4\textwidth]{ransomware.png}
		\end{center}
	\end{frame}
	
	% Slide 5 - Phishing
	\begin{frame}{Phishing}
		\begin{alertblock}{Tecnica di ingegneria sociale}
			Email fraudolente che imitano comunicazioni ufficiali
		\end{alertblock}
		\begin{columns}
			\column{0.6\textwidth}
			\includegraphics[width=\textwidth]{phishing.png}
			\column{0.4\textwidth}
			\textbf{Obiettivi:}
			\begin{itemize}
				\item Credenziali di accesso
				\item Dati bancari
				\item Informazioni sensibili
			\end{itemize}
		\end{columns}
		\textbf{Esempio:} Email apparentemente dall'IT interno richiede aggiornamento password urgente
	\end{frame}
	
	% Slide 6 - DDoS
	\begin{frame}{DDoS - Distributed Denial of Service}
		\begin{block}{Obiettivo}
			Sovraccaricare un servizio online con richieste massive
		\end{block}
		\begin{columns}
			\column{0.5\textwidth}
			\textbf{Metodo:}
			\begin{itemize}
				\item Botnet di migliaia di dispositivi
				\item Richieste simultanee
				\item Blocco del servizio
			\end{itemize}
			\column{0.5\textwidth}
			\textbf{Conseguenze:}
			\begin{itemize}
				\item Inaccessibilità ore/giorni
				\item Perdite economiche
				\item Danno reputazionale
			\end{itemize}
		\end{columns}
		\begin{center}
			\includegraphics[width=0.5\textwidth]{ddos.png}
		\end{center}
	\end{frame}
	
	% Slide 7 - Supply Chain Attack
	\begin{frame}{Supply Chain Attack}
		\begin{exampleblock}{Strategia}
			Attacco al fornitore per propagarsi all'organizzazione principale
		\end{exampleblock}
		\textbf{Caso SolarWinds 2020:}
		\begin{itemize}
			\item Software di gestione IT compromesso
			\item Migliaia di clienti infettati
			\item Agenzie governative coinvolte
		\end{itemize}
		\begin{center}

		\end{center}
	\end{frame}
	
	% Slide 8 - IA nelle minacce
	\section{Intelligenza Artificiale e Cybersecurity}
	\begin{frame}{Il Ruolo dell'IA nelle Minacce}
		\begin{columns}
			\column{0.6\textwidth}
			\textbf{Automazione degli attacchi:}
			\begin{itemize}
				\item Phishing personalizzato
				\item Imitazione stili di scrittura
				\item Analisi vulnerabilità real-time
			\end{itemize}
			\column{0.4\textwidth}
		
		\end{columns}
		\textbf{Botnet intelligenti:}
		\begin{itemize}
			\item Adattamento dinamico
			\item Elusione rilevamento
			\item Selezione bersagli redditizi
		\end{itemize}
	\end{frame}
	
	% Slide 9 - Introduzione NIS2
	\section{La Direttiva NIS 2}
	\begin{frame}{Perché NIS 2?}
		\begin{alertblock}{Problematiche NIS 1 (2016)}
			\begin{itemize}
				\item Applicazione disomogenea
				\item Ambito troppo ristretto
				\item Requisiti generici
				\item Monitoraggio inefficace
			\end{itemize}
		\end{alertblock}
		\textbf{Fattori acceleranti:}
		\begin{itemize}
			\item Pandemia e smart working
			\item Guerra in Ucraina
			\item Aumento supply chain attacks
		\end{itemize}
	\end{frame}
	
	% Slide 10 - Timeline NIS2
	\begin{frame}{Timeline NIS 2}
		\begin{tikzpicture}
			\draw[->, thick] (0,0) -- (12,0);
			\draw[->, thick] (0,0) -- (0,3);
			
			\node at (2,1) [rectangle, fill=lightblue, text=white] {Dic 2022};
			\node at (2,0.3) {\tiny Approvazione};
			
			\node at (6,1) [rectangle, fill=darkblue, text=white] {Gen 2023};
			\node at (6,0.3) {\tiny Entrata in vigore};
			
			\node at (10,1) [rectangle, fill=red, text=white] {Ott 2024};
			\node at (10,0.3) {\tiny Recepimento};
			
			\draw[thick, dashed] (2,0) -- (2,2);
			\draw[thick, dashed] (6,0) -- (6,2);
			\draw[thick, dashed] (10,0) -- (10,2);
		\end{tikzpicture}
		
		\textbf{Obiettivo:} Rafforzare resilienza sistemi informatici europei
	\end{frame}
	
	% Slide 11 - Novità NIS2
	\begin{frame}{Principali Novità di NIS 2}
		\begin{enumerate}
			\item \textbf{Ampliamento settori:}
			\begin{itemize}
				\item PA centrale e locale
				\item Acque reflue e rifiuti
				\item Spazio e satelliti
				\item Poste e corriere
				\item Produzione critica (microchip, farmaci)
			\end{itemize}
			\item \textbf{Obblighi di sicurezza specifici}
			\item \textbf{Notifica tempestiva incidenti}
			\item \textbf{Responsabilità dirigenziale}
			\item \textbf{Sanzioni severe}
		\end{enumerate}
	\end{frame}
	
	% Slide 12 - Obblighi Sicurezza
	\begin{frame}{Obblighi di Sicurezza Informatica}
		\begin{block}{Misure obbligatorie}
			\begin{itemize}
				\item Analisi rischi e vulnerabilità
				\item Gestione e risposta incidenti
				\item Continuità operativa e piani di ripristino
				\item Sicurezza supply chain
				\item Crittografia e controllo accessi
				\item Formazione del personale
			\end{itemize}
		\end{block}
		\begin{center}

		\end{center}
	\end{frame}
	
	% Slide 13 - Notifica Incidenti
	\begin{frame}{Notifica degli Incidenti}
		\begin{alertblock}{Tempistiche obbligatorie}
			\begin{description}
				\item[24h] Prima segnalazione
				\item[72h] Relazione tecnica dettagliata
				\item[30 giorni] Relazione finale
			\end{description}
		\end{alertblock}
		\textbf{Esempio:} Ospedale colpito da malware $\rightarrow$ informare immediatamente ACN
		\begin{center}

		\end{center}
	\end{frame}
	
	% Slide 14 - Responsabilità
	\begin{frame}{Responsabilità e Sanzioni}
		\begin{columns}
			\column{0.5\textwidth}
			\textbf{Responsabilità dirigenziale:}
			\begin{itemize}
				\item Approvazione misure sicurezza
				\item Formazione specifica
				\item Responsabilità per negligenza
			\end{itemize}
			\column{0.5\textwidth}
			\textbf{Sanzioni:}
			\begin{itemize}
				\item 10M€ o 2\% fatturato (essenziali)
				\item 7M€ o 1.4\% fatturato (altri)
			\end{itemize}
		\end{columns}
		\begin{center}

		\end{center}
	\end{frame}
	
	% Slide 15 - ACN
	\section{Implementazione in Italia}
	\begin{frame}{L'Agenzia per la Cybersicurezza Nazionale (ACN)}
		\begin{block}{Istituita nel 2021}
			Ente pubblico per la protezione e rafforzamento sicurezza cibernetica italiana
		\end{block}
		\textbf{Principali compiti:}
		\begin{itemize}
			\item Strategia nazionale cybersicurezza
			\item Coordinamento risposta incidenti
			\item Supporto tecnico-operativo
			\item Gestione notifiche NIS 2
			\item Promozione cultura cybersecurity
			\item Vigilanza e sanzioni
		\end{itemize}
	\end{frame}
	
	% Slide 16 - Recepimento Italia
	\begin{frame}{Recepimento in Italia}
		\begin{exampleblock}{Decreto Legislativo 138/2024}
			4 settembre 2024 - Entrato in vigore 16 ottobre 2024
		\end{exampleblock}
		\textbf{Adempimenti principali:}
		\begin{itemize}
			\item Registrazione piattaforma ACN
			\item Fornitura informazioni organizzazione
			\item Comunicazione servizi offerti
			\item Adempimento entro 28 febbraio 2025
		\end{itemize}
	\end{frame}
	
	% Slide 17 - Obblighi Imprese
	\begin{frame}{Obblighi per le Imprese}
		\begin{columns}
			\column{0.6\textwidth}
			\textbf{Ambito applicazione:}
			\begin{itemize}
				\item Settori ad alta criticità
				\item Aziende 50+ dipendenti
				\item Fatturato > 10M€
				\item Servizi essenziali
			\end{itemize}
			\column{0.4\textwidth}

		\end{columns}
	\end{frame}
	
	% Slide 18 - Adempimenti Imprese
	\begin{frame}{Adempimenti Obbligatori per le Imprese}
		\begin{enumerate}
			\item Registrazione ACN entro 28/02/2025
			\item Misure gestione rischio
			\item Notifica incidenti entro 72h
			\item Designazione responsabile cybersecurity
			\item Formazione continua personale
			\item Gestione supply chain
			\item Supervisione organi direttivi
		\end{enumerate}
	\end{frame}
	
	% Slide 19 - Benefici Cittadini
	\section{Benefici per i Cittadini}
	\begin{frame}{Benefici Concreti per i Cittadini}
		\begin{columns}
			\column{0.5\textwidth}
			\includegraphics[width=\textwidth]{citizens.png}
			\column{0.5\textwidth}
			\textbf{Vantaggi:}
			\begin{itemize}
				\item Maggiore continuità servizi
				\item Protezione dati personali
				\item Fiducia servizi digitali
				\item Prevenzione truffe
				\item Risposte rapide incidenti
			\end{itemize}
		\end{columns}
	\end{frame}
	
	% Slide 20 - Esempi Benefici
	\begin{frame}{Esempi di Benefici per i Cittadini}
		\begin{exampleblock}{Servizi Sanitari}
			Riduzione rischi interruzione appuntamenti e cure urgenti per attacchi ransomware
		\end{exampleblock}
		\begin{exampleblock}{Servizi Bancari}
			Maggiore sicurezza home banking e protezione dati finanziari
		\end{exampleblock}
		\begin{exampleblock}{PA Digitale}
			Fiducia nell'utilizzo servizi online (fascicolo sanitario, portale PA)
		\end{exampleblock}
	\end{frame}
	
	% Slide 21 - Conclusioni
	\section{Conclusioni}
	\begin{frame}{Conclusioni}
		\begin{block}{NIS 2: Un passo fondamentale}
			Verso un'Europa più sicura digitalmente
		\end{block}
		\textbf{Imperativo per le organizzazioni:}
		\begin{itemize}
			\item Prepararsi a nuove responsabilità
			\item Adottare misure sicurezza adeguate
			\item Contribuire ecosistema digitale resiliente
		\end{itemize}
		\begin{center}

		\end{center}
		\textbf{Obiettivo finale:} Cybersicurezza condivisa tra pubblico e privato
	\end{frame}
	
	% Slide 22 - Riferimenti
	\begin{frame}{Riferimenti Bibliografici}
		\begin{tiny}
			\begin{itemize}
				\item Direttiva (UE) 2022/2555 del Parlamento Europeo e del Consiglio
				\item https://digital-strategy.ec.europa.eu/it/policies/nis2-directive
				\item https://direttivanis2.eu
				\item https://www.akamai.com/it/glossary/what-is-nis2
				\item Decreto Legislativo 4 settembre 2024, n. 138
				\item https://www.confindustriaemilia.it/flex/files/1/1/2/D.90e6cec33a853f3a4ea7/sicurezza\_informatica.pdf
			\end{itemize}
		\end{tiny}
		\begin{center}
			\Huge{\textbf{Grazie per l'attenzione!}}
		\end{center}
	\end{frame}
	
\end{document}