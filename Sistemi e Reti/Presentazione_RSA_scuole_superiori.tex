\documentclass[aspectratio=169]{beamer}
\usepackage[utf8]{inputenc}
\usepackage[italian]{babel}
\usepackage{amsmath}
\usepackage{amssymb}
\usepackage{amsthm}
\usepackage{graphicx}
\usepackage{booktabs}
\usepackage{tikz}

% Theme
\usetheme{Madrid}
\usecolortheme{default}

% Title page information
\title{Il Criptosistema RSA}
\subtitle{Come Funziona la Crittografia Moderna}
\author{Prof. Fedeli Massimo}
\institute{IIS Fermi Sacconi Cpia\\Ascoli Piceno}
\date{\today}

\begin{document}

% Slide 1: Title
\frame{\titlepage}

% Slide 2: Introduzione - Perché la Crittografia?
\begin{frame}{Perché Abbiamo Bisogno della Crittografia?}

\textbf{Ogni giorno usiamo la crittografia:}
\begin{itemize}
\item Quando compriamo online (carta di credito)
\item Quando chattiamo su WhatsApp
\item Quando facciamo home banking
\item Quando inviamo email
\end{itemize}

\vspace{0.5cm}

\begin{alertblock}{Obiettivo}
Proteggere i nostri messaggi da occhi indiscreti!
\end{alertblock}
\end{frame}

% Slide 3: Il Cifrario di Cesare
\begin{frame}{Il Cifrario di Cesare: Un Esempio Semplice}
\textbf{Giulio Cesare} usava un metodo semplicissimo per cifrare i suoi messaggi:

\vspace{0.5cm}

\textbf{Regola:} Spostare ogni lettera di un certo numero di posizioni in avanti  (es. 3)

\vspace{0.3cm}

\begin{center}
\begin{tikzpicture}
\node[draw, rectangle, fill=blue!20] at (0,0) {A};
\node at (1,0) {$\rightarrow$};
\node[draw, rectangle, fill=green!20] at (2,0) {D};

\node[draw, rectangle, fill=blue!20] at (3.5,0) {B};
\node at (4.5,0) {$\rightarrow$};
\node[draw, rectangle, fill=green!20] at (5.5,0) {E};

\node[draw, rectangle, fill=blue!20] at (7,0) {C};
\node at (8,0) {$\rightarrow$};
\node[draw, rectangle, fill=green!20] at (9,0) {F};
\end{tikzpicture}
\end{center}

\vspace{0.5cm}

\textbf{Esempio:}
\begin{itemize}
\item Messaggio originale: \alert{CIAO}
\item Messaggio cifrato: \alert{FLDR}
\end{itemize}

\vspace{0.3cm}

\textbf{Problema:} Se scopro la chiave (3), posso decifrare tutto!
\end{frame}

% Slide 4: Il Problema della Chiave Segreta
\begin{frame}{Il Grande Problema delle Chiavi Segrete}
\textbf{Con il Cifrario di Cesare:}

\begin{center}
\begin{tikzpicture}[scale=0.8]
% Alice
\node[circle, draw, fill=pink!30, minimum size=1.5cm] at (0,0) {Alice};
% Bob
\node[circle, draw, fill=blue!30, minimum size=1.5cm] at (8,0) {Bob};
% Freccia
\draw[->, very thick, red] (1.2,0.3) -- (6.8,0.3) node[midway, above] {Chiave segreta (3)};
\draw[->, very thick, green] (1.2,-0.3) -- (6.8,-0.3) node[midway, below] {Messaggio cifrato};
\end{tikzpicture}
\end{center}

\vspace{0.5cm}

\begin{block}{Problemi}
\begin{enumerate}
\item Come faccio a dare la chiave a Bob in modo sicuro?
\item Se qualcuno intercetta la chiave, può leggere tutto!
\item Chi sa cifrare, sa anche decifrare
\end{enumerate}
\end{block}

\vspace{0.3cm}

\textbf{Oggi:} Vogliamo comunicare con milioni di persone su Internet. Impossibile scambiare chiavi segrete con tutti!
\end{frame}

% Slide 5: La Grande Idea: Due Chiavi!
\begin{frame}{La Rivoluzione: Crittografia a Chiave Pubblica}
\begin{center}
\textbf{L'idea geniale (1977):}
\end{center}

Invece di una chiave, usiamo \alert{DUE chiavi diverse}:

\vspace{0.5cm}

\begin{columns}
\column{0.5\textwidth}
\begin{block}{Chiave Pubblica}
\begin{itemize}
\item La do a TUTTI
\item Serve per CIFRARE
\end{itemize}
\end{block}

\column{0.5\textwidth}
\begin{block}{Chiave Privata}
\begin{itemize}
\item La tengo SOLO IO
\item Serve per DECIFRARE
\end{itemize}
\end{block}
\end{columns}

\vspace{0.5cm}

\begin{alertblock}{Magia della Matematica!}
Cifrare è facile per tutti, decifrare è facile SOLO per me!
\end{alertblock}
\end{frame}

% Slide 6: Come Funziona in Pratica
\begin{frame}{Come Funziona nella Vita Reale?}
\textbf{Esempio: Alice vuole ricevere messaggi segreti}

\vspace{0.3cm}

\begin{enumerate}
\item Alice crea due chiavi:
\begin{itemize}
\item Chiave pubblica: la pubblica su Internet
\item Chiave privata: la tiene nel suo computer
\end{itemize}

\vspace{0.3cm}

\item Bob vuole scrivere ad Alice:
\begin{itemize}
\item Prende la chiave pubblica di Alice
\item Cifra il messaggio
\item Invia il messaggio cifrato
\end{itemize}

\vspace{0.3cm}

\item Alice riceve il messaggio cifrato:
\begin{itemize}
\item Usa la sua chiave privata
\item Decifra il messaggio
\item Legge il contenuto
\end{itemize}
\end{enumerate}

\vspace{0.3cm}

\textbf{Anche se un hacker intercetta il messaggio, non può decifrarlo!}
\end{frame}

% Slide 7: RSA - I Numeri Primi
\begin{frame}{RSA: La Base Matematica}
\textbf{RSA usa i numeri primi!}

\vspace{0.5cm}

\begin{block}{Cos'è un numero primo?}
Un numero divisibile solo per 1 e per se stesso.
\end{block}

\vspace{0.3cm}

\textbf{Esempi di numeri primi:}
\begin{center}
2, 3, 5, 7, 11, 13, 17, 19, 23, 29, 31, 37, 41, 43, 47...
\end{center}

\vspace{0.5cm}

\begin{alertblock}{Il Trucco di RSA}
\begin{itemize}
\item \textbf{Facile:} Moltiplicare due numeri primi grandi
\item \textbf{Difficilissimo:} Fattorizzare il risultato (trovare i due numeri originali)
\end{itemize}
\end{alertblock}


\end{frame}


\begin{frame}
	\vspace{0.3cm}
	
	\textbf{Esempio semplice:}
	\begin{itemize}
		\item Facile: $13 \times 17 = 221$
		\item Più difficile: $221 = ? \times ?$ (devi provare vari numeri)
	\end{itemize}
\end{frame}


% Slide 8: Un Esempio Semplificato
\begin{frame}{RSA: Un Esempio con Numeri Piccoli}
\textbf{Alice vuole creare le sue chiavi:}

\vspace{0.3cm}

\textbf{Passo 1:} Sceglie due numeri primi piccoli
\begin{itemize}
\item $p_1 = 3$ e $p_2 = 11$
\item Calcola: $q = 3 \times 11 = 33$
\end{itemize}

\vspace{0.3cm}

\textbf{Passo 2:} Sceglie due numeri speciali $s$ e $t$
\begin{itemize}
\item Sceglie $s = 3$ (chiave pubblica)
\item Calcola $t = 7$ (chiave privata)
\end{itemize}

\vspace{0.3cm}

\textbf{Risultato:}
\begin{itemize}
\item \alert{Chiave pubblica di Alice:} $(q=33, s=3)$ → pubblicata online
\item \alert{Chiave privata di Alice:} $t=7$ → segreta!
\end{itemize}

\vspace{0.3cm}

\textbf{Nota:} Nella realtà si usano numeri ENORMI (centinaia di cifre)!
\end{frame}

% Slide 9: Cifrare un Messaggio
\begin{frame}{Come Bob Cifra un Messaggio}
\textbf{Bob vuole inviare ad Alice il numero 4}

\vspace{0.5cm}

Bob usa la chiave pubblica di Alice: $(q=33, s=3)$

\vspace{0.3cm}

\textbf{Formula di cifratura:}
\[
\text{messaggio cifrato} = 4^3 \mod 33
\]

\vspace{0.3cm}

\textbf{Calcolo:}
\begin{align*}
4^3 &= 64 \\
64 \div 33 &= 1 \text{ resto } 31 \\
\text{Quindi: } 64 \mod 33 &= 31
\end{align*}

\vspace{0.5cm}

\begin{block}{Risultato}
Bob invia ad Alice il numero \alert{31}
\end{block}
\end{frame}

% Slide 10: Decifrare il Messaggio
\begin{frame}{Come Alice Decifra il Messaggio}
\textbf{Alice riceve il numero cifrato: 31}

\vspace{0.5cm}

Alice usa la sua chiave privata: $t=7$

\vspace{0.3cm}

\textbf{Formula di decifratura:}
\[
\text{messaggio originale} = 31^7 \mod 33
\]

\vspace{0.3cm}

\textbf{Calcolo (semplificato):}
\begin{align*}
31^7 \mod 33 &= \text{(calcolo complesso)} \\
&= 4
\end{align*}

\vspace{0.5cm}

\begin{block}{Magia!}
Alice recupera il messaggio originale: \alert{4}
\end{block}

\vspace{0.3cm}

\textbf{Perché funziona?} Grazie a proprietà matematiche speciali dei numeri primi!
\end{frame}

% Slide 11: Perché è Sicuro?
\begin{frame}{Perché RSA è Sicuro?}
\textbf{Un hacker intercetta:}
\begin{itemize}
\item Il messaggio cifrato: 31
\item La chiave pubblica: $(q=33, s=3)$
\end{itemize}

\vspace{0.5cm}

\textbf{Per decifrare, l'hacker deve:}
\begin{enumerate}
\item Fattorizzare 33 per trovare $3 \times 11$
\item Calcolare la chiave privata $t=7$
\end{enumerate}

\vspace{0.3cm}

\begin{alertblock}{Con numeri piccoli (33) è facile!}
Ma RSA usa numeri con centinaia di cifre...
\end{alertblock}

\vspace{0.5cm}

\textbf{Esempio reale:}
\begin{itemize}
\item $q$ ha circa 600 cifre (2048 bit)
\item Fattorizzarlo richiederebbe miliardi di anni anche ai supercomputer più potenti!
\end{itemize}
\end{frame}

% Slide 12: RSA nella Vita Reale
\begin{frame}{RSA nella Vita Quotidiana}
\textbf{Dove usi RSA senza saperlo?}

\vspace{0.5cm}

\begin{columns}
\column{0.5\textwidth}
\begin{block}{Browser Web (HTTPS)}
\begin{itemize}
\item Quando vedi il lucchetto 🔒
\item Protegge password e dati
\end{itemize}
\end{block}

\begin{block}{Email Sicure}
\begin{itemize}
\item PGP, S/MIME
\item Firma digitale
\end{itemize}
\end{block}

\column{0.5\textwidth}
\begin{block}{Messaggistica}
\begin{itemize}
\item WhatsApp, Signal
\item Crittografia end-to-end
\end{itemize}
\end{block}

\begin{block}{Bitcoin \& Crypto}
\begin{itemize}
\item Portafogli digitali
\item Transazioni sicure
\end{itemize}
\end{block}
\end{columns}

\vspace{0.5cm}

\begin{center}
\textbf{RSA protegge miliardi di transazioni ogni giorno!}
\end{center}
\end{frame}

% Slide 13: Punti di Forza e Limitazioni
\begin{frame}{Punti di Forza e Limitazioni di RSA}
\begin{columns}
\column{0.5\textwidth}
\begin{block}{✓ Punti di Forza}
\begin{itemize}
\item Molto sicuro (se ben implementato)
\item Non serve scambio di chiavi segrete
\item Permette firma digitale
\item Standard mondiale
\end{itemize}
\end{block}

\column{0.5\textwidth}
\begin{block}{⚠ Limitazioni}
\begin{itemize}
\item Più lento della crittografia simmetrica
\item Richiede chiavi lunghe (2048+ bit)
\item Vulnerabile ai computer quantistici (futuro)
\end{itemize}
\end{block}
\end{columns}

\vspace{0.5cm}

\begin{alertblock}{Nella Pratica}
Si usa spesso RSA + crittografia simmetrica insieme:
\begin{itemize}
\item RSA per scambiare una chiave segreta
\item Crittografia simmetrica (AES) per i dati veri e propri
\end{itemize}
\end{alertblock}
\end{frame}

% Slide 14: Conclusioni
\begin{frame}{Conclusioni}
\begin{block}{Cosa Abbiamo Imparato}
\begin{itemize}
\item La crittografia moderna usa \alert{due chiavi}: pubblica e privata
\item RSA si basa sulla difficoltà di \alert{fattorizzare numeri grandi}
\item Cifrare è facile, decifrare senza la chiave è quasi impossibile
\item RSA protegge la nostra vita digitale quotidiana
\end{itemize}
\end{block}

\vspace{0.5cm}

\begin{center}
\textbf{Chi ha inventato RSA?}

\vspace{0.3cm}

\textbf{Rivest, Shamir, Adleman (1977)}

\vspace{0.3cm}

Il nome RSA viene dalle iniziali dei tre inventori!
\end{center}

\begin{alertblock}{Messaggio Finale}
La matematica che studiate non è solo teoria: protegge il mondo digitale!
\end{alertblock}
\end{frame}

% Slide 15: Domande?
\begin{frame}
\begin{center}
{\Huge Domande?}



\vspace{1cm}

\texttt{fedeli.massimo@iisfermisacconiceciap.edu.it}
\end{center}
\end{frame}

\end{document}
