\documentclass{beamer}
\usepackage[utf8]{inputenc}
\usepackage[italian]{babel}
\usepackage{amsmath}
\usepackage{amssymb}
\usepackage{array}
\usepackage{booktabs}

\usetheme{Madrid}
\usecolortheme{default}

\title{Circuiti Logici Digitali}
\subtitle{Operatori Logici e Porte Logiche}
\author{Corso di Scienze e Tecnologie Applicate}
\institute{IIS Fermi Sacconi Ceci}
\date{\today}

\begin{document}
	
	% Slide 1: Titolo
	\frame{\titlepage}
	
	% Slide 2: Indice
	\begin{frame}
		\frametitle{Indice}
		\tableofcontents
	\end{frame}
	
	% Slide 3: Introduzione
	\section{Introduzione}
	\begin{frame}
		\frametitle{Introduzione alla Logica Digitale}
		\begin{itemize}
			\item La logica formale studia le proposizioni dichiarative
			\item Una proposizione può essere \textbf{Vera} o \textbf{Falsa}
			\item Rappresentazione numerica:
			\begin{itemize}
				\item Falso = 0 = 0V
				\item Vero = 1 = tensione di alimentazione
			\end{itemize}
			\item Le variabili logiche possono assumere solo valori booleani
			\item Gli operatori logici connettono i valori tra loro
		\end{itemize}
	\end{frame}
	
	% Slide 4: Operatori Unari
	\section{Operatori Logici}
	\begin{frame}
		\frametitle{Operatori Unari}
		\begin{block}{Operatore NOT}
			\begin{itemize}
				\item Interviene su un solo valore logico
				\item Inverte il valore in ingresso
				\item Simbolo: NOT, $\neg$, ' (apice)
			\end{itemize}
		\end{block}
		
		\begin{center}
			\begin{tabular}{c|c}
				\hline
				A & NOT A \\
				\hline
				0 & 1 \\
				1 & 0 \\
				\hline
			\end{tabular}
		\end{center}
		
		\textbf{Esempio:} Se A = ``Antonio mangia'', NOT A = ``Antonio non mangia''
	\end{frame}
	
	% Slide 5: Connettivo AND
	\begin{frame}
		\frametitle{Connettivo AND}
		\begin{block}{Operatore AND}
			\begin{itemize}
				\item Opera su due o più ingressi
				\item Restituisce Vero solo se \textbf{tutti} gli ingressi sono Vero
				\item Simboli: AND, $\land$, $\cdot$
			\end{itemize}
		\end{block}
		
		\begin{center}
			\begin{tabular}{cc|c}
				\hline
				A & B & A AND B \\
				\hline
				0 & 0 & 0 \\
				0 & 1 & 0 \\
				1 & 0 & 0 \\
				1 & 1 & 1 \\
				\hline
			\end{tabular}
		\end{center}
		
		\textbf{Esempio:} ``Antonio mangia E Piero legge''
	\end{frame}
	
	% Slide 6: Connettivo OR
	\begin{frame}
		\frametitle{Connettivo OR}
		\begin{block}{Operatore OR}
			\begin{itemize}
				\item Opera su due o più ingressi
				\item Restituisce Vero se \textbf{almeno uno} degli ingressi è Vero
				\item Simboli: OR, $\lor$, +
			\end{itemize}
		\end{block}
		
		\begin{center}
			\begin{tabular}{cc|c}
				\hline
				A & B & A OR B \\
				\hline
				0 & 0 & 0 \\
				0 & 1 & 1 \\
				1 & 0 & 1 \\
				1 & 1 & 1 \\
				\hline
			\end{tabular}
		\end{center}
		
		\textbf{Esempio:} ``Antonio mangia E/O Piero legge''
	\end{frame}
	
	% Slide 7: Connettivo XOR
	\begin{frame}
		\frametitle{Connettivo XOR (OR Esclusivo)}
		\begin{block}{Operatore XOR}
			\begin{itemize}
				\item Opera su due ingressi
				\item Restituisce Vero se \textbf{solo uno} degli ingressi è Vero
				\item Simboli: XOR, $\oplus$
			\end{itemize}
		\end{block}
		
		\begin{center}
			\begin{tabular}{cc|c}
				\hline
				A & B & A XOR B \\
				\hline
				0 & 0 & 0 \\
				0 & 1 & 1 \\
				1 & 0 & 1 \\
				1 & 1 & 0 \\
				\hline
			\end{tabular}
		\end{center}
		
		\textbf{Esempio:} ``Antonio mangia OPPURE Piero legge'' (ma non entrambi)
	\end{frame}
	
	% Slide 8: Altri operatori
	\begin{frame}
		\frametitle{Altri Operatori Logici}
		\begin{columns}[T]
			\column{0.5\textwidth}
			\textbf{NAND} (NOT AND)
			\begin{center}
				\begin{tabular}{cc|c}
					\hline
					A & B & NAND \\
					\hline
					0 & 0 & 1 \\
					0 & 1 & 1 \\
					1 & 0 & 1 \\
					1 & 1 & 0 \\
					\hline
				\end{tabular}
			\end{center}
			
			\column{0.5\textwidth}
			\textbf{NOR} (NOT OR)
			\begin{center}
				\begin{tabular}{cc|c}
					\hline
					A & B & NOR \\
					\hline
					0 & 0 & 1 \\
					0 & 1 & 0 \\
					1 & 0 & 0 \\
					1 & 1 & 0 \\
					\hline
				\end{tabular}
			\end{center}
		\end{columns}
		
		\vspace{0.5cm}
		\textbf{Nota:} NAND e NOR sono operatori universali: qualsiasi funzione logica può essere realizzata usando solo porte NAND o solo porte NOR.
	\end{frame}
	
	% Slide 9: Ordine di precedenza
	\begin{frame}
		\frametitle{Ordine di Precedenza}
		\begin{block}{Regole di Valutazione}
			Le espressioni logiche seguono un ordine di precedenza:
			\begin{enumerate}
				\item \textbf{NOT} (negazione)
				\item \textbf{AND} (prodotto logico)
				\item \textbf{OR} (somma logica)
			\end{enumerate}
		\end{block}
		
		\textbf{Esempi:}
		\begin{itemize}
			\item $a + b \cdot c'$ si legge come $a + (b \cdot (c'))$
			\item Le parentesi modificano l'ordine di precedenza
			\item $A \cdot B + C$ significa $(A \cdot B) + C$
		\end{itemize}
	\end{frame}
	
	% Slide 10: Valori irrilevanti
	\section{Tabelle di Verità}
	\begin{frame}
		\frametitle{Valori Irrilevanti nelle Tabelle di Verità}
		\begin{itemize}
			\item In alcune situazioni, il valore di certe variabili non influenza il risultato
			\item Si indica con ``X'' o ``-'' (don't care)
			\item Utile per semplificare i circuiti
		\end{itemize}
		
		\textbf{Esempio:} $A \cdot (B + C)$
		
		\begin{columns}[T]
			\column{0.5\textwidth}
			\centering
			Tabella completa
			\begin{tabular}{ccc|c}
				\hline
				A & B & C & Risultato \\
				\hline
				0 & 0 & 0 & 0 \\
				0 & 0 & 1 & 0 \\
				0 & 1 & 0 & 0 \\
				0 & 1 & 1 & 0 \\
				1 & 0 & 0 & 0 \\
				1 & 0 & 1 & 1 \\
				1 & 1 & 0 & 1 \\
				1 & 1 & 1 & 1 \\
				\hline
			\end{tabular}
			
			\column{0.5\textwidth}
			\centering
			Tabella semplificata
			\begin{tabular}{ccc|c}
				\hline
				A & B & C & Risultato \\
				\hline
				0 & X & X & 0 \\
				1 & 0 & 0 & 0 \\
				1 & X & 1 & 1 \\
				1 & 1 & X & 1 \\
				\hline
			\end{tabular}
		\end{columns}
	\end{frame}
	
	% Slide 11: Equivalenze fondamentali
	\section{Algebra di Boole}
	\begin{frame}
		\frametitle{Equivalenze dell'Algebra di Boole (1/2)}
		\begin{block}{Proprietà Fondamentali}
			\begin{columns}[T]
				\column{0.5\textwidth}
				\begin{align*}
					a + 0 &= a \\
					a + 1 &= 1 \\
					a + a &= a \\
					a + a' &= 1 \\
					a + b &= b + a
				\end{align*}
				
				\column{0.5\textwidth}
				\begin{align*}
					a \cdot 0 &= 0 \\
					a \cdot 1 &= a \\
					a \cdot a &= a \\
					a \cdot a' &= 0 \\
					a \cdot b &= b \cdot a
				\end{align*}
			\end{columns}
		\end{block}
	\end{frame}
	
	% Slide 12: Teoremi di De Morgan
	\begin{frame}
		\frametitle{Teoremi di De Morgan}
		\begin{block}{Teoremi Fondamentali}
			\Large
			\begin{align*}
				(a + b)' &= a' \cdot b' \\[0.5cm]
				(a \cdot b)' &= a' + b'
			\end{align*}
		\end{block}
		
		\textbf{Applicazione:} Trasformare funzioni tra forme diverse
		\begin{itemize}
			\item Convertire OR in AND (e viceversa)
			\item Utile per implementare circuiti con un solo tipo di porta
			\item Esempio: realizzare circuiti con sole porte NAND
		\end{itemize}
	\end{frame}
	
	% Slide 13: Proprietà aggiuntive
	\begin{frame}
		\frametitle{Altre Proprietà dell'Algebra di Boole}
		\begin{block}{Associatività}
			\begin{align*}
				a + (b + c) &= (a + b) + c \\
				a \cdot (b \cdot c) &= (a \cdot b) \cdot c
			\end{align*}
		\end{block}
		
		\begin{block}{Distributività}
			\begin{align*}
				a \cdot (b + c) &= a \cdot b + a \cdot c \\
				a + (b \cdot c) &= (a + b) \cdot (a + c)
			\end{align*}
		\end{block}
		
		\begin{block}{Doppia Negazione}
			$$(a')' = a$$
		\end{block}
	\end{frame}
	
	% Slide 14: Somma dei prodotti
	\section{Forme Canoniche}
	\begin{frame}
		\frametitle{Somma dei Prodotti (SOP)}
		\begin{itemize}
			\item Ogni funzione logica può essere espressa come somma di prodotti
			\item Si parte dalla tabella di verità
			\item Si considerano solo le righe con output = 1
			\item Si crea un prodotto (AND) per ogni riga
			\item Si sommano (OR) tutti i prodotti
		\end{itemize}
		
		\textbf{Esempio:} Funzione NXOR
		\begin{center}
			\begin{tabular}{cc|c}
				\hline
				A & B & NXOR \\
				\hline
				0 & 0 & 1 \\
				0 & 1 & 0 \\
				1 & 0 & 0 \\
				1 & 1 & 1 \\
				\hline
			\end{tabular}
		\end{center}
		
		$$F = A' \cdot B' + A \cdot B$$
	\end{frame}
	
	% Slide 15: Mintermini
	\begin{frame}
		\frametitle{Mintermini e Prodotti Fondamentali}
		\begin{block}{Definizioni}
			\begin{itemize}
				\item \textbf{Letterale}: una variabile o la sua negazione (es. A, A')
				\item \textbf{Mintermine}: prodotto di letterali per tutte le variabili
				\item Ogni combinazione di input ha un mintermine corrispondente
			\end{itemize}
		\end{block}
		
		\textbf{Mintermini per 2 variabili:}
		\begin{center}
			\begin{tabular}{cc|c|c}
				\hline
				A & B & Mintermine & Notazione \\
				\hline
				0 & 0 & $A' \cdot B'$ & $m_0$ \\
				0 & 1 & $A' \cdot B$ & $m_1$ \\
				1 & 0 & $A \cdot B'$ & $m_2$ \\
				1 & 1 & $A \cdot B$ & $m_3$ \\
				\hline
			\end{tabular}
		\end{center}
	\end{frame}
	
	% Slide 16: Introduzione alle mappe di Karnaugh
	\section{Mappe di Karnaugh}
	\begin{frame}
		\frametitle{Introduzione alle Mappe di Karnaugh}
		\begin{itemize}
			\item Metodo grafico per semplificare funzioni logiche
			\item Alternativa all'algebra di Boole
			\item Rappresentazione bidimensionale dei prodotti fondamentali
			\item Efficace fino a 4 variabili
			\item Per più variabili serve rappresentazione multidimensionale
		\end{itemize}
		
		\textbf{Vantaggi:}
		\begin{itemize}
			\item Visualizzazione immediata
			\item Identificazione rapida di semplificazioni
			\item Riduzione del numero di porte logiche necessarie
		\end{itemize}
	\end{frame}
	
	% Slide 17: Mappa a 2 variabili
	\begin{frame}
		\frametitle{Mappa di Karnaugh a 2 Variabili}
		\begin{columns}[T]
			\column{0.5\textwidth}
			\textbf{Struttura:}
			\begin{center}
				\begin{tabular}{c|c|c|}
					\multicolumn{1}{c}{} & \multicolumn{1}{c}{$B'$} & \multicolumn{1}{c}{$B$} \\
					\cline{2-3}
					$A'$ & $A'B'$ & $A'B$ \\
					\cline{2-3}
					$A$ & $AB'$ & $AB$ \\
					\cline{2-3}
				\end{tabular}
			\end{center}
			
			\column{0.5\textwidth}
			\textbf{Esempio: XOR}
			\begin{center}
				\begin{tabular}{c|c|c|}
					\multicolumn{1}{c}{} & \multicolumn{1}{c}{$B'$} & \multicolumn{1}{c}{$B$} \\
					\cline{2-3}
					$A'$ & 0 & 1 \\
					\cline{2-3}
					$A$ & 1 & 0 \\
					\cline{2-3}
				\end{tabular}
			\end{center}
			
			$$F = A \oplus B$$
		\end{columns}
		
		\vspace{0.5cm}
		\textbf{Nota:} Celle adiacenti differiscono per una sola variabile
	\end{frame}
	
	% Slide 18: Mappa a 3 variabili
	\begin{frame}
		\frametitle{Mappa di Karnaugh a 3 Variabili}
		\textbf{Attenzione all'ordine delle colonne: 00, 01, 11, 10 (Codice Gray)}
		
		\begin{center}
			\begin{tabular}{c|c|c|c|c|}
				\multicolumn{1}{c}{} & \multicolumn{1}{c}{$B'C'$} & \multicolumn{1}{c}{$B'C$} & \multicolumn{1}{c}{$BC$} & \multicolumn{1}{c}{$BC'$} \\
				\cline{2-5}
				$A'$ & $A'B'C'$ & $A'B'C$ & $A'BC$ & $A'BC'$ \\
				\cline{2-5}
				$A$ & $AB'C'$ & $AB'C$ & $ABC$ & $ABC'$ \\
				\cline{2-5}
			\end{tabular}
		\end{center}
		
		\textbf{Procedimento di semplificazione:}
		\begin{enumerate}
			\item Inserire 1 nelle celle corrispondenti ai mintermini
			\item Raggruppare celle adiacenti con 1 (gruppi di 1, 2, 4, 8)
			\item Ogni gruppo elimina una variabile
			\item Scrivere l'espressione minimizzata
		\end{enumerate}
	\end{frame}
	
	% Slide 19: Mappa a 4 variabili
	\begin{frame}
		\frametitle{Mappa di Karnaugh a 4 Variabili}
		\begin{center}
			\small
			\begin{tabular}{c|c|c|c|c|}
				\multicolumn{1}{c}{} & \multicolumn{1}{c}{$C'D'$} & \multicolumn{1}{c}{$C'D$} & \multicolumn{1}{c}{$CD$} & \multicolumn{1}{c}{$CD'$} \\
				\cline{2-5}
				$A'B'$ & 0 & 1 & 3 & 2 \\
				\cline{2-5}
				$A'B$ & 4 & 5 & 7 & 6 \\
				\cline{2-5}
				$AB$ & 12 & 13 & 15 & 14 \\
				\cline{2-5}
				$AB'$ & 8 & 9 & 11 & 10 \\
				\cline{2-5}
			\end{tabular}
		\end{center}
		
		\textbf{Regole di raggruppamento:}
		\begin{itemize}
			\item I gruppi devono essere rettangolari
			\item Dimensioni: 1, 2, 4, 8, 16 celle
			\item La mappa è toroidale: i bordi sono adiacenti
			\item Massimizzare la dimensione dei gruppi
		\end{itemize}
	\end{frame}
	
	% Slide 20: Esempi di semplificazione
	\begin{frame}
		\frametitle{Esempi di Semplificazione}
		\textbf{Regole importanti:}
		\begin{itemize}
			\item Gruppo di 2 celle adiacenti $\rightarrow$ elimina 1 variabile
			\item Gruppo di 4 celle adiacenti $\rightarrow$ elimina 2 variabili
			\item Gruppo di 8 celle adiacenti $\rightarrow$ elimina 3 variabili
		\end{itemize}
		
		\textbf{Processo:}
		\begin{enumerate}
			\item Formare i gruppi più grandi possibili
			\item Ogni 1 deve essere coperto da almeno un gruppo
			\item I gruppi possono sovrapporsi
			\item Minimizzare il numero totale di gruppi
		\end{enumerate}
		
		\textbf{Risultato:} Espressione logica minimizzata = somma dei termini rappresentati dai gruppi
	\end{frame}
	
	% Slide 21: Condizioni indifferenti
	\begin{frame}
		\frametitle{Condizioni Indifferenti (Don't Care)}
		\begin{itemize}
			\item In alcune situazioni, certe combinazioni di input non si verificano mai
			\item Oppure il valore di output è irrilevante
			\item Si indicano con ``X'' nella mappa
			\item Possono essere considerati come 0 o 1 per ottenere la migliore semplificazione
		\end{itemize}
		
		\textbf{Vantaggio:}
		\begin{itemize}
			\item Permettono ulteriori semplificazioni
			\item Riducono il numero di porte necessarie
			\item Ottimizzazione del circuito
		\end{itemize}
	\end{frame}
	
	% Slide 22: Mappe con XOR
	\begin{frame}
		\frametitle{Mappe di Karnaugh con Funzione XOR}
		\begin{block}{Pattern XOR nelle mappe}
			Le funzioni XOR hanno pattern caratteristici:
			\begin{itemize}
				\item Distribuzione a ``scacchiera''
				\item Nessun raggruppamento efficace possibile
				\item Indicazione che la funzione contiene XOR
			\end{itemize}
		\end{block}
		
		\textbf{Esempio XOR a 2 variabili:}
		\begin{center}
			\begin{tabular}{c|c|c|}
				\multicolumn{1}{c}{} & \multicolumn{1}{c}{$B'$} & \multicolumn{1}{c}{$B$} \\
				\cline{2-3}
				$A'$ & 0 & 1 \\
				\cline{2-3}
				$A$ & 1 & 0 \\
				\cline{2-3}
			\end{tabular}
		\end{center}
		
		$$F = A \oplus B = A'B + AB'$$
		
		La presenza di XOR rende difficile la semplificazione con Karnaugh
	\end{frame}
	
	% Slide 23: Applicazioni pratiche
	\section{Applicazioni}
	\begin{frame}
		\frametitle{Applicazioni dei Circuiti Combinatori}
		\begin{block}{Principali Applicazioni}
			\begin{itemize}
				\item \textbf{Decodificatori}: selezionano una linea di uscita in base all'input
				\item \textbf{Multiplexer}: selezionano uno tra più ingressi
				\item \textbf{Demultiplexer}: indirizzano un input verso una delle uscite
				\item \textbf{Addizionatori}: somma di numeri binari
				\item \textbf{Comparatori}: confronto tra valori
				\item \textbf{Unità logiche (ALU)}: operazioni aritmetiche e logiche
			\end{itemize}
		\end{block}
	\end{frame}
	
	% Slide 24: Vantaggi della minimizzazione
	\begin{frame}
		\frametitle{Vantaggi della Minimizzazione}
		\begin{block}{Perché semplificare i circuiti?}
			\begin{itemize}
				\item \textbf{Costo}: meno componenti $\rightarrow$ costo inferiore
				\item \textbf{Spazio}: circuiti più compatti
				\item \textbf{Velocità}: meno porte $\rightarrow$ minor ritardo di propagazione
				\item \textbf{Consumo}: minore dissipazione di potenza
				\item \textbf{Affidabilità}: meno componenti $\rightarrow$ minor probabilità di guasto
			\end{itemize}
		\end{block}
		
		\textbf{Obiettivo:} Trovare il miglior compromesso tra:
		\begin{itemize}
			\item Numero di porte logiche
			\item Profondità del circuito (livelli di porte)
			\item Numero di connessioni
		\end{itemize}
	\end{frame}
	
	% Slide 25: Conclusioni
	\begin{frame}
		\frametitle{Conclusioni}
		\begin{block}{Riepilogo}
			\begin{itemize}
				\item Gli operatori logici sono i mattoni dei circuiti digitali
				\item L'algebra di Boole fornisce le regole di manipolazione
				\item Le mappe di Karnaugh offrono un metodo grafico di semplificazione
				\item La minimizzazione è fondamentale per circuiti efficienti
			\end{itemize}
		\end{block}
		
		\textbf{Prossimi passi:}
		\begin{itemize}
			\item Circuiti combinatori complessi
			\item Circuiti sequenziali (con memoria)
			\item Flip-flop e registri
			\item Progettazione di sistemi digitali
		\end{itemize}
	\end{frame}
	
	% Slide 26: Bibliografia
	\begin{frame}
		\frametitle{Bibliografia e Risorse}
		\begin{itemize}
			\item Materiale didattico del corso
			\item ``Appunti di Informatica Libera'' - Daniele Giacomini
			\item Simulatori di circuiti logici: Tkgate, Logisim
			\item Risorse online per esercitazioni
		\end{itemize}
		
		\vspace{1cm}
		\centering
		\Large{Grazie per l'attenzione!}
	\end{frame}
	
\end{document}