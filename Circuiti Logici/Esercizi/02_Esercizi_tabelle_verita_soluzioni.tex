\documentclass[a4paper,11pt]{article}
\usepackage[utf8]{inputenc}
\usepackage[italian]{babel}
\usepackage{amsmath}
\usepackage{amssymb}
\usepackage{geometry}
\usepackage{fancyhdr}
\usepackage{enumitem}
\usepackage{array}
\usepackage{booktabs}
\usepackage{xcolor}
\usepackage{colortbl}

\geometry{margin=2cm}
\pagestyle{fancy}
\fancyhf{}
\fancyhead[L]{Soluzioni: Tabelle di Verità}
\fancyhead[R]{IIS Fermi Sacconi Ceci}
\fancyfoot[C]{\thepage}

\title{\textbf{Soluzioni Complete}\\ \large Esercizi sulle Tabelle di Verità}
\author{Prof. Massimo Fedeli}
\date{}

\begin{document}

\maketitle

\section*{Esercizio 1}
\subsection*{Funzione}
\[ F(A, B) = A \cdot B + \overline{A} \]

\subsection*{Procedimento}
La funzione è composta da due termini in OR:
\begin{itemize}
    \item $A \cdot B$ (prodotto logico di A e B)
    \item $\overline{A}$ (negazione di A)
\end{itemize}

La funzione è vera quando almeno uno dei due termini è vero.

\subsection*{Tabella di Verità Completa}
\begin{center}
\begin{tabular}{|c|c|c|c|c|}
\hline
\rowcolor{lightgray}
\textbf{A} & \textbf{B} & $A \cdot B$ & $\overline{A}$ & \textbf{F(A,B)} \\
\hline
0 & 0 & 0 & 1 & \textbf{1} \\
\hline
0 & 1 & 0 & 1 & \textbf{1} \\
\hline
1 & 0 & 0 & 0 & \textbf{0} \\
\hline
1 & 1 & 1 & 0 & \textbf{1} \\
\hline
\end{tabular}
\end{center}

\subsection*{Analisi del Risultato}
La funzione è falsa solo quando $A=1$ e $B=0$. In tutti gli altri casi è vera.

\vspace{1cm}

\section*{Esercizio 2}
\subsection*{Funzione}
\[ F(A, B) = \overline{A \cdot B} \]

\subsection*{Procedimento}
Questa è la funzione \textbf{NAND} (NOT-AND):
\begin{enumerate}
    \item Si calcola prima il prodotto $A \cdot B$
    \item Poi si nega il risultato
\end{enumerate}

\subsection*{Tabella di Verità Completa}
\begin{center}
\begin{tabular}{|c|c|c|c|}
\hline
\rowcolor{lightgray}
\textbf{A} & \textbf{B} & $A \cdot B$ & \textbf{F(A,B)} \\
\hline
0 & 0 & 0 & \textbf{1} \\
\hline
0 & 1 & 0 & \textbf{1} \\
\hline
1 & 0 & 0 & \textbf{1} \\
\hline
1 & 1 & 1 & \textbf{0} \\
\hline
\end{tabular}
\end{center}

\subsection*{Analisi del Risultato}
La funzione NAND è vera quando almeno uno dei due input è falso. È l'opposto della funzione AND.

\newpage

\section*{Esercizio 3}
\subsection*{Funzione}
\[ F(A, B, C) = A \cdot B + B \cdot C \]

\subsection*{Procedimento}
La funzione è composta da due prodotti in OR:
\begin{itemize}
    \item $A \cdot B$ (AND tra A e B)
    \item $B \cdot C$ (AND tra B e C)
\end{itemize}

Si nota che la variabile B è presente in entrambi i termini.

\subsection*{Tabella di Verità Completa}
\begin{center}
\begin{tabular}{|c|c|c|c|c|c|}
\hline
\rowcolor{lightgray}
\textbf{A} & \textbf{B} & \textbf{C} & $A \cdot B$ & $B \cdot C$ & \textbf{F(A,B,C)} \\
\hline
0 & 0 & 0 & 0 & 0 & \textbf{0} \\
\hline
0 & 0 & 1 & 0 & 0 & \textbf{0} \\
\hline
0 & 1 & 0 & 0 & 0 & \textbf{0} \\
\hline
0 & 1 & 1 & 0 & 1 & \textbf{1} \\
\hline
1 & 0 & 0 & 0 & 0 & \textbf{0} \\
\hline
1 & 0 & 1 & 0 & 0 & \textbf{0} \\
\hline
1 & 1 & 0 & 1 & 0 & \textbf{1} \\
\hline
1 & 1 & 1 & 1 & 1 & \textbf{1} \\
\hline
\end{tabular}
\end{center}

\subsection*{Analisi del Risultato}
La funzione è vera in 3 casi su 8:
\begin{itemize}
    \item Quando $B=1$ e $C=1$ (indipendentemente da A)
    \item Quando $A=1$, $B=1$ e $C=0$
    \item Quando tutti sono a 1
\end{itemize}

Si può semplificare come: $F = B \cdot (A + C)$

\vspace{1cm}

\section*{Esercizio 4}
\subsection*{Funzione}
\[ F(A, B, C) = A \cdot (\overline{B} + C) \]

\subsection*{Procedimento}
La funzione richiede:
\begin{enumerate}
    \item Calcolare $\overline{B}$ (negazione di B)
    \item Calcolare $\overline{B} + C$ (OR tra $\overline{B}$ e C)
    \item Moltiplicare il risultato per A
\end{enumerate}

La funzione è vera solo quando A è vero E almeno una tra $\overline{B}$ o C è vera.

\subsection*{Tabella di Verità Completa}
\begin{center}
\begin{tabular}{|c|c|c|c|c|c|}
\hline
\rowcolor{lightgray}
\textbf{A} & \textbf{B} & \textbf{C} & $\overline{B}$ & $\overline{B} + C$ & \textbf{F(A,B,C)} \\
\hline
0 & 0 & 0 & 1 & 1 & \textbf{0} \\
\hline
0 & 0 & 1 & 1 & 1 & \textbf{0} \\
\hline
0 & 1 & 0 & 0 & 0 & \textbf{0} \\
\hline
0 & 1 & 1 & 0 & 1 & \textbf{0} \\
\hline
1 & 0 & 0 & 1 & 1 & \textbf{1} \\
\hline
1 & 0 & 1 & 1 & 1 & \textbf{1} \\
\hline
1 & 1 & 0 & 0 & 0 & \textbf{0} \\
\hline
1 & 1 & 1 & 0 & 1 & \textbf{1} \\
\hline
\end{tabular}
\end{center}

\subsection*{Analisi del Risultato}
La funzione è vera solo quando $A=1$ E ($B=0$ OPPURE $C=1$). Applicando De Morgan: $F = A \cdot \overline{B \cdot \overline{C}}$

\newpage

\section*{Esercizio 5}
\subsection*{Funzione}
\[ F(A, B, C) = \overline{A + B} \cdot C \]

\subsection*{Procedimento}
\textbf{Attenzione}: La negazione si applica all'intero termine $(A + B)$:
\begin{enumerate}
    \item Calcolare $A + B$ (OR tra A e B)
    \item Negare il risultato: $\overline{A + B}$
    \item Moltiplicare per C
\end{enumerate}

Per la legge di De Morgan: $\overline{A + B} = \overline{A} \cdot \overline{B}$

\subsection*{Tabella di Verità Completa}
\begin{center}
\begin{tabular}{|c|c|c|c|c|c|}
\hline
\rowcolor{lightgray}
\textbf{A} & \textbf{B} & \textbf{C} & $A + B$ & $\overline{A + B}$ & \textbf{F(A,B,C)} \\
\hline
0 & 0 & 0 & 0 & 1 & \textbf{0} \\
\hline
0 & 0 & 1 & 0 & 1 & \textbf{1} \\
\hline
0 & 1 & 0 & 1 & 0 & \textbf{0} \\
\hline
0 & 1 & 1 & 1 & 0 & \textbf{0} \\
\hline
1 & 0 & 0 & 1 & 0 & \textbf{0} \\
\hline
1 & 0 & 1 & 1 & 0 & \textbf{0} \\
\hline
1 & 1 & 0 & 1 & 0 & \textbf{0} \\
\hline
1 & 1 & 1 & 1 & 0 & \textbf{0} \\
\hline
\end{tabular}
\end{center}

\subsection*{Analisi del Risultato}
La funzione è vera in un solo caso: quando $A=0$, $B=0$ e $C=1$.

Forma equivalente: $F = \overline{A} \cdot \overline{B} \cdot C$

\vspace{1cm}

\section*{Esercizio 6}
\subsection*{Funzione}
\[ F(A, B, C) = A \cdot \overline{B} \cdot C + \overline{A} \cdot B \cdot \overline{C} \]

\subsection*{Procedimento}
Questa è una funzione in \textbf{forma canonica SOP} (Sum of Products):
\begin{itemize}
    \item Primo termine: $A \cdot \overline{B} \cdot C$ (vero quando $A=1$, $B=0$, $C=1$)
    \item Secondo termine: $\overline{A} \cdot B \cdot \overline{C}$ (vero quando $A=0$, $B=1$, $C=0$)
\end{itemize}

La funzione è vera quando almeno uno dei due termini è vero.

\subsection*{Tabella di Verità Completa}
\begin{center}
\begin{tabular}{|c|c|c|c|c|c|}
\hline
\rowcolor{lightgray}
\textbf{A} & \textbf{B} & \textbf{C} & $A \cdot \overline{B} \cdot C$ & $\overline{A} \cdot B \cdot \overline{C}$ & \textbf{F(A,B,C)} \\
\hline
0 & 0 & 0 & 0 & 0 & \textbf{0} \\
\hline
0 & 0 & 1 & 0 & 0 & \textbf{0} \\
\hline
0 & 1 & 0 & 0 & 1 & \textbf{1} \\
\hline
0 & 1 & 1 & 0 & 0 & \textbf{0} \\
\hline
1 & 0 & 0 & 0 & 0 & \textbf{0} \\
\hline
1 & 0 & 1 & 1 & 0 & \textbf{1} \\
\hline
1 & 1 & 0 & 0 & 0 & \textbf{0} \\
\hline
1 & 1 & 1 & 0 & 0 & \textbf{0} \\
\hline
\end{tabular}
\end{center}

\subsection*{Analisi del Risultato}
La funzione è vera in esattamente 2 casi su 8:
\begin{itemize}
    \item Riga 3: $(0,1,0)$ - mintermino $\overline{A} \cdot B \cdot \overline{C}$
    \item Riga 6: $(1,0,1)$ - mintermino $A \cdot \overline{B} \cdot C$
\end{itemize}

Notazione decimale: $F = \sum m(2,5)$

\newpage

\section*{Esercizio 7}
\subsection*{Funzione}
\[ F(A, B, C, D) = A \cdot B + C \cdot D \]

\subsection*{Procedimento}
Con 4 variabili abbiamo 16 combinazioni possibili ($2^4=16$).

La funzione è vera quando:
\begin{itemize}
    \item $A \cdot B = 1$ (A e B entrambi a 1), oppure
    \item $C \cdot D = 1$ (C e D entrambi a 1), oppure
    \item Entrambe le condizioni
\end{itemize}

\subsection*{Tabella di Verità Completa}
\begin{center}
\begin{tabular}{|c|c|c|c|c|c|c|}
\hline
\rowcolor{lightgray}
\textbf{A} & \textbf{B} & \textbf{C} & \textbf{D} & $A \cdot B$ & $C \cdot D$ & \textbf{F(A,B,C,D)} \\
\hline
0 & 0 & 0 & 0 & 0 & 0 & \textbf{0} \\
\hline
0 & 0 & 0 & 1 & 0 & 0 & \textbf{0} \\
\hline
0 & 0 & 1 & 0 & 0 & 0 & \textbf{0} \\
\hline
0 & 0 & 1 & 1 & 0 & 1 & \textbf{1} \\
\hline
0 & 1 & 0 & 0 & 0 & 0 & \textbf{0} \\
\hline
0 & 1 & 0 & 1 & 0 & 0 & \textbf{0} \\
\hline
0 & 1 & 1 & 0 & 0 & 0 & \textbf{0} \\
\hline
0 & 1 & 1 & 1 & 0 & 1 & \textbf{1} \\
\hline
1 & 0 & 0 & 0 & 0 & 0 & \textbf{0} \\
\hline
1 & 0 & 0 & 1 & 0 & 0 & \textbf{0} \\
\hline
1 & 0 & 1 & 0 & 0 & 0 & \textbf{0} \\
\hline
1 & 0 & 1 & 1 & 0 & 1 & \textbf{1} \\
\hline
1 & 1 & 0 & 0 & 1 & 0 & \textbf{1} \\
\hline
1 & 1 & 0 & 1 & 1 & 0 & \textbf{1} \\
\hline
1 & 1 & 1 & 0 & 1 & 0 & \textbf{1} \\
\hline
1 & 1 & 1 & 1 & 1 & 1 & \textbf{1} \\
\hline
\end{tabular}
\end{center}

\subsection*{Analisi del Risultato}
La funzione è vera in 7 casi su 16:
\begin{itemize}
    \item Righe 4, 8, 12: $C=1$ e $D=1$ (ma $A \cdot B = 0$)
    \item Righe 13, 14, 15: $A=1$ e $B=1$ (con qualsiasi valore di C e D)
    \item Riga 16: Tutti a 1
\end{itemize}

\newpage

\section*{Esercizio 8}
\subsection*{Funzione}
\[ F(A, B, C, D) = \overline{A \cdot B} \cdot (C + D) \]

\subsection*{Procedimento}
Questa funzione combina NAND e OR:
\begin{enumerate}
    \item Calcolare $A \cdot B$ (AND)
    \item Negare: $\overline{A \cdot B}$ (NAND)
    \item Calcolare $C + D$ (OR)
    \item Moltiplicare i due risultati (AND finale)
\end{enumerate}

\subsection*{Tabella di Verità Completa}
\begin{center}
\begin{tabular}{|c|c|c|c|c|c|c|}
\hline
\rowcolor{lightgray}
\textbf{A} & \textbf{B} & \textbf{C} & \textbf{D} & $\overline{A \cdot B}$ & $C + D$ & \textbf{F(A,B,C,D)} \\
\hline
0 & 0 & 0 & 0 & 1 & 0 & \textbf{0} \\
\hline
0 & 0 & 0 & 1 & 1 & 1 & \textbf{1} \\
\hline
0 & 0 & 1 & 0 & 1 & 1 & \textbf{1} \\
\hline
0 & 0 & 1 & 1 & 1 & 1 & \textbf{1} \\
\hline
0 & 1 & 0 & 0 & 1 & 0 & \textbf{0} \\
\hline
0 & 1 & 0 & 1 & 1 & 1 & \textbf{1} \\
\hline
0 & 1 & 1 & 0 & 1 & 1 & \textbf{1} \\
\hline
0 & 1 & 1 & 1 & 1 & 1 & \textbf{1} \\
\hline
1 & 0 & 0 & 0 & 1 & 0 & \textbf{0} \\
\hline
1 & 0 & 0 & 1 & 1 & 1 & \textbf{1} \\
\hline
1 & 0 & 1 & 0 & 1 & 1 & \textbf{1} \\
\hline
1 & 0 & 1 & 1 & 1 & 1 & \textbf{1} \\
\hline
1 & 1 & 0 & 0 & 0 & 0 & \textbf{0} \\
\hline
1 & 1 & 0 & 1 & 0 & 1 & \textbf{0} \\
\hline
1 & 1 & 1 & 0 & 0 & 1 & \textbf{0} \\
\hline
1 & 1 & 1 & 1 & 0 & 1 & \textbf{0} \\
\hline
\end{tabular}
\end{center}

\subsection*{Analisi del Risultato}
La funzione è vera in 9 casi su 16. È vera quando:
\begin{itemize}
    \item $\overline{A \cdot B} = 1$ (almeno uno tra A o B è 0) \textbf{E}
    \item $C + D = 1$ (almeno uno tra C o D è 1)
\end{itemize}

È falsa quando: $A=1$ E $B=1$ (indipendentemente da C e D), oppure quando $C=0$ E $D=0$ (indipendentemente da A e B).

\newpage

\section*{Esercizio 9}
\subsection*{Funzione}
\[ F(A, B, C) = A \oplus B \oplus C \]

\subsection*{Procedimento}
Lo XOR (OR esclusivo) è definito come:
\[ X \oplus Y = X \cdot \overline{Y} + \overline{X} \cdot Y \]

Per lo XOR a cascata:
\begin{enumerate}
    \item Prima si calcola $A \oplus B$
    \item Poi si calcola $(A \oplus B) \oplus C$
\end{enumerate}

\textbf{Proprietà importante}: Lo XOR multiplo è vero quando c'è un numero \textbf{dispari} di 1 negli input.

\subsection*{Tabella di Verità Completa}
\begin{center}
\begin{tabular}{|c|c|c|c|c|c|}
\hline
\rowcolor{lightgray}
\textbf{A} & \textbf{B} & \textbf{C} & $A \oplus B$ & \textbf{F(A,B,C)} & \textbf{N° di 1} \\
\hline
0 & 0 & 0 & 0 & \textbf{0} & 0 (pari) \\
\hline
0 & 0 & 1 & 0 & \textbf{1} & 1 (dispari) \\
\hline
0 & 1 & 0 & 1 & \textbf{1} & 1 (dispari) \\
\hline
0 & 1 & 1 & 1 & \textbf{0} & 2 (pari) \\
\hline
1 & 0 & 0 & 1 & \textbf{1} & 1 (dispari) \\
\hline
1 & 0 & 1 & 1 & \textbf{0} & 2 (pari) \\
\hline
1 & 1 & 0 & 0 & \textbf{0} & 2 (pari) \\
\hline
1 & 1 & 1 & 0 & \textbf{1} & 3 (dispari) \\
\hline
\end{tabular}
\end{center}

\subsection*{Analisi del Risultato}
Come previsto, la funzione è vera esattamente quando c'è un numero dispari di 1:
\begin{itemize}
    \item 1 uno: righe 2, 3, 5 $\rightarrow$ F = 1
    \item 3 uni: riga 8 $\rightarrow$ F = 1
    \item 0 o 2 uni: righe 1, 4, 6, 7 $\rightarrow$ F = 0
\end{itemize}

\textbf{Applicazione}: Questa funzione è usata nei circuiti di controllo di parità.

\newpage

\section*{Esercizio 10}
\subsection*{Funzione}
\[ F(A, B, C, D) = (A + B) \cdot (\overline{C} + D) \cdot (\overline{A} + C) \]

\subsection*{Procedimento}
Questa è una funzione in \textbf{forma POS} (Product of Sums):
\begin{enumerate}
    \item Calcolare ogni clausola separatamente:
    \begin{itemize}
        \item $(A + B)$ - vero quando almeno uno tra A o B è vero
        \item $(\overline{C} + D)$ - vero quando C è falso o D è vero
        \item $(\overline{A} + C)$ - vero quando A è falso o C è vero
    \end{itemize}
    \item Moltiplicare (AND) tutti i risultati
    \item La funzione è vera solo quando tutte e tre le clausole sono vere
\end{enumerate}

\subsection*{Tabella di Verità Completa}
\begin{center}
\begin{tabular}{|c|c|c|c|c|c|c|c|}
\hline
\rowcolor{lightgray}
\textbf{A} & \textbf{B} & \textbf{C} & \textbf{D} & $A+B$ & $\overline{C}+D$ & $\overline{A}+C$ & \textbf{F} \\
\hline
0 & 0 & 0 & 0 & 0 & 1 & 1 & \textbf{0} \\
\hline
0 & 0 & 0 & 1 & 0 & 1 & 1 & \textbf{0} \\
\hline
0 & 0 & 1 & 0 & 0 & 0 & 1 & \textbf{0} \\
\hline
0 & 0 & 1 & 1 & 0 & 1 & 1 & \textbf{0} \\
\hline
0 & 1 & 0 & 0 & 1 & 1 & 1 & \textbf{1} \\
\hline
0 & 1 & 0 & 1 & 1 & 1 & 1 & \textbf{1} \\
\hline
0 & 1 & 1 & 0 & 1 & 0 & 1 & \textbf{0} \\
\hline
0 & 1 & 1 & 1 & 1 & 1 & 1 & \textbf{1} \\
\hline
1 & 0 & 0 & 0 & 1 & 1 & 0 & \textbf{0} \\
\hline
1 & 0 & 0 & 1 & 1 & 1 & 0 & \textbf{0} \\
\hline
1 & 0 & 1 & 0 & 1 & 0 & 1 & \textbf{0} \\
\hline
1 & 0 & 1 & 1 & 1 & 1 & 1 & \textbf{1} \\
\hline
1 & 1 & 0 & 0 & 1 & 1 & 0 & \textbf{0} \\
\hline
1 & 1 & 0 & 1 & 1 & 1 & 0 & \textbf{0} \\
\hline
1 & 1 & 1 & 0 & 1 & 0 & 1 & \textbf{0} \\
\hline
1 & 1 & 1 & 1 & 1 & 1 & 1 & \textbf{1} \\
\hline
\end{tabular}
\end{center}

\subsection*{Analisi del Risultato}
La funzione è vera in 5 casi su 16 (righe 5, 6, 8, 12, 16).

Osservazioni:
\begin{itemize}
    \item Quando $A=0$ e $B=0$: $(A+B)=0$ quindi F è sempre falso
    \item Quando $A=1$ e $C=0$: $(\overline{A}+C)=0$ quindi F è sempre falso
    \item La funzione richiede che tutte le clausole siano contemporaneamente vere
\end{itemize}

In forma SOP (mintermini): $F = \sum m(4,5,7,11,15)$

\newpage

\section*{Riepilogo e Note Conclusive}

\subsection*{Distribuzione dei Risultati}

\begin{center}
\begin{tabular}{|l|c|c|}
\hline
\rowcolor{lightgray}
\textbf{Esercizio} & \textbf{Casi Veri} & \textbf{Casi Totali} \\
\hline
Esercizio 1 & 3 & 4 \\
\hline
Esercizio 2 (NAND) & 3 & 4 \\
\hline
Esercizio 3 & 3 & 8 \\
\hline
Esercizio 4 & 3 & 8 \\
\hline
Esercizio 5 & 1 & 8 \\
\hline
Esercizio 6 & 2 & 8 \\
\hline
Esercizio 7 & 7 & 16 \\
\hline
Esercizio 8 & 9 & 16 \\
\hline
Esercizio 9 (XOR) & 4 & 8 \\
\hline
Esercizio 10 (POS) & 5 & 16 \\
\hline
\end{tabular}
\end{center}

\subsection*{Concetti Chiave Applicati}

\subsubsection*{1. Precedenza degli Operatori}
\begin{itemize}
    \item NOT (negazione) - massima precedenza
    \item AND (prodotto logico) - precedenza intermedia
    \item OR (somma logica) - minima precedenza
    \item Le parentesi modificano l'ordine di valutazione
\end{itemize}

\subsubsection*{2. Leggi di De Morgan}
\begin{itemize}
    \item $\overline{A \cdot B} = \overline{A} + \overline{B}$
    \item $\overline{A + B} = \overline{A} \cdot \overline{B}$
\end{itemize}

\subsubsection*{3. Forme Canoniche}
\begin{itemize}
    \item \textbf{SOP (Sum of Products)}: Somma di prodotti (es. Esercizio 6)
    \item \textbf{POS (Product of Sums)}: Prodotto di somme (es. Esercizio 10)
\end{itemize}

\subsubsection*{4. Funzioni Speciali}
\begin{itemize}
    \item \textbf{NAND}: $\overline{A \cdot B}$ - universale (può generare tutte le altre)
    \item \textbf{XOR}: $A \oplus B$ - vero per numero dispari di 1
    \item \textbf{NOR}: $\overline{A + B}$ - universale
\end{itemize}

\subsection*{Metodo di Verifica}
Per verificare i propri risultati:
\begin{enumerate}
    \item Controllare il numero di righe: deve essere $2^n$ dove n è il numero di variabili
    \item Verificare che tutte le combinazioni siano presenti (ordine binario)
    \item Ricontrollare le righe dove la funzione è vera
    \item Controllare casi limite (tutti 0, tutti 1)
\end{enumerate}

\subsection*{Esercizi di Approfondimento}
Dopo aver compreso questi esercizi, si suggerisce di:
\begin{enumerate}
    \item Semplificare le funzioni usando le mappe di Karnaugh
    \item Convertire tra forme SOP e POS
    \item Disegnare i circuiti logici corrispondenti
    \item Implementare le funzioni usando solo porte NAND o solo NOR
\end{enumerate}

\end{document}
