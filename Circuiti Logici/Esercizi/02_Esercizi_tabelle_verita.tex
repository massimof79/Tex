\documentclass[a4paper,11pt]{article}
\usepackage[utf8]{inputenc}
\usepackage[italian]{babel}
\usepackage{amsmath}
\usepackage{amssymb}
\usepackage{geometry}
\usepackage{fancyhdr}
\usepackage{enumitem}
\usepackage{array}
\usepackage{booktabs}

\geometry{margin=2.5cm}
\pagestyle{fancy}
\fancyhf{}
\fancyhead[L]{Esercizi: Tabelle di Verità}
\fancyhead[R]{IIS Fermi Sacconi Ceci}
\fancyfoot[C]{\thepage}

\title{\textbf{Esercizi sulle Tabelle di Verità}\\ \large Dall'Espressione Booleana alla Tabella di Verità}
\author{Prof. Massimo Fedeli}
\date{}

\begin{document}

\maketitle

\section*{Istruzioni}
Per ogni funzione booleana fornita:
\begin{enumerate}
    \item Identifica le variabili di input
    \item Costruisci la tabella di verità completa
    \item Calcola il valore della funzione per ogni combinazione di input
\end{enumerate}

\vspace{0.5cm}

\section*{Esercizi}

\subsection*{Esercizio 1 - Livello Base (2 variabili)}
Costruire la tabella di verità per la seguente funzione:
\[ F(A, B) = A \cdot B + \overline{A} \]

\vspace{0.3cm}
\noindent\textit{Suggerimento: La funzione ha 2 variabili, quindi la tabella avrà $2^2 = 4$ righe.}

\vspace{1cm}

\subsection*{Esercizio 2 - Livello Base (2 variabili)}
Costruire la tabella di verità per la seguente funzione:
\[ F(A, B) = \overline{A \cdot B} \]

\vspace{0.3cm}
\noindent\textit{Nota: Questa è la funzione NAND.}

\vspace{1cm}

\subsection*{Esercizio 3 - Livello Intermedio (3 variabili)}
Costruire la tabella di verità per la seguente funzione:
\[ F(A, B, C) = A \cdot B + B \cdot C \]

\vspace{0.3cm}
\noindent\textit{Suggerimento: Con 3 variabili avrai $2^3 = 8$ righe.}

\vspace{1cm}

\subsection*{Esercizio 4 - Livello Intermedio (3 variabili)}
Costruire la tabella di verità per la seguente funzione:
\[ F(A, B, C) = A \cdot (\overline{B} + C) \]

\vspace{0.3cm}
\noindent\textit{Suggerimento: Ricorda le precedenze: NOT $>$ AND $>$ OR.}

\vspace{1cm}

\subsection*{Esercizio 5 - Livello Intermedio (3 variabili)}
Costruire la tabella di verità per la seguente funzione:
\[ F(A, B, C) = \overline{A + B} \cdot C \]

\vspace{0.3cm}
\noindent\textit{Nota: Il NOT si applica a tutta l'espressione $(A + B)$.}

\newpage

\subsection*{Esercizio 6 - Livello Avanzato (3 variabili)}
Costruire la tabella di verità per la seguente funzione:
\[ F(A, B, C) = A \cdot \overline{B} \cdot C + \overline{A} \cdot B \cdot \overline{C} \]

\vspace{0.3cm}
\noindent\textit{Suggerimento: Questa funzione è vera solo per combinazioni specifiche.}

\vspace{1cm}

\subsection*{Esercizio 7 - Livello Avanzato (4 variabili)}
Costruire la tabella di verità per la seguente funzione:
\[ F(A, B, C, D) = A \cdot B + C \cdot D \]

\vspace{0.3cm}
\noindent\textit{Attenzione: Con 4 variabili la tabella avrà $2^4 = 16$ righe!}

\vspace{1cm}

\subsection*{Esercizio 8 - Livello Avanzato (4 variabili)}
Costruire la tabella di verità per la seguente funzione:
\[ F(A, B, C, D) = \overline{A \cdot B} \cdot (C + D) \]

\vspace{0.3cm}
\noindent\textit{Suggerimento: Prima calcola $\overline{A \cdot B}$, poi moltiplica per $(C + D)$.}

\vspace{1cm}

\subsection*{Esercizio 9 - Livello Esperto (3 variabili - XOR)}
Costruire la tabella di verità per la seguente funzione:
\[ F(A, B, C) = A \oplus B \oplus C \]

\vspace{0.3cm}
\noindent\textit{Nota: Il simbolo $\oplus$ rappresenta lo XOR (OR esclusivo). \\
Ricorda: $A \oplus B = A \cdot \overline{B} + \overline{A} \cdot B$}

\vspace{1cm}

\subsection*{Esercizio 10 - Livello Esperto (4 variabili)}
Costruire la tabella di verità per la seguente funzione:
\[ F(A, B, C, D) = (A + B) \cdot (\overline{C} + D) \cdot (\overline{A} + C) \]

\vspace{0.3cm}
\noindent\textit{Suggerimento: Questa è una forma POS (Product of Sums). Calcola prima ogni termine tra parentesi.}

\newpage

\section*{Template per le Tabelle di Verità}

\subsection*{Template per 2 variabili}
\begin{center}
\begin{tabular}{|c|c|c|}
\hline
\textbf{A} & \textbf{B} & \textbf{F(A,B)} \\
\hline
0 & 0 & \\
\hline
0 & 1 & \\
\hline
1 & 0 & \\
\hline
1 & 1 & \\
\hline
\end{tabular}
\end{center}

\subsection*{Template per 3 variabili}
\begin{center}
\begin{tabular}{|c|c|c|c|}
\hline
\textbf{A} & \textbf{B} & \textbf{C} & \textbf{F(A,B,C)} \\
\hline
0 & 0 & 0 & \\
\hline
0 & 0 & 1 & \\
\hline
0 & 1 & 0 & \\
\hline
0 & 1 & 1 & \\
\hline
1 & 0 & 0 & \\
\hline
1 & 0 & 1 & \\
\hline
1 & 1 & 0 & \\
\hline
1 & 1 & 1 & \\
\hline
\end{tabular}
\end{center}

\subsection*{Template per 4 variabili}
\begin{center}
\begin{tabular}{|c|c|c|c|c|}
\hline
\textbf{A} & \textbf{B} & \textbf{C} & \textbf{D} & \textbf{F(A,B,C,D)} \\
\hline
0 & 0 & 0 & 0 & \\
\hline
0 & 0 & 0 & 1 & \\
\hline
0 & 0 & 1 & 0 & \\
\hline
0 & 0 & 1 & 1 & \\
\hline
0 & 1 & 0 & 0 & \\
\hline
0 & 1 & 0 & 1 & \\
\hline
0 & 1 & 1 & 0 & \\
\hline
0 & 1 & 1 & 1 & \\
\hline
1 & 0 & 0 & 0 & \\
\hline
1 & 0 & 0 & 1 & \\
\hline
1 & 0 & 1 & 0 & \\
\hline
1 & 0 & 1 & 1 & \\
\hline
1 & 1 & 0 & 0 & \\
\hline
1 & 1 & 0 & 1 & \\
\hline
1 & 1 & 1 & 0 & \\
\hline
1 & 1 & 1 & 1 & \\
\hline
\end{tabular}
\end{center}

\newpage

\section*{Soluzioni}

\subsection*{Soluzione Esercizio 1}
$F(A, B) = A \cdot B + \overline{A}$

\begin{center}
\begin{tabular}{|c|c|c|c|c|}
\hline
\textbf{A} & \textbf{B} & $A \cdot B$ & $\overline{A}$ & \textbf{F} \\
\hline
0 & 0 & 0 & 1 & 1 \\
\hline
0 & 1 & 0 & 1 & 1 \\
\hline
1 & 0 & 0 & 0 & 0 \\
\hline
1 & 1 & 1 & 0 & 1 \\
\hline
\end{tabular}
\end{center}

\subsection*{Soluzione Esercizio 2}
$F(A, B) = \overline{A \cdot B}$ (NAND)

\begin{center}
\begin{tabular}{|c|c|c|c|}
\hline
\textbf{A} & \textbf{B} & $A \cdot B$ & \textbf{F} \\
\hline
0 & 0 & 0 & 1 \\
\hline
0 & 1 & 0 & 1 \\
\hline
1 & 0 & 0 & 1 \\
\hline
1 & 1 & 1 & 0 \\
\hline
\end{tabular}
\end{center}

\subsection*{Soluzione Esercizio 3}
$F(A, B, C) = A \cdot B + B \cdot C$

\begin{center}
\begin{tabular}{|c|c|c|c|c|c|}
\hline
\textbf{A} & \textbf{B} & \textbf{C} & $A \cdot B$ & $B \cdot C$ & \textbf{F} \\
\hline
0 & 0 & 0 & 0 & 0 & 0 \\
\hline
0 & 0 & 1 & 0 & 0 & 0 \\
\hline
0 & 1 & 0 & 0 & 0 & 0 \\
\hline
0 & 1 & 1 & 0 & 1 & 1 \\
\hline
1 & 0 & 0 & 0 & 0 & 0 \\
\hline
1 & 0 & 1 & 0 & 0 & 0 \\
\hline
1 & 1 & 0 & 1 & 0 & 1 \\
\hline
1 & 1 & 1 & 1 & 1 & 1 \\
\hline
\end{tabular}
\end{center}

\subsection*{Soluzione Esercizio 4}
$F(A, B, C) = A \cdot (\overline{B} + C)$

\begin{center}
\begin{tabular}{|c|c|c|c|c|c|}
\hline
\textbf{A} & \textbf{B} & \textbf{C} & $\overline{B}$ & $\overline{B} + C$ & \textbf{F} \\
\hline
0 & 0 & 0 & 1 & 1 & 0 \\
\hline
0 & 0 & 1 & 1 & 1 & 0 \\
\hline
0 & 1 & 0 & 0 & 0 & 0 \\
\hline
0 & 1 & 1 & 0 & 1 & 0 \\
\hline
1 & 0 & 0 & 1 & 1 & 1 \\
\hline
1 & 0 & 1 & 1 & 1 & 1 \\
\hline
1 & 1 & 0 & 0 & 0 & 0 \\
\hline
1 & 1 & 1 & 0 & 1 & 1 \\
\hline
\end{tabular}
\end{center}

\subsection*{Soluzione Esercizio 5}
$F(A, B, C) = \overline{A + B} \cdot C$

\begin{center}
\begin{tabular}{|c|c|c|c|c|c|}
\hline
\textbf{A} & \textbf{B} & \textbf{C} & $A + B$ & $\overline{A + B}$ & \textbf{F} \\
\hline
0 & 0 & 0 & 0 & 1 & 0 \\
\hline
0 & 0 & 1 & 0 & 1 & 1 \\
\hline
0 & 1 & 0 & 1 & 0 & 0 \\
\hline
0 & 1 & 1 & 1 & 0 & 0 \\
\hline
1 & 0 & 0 & 1 & 0 & 0 \\
\hline
1 & 0 & 1 & 1 & 0 & 0 \\
\hline
1 & 1 & 0 & 1 & 0 & 0 \\
\hline
1 & 1 & 1 & 1 & 0 & 0 \\
\hline
\end{tabular}
\end{center}

\subsection*{Soluzione Esercizio 6}
$F(A, B, C) = A \cdot \overline{B} \cdot C + \overline{A} \cdot B \cdot \overline{C}$

\begin{center}
\begin{tabular}{|c|c|c|c|c|c|}
\hline
\textbf{A} & \textbf{B} & \textbf{C} & $A \cdot \overline{B} \cdot C$ & $\overline{A} \cdot B \cdot \overline{C}$ & \textbf{F} \\
\hline
0 & 0 & 0 & 0 & 0 & 0 \\
\hline
0 & 0 & 1 & 0 & 0 & 0 \\
\hline
0 & 1 & 0 & 0 & 1 & 1 \\
\hline
0 & 1 & 1 & 0 & 0 & 0 \\
\hline
1 & 0 & 0 & 0 & 0 & 0 \\
\hline
1 & 0 & 1 & 1 & 0 & 1 \\
\hline
1 & 1 & 0 & 0 & 0 & 0 \\
\hline
1 & 1 & 1 & 0 & 0 & 0 \\
\hline
\end{tabular}
\end{center}

\subsection*{Soluzione Esercizio 7}
$F(A, B, C, D) = A \cdot B + C \cdot D$

\begin{center}
\begin{tabular}{|c|c|c|c|c|c|c|}
\hline
\textbf{A} & \textbf{B} & \textbf{C} & \textbf{D} & $A \cdot B$ & $C \cdot D$ & \textbf{F} \\
\hline
0 & 0 & 0 & 0 & 0 & 0 & 0 \\
\hline
0 & 0 & 0 & 1 & 0 & 0 & 0 \\
\hline
0 & 0 & 1 & 0 & 0 & 0 & 0 \\
\hline
0 & 0 & 1 & 1 & 0 & 1 & 1 \\
\hline
0 & 1 & 0 & 0 & 0 & 0 & 0 \\
\hline
0 & 1 & 0 & 1 & 0 & 0 & 0 \\
\hline
0 & 1 & 1 & 0 & 0 & 0 & 0 \\
\hline
0 & 1 & 1 & 1 & 0 & 1 & 1 \\
\hline
1 & 0 & 0 & 0 & 0 & 0 & 0 \\
\hline
1 & 0 & 0 & 1 & 0 & 0 & 0 \\
\hline
1 & 0 & 1 & 0 & 0 & 0 & 0 \\
\hline
1 & 0 & 1 & 1 & 0 & 1 & 1 \\
\hline
1 & 1 & 0 & 0 & 1 & 0 & 1 \\
\hline
1 & 1 & 0 & 1 & 1 & 0 & 1 \\
\hline
1 & 1 & 1 & 0 & 1 & 0 & 1 \\
\hline
1 & 1 & 1 & 1 & 1 & 1 & 1 \\
\hline
\end{tabular}
\end{center}

\subsection*{Soluzione Esercizio 8}
$F(A, B, C, D) = \overline{A \cdot B} \cdot (C + D)$

\begin{center}
\begin{tabular}{|c|c|c|c|c|c|c|}
\hline
\textbf{A} & \textbf{B} & \textbf{C} & \textbf{D} & $\overline{A \cdot B}$ & $C + D$ & \textbf{F} \\
\hline
0 & 0 & 0 & 0 & 1 & 0 & 0 \\
\hline
0 & 0 & 0 & 1 & 1 & 1 & 1 \\
\hline
0 & 0 & 1 & 0 & 1 & 1 & 1 \\
\hline
0 & 0 & 1 & 1 & 1 & 1 & 1 \\
\hline
0 & 1 & 0 & 0 & 1 & 0 & 0 \\
\hline
0 & 1 & 0 & 1 & 1 & 1 & 1 \\
\hline
0 & 1 & 1 & 0 & 1 & 1 & 1 \\
\hline
0 & 1 & 1 & 1 & 1 & 1 & 1 \\
\hline
1 & 0 & 0 & 0 & 1 & 0 & 0 \\
\hline
1 & 0 & 0 & 1 & 1 & 1 & 1 \\
\hline
1 & 0 & 1 & 0 & 1 & 1 & 1 \\
\hline
1 & 0 & 1 & 1 & 1 & 1 & 1 \\
\hline
1 & 1 & 0 & 0 & 0 & 0 & 0 \\
\hline
1 & 1 & 0 & 1 & 0 & 1 & 0 \\
\hline
1 & 1 & 1 & 0 & 0 & 1 & 0 \\
\hline
1 & 1 & 1 & 1 & 0 & 1 & 0 \\
\hline
\end{tabular}
\end{center}

\subsection*{Soluzione Esercizio 9}
$F(A, B, C) = A \oplus B \oplus C$

\begin{center}
\begin{tabular}{|c|c|c|c|c|}
\hline
\textbf{A} & \textbf{B} & \textbf{C} & $A \oplus B$ & \textbf{F} \\
\hline
0 & 0 & 0 & 0 & 0 \\
\hline
0 & 0 & 1 & 0 & 1 \\
\hline
0 & 1 & 0 & 1 & 1 \\
\hline
0 & 1 & 1 & 1 & 0 \\
\hline
1 & 0 & 0 & 1 & 1 \\
\hline
1 & 0 & 1 & 1 & 0 \\
\hline
1 & 1 & 0 & 0 & 0 \\
\hline
1 & 1 & 1 & 0 & 1 \\
\hline
\end{tabular}
\end{center}

\noindent\textit{Nota: Lo XOR a cascata è vero quando c'è un numero dispari di 1 negli input.}

\subsection*{Soluzione Esercizio 10}
$F(A, B, C, D) = (A + B) \cdot (\overline{C} + D) \cdot (\overline{A} + C)$

\begin{center}
\begin{tabular}{|c|c|c|c|c|c|c|c|}
\hline
\textbf{A} & \textbf{B} & \textbf{C} & \textbf{D} & $A+B$ & $\overline{C}+D$ & $\overline{A}+C$ & \textbf{F} \\
\hline
0 & 0 & 0 & 0 & 0 & 1 & 1 & 0 \\
\hline
0 & 0 & 0 & 1 & 0 & 1 & 1 & 0 \\
\hline
0 & 0 & 1 & 0 & 0 & 0 & 1 & 0 \\
\hline
0 & 0 & 1 & 1 & 0 & 1 & 1 & 0 \\
\hline
0 & 1 & 0 & 0 & 1 & 1 & 1 & 1 \\
\hline
0 & 1 & 0 & 1 & 1 & 1 & 1 & 1 \\
\hline
0 & 1 & 1 & 0 & 1 & 0 & 1 & 0 \\
\hline
0 & 1 & 1 & 1 & 1 & 1 & 1 & 1 \\
\hline
1 & 0 & 0 & 0 & 1 & 1 & 0 & 0 \\
\hline
1 & 0 & 0 & 1 & 1 & 1 & 0 & 0 \\
\hline
1 & 0 & 1 & 0 & 1 & 0 & 1 & 0 \\
\hline
1 & 0 & 1 & 1 & 1 & 1 & 1 & 1 \\
\hline
1 & 1 & 0 & 0 & 1 & 1 & 0 & 0 \\
\hline
1 & 1 & 0 & 1 & 1 & 1 & 0 & 0 \\
\hline
1 & 1 & 1 & 0 & 1 & 0 & 1 & 0 \\
\hline
1 & 1 & 1 & 1 & 1 & 1 & 1 & 1 \\
\hline
\end{tabular}
\end{center}

\section*{Note Didattiche}

\subsection*{Metodologia Suggerita}
\begin{enumerate}
    \item Iniziare con le funzioni a 2 variabili per familiarizzare con il procedimento
    \item Procedere gradualmente verso funzioni più complesse
    \item Incoraggiare gli studenti a verificare i propri risultati
    \item Utilizzare colonne intermedie per calcolare i termini parziali
\end{enumerate}

\subsection*{Errori Comuni da Evitare}
\begin{itemize}
    \item Non rispettare le precedenze degli operatori
    \item Dimenticare righe nella tabella
    \item Confondere AND con OR
    \item Non considerare correttamente la negazione
\end{itemize}

\subsection*{Estensioni Possibili}
\begin{itemize}
    \item Dalla tabella di verità all'espressione booleana (percorso inverso)
    \item Semplificazione delle funzioni usando le mappe di Karnaugh
    \item Implementazione circuitale delle funzioni
    \item Analisi della complessità (numero di porte logiche)
\end{itemize}

\end{document}
