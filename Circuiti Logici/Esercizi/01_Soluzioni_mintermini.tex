\documentclass[a4paper,12pt]{article}
\usepackage[utf8]{inputenc}
\usepackage[italian]{babel}
\usepackage{geometry}
\usepackage{array}
\usepackage{booktabs}
\usepackage{amsmath}
\usepackage{enumitem}
\usepackage{fancyhdr}
\usepackage{xcolor}
\usepackage{tcolorbox}

\geometry{margin=2cm}
\pagestyle{fancy}
\fancyhf{}
\lhead{Soluzioni Esercizi sui Mintermini}
\rhead{A.S. 2024/2025}
\cfoot{\thepage}

\title{\textbf{Soluzioni Dettagliate\\Esercizi sui Mintermini}}
\author{Prof. Massimo Fedeli\\IIS Fermi Sacconi Ceci - Ascoli Piceno}
\date{}

\begin{document}

\maketitle

\section*{Metodologia di Risoluzione}

Per determinare la funzione logica a partire dalla tabella di verità utilizzando i mintermini, seguire questi passaggi:

\begin{enumerate}
    \item \textbf{Numerare le righe}: Assegnare a ciascuna riga della tabella un numero decimale (da 0 a $2^n-1$)
    \item \textbf{Identificare le righe con uscita 1}: Individuare tutte le righe in cui $F = 1$
    \item \textbf{Scrivere i mintermini}: Per ogni riga identificata, scrivere il prodotto logico di tutte le variabili:
    \begin{itemize}
        \item Variabile in forma \textbf{diretta} se il valore è 1
        \item Variabile in forma \textbf{complementata} ($\overline{}$) se il valore è 0
    \end{itemize}
    \item \textbf{Sommare i mintermini}: Scrivere la funzione come OR (somma) dei mintermini
\end{enumerate}

\vspace{0.5cm}

\section*{Soluzioni degli Esercizi}

\subsection*{Esercizio 1}

\begin{center}
\begin{tabular}{ccc|c}
\toprule
\textbf{Riga} & \textbf{A} & \textbf{B} & \textbf{F} \\
\midrule
0 & 0 & 0 & 0 \\
1 & 0 & 1 & \textcolor{red}{\textbf{1}} \\
2 & 1 & 0 & \textcolor{red}{\textbf{1}} \\
3 & 1 & 1 & 0 \\
\bottomrule
\end{tabular}
\end{center}

\textbf{Passaggio 1: Identificazione righe con F=1}
\begin{itemize}
    \item Riga 1: $A=0, B=1 \Rightarrow F=1$
    \item Riga 2: $A=1, B=0 \Rightarrow F=1$
\end{itemize}

\textbf{Passaggio 2: Scrittura dei mintermini}
\begin{itemize}
    \item $m_1$: $A=0$ (complementato), $B=1$ (diretto) $\Rightarrow m_1 = \overline{A} \cdot B$
    \item $m_2$: $A=1$ (diretto), $B=0$ (complementato) $\Rightarrow m_2 = A \cdot \overline{B}$
\end{itemize}

\textbf{Passaggio 3: Funzione logica}
\begin{tcolorbox}[colback=blue!5!white,colframe=blue!75!black,title=Soluzione]
$$F(A,B) = \overline{A} \cdot B + A \cdot \overline{B} = \sum(1,2)$$
\end{tcolorbox}

\textbf{Nota}: Questa è la funzione XOR (OR esclusivo).

\vspace{1cm}

\subsection*{Esercizio 2}

\begin{center}
\begin{tabular}{ccc|c}
\toprule
\textbf{Riga} & \textbf{X} & \textbf{Y} & \textbf{F} \\
\midrule
0 & 0 & 0 & \textcolor{red}{\textbf{1}} \\
1 & 0 & 1 & 0 \\
2 & 1 & 0 & 0 \\
3 & 1 & 1 & \textcolor{red}{\textbf{1}} \\
\bottomrule
\end{tabular}
\end{center}

\textbf{Passaggio 1: Identificazione righe con F=1}
\begin{itemize}
    \item Riga 0: $X=0, Y=0 \Rightarrow F=1$
    \item Riga 3: $X=1, Y=1 \Rightarrow F=1$
\end{itemize}

\textbf{Passaggio 2: Scrittura dei mintermini}
\begin{itemize}
    \item $m_0$: $X=0$ (complementato), $Y=0$ (complementato) $\Rightarrow m_0 = \overline{X} \cdot \overline{Y}$
    \item $m_3$: $X=1$ (diretto), $Y=1$ (diretto) $\Rightarrow m_3 = X \cdot Y$
\end{itemize}

\textbf{Passaggio 3: Funzione logica}
\begin{tcolorbox}[colback=blue!5!white,colframe=blue!75!black,title=Soluzione]
$$F(X,Y) = \overline{X} \cdot \overline{Y} + X \cdot Y = \sum(0,3)$$
\end{tcolorbox}

\textbf{Nota}: Questa è la funzione XNOR (equivalenza logica).

\newpage

\subsection*{Esercizio 3}

\begin{center}
\begin{tabular}{cccc|c}
\toprule
\textbf{Riga} & \textbf{A} & \textbf{B} & \textbf{C} & \textbf{F} \\
\midrule
0 & 0 & 0 & 0 & \textcolor{red}{\textbf{1}} \\
1 & 0 & 0 & 1 & 0 \\
2 & 0 & 1 & 0 & \textcolor{red}{\textbf{1}} \\
3 & 0 & 1 & 1 & 0 \\
4 & 1 & 0 & 0 & 0 \\
5 & 1 & 0 & 1 & \textcolor{red}{\textbf{1}} \\
6 & 1 & 1 & 0 & \textcolor{red}{\textbf{1}} \\
7 & 1 & 1 & 1 & 0 \\
\bottomrule
\end{tabular}
\end{center}

\textbf{Passaggio 1: Identificazione righe con F=1}
\begin{itemize}
    \item Riga 0: $A=0, B=0, C=0$
    \item Riga 2: $A=0, B=1, C=0$
    \item Riga 5: $A=1, B=0, C=1$
    \item Riga 6: $A=1, B=1, C=0$
\end{itemize}

\textbf{Passaggio 2: Scrittura dei mintermini}
\begin{itemize}
    \item $m_0$: $A=0, B=0, C=0 \Rightarrow m_0 = \overline{A} \cdot \overline{B} \cdot \overline{C}$
    \item $m_2$: $A=0, B=1, C=0 \Rightarrow m_2 = \overline{A} \cdot B \cdot \overline{C}$
    \item $m_5$: $A=1, B=0, C=1 \Rightarrow m_5 = A \cdot \overline{B} \cdot C$
    \item $m_6$: $A=1, B=1, C=0 \Rightarrow m_6 = A \cdot B \cdot \overline{C}$
\end{itemize}

\textbf{Passaggio 3: Funzione logica}
\begin{tcolorbox}[colback=blue!5!white,colframe=blue!75!black,title=Soluzione]
$$F(A,B,C) = \overline{A} \cdot \overline{B} \cdot \overline{C} + \overline{A} \cdot B \cdot \overline{C} + A \cdot \overline{B} \cdot C + A \cdot B \cdot \overline{C}$$
$$F(A,B,C) = \sum(0,2,5,6)$$
\end{tcolorbox}

\textbf{Osservazione}: Si può notare che $C=0$ in tre mintermini su quattro, questo suggerisce una possibile semplificazione.

\vspace{1cm}

\subsection*{Esercizio 4}

\begin{center}
\begin{tabular}{ccccc|c}
\toprule
\textbf{Riga} & \textbf{X} & \textbf{Y} & \textbf{Z} & \textbf{F} \\
\midrule
0 & 0 & 0 & 0 & 0 \\
1 & 0 & 0 & 1 & \textcolor{red}{\textbf{1}} \\
2 & 0 & 1 & 0 & 0 \\
3 & 0 & 1 & 1 & \textcolor{red}{\textbf{1}} \\
4 & 1 & 0 & 0 & \textcolor{red}{\textbf{1}} \\
5 & 1 & 0 & 1 & 0 \\
6 & 1 & 1 & 0 & \textcolor{red}{\textbf{1}} \\
7 & 1 & 1 & 1 & \textcolor{red}{\textbf{1}} \\
\bottomrule
\end{tabular}
\end{center}

\textbf{Passaggio 1: Identificazione righe con F=1}

Righe: 1, 3, 4, 6, 7

\textbf{Passaggio 2: Scrittura dei mintermini}
\begin{itemize}
    \item $m_1 = \overline{X} \cdot \overline{Y} \cdot Z$
    \item $m_3 = \overline{X} \cdot Y \cdot Z$
    \item $m_4 = X \cdot \overline{Y} \cdot \overline{Z}$
    \item $m_6 = X \cdot Y \cdot \overline{Z}$
    \item $m_7 = X \cdot Y \cdot Z$
\end{itemize}

\textbf{Passaggio 3: Funzione logica}
\begin{tcolorbox}[colback=blue!5!white,colframe=blue!75!black,title=Soluzione]
\begin{align*}
F(X,Y,Z) = &\overline{X} \cdot \overline{Y} \cdot Z + \overline{X} \cdot Y \cdot Z + X \cdot \overline{Y} \cdot \overline{Z} + \\
           &X \cdot Y \cdot \overline{Z} + X \cdot Y \cdot Z
\end{align*}
$$F(X,Y,Z) = \sum(1,3,4,6,7)$$
\end{tcolorbox}

\newpage

\subsection*{Esercizio 5}

\begin{center}
\begin{tabular}{ccccc|c}
\toprule
\textbf{Riga} & \textbf{A} & \textbf{B} & \textbf{C} & \textbf{F} \\
\midrule
0 & 0 & 0 & 0 & 0 \\
1 & 0 & 0 & 1 & 0 \\
2 & 0 & 1 & 0 & \textcolor{red}{\textbf{1}} \\
3 & 0 & 1 & 1 & \textcolor{red}{\textbf{1}} \\
4 & 1 & 0 & 0 & \textcolor{red}{\textbf{1}} \\
5 & 1 & 0 & 1 & 0 \\
6 & 1 & 1 & 0 & 0 \\
7 & 1 & 1 & 1 & \textcolor{red}{\textbf{1}} \\
\bottomrule
\end{tabular}
\end{center}

\textbf{Passaggio 1: Identificazione righe con F=1}

Righe: 2, 3, 4, 7

\textbf{Passaggio 2: Scrittura dei mintermini}
\begin{itemize}
    \item $m_2$: $A=0, B=1, C=0 \Rightarrow \overline{A} \cdot B \cdot \overline{C}$
    \item $m_3$: $A=0, B=1, C=1 \Rightarrow \overline{A} \cdot B \cdot C$
    \item $m_4$: $A=1, B=0, C=0 \Rightarrow A \cdot \overline{B} \cdot \overline{C}$
    \item $m_7$: $A=1, B=1, C=1 \Rightarrow A \cdot B \cdot C$
\end{itemize}

\textbf{Passaggio 3: Funzione logica}
\begin{tcolorbox}[colback=blue!5!white,colframe=blue!75!black,title=Soluzione]
$$F(A,B,C) = \overline{A} \cdot B \cdot \overline{C} + \overline{A} \cdot B \cdot C + A \cdot \overline{B} \cdot \overline{C} + A \cdot B \cdot C$$
$$F(A,B,C) = \sum(2,3,4,7)$$
\end{tcolorbox}

\textbf{Semplificazione possibile}:
\begin{itemize}
    \item $m_2 + m_3 = \overline{A} \cdot B \cdot (\overline{C} + C) = \overline{A} \cdot B$
    \item Funzione semplificata: $F = \overline{A} \cdot B + A \cdot \overline{B} \cdot \overline{C} + A \cdot B \cdot C$
\end{itemize}

\vspace{1cm}

\subsection*{Esercizio 6}

\begin{center}
\begin{tabular}{cccccc|c}
\toprule
\textbf{Riga} & \textbf{A} & \textbf{B} & \textbf{C} & \textbf{D} & \textbf{F} \\
\midrule
0 & 0 & 0 & 0 & 0 & \textcolor{red}{\textbf{1}} \\
1 & 0 & 0 & 0 & 1 & 0 \\
2 & 0 & 0 & 1 & 0 & \textcolor{red}{\textbf{1}} \\
3 & 0 & 0 & 1 & 1 & 0 \\
4 & 0 & 1 & 0 & 0 & 0 \\
5 & 0 & 1 & 0 & 1 & \textcolor{red}{\textbf{1}} \\
6 & 0 & 1 & 1 & 0 & 0 \\
7 & 0 & 1 & 1 & 1 & \textcolor{red}{\textbf{1}} \\
8 & 1 & 0 & 0 & 0 & 0 \\
9 & 1 & 0 & 0 & 1 & 0 \\
10 & 1 & 0 & 1 & 0 & \textcolor{red}{\textbf{1}} \\
11 & 1 & 0 & 1 & 1 & \textcolor{red}{\textbf{1}} \\
12 & 1 & 1 & 0 & 0 & \textcolor{red}{\textbf{1}} \\
13 & 1 & 1 & 0 & 1 & 0 \\
14 & 1 & 1 & 1 & 0 & 0 \\
15 & 1 & 1 & 1 & 1 & \textcolor{red}{\textbf{1}} \\
\bottomrule
\end{tabular}
\end{center}

\textbf{Passaggio 1: Identificazione righe con F=1}

Righe: 0, 2, 5, 7, 10, 11, 12, 15

\textbf{Passaggio 2: Scrittura dei mintermini}
\begin{itemize}
    \item $m_0 = \overline{A} \cdot \overline{B} \cdot \overline{C} \cdot \overline{D}$
    \item $m_2 = \overline{A} \cdot \overline{B} \cdot C \cdot \overline{D}$
    \item $m_5 = \overline{A} \cdot B \cdot \overline{C} \cdot D$
    \item $m_7 = \overline{A} \cdot B \cdot C \cdot D$
    \item $m_{10} = A \cdot \overline{B} \cdot C \cdot \overline{D}$
    \item $m_{11} = A \cdot \overline{B} \cdot C \cdot D$
    \item $m_{12} = A \cdot B \cdot \overline{C} \cdot \overline{D}$
    \item $m_{15} = A \cdot B \cdot C \cdot D$
\end{itemize}

\textbf{Passaggio 3: Funzione logica}
\begin{tcolorbox}[colback=blue!5!white,colframe=blue!75!black,title=Soluzione]
\begin{align*}
F(A,B,C,D) = &\overline{A} \cdot \overline{B} \cdot \overline{C} \cdot \overline{D} + \overline{A} \cdot \overline{B} \cdot C \cdot \overline{D} + \overline{A} \cdot B \cdot \overline{C} \cdot D + \\
             &\overline{A} \cdot B \cdot C \cdot D + A \cdot \overline{B} \cdot C \cdot \overline{D} + A \cdot \overline{B} \cdot C \cdot D + \\
             &A \cdot B \cdot \overline{C} \cdot \overline{D} + A \cdot B \cdot C \cdot D
\end{align*}
$$F(A,B,C,D) = \sum(0,2,5,7,10,11,12,15)$$
\end{tcolorbox}

\newpage

\subsection*{Esercizio 7}

\begin{center}
\begin{tabular}{ccccc|c}
\toprule
\textbf{Riga} & \textbf{P} & \textbf{Q} & \textbf{R} & \textbf{F} \\
\midrule
0 & 0 & 0 & 0 & \textcolor{red}{\textbf{1}} \\
1 & 0 & 0 & 1 & \textcolor{red}{\textbf{1}} \\
2 & 0 & 1 & 0 & 0 \\
3 & 0 & 1 & 1 & 0 \\
4 & 1 & 0 & 0 & \textcolor{red}{\textbf{1}} \\
5 & 1 & 0 & 1 & \textcolor{red}{\textbf{1}} \\
6 & 1 & 1 & 0 & 0 \\
7 & 1 & 1 & 1 & \textcolor{red}{\textbf{1}} \\
\bottomrule
\end{tabular}
\end{center}

\textbf{Passaggio 1: Identificazione righe con F=1}

Righe: 0, 1, 4, 5, 7

\textbf{Passaggio 2: Scrittura dei mintermini}
\begin{itemize}
    \item $m_0 = \overline{P} \cdot \overline{Q} \cdot \overline{R}$
    \item $m_1 = \overline{P} \cdot \overline{Q} \cdot R$
    \item $m_4 = P \cdot \overline{Q} \cdot \overline{R}$
    \item $m_5 = P \cdot \overline{Q} \cdot R$
    \item $m_7 = P \cdot Q \cdot R$
\end{itemize}

\textbf{Passaggio 3: Funzione logica}
\begin{tcolorbox}[colback=blue!5!white,colframe=blue!75!black,title=Soluzione]
\begin{align*}
F(P,Q,R) = &\overline{P} \cdot \overline{Q} \cdot \overline{R} + \overline{P} \cdot \overline{Q} \cdot R + P \cdot \overline{Q} \cdot \overline{R} + \\
           &P \cdot \overline{Q} \cdot R + P \cdot Q \cdot R
\end{align*}
$$F(P,Q,R) = \sum(0,1,4,5,7)$$
\end{tcolorbox}

\textbf{Semplificazione possibile}:
\begin{itemize}
    \item $m_0 + m_1 = \overline{P} \cdot \overline{Q}$
    \item $m_4 + m_5 = P \cdot \overline{Q}$
    \item Funzione semplificata: $F = \overline{Q} + P \cdot Q \cdot R$
\end{itemize}

\vspace{1cm}

\subsection*{Esercizio 8}

\begin{center}
\begin{tabular}{ccccc|c}
\toprule
\textbf{Riga} & \textbf{A} & \textbf{B} & \textbf{C} & \textbf{F} \\
\midrule
0 & 0 & 0 & 0 & 0 \\
1 & 0 & 0 & 1 & \textcolor{red}{\textbf{1}} \\
2 & 0 & 1 & 0 & \textcolor{red}{\textbf{1}} \\
3 & 0 & 1 & 1 & 0 \\
4 & 1 & 0 & 0 & 0 \\
5 & 1 & 0 & 1 & 0 \\
6 & 1 & 1 & 0 & \textcolor{red}{\textbf{1}} \\
7 & 1 & 1 & 1 & \textcolor{red}{\textbf{1}} \\
\bottomrule
\end{tabular}
\end{center}

\textbf{Passaggio 1: Identificazione righe con F=1}

Righe: 1, 2, 6, 7

\textbf{Passaggio 2: Scrittura dei mintermini}
\begin{itemize}
    \item $m_1 = \overline{A} \cdot \overline{B} \cdot C$
    \item $m_2 = \overline{A} \cdot B \cdot \overline{C}$
    \item $m_6 = A \cdot B \cdot \overline{C}$
    \item $m_7 = A \cdot B \cdot C$
\end{itemize}

\textbf{Passaggio 3: Funzione logica}
\begin{tcolorbox}[colback=blue!5!white,colframe=blue!75!black,title=Soluzione]
$$F(A,B,C) = \overline{A} \cdot \overline{B} \cdot C + \overline{A} \cdot B \cdot \overline{C} + A \cdot B \cdot \overline{C} + A \cdot B \cdot C$$
$$F(A,B,C) = \sum(1,2,6,7)$$
\end{tcolorbox}

\textbf{Semplificazione possibile}:
\begin{itemize}
    \item $m_6 + m_7 = A \cdot B \cdot (\overline{C} + C) = A \cdot B$
    \item Funzione semplificata: $F = \overline{A} \cdot \overline{B} \cdot C + \overline{A} \cdot B \cdot \overline{C} + A \cdot B$
\end{itemize}

\newpage

\subsection*{Esercizio 9}

\begin{center}
\begin{tabular}{ccccccc|c}
\toprule
\textbf{Riga} & \textbf{W} & \textbf{X} & \textbf{Y} & \textbf{Z} & \textbf{F} \\
\midrule
0 & 0 & 0 & 0 & 0 & 0 \\
1 & 0 & 0 & 0 & 1 & \textcolor{red}{\textbf{1}} \\
2 & 0 & 0 & 1 & 0 & 0 \\
3 & 0 & 0 & 1 & 1 & \textcolor{red}{\textbf{1}} \\
4 & 0 & 1 & 0 & 0 & \textcolor{red}{\textbf{1}} \\
5 & 0 & 1 & 0 & 1 & 0 \\
6 & 0 & 1 & 1 & 0 & \textcolor{red}{\textbf{1}} \\
7 & 0 & 1 & 1 & 1 & 0 \\
8 & 1 & 0 & 0 & 0 & 0 \\
9 & 1 & 0 & 0 & 1 & 0 \\
10 & 1 & 0 & 1 & 0 & \textcolor{red}{\textbf{1}} \\
11 & 1 & 0 & 1 & 1 & \textcolor{red}{\textbf{1}} \\
12 & 1 & 1 & 0 & 0 & 0 \\
13 & 1 & 1 & 0 & 1 & \textcolor{red}{\textbf{1}} \\
14 & 1 & 1 & 1 & 0 & 0 \\
15 & 1 & 1 & 1 & 1 & 0 \\
\bottomrule
\end{tabular}
\end{center}

\textbf{Passaggio 1: Identificazione righe con F=1}

Righe: 1, 3, 4, 6, 10, 11, 13

\textbf{Passaggio 2: Scrittura dei mintermini}
\begin{itemize}
    \item $m_1 = \overline{W} \cdot \overline{X} \cdot \overline{Y} \cdot Z$
    \item $m_3 = \overline{W} \cdot \overline{X} \cdot Y \cdot Z$
    \item $m_4 = \overline{W} \cdot X \cdot \overline{Y} \cdot \overline{Z}$
    \item $m_6 = \overline{W} \cdot X \cdot Y \cdot \overline{Z}$
    \item $m_{10} = W \cdot \overline{X} \cdot Y \cdot \overline{Z}$
    \item $m_{11} = W \cdot \overline{X} \cdot Y \cdot Z$
    \item $m_{13} = W \cdot X \cdot \overline{Y} \cdot Z$
\end{itemize}

\textbf{Passaggio 3: Funzione logica}
\begin{tcolorbox}[colback=blue!5!white,colframe=blue!75!black,title=Soluzione]
\begin{align*}
F(W,X,Y,Z) = &\overline{W} \cdot \overline{X} \cdot \overline{Y} \cdot Z + \overline{W} \cdot \overline{X} \cdot Y \cdot Z + \overline{W} \cdot X \cdot \overline{Y} \cdot \overline{Z} + \\
             &\overline{W} \cdot X \cdot Y \cdot \overline{Z} + W \cdot \overline{X} \cdot Y \cdot \overline{Z} + W \cdot \overline{X} \cdot Y \cdot Z + \\
             &W \cdot X \cdot \overline{Y} \cdot Z
\end{align*}
$$F(W,X,Y,Z) = \sum(1,3,4,6,10,11,13)$$
\end{tcolorbox}

\vspace{1cm}

\subsection*{Esercizio 10}

\begin{center}
\begin{tabular}{ccccc|c}
\toprule
\textbf{Riga} & \textbf{A} & \textbf{B} & \textbf{C} & \textbf{F} \\
\midrule
0 & 0 & 0 & 0 & \textcolor{red}{\textbf{1}} \\
1 & 0 & 0 & 1 & 0 \\
2 & 0 & 1 & 0 & 0 \\
3 & 0 & 1 & 1 & \textcolor{red}{\textbf{1}} \\
4 & 1 & 0 & 0 & \textcolor{red}{\textbf{1}} \\
5 & 1 & 0 & 1 & \textcolor{red}{\textbf{1}} \\
6 & 1 & 1 & 0 & 0 \\
7 & 1 & 1 & 1 & 0 \\
\bottomrule
\end{tabular}
\end{center}

\textbf{Passaggio 1: Identificazione righe con F=1}

Righe: 0, 3, 4, 5

\textbf{Passaggio 2: Scrittura dei mintermini}
\begin{itemize}
    \item $m_0 = \overline{A} \cdot \overline{B} \cdot \overline{C}$
    \item $m_3 = \overline{A} \cdot B \cdot C$
    \item $m_4 = A \cdot \overline{B} \cdot \overline{C}$
    \item $m_5 = A \cdot \overline{B} \cdot C$
\end{itemize}

\textbf{Passaggio 3: Funzione logica}
\begin{tcolorbox}[colback=blue!5!white,colframe=blue!75!black,title=Soluzione]
$$F(A,B,C) = \overline{A} \cdot \overline{B} \cdot \overline{C} + \overline{A} \cdot B \cdot C + A \cdot \overline{B} \cdot \overline{C} + A \cdot \overline{B} \cdot C$$
$$F(A,B,C) = \sum(0,3,4,5)$$
\end{tcolorbox}

\textbf{Semplificazione possibile}:
\begin{itemize}
    \item $m_0 + m_4 = \overline{B} \cdot \overline{C}$
    \item $m_4 + m_5 = A \cdot \overline{B}$
    \item Funzione semplificata: $F = \overline{B} \cdot \overline{C} + \overline{A} \cdot B \cdot C + A \cdot \overline{B} \cdot C$
    
    oppure
    
    $F = A \cdot \overline{B} + \overline{A} \cdot \overline{B} \cdot \overline{C} + \overline{A} \cdot B \cdot C$
\end{itemize}

\newpage

\section*{Riepilogo delle Notazioni}

\subsection*{Forma Estesa}
La funzione viene scritta esplicitando tutti i prodotti logici:
$$F = \overline{A} \cdot B + A \cdot \overline{B}$$

\subsection*{Forma Compatta (Notazione Sigma)}
La funzione viene scritta come somma dei mintermini indicati dai loro indici:
$$F = \sum(1,2)$$

dove i numeri indicano le righe della tabella di verità con uscita 1.

\subsection*{Conversione Decimale-Binario}

Per determinare quale mintermine corrisponde a un certo indice:

\begin{itemize}
    \item Convertire l'indice decimale in binario
    \item Il numero di bit deve essere uguale al numero di variabili
    \item 0 → variabile complementata
    \item 1 → variabile diretta
\end{itemize}

\textbf{Esempio con 3 variabili A, B, C:}
\begin{itemize}
    \item $m_5 = ?$
    \item $5_{10} = 101_2$
    \item $A=1$ (diretto), $B=0$ (complementato), $C=1$ (diretto)
    \item $m_5 = A \cdot \overline{B} \cdot C$
\end{itemize}

\section*{Consigli per la Risoluzione}

\begin{enumerate}
    \item \textbf{Organizzazione}: Numerare sempre le righe partendo da 0
    \item \textbf{Evidenziazione}: Evidenziare o marcare le righe con $F=1$
    \item \textbf{Sistematicità}: Procedere in ordine, senza saltare righe
    \item \textbf{Verifica}: Controllare che ogni variabile compaia in ogni mintermine
    \item \textbf{Semplificazione}: Dopo aver scritto la forma SOP, cercare possibili semplificazioni usando:
    \begin{itemize}
        \item Teorema del complemento: $X + \overline{X} = 1$
        \item Fattorizzazione
        \item Mappe di Karnaugh (argomento successivo)
    \end{itemize}
\end{enumerate}

\end{document}
