\documentclass[a4paper,12pt]{article}
\usepackage[italian]{babel}
\usepackage[T1]{fontenc}
\usepackage[utf8]{inputenc}
\usepackage{amsmath}
\usepackage{geometry}
\geometry{margin=2.5cm}

\title{Circuiti Combinatori \\ Esercizi}
\author{Prof. Fedeli Massimo - IIS Fermi Sacconi Cpia}
\date{}

\begin{document}
	\maketitle
	
	\section*{Esercizi}
	
	\subsection*{Esercizio 1}
	È data la seguente tabella di verità per la funzione $F(A,B)$:
	
	\begin{center}
		\begin{tabular}{c c c}
			A & B & F \\
			0 & 0 & 0 \\
			0 & 1 & 1 \\
			1 & 0 & 1 \\
			1 & 1 & 0 \\
		\end{tabular}
	\end{center}
	
	\begin{enumerate}
		\item Scrivere la funzione logica in forma canonica somma di prodotti.
		\item Disegnare il circuito logico equivalente.
	\end{enumerate}
	
	\subsection*{Esercizio 2}
	La funzione logica $F(A,B,C)$ vale 0 per le combinazioni:
	\[
	000,\;011,\;101
	\]
	
	Scrivere la funzione in forma canonica prodotto di somme.
	
	\subsection*{Esercizio 3}
	Data la funzione:
	\[
	F(A,B,C) = \overline{A}BC + A\overline{B}C
	\]
	
	\begin{enumerate}
		\item Indicare il numero e il tipo di porte logiche necessarie.
		\item Disegnare il circuito logico equivalente.
	\end{enumerate}
	
	\subsection*{Esercizio 4}
	Semplificare la seguente funzione logica utilizzando l’algebra di Boole:
	\[
	F(A,B) = \overline{A}B + AB + \overline{A}\overline{B}
	\]
	
	\subsection*{Esercizio 5}
	Data la funzione:
	\[
	F(A,B,C) = A(B + C)
	\]
	
	Costruire la tabella di verità completa.
	
	\subsection*{Esercizio 6}
	È data la funzione:
	\[
	F(A,B) = \overline{A}B + A\overline{B}
	\]
	
	\begin{enumerate}
		\item Identificare il tipo di funzione logica.
		\item Indicare un possibile circuito equivalente.
	\end{enumerate}
	
	\subsection*{Esercizio 7}
	Un sistema di allarme si attiva se il sensore $A$ è attivo e almeno uno tra i sensori $B$ o $C$ è attivo.
	
	\begin{enumerate}
		\item Scrivere la funzione logica.
		\item Disegnare il circuito.
	\end{enumerate}
	
	\subsection*{Esercizio 8}
	Data la funzione:
	\[
	F(A,B,C,D) = \Sigma m(1,3,7,11)
	\]
	
	Scrivere la funzione in forma canonica somma di prodotti.
	
	\subsection*{Esercizio 9}
	Un circuito è composto da:
	\begin{itemize}
		\item una porta AND con ingressi $A$ e $B$
		\item una porta NOT sull’ingresso $C$
		\item una porta OR che unisce le due uscite
	\end{itemize}
	
	Scrivere la funzione logica realizzata.
	
	\subsection*{Esercizio 10}
	Data la funzione:
	\[
	F(A,B,C) = \overline{A}BC + ABC + \overline{A}\overline{B}C
	\]
	
	Semplificare la funzione.
	
\end{document}
