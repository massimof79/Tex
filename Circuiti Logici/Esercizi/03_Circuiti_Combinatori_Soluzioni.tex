\documentclass[a4paper,12pt]{article}
\usepackage[italian]{babel}
\usepackage[T1]{fontenc}
\usepackage[utf8]{inputenc}
\usepackage{amsmath}
\usepackage{geometry}
\geometry{margin=2.5cm}

\title{Circuiti Combinatori \\ Esercizi con Soluzioni}
\author{Prof. Fedeli Massimo - IIS Fermi Sacconi Cpia}
\date{}

\begin{document}
	\maketitle
	
	\section*{Esercizi svolti}
	
	\subsection*{Esercizio 1}
	La funzione vale 1 per le combinazioni $01$ e $10$.
	
	\[
	F(A,B) = \overline{A}B + A\overline{B}
	\]
	
	Il circuito è composto da due NOT, due AND e una OR.
	
	\subsection*{Esercizio 2}
	I maxtermini corrispondenti sono:
	\[
	(A + B + C)(A + \overline{B} + \overline{C})(\overline{A} + B + \overline{C})
	\]
	
	\subsection*{Esercizio 3}
	La funzione è:
	\[
	F(A,B,C) = \overline{A}BC + A\overline{B}C
	\]
	
	Sono necessarie:
	\begin{itemize}
		\item 2 porte NOT
		\item 2 porte AND a tre ingressi
		\item 1 porta OR
	\end{itemize}
	
	\subsection*{Esercizio 4}
	\[
	F = \overline{A}B + AB + \overline{A}\overline{B}
	\]
	\[
	F = \overline{A}(B + \overline{B}) + AB = \overline{A} + AB
	\]
	\[
	F = \overline{A} + B
	\]
	
	\subsection*{Esercizio 5}
	La funzione vale 1 solo se $A=1$ e almeno uno tra $B$ e $C$ vale 1.  
	La tabella di verità si ottiene valutando prima $(B + C)$ e poi il prodotto con $A$.
	
	\subsection*{Esercizio 6}
	La funzione:
	\[
	F(A,B) = \overline{A}B + A\overline{B}
	\]
	è una XOR.  
	Può essere realizzata con una sola porta XOR.
	
	\subsection*{Esercizio 7}
	La funzione logica è:
	\[
	F(A,B,C) = A(B + C)
	\]
	
	\subsection*{Esercizio 8}
	\[
	F = \overline{A}\overline{B}\overline{C}D +
	\overline{A}\overline{B}CD +
	\overline{A}BCD +
	A\overline{B}CD
	\]
	
	\subsection*{Esercizio 9}
	La funzione realizzata dal circuito è:
	\[
	F(A,B,C) = AB + \overline{C}
	\]
	
	\subsection*{Esercizio 10}
	\[
	F = \overline{A}BC + ABC + \overline{A}\overline{B}C
	\]
	\[
	F = C(\overline{A}B + AB + \overline{A}\overline{B})
	\]
	\[
	F = C(\overline{A} + B)
	\]
	
\end{document}
