\documentclass[aspectratio=169,14pt]{beamer}
\usepackage[utf8]{inputenc}
\usepackage[italian]{babel}
\usepackage{graphicx}
\usepackage{tikz}
\usepackage{booktabs}
\usepackage{array}
\usepackage{colortbl}
\usepackage{xcolor}

% Tema e colori
\usetheme{Madrid}
\usecolortheme{default}

% Definizione colori personalizzati
\definecolor{primaryblue}{RGB}{0,51,102}
\definecolor{secondaryblue}{RGB}{51,102,153}
\definecolor{accentorange}{RGB}{255,102,0}
\definecolor{lightgray}{RGB}{240,240,240}

\setbeamercolor{structure}{fg=primaryblue}
\setbeamercolor{title}{fg=white,bg=primaryblue}
\setbeamercolor{frametitle}{fg=white,bg=secondaryblue}
\setbeamercolor{block title}{fg=white,bg=primaryblue}
\setbeamercolor{block body}{bg=lightgray}

% Impostazioni piè di pagina
\setbeamertemplate{navigation symbols}{}
\setbeamertemplate{footline}[frame number]

% Logo nel footer
\logo{\includegraphics[height=0.6cm]{logo_fermi_sacconi.png}\hspace{0.3cm}\includegraphics[height=0.6cm]{logo_its_turismo.png}}

% Informazioni presentazione
\title[Filiera 4+2 Informatica]{Filiera Formativa Tecnologico-Professionale 4+2}
\subtitle{Indirizzo Informatica e Telecomunicazioni\\Intelligenza Artificiale e Cybersecurity}
\author{IIS Fermi Sacconi CPIA -- Ascoli Piceno}
\institute{In collaborazione con ITS Turismo Marche}
\date{Anno Scolastico 2026/2027}

\begin{document}

% Slide 1: Titolo
\begin{frame}
\titlepage
\begin{center}

\end{center}
\end{frame}

% Slide 2: Monte ore quinquennale vs quadriennale
\begin{frame}{Monte Ore: Quinquennale vs Quadriennale}
\begin{block}{Monte ore su 5 anni (tradizionale) 26-27}
32x4 + 30x1 h/settimana × 33 settimane × 5 anni = \textbf{5.214 ore}
\end{block}

\begin{block}{Rimodulazione su 4 anni}
	\begin{itemize}
		\item \textbf{5214 ore} - di cui -
		\item 36 h/settimana × 33 settimane × 4 anni = \textbf{4752 ore frontali}
	   \item 5214-4752 = 462		 
	\end{itemize}
\end{block}

\vspace{0.3cm}
\begin{alertblock}{Ore da redistribuire}
\textbf{462 ore} da distribuire con modalità innovative
\end{alertblock}


\end{frame}


\begin{frame}{Ore da redistribuire}
	\vspace{0.3cm}
	\small
	\textit{Le 462 ore saranno redistribuite attraverso:}
	\begin{itemize}
		\item Attività pomeridiane e FAD
		\item FSL potenziata
	\end{itemize}
\end{frame}
% Slide 3: Distribuzione delle 528 ore
\begin{frame}{Distribuzione delle 5181 Ore su 4 Anni}
\small
\begin{table}
\centering
\begin{tabular}{|l|c|p{10cm}|}
\hline
\rowcolor{primaryblue}
\textcolor{white}{\textbf{Anno}} & \textcolor{white}{\textbf{Ore}} & \textcolor{white}{\textbf{Modalità}} \\
\hline
\textbf{1° anno} & 1122 & 34h p.sett. + 20h FSL + 66 potenz. \\
\hline
\textbf{2° anno} & 1221& 37h p.sett (di cui) 2h FAD pom. + 20h FSL  + 99 potenz. \\
\hline
\textbf{3° anno} & 1221 & 37h p.sett (di cui) 2h FAD pom. + 100h FSL + 66 potenz. 	\\
\hline
\textbf{4° anno} & 1254 & 38h p.sett (di cui) 2h FAD pom. + 100h FSL + 198 potenz. \\
\hline
\rowcolor{lightgray}
\textbf{TOTALE} & \textbf{5181} & + 240 FSL\\ 
\hline
\end{tabular}
\end{table}

\vspace{0.3cm}
\textcolor{accentorange}{\textbf{Nota:}} Tutte le attività sono co-progettate con ITS e partner della filiera
\end{frame}


% Slide 5: Ore vincolate delle discipline di base

% Slide 6: Ore di compresenza e potenziamento
\begin{frame}{Ore di Compresenza, Potenziamento e Orientamento}
\framesubtitle{Dal monte ore del V anno}

\begin{table}
\centering
\begin{tabular}{|l|c|p{10.5cm}|}
\hline
\rowcolor{primaryblue}
\textcolor{white}{\textbf{Anno}} & \textcolor{white}{\textbf{Ore}} & \textcolor{white}{\textbf{Attività}} \\
\hline
\textbf{1° anno} & 66 & Discipline di indirizzo + Lingua Inglese + Storia (2h/sett.)\\
\hline
\textbf{2° anno} & 99 & Discipline di indirizzo + compresenza Scienze motorie + Inglese (3h/sett.) \\
\hline
\textbf{3° anno} & 66 & Discipline di indirizzo + Lingua Italiana - Informatica(2h/sett.)  \\
\hline
\textbf{4° anno} & 198 & Discipline di indirizzo + Lingua Italiana - Inglese + Matematica (6h/sett.) \\
\hline
\end{tabular}
\end{table}
\end{frame}

% Slide 7: L'offerta formativa integrata
\begin{frame}{L'Offerta Formativa Integrata della Filiera}
\small
\begin{block}{Percorso Quadriennale}
\textbf{Informatica e Telecomunicazioni} -- Articolazione Informatica\\
Curvatura: \textcolor{accentorange}{\textbf{Intelligenza Artificiale e Cybersecurity}}
\end{block}

\begin{block}{Percorso Biennale ITS}
\textbf{ITS Turismo Marche} -- Area Tecnologica 10: Informatica\\
Corso: \textit{Full Stack Developer} (sede San Benedetto del Tronto)
\end{block}


\end{frame}

\begin{frame}{L'Offerta Formativa Integrata della Filiera}
\begin{block}{Prosecuzione universitaria}
	Lauree triennali e magistrali in:
	\begin{itemize}
		\item Informatica / Informatica per la comunicazione digitale
		\item Ingegneria Informatica / Computer Science
	\end{itemize}
	presso \textbf{Università di Camerino} e \textbf{Università Politecnica delle Marche}
\end{block}
\end{frame}


% Slide 8: Partner della filiera
\begin{frame}{Partner Strategici della Filiera}
\small
\begin{columns}[t]
\column{0.5\textwidth}
\textbf{Partner istituzionali:}
\begin{itemize}
\item \textcolor{primaryblue}{ITS Turismo Marche}
\item Università di Camerino
\item Università Politecnica Marche
\item IIS Ulpiani (Service Learning)
\item CPIA Ascoli Piceno
\end{itemize}

\column{0.5\textwidth}
\textbf{Partner professionali:}
\begin{itemize}
\item \textcolor{accentorange}{Confindustria Servizi} (IFTS e formazione docenti)
\item \textcolor{accentorange}{Cyber Evolution S.r.l.} (FSL e formazione)
\item Ordine Ingegneri Ascoli Piceno
\item Associazione Ex Allievi OdV
\end{itemize}
\end{columns}

\vspace{0.5cm}
\begin{center}
\textcolor{primaryblue}{\textbf{Un ecosistema continuo e permeabile}}\\
\textit{dalla scuola secondaria all'istruzione terziaria}
\end{center}
\end{frame}

% Slide 9: Vantaggi della filiera
\begin{frame}{Vantaggi della Filiera 4+2}
\begin{itemize}
\item[\checkmark] \textbf{Co-progettazione continua} con ITS e imprese del territorio
\item[\checkmark] \textbf{Contatti precoci} con realtà lavorative (dal 1° anno)
\item[\checkmark] \textbf{Orientamento strutturato} per il percorso ITS/universitario
\item[\checkmark] \textbf{Compresenze potenziate} fin dal 1° anno
\item[\checkmark] \textbf{Organico potenziato} 
\item[\checkmark] \textbf{240 ore di FSL} con possibilità di apprendistato formativo
\item[\checkmark] \textbf{Allineamento europeo} nel conseguimento del titolo
\item[\checkmark] \textbf{Maggiore esposizione} a competenze STEM e laboratorialità
\end{itemize}

\vspace{0.3cm}
\begin{center}
\textcolor{accentorange}{\textbf{Innovazione didattica radicale}} attraverso UDA, CLIL, metodologie attive
\end{center}
\end{frame}

% Slide 10: Metodologie didattiche innovative
\begin{frame}{Metodologie Didattiche Innovative}
\begin{columns}
\column{0.4\textwidth}
\begin{block}{Approcci metodologici}
\begin{itemize}
\item UDA interdisciplinari
\item CLIL (inglese + indirizzo)
\item Project-based learning
\item Service learning
\item Apprendimento in contesti produttivi
\end{itemize}
\end{block}

\column{0.4\textwidth}
\begin{block}{Focus competenze}
\begin{itemize}
\item Intelligenza Artificiale
\item Cybersecurity
\item Cloud Computing
\item Full Stack Development
\item Big Data \& Machine Learning
\end{itemize}
\end{block}
\end{columns}
\end{frame}

% Slide 10: Metodologie didattiche innovative
\begin{frame}{Metodologie Didattiche Innovative}
	\begin{alertblock}{Obiettivo}
		Eliminare sovrapposizioni e duplicazioni tra discipline, valorizzando l'apprendimento attivo e contestualizzato
	\end{alertblock}
\end{frame}


% Slide 11: Profili professionali in uscita
\begin{frame}{Profili Professionali in Uscita}
	\framesubtitle{Dopo il biennio ITS}
	\begin{block}{Profili tecnici}
		\begin{itemize}
			\item {10.1.1 Tecnico superiore "Full Stack Developer e Cloud Specialist"}
			\item {10.2.2 Tecnico superiore "System Cybersecurity"}
			\item {10.3.1 Tecnico superiore per la "Digitalizzazione dei processi con soluzioni Artificial Intelligence based"}
			\item {10.4.1 Tecnico Superiore in "Augmented, Virtual e Mixed Reality"}
		\end{itemize}
	\end{block}
\end{frame}

% Slide 12: Monitoraggio e valutazione
\begin{frame}{Sistema di Monitoraggio e Valutazione}
\begin{block}{Coordinamento}
Sistema strutturato coordinato da IIS Fermi-Sacconi-CPIA\\
con supporto dell'ITS Turismo Marche
\end{block}

\begin{columns}[t]
\column{0.5\textwidth}
\textbf{Ambiti di monitoraggio:}
\begin{itemize}
\item Qualità organizzazione didattica
\item Efficacia moduli laboratoriali
\item Esperienze on the job
\item Esiti formativi studenti
\end{itemize}

\column{0.5\textwidth}
\textbf{Strumenti:}
\begin{itemize}
\item Report trimestrali digitali
\item Indicatori quanti/qualitativi
\item Incontri periodici
\item Revisione progettuale continua
\end{itemize}
\end{columns}

\vspace{0.3cm}
\textcolor{primaryblue}{\textbf{Cadenza:}} ottobre-dicembre / gennaio-marzo / aprile-giugno / luglio-settembre
\end{frame}

% Slide 13: Opportunità e impegni
\begin{frame}{Opportunità e Adempimenti}
\begin{block}{Innovazione didattica}
\begin{itemize}
\item Riprogrammazione disciplinare su 4 anni (UDA)
\item Revisione curricolare verticale
\item Adozione metodologie innovative
\item Formazione continua docenti
\end{itemize}
\end{block}

\begin{block}{Collaborazione territoriale}
\begin{itemize}
\item Accordi di rete con tutti i partner
\item Co-progettazione costante con ITS Academy
\item Partnership con almeno 1 azienda del settore
\item Coinvolgimento consorzi di formazione
\end{itemize}
\end{block}

\begin{alertblock}{Vincolo di accesso}
Iscrizioni riservate a studenti con pregresso percorso scolastico di \textbf{almeno 8 anni}
\end{alertblock}
\end{frame}

% Slide 14: Conclusioni e contatti
\begin{frame}{Filiera 4+2 Informatica: Un Progetto Territoriale}
\begin{center}
\Large
\textcolor{primaryblue}{\textbf{Un ecosistema formativo integrato}}\\[0.05cm]
\normalsize
che accompagna gli studenti dalla scuola secondaria\\
all'istruzione terziaria non accademica
\end{center}

\vspace{0.1cm}
\begin{block}{Contatti}
\textbf{IIS Fermi Sacconi CPIA}\\
Via della Repubblica, 31/A -- 63100 Ascoli Piceno\\
\texttt{apis01100a@istruzione.it}\\
\texttt{apis01100a@pec.istruzione.it}
\end{block}

\vspace{0.1cm}
\begin{center}
\includegraphics[height=1.2cm]{logo_fermi_sacconi.png}\hspace{1.5cm}
\includegraphics[height=1.2cm]{logo_its_turismo.png}
\end{center}

\vspace{0.2cm}
\begin{center}
\textcolor{accentorange}{\textit{Avvio percorso: Anno Scolastico 2026/2027}}
\end{center}
\end{frame}

\end{document}
