\documentclass[aspectratio=169]{beamer}
\usetheme{Madrid}
\usecolortheme{default}

% Pacchetti
\usepackage[utf8]{inputenc}
\usepackage[italian]{babel}
\usepackage{graphicx}
\usepackage{tikz}
\usepackage{listings}
\usepackage{booktabs}
\usepackage{array}
\usepackage{colortbl}
\usepackage{xcolor}

% Configurazione listing per SQL
\lstdefinestyle{sqlstyle}{
    language=SQL,
    basicstyle=\ttfamily\small,
    keywordstyle=\color{blue}\bfseries,
    commentstyle=\color{gray}\itshape,
    stringstyle=\color{red},
    showstringspaces=false,
    breaklines=true,
    frame=single,
    numbers=left,
    numberstyle=\tiny\color{gray},
    backgroundcolor=\color{gray!10}
}

\lstset{style=sqlstyle}

% Colori personalizzati
\definecolor{mainblue}{RGB}{0,102,204}
\definecolor{lightblue}{RGB}{173,216,230}
\definecolor{darkgreen}{RGB}{0,128,0}

% Informazioni presentazione
\title{Query Language e i Raggruppamenti}
\subtitle{Operatori Aggregati e Clausole SQL}
\author{Informatica - IIS Fermi Sacconi Ceci}
\date{A.S. 2024/2025}
\institute{Ascoli Piceno}

\begin{document}

% Slide 1 - Titolo
\begin{frame}
\titlepage
\end{frame}

% Slide 2 - Indice
\begin{frame}{Indice}
\begin{center}
\begin{tikzpicture}[scale=0.8]
    % Database icon
    \draw[fill=mainblue!20] (0,0) ellipse (2cm and 0.5cm);
    \draw[fill=mainblue!30] (-2,0) -- (-2,-1) arc (180:360:2cm and 0.5cm) -- (2,0);
    \draw[fill=mainblue!20] (0,-1) ellipse (2cm and 0.5cm);
    \draw[fill=mainblue!30] (-2,-1) -- (-2,-2) arc (180:360:2cm and 0.5cm) -- (2,-1);
    \draw[fill=mainblue!20] (0,-2) ellipse (2cm and 0.5cm);
    
    % Query arrows
    \draw[->,thick,mainblue] (3,-1) -- (5,-1);
    \node[align=center] at (7,-1) {\textbf{Operatori}\\\textbf{Aggregati}};
    
    % Results
    \draw[fill=lightblue,rounded corners] (9,-1.5) rectangle (11,-0.5);
    \node at (10,-1) {\textbf{Risultati}};
\end{tikzpicture}
\end{center}

\begin{itemize}
    \item Operatori Aggregati
    \item Clausola GROUP BY
    \item Clausola HAVING
    \item Limitazione dei risultati
\end{itemize}
\end{frame}

% Slide 3 - Introduzione agli operatori aggregati
\begin{frame}{Gli Operatori Aggregati}
\begin{block}{Estensione di SQL}
Rispetto all'algebra relazionale, la più importante estensione che introduce SQL è quella degli \textbf{operatori aggregati}.
\end{block}

\vspace{0.3cm}

\textbf{Caratteristiche principali:}
\begin{itemize}
    \item Aggregano più righe della tabella
    \item Compiono operazioni a livello di \emph{relazione} e non di tupla
    \item Restituiscono un \textbf{singolo valore aggregato}
    \item Non selezionano un sottoinsieme di righe, ma calcolano valori
\end{itemize}
\end{frame}

% Slide 4 - Operazioni di aggregazione
\begin{frame}{Gli Operatori Aggregati}
\begin{block}{Funzionalità}
Le operazioni di aggregazione consentono di:
\begin{itemize}
    \item \textbf{Raggruppare} le tuple
    \item Effettuare \textbf{calcoli specifici} (somme, conteggi, statistiche)
    \item \textbf{Dividere} la tabella in sottoinsiemi
    \item Raggruppare tuple con \textbf{stessi valori} per attributi specifici
\end{itemize}
\end{block}

\vspace{0.3cm}

\begin{center}
\begin{tikzpicture}[scale=0.7]
    % Tabella originale
    \draw[fill=lightblue!30] (0,0) rectangle (2,2);
    \node at (1,2.3) {\footnotesize Tabella};
    
    % Freccia
    \draw[->,thick] (2.5,1) -- (3.5,1);
    \node[above] at (3,1.2) {\tiny GROUP BY};
    
    % Gruppi
    \draw[fill=green!20] (4,1.5) rectangle (5,2);
    \draw[fill=yellow!20] (4,1) rectangle (5,1.5);
    \draw[fill=red!20] (4,0) rectangle (5,1);
    \node at (4.5,2.3) {\footnotesize Gruppi};
\end{tikzpicture}
\end{center}
\end{frame}

% Slide 5 - Caratteristiche operatori
\begin{frame}{Gli Operatori Aggregati}
\begin{alertblock}{Differenza fondamentale}
Gli operatori di aggregazione, a differenza degli operatori matematici che agiscono soltanto sulla tupla in corso di elaborazione:
\end{alertblock}

\vspace{0.3cm}

\textbf{Possono essere applicati a più tuple} della tabella e si caratterizzano per:
\begin{itemize}
    \item Restituire un \textbf{valore unico}
    \item Operare su un \textbf{gruppo di valori}
    \item Lavorare sui valori che formano una \textbf{colonna}
\end{itemize}

\vspace{0.3cm}

\begin{center}
\begin{tikzpicture}[scale=0.6]
    \foreach \y in {0,0.5,1,1.5,2} {
        \draw[fill=lightblue!20] (0,\y) rectangle (1,\y+0.4);
    }
    \draw[->,thick,red] (1.5,1) -- (2.5,1);
    \node[red] at (2,1.3) {\tiny Aggregazione};
    \draw[fill=green!30] (3,0.8) rectangle (4,1.2);
    \node at (3.5,1) {\tiny 42};
\end{tikzpicture}
\end{center}
\end{frame}

% Slide 6 - Elenco operatori
\begin{frame}[fragile]{Gli Operatori Aggregati}
\begin{block}{Principali operatori di aggregazione}
\begin{description}
    \item[\texttt{AVG(campo)}] Calcola la \textbf{media aritmetica}
    \item[\texttt{COUNT(expr|*)}] \textbf{Conta} le righe
    \item[\texttt{MAX(expr)}] Calcola il valore \textbf{massimo}
    \item[\texttt{MIN(expr)}] Calcola il valore \textbf{minimo}
    \item[\texttt{SUM(campo)}] Calcola la \textbf{somma totale}
    \item[\texttt{STDDEV(campo)}] Calcola la \textbf{deviazione standard}
\end{description}
\end{block}

\vspace{0.2cm}

\begin{center}
\begin{tikzpicture}[scale=0.7]
    \draw[fill=mainblue!20,rounded corners] (0,0) rectangle (3,0.6);
    \node at (1.5,0.3) {\texttt{SELECT AVG(prezzo)}};
    \draw[->,thick] (3.5,0.3) -- (4.5,0.3);
    \draw[fill=green!20,rounded corners] (5,0) rectangle (6.5,0.6);
    \node at (5.75,0.3) {\texttt{234.56}};
\end{tikzpicture}
\end{center}
\end{frame}

% Slide 7 - GROUP BY e posizionamento
\begin{frame}[fragile]{Gli Operatori Aggregati}
\begin{block}{Clausola GROUP BY}
Specifica su quali campi effettuare i raggruppamenti.
\end{block}

\vspace{0.3cm}

\begin{alertblock}{Importante}
Gli operatori aggregati possono comparire solo dopo le clausole:
\begin{itemize}
    \item \texttt{SELECT}
    \item \texttt{HAVING}
\end{itemize}
\end{alertblock}

\vspace{0.3cm}

\begin{lstlisting}
SELECT categoria, COUNT(*) AS numero
FROM prodotti
GROUP BY categoria;
\end{lstlisting}
\end{frame}

% Slide 8 - Clausola HAVING
\begin{frame}[fragile]{La Clausola HAVING}
\begin{block}{Funzionamento}
La clausola \texttt{HAVING} permette di inserire un vincolo sui dati risultanti dall'operazione di raggruppamento \texttt{GROUP BY}.
\end{block}

\vspace{0.3cm}

\textbf{Differenza con WHERE:}
\begin{itemize}
    \item \texttt{WHERE} opera sui \textbf{campi del database} (prima del raggruppamento)
    \item \texttt{HAVING} opera sui \textbf{campi risultanti dai raggruppamenti} (dopo)
\end{itemize}

\vspace{0.3cm}

\begin{lstlisting}
SELECT reparto, AVG(stipendio) AS media
FROM impiegati
GROUP BY reparto
HAVING AVG(stipendio) > 30000;
\end{lstlisting}
\end{frame}

% Slide 9 - Operatore COUNT
\begin{frame}[fragile]{Operatore COUNT}
\begin{block}{Sintassi}
\texttt{COUNT(espressione | *)}
\end{block}

\vspace{0.3cm}

\begin{lstlisting}
SELECT COUNT(*) AS totale_auto
FROM auto;
\end{lstlisting}

\vspace{0.3cm}

\begin{center}
\begin{tabular}{|c|}
\hline
\rowcolor{lightblue}
\textbf{totale\_auto} \\
\hline
12 \\
\hline
\end{tabular}
\end{center}

\vspace{0.2cm}

\textbf{Nota:} \texttt{COUNT(*)} conta tutte le righe, inclusi i duplicati.
\end{frame}

% Slide 10 - COUNT con DISTINCT
\begin{frame}[fragile]{Operatore COUNT}
\begin{block}{Risultato}
Il risultato è \textbf{12}: vengono contati tutti i record che compongono la tabella e che non sono NULL, a prescindere dal loro contenuto.
\end{block}

\vspace{0.2cm}

\begin{alertblock}{Importante}
\texttt{COUNT} è l'unico operatore di aggregazione che considera i valori nulli ai fini del calcolo.
\end{alertblock}

\vspace{0.2cm}

\textbf{Per contare valori distinti:}
\begin{lstlisting}
SELECT COUNT(DISTINCT marca) AS marche_diverse
FROM auto;
\end{lstlisting}

\begin{center}
\begin{tabular}{|c|}
\hline
\rowcolor{lightblue}
\textbf{marche\_diverse} \\
\hline
8 \\
\hline
\end{tabular}
\end{center}
\end{frame}

% Slide 11 - Operatore AVG
\begin{frame}[fragile]{Operatore AVG}
\begin{block}{Media Aritmetica}
Calcola la media dei valori di una colonna numerica.
\end{block}

\vspace{0.3cm}

\begin{lstlisting}
SELECT AVG(prezzo) AS prezzo_medio
FROM auto;
\end{lstlisting}

\vspace{0.3cm}

\begin{center}
\begin{tikzpicture}[scale=0.8]
    % Tabella input
    \draw[fill=lightblue!30] (0,0) rectangle (2,3);
    \node at (1,3.3) {\footnotesize Prezzi};
    \node at (1,2.5) {25000};
    \node at (1,2) {30000};
    \node at (1,1.5) {28000};
    \node at (1,1) {35000};
    \node at (1,0.5) {27000};
    
    % Freccia
    \draw[->,thick] (2.5,1.5) -- (3.5,1.5);
    \node[above] at (3,1.7) {\tiny AVG};
    
    % Risultato
    \draw[fill=green!20] (4,1.2) rectangle (5.5,1.8);
    \node at (4.75,1.5) {29000};
\end{tikzpicture}
\end{center}
\end{frame}

% Slide 12 - Operatori MIN e MAX intro
\begin{frame}[fragile]{Operatori MIN e MAX}
\begin{block}{Funzionamento}
\texttt{MAX} e \texttt{MIN} restituiscono i valori più alti e più bassi contenuti all'interno di una colonna.
\end{block}

\vspace{0.3cm}

\textbf{Comportamento per tipo di dato:}
\begin{itemize}
    \item \textbf{Campi numerici:} restituisce il valore numerico
    \item \textbf{Campi testuali:} individua il campo che, in ordine alfabetico, è ultimo (MAX) o primo (MIN)
    \item Ordinamento secondo la tabella dei codici \textbf{ASCII}
    \item Ordinamento \textbf{lessicografico}
\end{itemize}

\vspace{0.3cm}

\begin{center}
\begin{tikzpicture}[scale=0.6]
    \draw[<->,thick] (0,0) -- (8,0);
    \node[below] at (0,-0.2) {A};
    \node[below] at (8,-0.2) {Z};
    \draw[fill=red] (1,0) circle (0.15) node[above] {\tiny MIN};
    \draw[fill=blue] (7,0) circle (0.15) node[above] {\tiny MAX};
\end{tikzpicture}
\end{center}
\end{frame}

% Slide 13 - MIN e MAX con espressioni
\begin{frame}[fragile]{Operatori MIN e MAX}
\begin{block}{Espressioni matematiche}
È possibile usare come argomento degli operatori \texttt{MIN} e \texttt{MAX} anche espressioni matematiche.
\end{block}

\vspace{0.3cm}

\begin{lstlisting}
SELECT MAX(prezzo * 1.22) AS prezzo_max_iva
FROM auto;
\end{lstlisting}

\vspace{0.3cm}

\begin{lstlisting}
SELECT MIN(prezzo - sconto) AS prezzo_min_scontato
FROM auto;
\end{lstlisting}

\vspace{0.3cm}

\begin{center}
\begin{tikzpicture}[scale=0.7]
    \node at (0,0.5) {\footnotesize Prezzo};
    \draw[->,thick] (1,0.5) -- (2,0.5);
    \node at (2.5,0.5) {\footnotesize * 1.22};
    \draw[->,thick] (3,0.5) -- (4,0.5);
    \node at (5,0.5) {\footnotesize MAX()};
\end{tikzpicture}
\end{center}
\end{frame}

% Slide 14 - Esempio MIN
\begin{frame}[fragile]{Operatore MIN e MAX}
\begin{lstlisting}
SELECT MIN(stipendio) AS stipendio_minimo
FROM impiegati;
\end{lstlisting}

\vspace{0.3cm}

\begin{center}
\begin{tabular}{|l|c|}
\hline
\rowcolor{lightblue}
\textbf{Nome} & \textbf{Stipendio} \\
\hline
Mario Rossi & 28000 \\
Laura Bianchi & 32000 \\
Giuseppe Verdi & \cellcolor{yellow}\textbf{25000} \\
Anna Neri & 35000 \\
\hline
\end{tabular}

\vspace{0.5cm}

\begin{tabular}{|c|}
\hline
\rowcolor{green!30}
\textbf{stipendio\_minimo} \\
\hline
\textbf{25000} \\
\hline
\end{tabular}
\end{center}
\end{frame}

% Slide 15 - Esempio MAX
\begin{frame}[fragile]{Operatore MIN e MAX}
\begin{lstlisting}
SELECT MAX(stipendio) AS stipendio_massimo
FROM impiegati;
\end{lstlisting}

\vspace{0.3cm}

\begin{center}
\begin{tabular}{|l|c|}
\hline
\rowcolor{lightblue}
\textbf{Nome} & \textbf{Stipendio} \\
\hline
Mario Rossi & 28000 \\
Laura Bianchi & 32000 \\
Giuseppe Verdi & 25000 \\
Anna Neri & \cellcolor{yellow}\textbf{35000} \\
\hline
\end{tabular}

\vspace{0.5cm}

\begin{tabular}{|c|}
\hline
\rowcolor{green!30}
\textbf{stipendio\_massimo} \\
\hline
\textbf{35000} \\
\hline
\end{tabular}
\end{center}
\end{frame}

% Slide 16 - MAX su campi testuali
\begin{frame}[fragile]{Operatore MIN e MAX}
\begin{lstlisting}
SELECT MAX(cognome) AS cognome_max
FROM impiegati;
\end{lstlisting}

\vspace{0.3cm}

\begin{center}
\begin{tabular}{|l|}
\hline
\rowcolor{lightblue}
\textbf{Cognome} \\
\hline
Bianchi \\
Neri \\
Rossi \\
\cellcolor{yellow}\textbf{Verdi} \\
\hline
\end{tabular}

\vspace{0.5cm}

\begin{tabular}{|c|}
\hline
\rowcolor{green!30}
\textbf{cognome\_max} \\
\hline
\textbf{Verdi} \\
\hline
\end{tabular}
\end{center}

\vspace{0.2cm}
\footnotesize{Ordinamento alfabetico: B < N < R < V}
\end{frame}

% Slide 17 - MIN su campi testuali
\begin{frame}[fragile]{Operatore MIN e MAX}
\begin{lstlisting}
SELECT MIN(cognome) AS cognome_min
FROM impiegati;
\end{lstlisting}

\vspace{0.3cm}

\begin{center}
\begin{tabular}{|l|}
\hline
\rowcolor{lightblue}
\textbf{Cognome} \\
\hline
\cellcolor{yellow}\textbf{Bianchi} \\
Neri \\
Rossi \\
Verdi \\
\hline
\end{tabular}

\vspace{0.5cm}

\begin{tabular}{|c|}
\hline
\rowcolor{green!30}
\textbf{cognome\_min} \\
\hline
\textbf{Bianchi} \\
\hline
\end{tabular}
\end{center}

\vspace{0.2cm}
\footnotesize{Ordinamento alfabetico: B < N < R < V}
\end{frame}

% Slide 18 - Espressioni complesse
\begin{frame}[fragile]{Operatore MIN e MAX}
\begin{block}{Espressioni complesse}
Vediamo come applicare operatori aritmetici nella query. Per conoscere la differenza tra stipendio minimo e massimo degli impiegati, rapportata in percentuale:
\end{block}

\vspace{0.3cm}

\begin{lstlisting}
SELECT 
    (MAX(stipendio) - MIN(stipendio)) / 
    MAX(stipendio) * 100 AS diff_percentuale
FROM impiegati;
\end{lstlisting}

\vspace{0.3cm}

\begin{center}
\begin{tabular}{|c|}
\hline
\rowcolor{green!30}
\textbf{diff\_percentuale} \\
\hline
28.57 \\
\hline
\end{tabular}
\end{center}

\footnotesize{La differenza tra lo stipendio massimo e minimo è del 28.57\%}
\end{frame}

% Slide 19 - Operatore SUM intro
\begin{frame}[fragile]{Operatore SUM}
\begin{block}{Somma totale}
Calcola la somma di tutti i valori di una colonna numerica.
\end{block}

\vspace{0.3cm}

\begin{lstlisting}
SELECT SUM(stipendio) AS totale_stipendi
FROM impiegati;
\end{lstlisting}

\vspace{0.3cm}

\begin{center}
\begin{tikzpicture}[scale=0.8]
    % Valori
    \foreach \y/\val in {3/28000, 2.3/32000, 1.6/25000, 0.9/35000} {
        \draw[fill=lightblue!30] (0,\y-0.3) rectangle (2,\y+0.2);
        \node at (1,\y) {\footnotesize \val};
    }
    
    % Simbolo somma
    \node[scale=2] at (3,1.8) {$\sum$};
    
    % Risultato
    \draw[fill=green!30] (4.5,1.5) rectangle (6.5,2.1);
    \node at (5.5,1.8) {\textbf{120000}};
\end{tikzpicture}
\end{center}
\end{frame}

% Slide 20 - SUM esempio
\begin{frame}[fragile]{Operatore SUM}
\begin{lstlisting}
SELECT 
    SUM(importo) AS totale_vendite,
    SUM(quantita) AS totale_pezzi
FROM vendite
WHERE anno = 2024;
\end{lstlisting}

\vspace{0.3cm}

\begin{center}
\begin{tabular}{|c|c|}
\hline
\rowcolor{lightblue}
\textbf{totale\_vendite} & \textbf{totale\_pezzi} \\
\hline
158750.50 & 1243 \\
\hline
\end{tabular}
\end{center}

\vspace{0.3cm}

\begin{alertblock}{Nota}
\texttt{SUM} ignora i valori NULL nel calcolo.
\end{alertblock}
\end{frame}

% Slide 21 - AVG e STDDEV
\begin{frame}[fragile]{Operatori AVG e STDDEV}
\begin{lstlisting}
SELECT 
    AVG(stipendio) AS media_stipendio,
    STDDEV(stipendio) AS deviazione_standard
FROM impiegati;
\end{lstlisting}

\vspace{0.3cm}

\begin{center}
\begin{tabular}{|c|c|}
\hline
\rowcolor{lightblue}
\textbf{media\_stipendio} & \textbf{deviazione\_standard} \\
\hline
30000.00 & 4082.48 \\
\hline
\end{tabular}
\end{center}

\vspace{0.3cm}

\begin{center}
\begin{tikzpicture}[scale=0.7]
    % Curva gaussiana semplificata
    \draw[->,thick] (-3,0) -- (3,0);
    \draw[->,thick] (0,0) -- (0,2.5);
    \draw[blue,thick,domain=-2.5:2.5,samples=50] plot (\x,{2*exp(-\x*\x/2)});
    \draw[dashed,red] (0,0) -- (0,2);
    \node[below] at (0,-0.2) {$\mu$};
    \draw[<->,green,thick] (-1,0.3) -- (1,0.3);
    \node[above,green] at (0,0.3) {$\sigma$};
\end{tikzpicture}
\end{center}
\end{frame}

% Slide 22 - GROUP BY sintassi
\begin{frame}[fragile]{La Clausola GROUP BY}
\begin{block}{Funzione}
La clausola \texttt{GROUP BY} serve per raggruppare ed elaborare in modo uniforme diverse righe che nella tabella origine hanno valori uguali in una determinata colonna.
\end{block}

\vspace{0.3cm}

\textbf{Sintassi:}
\begin{lstlisting}
SELECT colonna1, funzione_aggregata(colonna2)
FROM tabella
WHERE condizione
GROUP BY colonna1;
\end{lstlisting}

\vspace{0.3cm}

\begin{center}
\begin{tikzpicture}[scale=0.6]
    \draw[fill=lightblue!20] (0,0) rectangle (1.5,3);
    \node[rotate=90] at (0.75,1.5) {Gruppo A};
    \draw[fill=yellow!20] (2,0) rectangle (3.5,3);
    \node[rotate=90] at (2.75,1.5) {Gruppo B};
    \draw[fill=green!20] (4,0) rectangle (5.5,3);
    \node[rotate=90] at (4.75,1.5) {Gruppo C};
\end{tikzpicture}
\end{center}
\end{frame}

% Slide 23 - Priorità esecuzione 1
\begin{frame}{La Clausola GROUP BY}
\begin{block}{Ordine di esecuzione}
Tenendo conto delle considerazioni che seguono, facciamo chiarezza sulle priorità durante l'esecuzione di una selezione con raggruppamento.
\end{block}

\vspace{0.3cm}

\textbf{Fasi di elaborazione:}
\begin{enumerate}
    \item Esecuzione della \texttt{JOIN}, se presente
    \item Selezione sulle tuple con la condizione dopo \texttt{WHERE}
\end{enumerate}

\vspace{0.3cm}

\begin{center}
\begin{tikzpicture}[scale=0.7]
    \node[draw,rectangle,fill=lightblue!30] (t1) at (0,2) {Tabella 1};
    \node[draw,rectangle,fill=lightblue!30] (t2) at (3,2) {Tabella 2};
    \node[draw,rectangle,fill=yellow!30] (join) at (1.5,0.5) {JOIN};
    \node[draw,rectangle,fill=green!30] (where) at (1.5,-1) {WHERE};
    \draw[->,thick] (t1) -- (join);
    \draw[->,thick] (t2) -- (join);
    \draw[->,thick] (join) -- (where);
\end{tikzpicture}
\end{center}
\end{frame}

% Slide 24 - Priorità esecuzione 2
\begin{frame}[fragile]{La Clausola GROUP BY}
\textbf{Fasi di elaborazione (continua):}
\begin{enumerate}
    \setcounter{enumi}{2}
    \item Raggruppamento in base ai campi specificati dopo \texttt{GROUP BY}
    \item Lista dei valori specificati dopo la \texttt{SELECT} (target list)
    \item Selezione delle tuple che soddisfano la condizione dopo \texttt{HAVING}
\end{enumerate}

\vspace{0.3cm}

\begin{center}
\begin{tikzpicture}[scale=0.7]
    \node[draw,rectangle,fill=yellow!30] (group) at (0,1.5) {GROUP BY};
    \node[draw,rectangle,fill=orange!30] (select) at (0,0) {SELECT};
    \node[draw,rectangle,fill=red!30] (having) at (0,-1.5) {HAVING};
    \draw[->,thick] (group) -- (select);
    \draw[->,thick] (select) -- (having);
    \node[right] at (0.5,1.5) {\footnotesize Raggruppamento};
    \node[right] at (0.5,0) {\footnotesize Proiezione};
    \node[right] at (0.5,-1.5) {\footnotesize Filtro gruppi};
\end{tikzpicture}
\end{center}
\end{frame}

% Slide 25 - Esempio GROUP BY 1
\begin{frame}[fragile]{La Clausola GROUP BY}
\begin{lstlisting}
SELECT reparto, COUNT(*) AS numero_impiegati
FROM impiegati
GROUP BY reparto;
\end{lstlisting}

\vspace{0.3cm}

\begin{center}
\textbf{Tabella originale:}
\begin{tabular}{|l|c|}
\hline
\rowcolor{lightblue}
\textbf{Nome} & \textbf{Reparto} \\
\hline
Mario & Vendite \\
Laura & IT \\
Giuseppe & Vendite \\
Anna & IT \\
Carlo & Marketing \\
\hline
\end{tabular}

\vspace{0.3cm}

\textbf{Risultato:}
\begin{tabular}{|l|c|}
\hline
\rowcolor{green!30}
\textbf{reparto} & \textbf{numero\_impiegati} \\
\hline
Vendite & 2 \\
IT & 2 \\
Marketing & 1 \\
\hline
\end{tabular}
\end{center}
\end{frame}

% Slide 26 - Esempio GROUP BY 2
\begin{frame}[fragile]{La Clausola GROUP BY}
\begin{lstlisting}
SELECT categoria, AVG(prezzo) AS prezzo_medio
FROM prodotti
GROUP BY categoria;
\end{lstlisting}

\vspace{0.3cm}

\begin{center}
\begin{tikzpicture}[scale=0.8]
    % Tabella originale
    \node at (0,3) {\footnotesize Prodotti};
    \foreach \y/\cat/\pr in {2.5/Elettr./500, 2/Elettr./700, 1.5/Libri/15, 1/Libri/25, 0.5/Sport/80} {
        \draw[fill=lightblue!20] (-0.8,\y-0.2) rectangle (0.8,\y+0.2);
        \node[font=\tiny] at (0,\y) {\cat: \pr€};
    }
    
    % Freccia
    \draw[->,thick] (1.5,1.5) -- (2.5,1.5);
    \node[above,font=\tiny] at (2,1.7) {GROUP BY};
    
    % Risultato
    \node at (4.5,3) {\footnotesize Risultato};
    \draw[fill=green!20] (3.5,2.3) rectangle (5.5,2.7);
    \node[font=\tiny] at (4.5,2.5) {Elettronica: 600€};
    \draw[fill=green!20] (3.5,1.7) rectangle (5.5,2.1);
    \node[font=\tiny] at (4.5,1.9) {Libri: 20€};
    \draw[fill=green!20] (3.5,1.1) rectangle (5.5,1.5);
    \node[font=\tiny] at (4.5,1.3) {Sport: 80€};
\end{tikzpicture}
\end{center}
\end{frame}

% Slide 27 - Esempio GROUP BY 3
\begin{frame}[fragile]{La Clausola GROUP BY}
\begin{lstlisting}
SELECT 
    anno, 
    mese, 
    SUM(vendite) AS totale_vendite
FROM vendite_mensili
GROUP BY anno, mese
ORDER BY anno, mese;
\end{lstlisting}

\vspace{0.3cm}

\begin{center}
\begin{tabular}{|c|c|c|}
\hline
\rowcolor{lightblue}
\textbf{anno} & \textbf{mese} & \textbf{totale\_vendite} \\
\hline
2023 & 1 & 45000 \\
2023 & 2 & 52000 \\
2024 & 1 & 48000 \\
2024 & 2 & 55000 \\
\hline
\end{tabular}
\end{center}

\vspace{0.2cm}
\footnotesize{Raggruppamento su \textbf{più colonne}}
\end{frame}

% Slide 28 - Esempio GROUP BY 4
\begin{frame}[fragile]{La Clausola GROUP BY}
\begin{lstlisting}
SELECT 
    citta,
    COUNT(*) AS numero_clienti,
    SUM(ordini) AS totale_ordini
FROM clienti
GROUP BY citta;
\end{lstlisting}

\vspace{0.3cm}

\begin{center}
\begin{tabular}{|l|c|c|}
\hline
\rowcolor{lightblue}
\textbf{citta} & \textbf{numero\_clienti} & \textbf{totale\_ordini} \\
\hline
Roma & 15 & 234 \\
Milano & 22 & 387 \\
Napoli & 8 & 145 \\
Torino & 12 & 198 \\
\hline
\end{tabular}
\end{center}
\end{frame}

% Slide 29 - Esempio GROUP BY 5
\begin{frame}[fragile]{La Clausola GROUP BY}
\begin{lstlisting}
SELECT 
    marca,
    modello,
    COUNT(*) AS unita_vendute,
    AVG(prezzo) AS prezzo_medio
FROM auto_vendute
GROUP BY marca, modello
ORDER BY unita_vendute DESC;
\end{lstlisting}

\vspace{0.3cm}

\begin{center}
\begin{tabular}{|l|l|c|c|}
\hline
\rowcolor{lightblue}
\textbf{marca} & \textbf{modello} & \textbf{unità} & \textbf{prezzo} \\
\hline
Fiat & Panda & 45 & 12500 \\
Volkswagen & Golf & 38 & 22000 \\
Ford & Fiesta & 32 & 15000 \\
\hline
\end{tabular}
\end{center}
\end{frame}

% Slide 30 - Esempio GROUP BY 6
\begin{frame}[fragile]{La Clausola GROUP BY}
\begin{lstlisting}
SELECT 
    YEAR(data_vendita) AS anno,
    MONTH(data_vendita) AS mese,
    COUNT(*) AS numero_transazioni,
    SUM(importo) AS fatturato
FROM transazioni
GROUP BY YEAR(data_vendita), MONTH(data_vendita);
\end{lstlisting}

\vspace{0.2cm}

\begin{center}
\begin{tabular}{|c|c|c|c|}
\hline
\rowcolor{lightblue}
\textbf{anno} & \textbf{mese} & \textbf{transazioni} & \textbf{fatturato} \\
\hline
2024 & 10 & 152 & 45200.00 \\
2024 & 11 & 178 & 52800.00 \\
2024 & 12 & 201 & 68500.00 \\
\hline
\end{tabular}
\end{center}

\vspace{0.2cm}
\footnotesize{Raggruppamento con \textbf{funzioni} nelle colonne}
\end{frame}

% Slide 31 - HAVING esempio 1
\begin{frame}[fragile]{La Clausola HAVING}
\begin{lstlisting}
SELECT reparto, AVG(stipendio) AS media_stipendio
FROM impiegati
GROUP BY reparto
HAVING AVG(stipendio) > 30000;
\end{lstlisting}

\vspace{0.3cm}

\begin{center}
\textbf{Tutti i gruppi:}
\begin{tabular}{|l|c|}
\hline
\rowcolor{lightblue}
\textbf{reparto} & \textbf{media\_stipendio} \\
\hline
Vendite & 28000 \\
IT & \cellcolor{yellow}35000 \\
Marketing & 27000 \\
Produzione & \cellcolor{yellow}32000 \\
\hline
\end{tabular}

\vspace{0.3cm}

\textbf{Risultato con HAVING:}
\begin{tabular}{|l|c|}
\hline
\rowcolor{green!30}
\textbf{reparto} & \textbf{media\_stipendio} \\
\hline
IT & 35000 \\
Produzione & 32000 \\
\hline
\end{tabular}
\end{center}
\end{frame}

% Slide 32 - HAVING esempio 2
\begin{frame}[fragile]{La Clausola HAVING}
\begin{lstlisting}
SELECT categoria, COUNT(*) AS numero_prodotti
FROM prodotti
GROUP BY categoria
HAVING COUNT(*) >= 5;
\end{lstlisting}

\vspace{0.3cm}

\begin{center}
\begin{tabular}{|l|c|}
\hline
\rowcolor{green!30}
\textbf{categoria} & \textbf{numero\_prodotti} \\
\hline
Elettronica & 12 \\
Abbigliamento & 8 \\
Casa & 6 \\
\hline
\end{tabular}
\end{center}

\vspace{0.3cm}

\begin{alertblock}{Nota}
\texttt{HAVING} filtra i gruppi \textbf{dopo} l'aggregazione, mentre \texttt{WHERE} filtra le righe \textbf{prima}.
\end{alertblock}
\end{frame}

% Slide 33 - HAVING esempio 3
\begin{frame}[fragile]{La Clausola HAVING}
\begin{lstlisting}
SELECT 
    cliente_id,
    SUM(importo) AS totale_acquisti
FROM ordini
WHERE anno = 2024
GROUP BY cliente_id
HAVING SUM(importo) > 10000;
\end{lstlisting}

\vspace{0.3cm}

\begin{center}
\begin{tikzpicture}[scale=0.7]
    \node[draw,rectangle,fill=lightblue!30] at (0,3) {Tutte le righe};
    \draw[->,thick] (0,2.7) -- (0,2.3);
    \node[draw,rectangle,fill=yellow!30] at (0,2) {WHERE};
    \node[right,font=\tiny] at (0.3,2) {anno=2024};
    \draw[->,thick] (0,1.7) -- (0,1.3);
    \node[draw,rectangle,fill=orange!30] at (0,1) {GROUP BY};
    \node[right,font=\tiny] at (0.3,1) {cliente\_id};
    \draw[->,thick] (0,0.7) -- (0,0.3);
    \node[draw,rectangle,fill=red!30] at (0,0) {HAVING};
    \node[right,font=\tiny] at (0.3,0) {SUM>10000};
\end{tikzpicture}
\end{center}
\end{frame}

% Slide 34 - HAVING esempio 4
\begin{frame}[fragile]{La Clausola HAVING}
\begin{lstlisting}
SELECT 
    citta,
    COUNT(*) AS numero_ordini,
    AVG(importo) AS importo_medio
FROM ordini
GROUP BY citta
HAVING COUNT(*) > 10 AND AVG(importo) > 500;
\end{lstlisting}

\vspace{0.3cm}

\begin{center}
\begin{tabular}{|l|c|c|}
\hline
\rowcolor{green!30}
\textbf{citta} & \textbf{numero\_ordini} & \textbf{importo\_medio} \\
\hline
Milano & 25 & 675.50 \\
Roma & 18 & 820.30 \\
\hline
\end{tabular}
\end{center}

\vspace{0.2cm}
\footnotesize{Condizioni \textbf{multiple} in HAVING}
\end{frame}

% Slide 35 - HAVING esempio 5
\begin{frame}[fragile]{La Clausola HAVING}
\begin{lstlisting}
SELECT 
    marca,
    COUNT(*) AS modelli,
    MIN(prezzo) AS prezzo_min,
    MAX(prezzo) AS prezzo_max
FROM automobili
GROUP BY marca
HAVING MAX(prezzo) - MIN(prezzo) > 20000;
\end{lstlisting}

\vspace{0.3cm}

\begin{center}
\begin{tabular}{|l|c|c|c|}
\hline
\rowcolor{green!30}
\textbf{marca} & \textbf{modelli} & \textbf{prezzo\_min} & \textbf{prezzo\_max} \\
\hline
BMW & 8 & 35000 & 85000 \\
Mercedes & 6 & 40000 & 95000 \\
\hline
\end{tabular}
\end{center}

\vspace{0.2cm}
\footnotesize{HAVING con \textbf{espressioni} tra aggregazioni}
\end{frame}

% Slide 36 - LIMIT intro
\begin{frame}[fragile]{Limitazione delle Tuple Risultato}
\begin{block}{Clausola LIMIT}
Permette di limitare il numero di righe restituite dalla query.
\end{block}

\vspace{0.3cm}

\textbf{Sintassi:}
\begin{lstlisting}
SELECT colonne
FROM tabella
WHERE condizione
ORDER BY colonna
LIMIT numero;
\end{lstlisting}

\vspace{0.3cm}

\textbf{Utilizzi comuni:}
\begin{itemize}
    \item Visualizzare solo i primi N risultati
    \item Implementare la paginazione
    \item Ottimizzare le performance delle query
\end{itemize}
\end{frame}

% Slide 37 - LIMIT esempio 1
\begin{frame}[fragile]{Limitazione delle Tuple Risultato}
\begin{lstlisting}
SELECT nome, cognome, stipendio
FROM impiegati
ORDER BY stipendio DESC
LIMIT 5;
\end{lstlisting}

\vspace{0.3cm}

\begin{center}
\textbf{Top 5 stipendi più alti:}
\begin{tabular}{|l|l|c|}
\hline
\rowcolor{lightblue}
\textbf{nome} & \textbf{cognome} & \textbf{stipendio} \\
\hline
Anna & Neri & 45000 \\
Mario & Rossi & 42000 \\
Laura & Bianchi & 38000 \\
Giuseppe & Verdi & 35000 \\
Carlo & Gialli & 33000 \\
\hline
\end{tabular}
\end{center}

\vspace{0.2cm}
\footnotesize{Restituisce solo le prime 5 righe ordinate}
\end{frame}

% Slide 38 - LIMIT con OFFSET
\begin{frame}[fragile]{Limitazione delle Tuple Risultato}
\begin{block}{Paginazione con OFFSET}
\texttt{OFFSET} permette di saltare un numero specificato di righe.
\end{block}

\vspace{0.3cm}

\begin{lstlisting}
SELECT nome, prezzo
FROM prodotti
ORDER BY prezzo DESC
LIMIT 10 OFFSET 20;
\end{lstlisting}

\vspace{0.3cm}

\begin{center}
\begin{tikzpicture}[scale=0.7]
    \foreach \x in {0,...,4} {
        \draw[fill=gray!30] (\x*0.4,2) rectangle (\x*0.4+0.3,2.5);
    }
    \node[right] at (2.2,2.25) {\tiny Skip 20};
    
    \foreach \x in {0,...,9} {
        \draw[fill=green!30] (\x*0.4,1) rectangle (\x*0.4+0.3,1.5);
    }
    \node[right] at (4,1.25) {\tiny Get 10};
    
    \foreach \x in {0,...,4} {
        \draw[fill=gray!30] (\x*0.4,0) rectangle (\x*0.4+0.3,0.5);
    }
    \node[right] at (2.2,0.25) {\tiny Altri...};
\end{tikzpicture}
\end{center}

\vspace{0.2cm}
\footnotesize{Utile per la paginazione: pagina 3 = LIMIT 10 OFFSET 20}
\end{frame}

% Slide 39 - LIMIT esempio pratico
\begin{frame}[fragile]{Limitazione delle Tuple Risultato}
\begin{lstlisting}
SELECT 
    prodotto,
    vendite,
    fatturato
FROM statistiche_vendite
ORDER BY fatturato DESC
LIMIT 3;
\end{lstlisting}

\vspace{0.3cm}

\begin{center}
\textbf{Top 3 prodotti per fatturato:}
\begin{tabular}{|l|c|c|}
\hline
\rowcolor{lightblue}
\textbf{prodotto} & \textbf{vendite} & \textbf{fatturato} \\
\hline
\cellcolor{yellow!50}Laptop Pro & 234 & 234000 € \\
\cellcolor{yellow!30}Smartphone X & 456 & 182400 € \\
\cellcolor{yellow!10}Tablet Plus & 198 & 99000 € \\
\hline
\end{tabular}
\end{center}

\vspace{0.3cm}

\begin{alertblock}{Best Practice}
Usare sempre \texttt{ORDER BY} con \texttt{LIMIT} per risultati consistenti.
\end{alertblock}
\end{frame}

% Slide 40 - LIMIT e performance
\begin{frame}[fragile]{Limitazione delle Tuple Risultato}
\begin{block}{Vantaggi di LIMIT}
\begin{itemize}
    \item \textbf{Performance:} riduce il carico sul database
    \item \textbf{Memoria:} trasferisce meno dati
    \item \textbf{Usabilità:} migliora l'esperienza utente
    \item \textbf{Testing:} utile per testare query su grandi dataset
\end{itemize}
\end{block}

\vspace{0.3cm}

\begin{lstlisting}
-- Query di test su tabella grande
SELECT * FROM log_eventi
WHERE data >= '2024-01-01'
LIMIT 100;
\end{lstlisting}

\vspace{0.3cm}

\begin{center}
\begin{tikzpicture}[scale=0.6]
    \draw[fill=red!30] (0,0) rectangle (4,1);
    \node at (2,0.5) {1.000.000 righe};
    \draw[->,thick] (4.5,0.5) -- (5.5,0.5);
    \node[above] at (5,0.7) {\tiny LIMIT 100};
    \draw[fill=green!30] (6,0.3) rectangle (7,0.7);
    \node[right] at (7.2,0.5) {\tiny 100 righe};
\end{tikzpicture}
\end{center}
\end{frame}

\end{document}
