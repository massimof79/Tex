\documentclass[aspectratio=169,10pt]{beamer}
\usepackage[utf8]{inputenc}
\usepackage[italian]{babel}
\usepackage{listings}
\usepackage{xcolor}
\usepackage{graphicx}
\usepackage{tikz}
\usepackage{booktabs}
\usepackage{array}

% Tema e colori
\usetheme{Madrid}
\usecolortheme{default}

% Definizione colori personalizzati
\definecolor{sqlblue}{RGB}{0,102,204}
\definecolor{sqlgreen}{RGB}{0,153,76}
\definecolor{sqlred}{RGB}{204,0,0}
\definecolor{codebg}{RGB}{245,245,245}
\definecolor{commentgreen}{RGB}{0,128,0}
\definecolor{keywordblue}{RGB}{0,0,255}

% Configurazione listing per SQL
\lstdefinestyle{sqlstyle}{
	language=SQL,
	basicstyle=\ttfamily\small,
	keywordstyle=\color{keywordblue}\bfseries,
	commentstyle=\color{commentgreen}\itshape,
	stringstyle=\color{sqlred},
	backgroundcolor=\color{codebg},
	frame=single,
	breaklines=true,
	showstringspaces=false,
	numbers=left,
	numberstyle=\tiny\color{gray},
	captionpos=b,
	tabsize=2,
	morekeywords={CONSTRAINT, REFERENCES, CASCADE, RESTRICT, CHECK, DEFAULT, AUTO_INCREMENT}
}

\lstset{style=sqlstyle}

% Informazioni sul documento
\title{Il Linguaggio SQL}
\subtitle{Costrutti DDL - Data Definition Language}
\author{Prof. Massimo Fedeli}
\institute{ITS Made in Italy - Fabbrica Digitale}
\date{\today}

\begin{document}
	
	% --- SLIDE 1: Titolo ---
	\begin{frame}
		\titlepage
	\end{frame}
	
	% --- SLIDE 2: Sommario ---
	\begin{frame}{Sommario}
		\tableofcontents
	\end{frame}
	
	% ===============================================
	% SEZIONE 1: INTRODUZIONE AL DDL
	% ===============================================
	\section{Introduzione al DDL}
	
	% --- SLIDE 3 ---
	\begin{frame}{Data Definition Language (DDL)}
		\begin{block}{Definizione}
			Il DDL è l'insieme di istruzioni utilizzate per \textbf{modificare la struttura} della base di dati.
		\end{block}
		
		\vspace{0.3cm}
		
		\begin{columns}[T]
			\column{0.48\textwidth}
			\textbf{Operazioni principali:}
			\begin{itemize}
				\item Definizione di schemi
				\item Creazione di tabelle
				\item Modifica di strutture
				\item Cancellazione di oggetti
			\end{itemize}
			
			\column{0.48\textwidth}
			\textbf{Elementi gestiti:}
			\begin{itemize}
				\item Tabelle (table)
				\item Vincoli (constraints)
				\item Domini
				\item Viste (view)
			\end{itemize}
		\end{columns}
	\end{frame}
	
	% --- SLIDE 4 ---
	\begin{frame}{DDL vs DML}
		\begin{center}
			\begin{tabular}{|p{0.45\textwidth}|p{0.45\textwidth}|}
				\hline
				\rowcolor{sqlblue!20}
				\textbf{DDL (Definition)} & \textbf{DML (Manipulation)} \\
				\hline
				Modifica la \textbf{struttura} & Modifica i \textbf{dati} \\
				\hline
				CREATE, ALTER, DROP & INSERT, UPDATE, DELETE, SELECT \\
				\hline
				Definisce schemi e tabelle & Opera sulle righe \\
				\hline
				Effetto permanente & Può essere annullato (ROLLBACK) \\
				\hline
			\end{tabular}
		\end{center}
		
		\vspace{0.3cm}
		
		\begin{alertblock}{Attenzione}
			Le operazioni DDL sono \textbf{irreversibili} e modificano la struttura del database in modo permanente.
		\end{alertblock}
	\end{frame}
	
	% ===============================================
	% SEZIONE 2: GESTIONE DEGLI SCHEMI
	% ===============================================
	\section{Gestione degli Schemi}
	
	% --- SLIDE 5 ---
	\begin{frame}[fragile]{Creazione di uno Schema}
		\begin{block}{Sintassi}
			\begin{lstlisting}
				CREATE SCHEMA [IF NOT EXISTS] NomeSchema
				[AUTHORIZATION utente];
			\end{lstlisting}
		\end{block}
		
		\textbf{Esempi pratici:}
		\begin{lstlisting}
			-- Creazione schema semplice
			CREATE SCHEMA IF NOT EXISTS UniversitaDB;
			
			-- Creazione con autorizzazione
			CREATE SCHEMA AziendaDB AUTHORIZATION amministratore;
		\end{lstlisting}
		
		\begin{block}{Note}
			\begin{itemize}
				\item La clausola \texttt{IF NOT EXISTS} evita errori se lo schema esiste già
				\item Uno schema è una collezione logica di tabelle e oggetti
			\end{itemize}
		\end{block}
	\end{frame}
	
	% --- SLIDE 6 ---
	\begin{frame}[fragile]{Selezione e Cancellazione Schema}
		\textbf{Selezionare uno schema:}
		\begin{lstlisting}
			USE NomeSchema;
		\end{lstlisting}
		
		\vspace{0.3cm}
		
		\textbf{Cancellare uno schema:}
		\begin{lstlisting}
			DROP SCHEMA [IF EXISTS] NomeSchema;
		\end{lstlisting}
		
		\begin{alertblock}{Attenzione!}
			Il comando \texttt{DROP SCHEMA} cancella \textbf{tutto il contenuto} dello schema, incluse tutte le tabelle e i dati in esse contenuti.
		\end{alertblock}
		
		\textbf{Esempio completo:}
		\begin{lstlisting}
			USE UniversitaDB;  -- Seleziona lo schema
			-- ... operazioni varie ...
			DROP SCHEMA IF EXISTS TestDB;  -- Cancella schema di test
		\end{lstlisting}
	\end{frame}
	
	% ===============================================
	% SEZIONE 3: CREAZIONE DELLE TABELLE
	% ===============================================
	\section{Creazione delle Tabelle}
	
	% --- SLIDE 7 ---
	\begin{frame}[fragile]{CREATE TABLE - Sintassi Base}
		\begin{block}{Sintassi}
			\begin{lstlisting}
				CREATE TABLE [IF NOT EXISTS] NomeTabella (
				nome_campo1 tipo_campo1 [vincoli],
				nome_campo2 tipo_campo2 [vincoli],
				...
				[vincoli_tabella]
				);
			\end{lstlisting}
		\end{block}
		
		\textbf{Esempio basilare:}
		\begin{lstlisting}
			CREATE TABLE Studenti (
			Matricola CHAR(10),
			Nome VARCHAR(20),
			Cognome VARCHAR(20),
			AnnoIscrizione INTEGER
			);
		\end{lstlisting}
	\end{frame}
	
	% --- SLIDE 8 ---
	\begin{frame}[fragile]{Tipi di Dato in SQL}
		\begin{columns}[T]
			\column{0.48\textwidth}
			\textbf{Tipi numerici:}
			\begin{itemize}
				\item \texttt{INTEGER} / \texttt{INT}
				\item \texttt{SMALLINT}
				\item \texttt{DECIMAL(p,s)} / \texttt{NUMERIC}
				\item \texttt{FLOAT} / \texttt{REAL}
				\item \texttt{DOUBLE}
			\end{itemize}
			
			\column{0.48\textwidth}
			\textbf{Tipi stringa:}
			\begin{itemize}
				\item \texttt{CHAR(n)} - lunghezza fissa
				\item \texttt{VARCHAR(n)} - lunghezza variabile
				\item \texttt{TEXT} - testo lungo
			\end{itemize}
			
			\textbf{Altri tipi:}
			\begin{itemize}
				\item \texttt{DATE}
				\item \texttt{TIME}
				\item \texttt{DATETIME} / \texttt{TIMESTAMP}
				\item \texttt{BOOLEAN}
			\end{itemize}
		\end{columns}
	\end{frame}
	
	% --- SLIDE 9 ---
	\begin{frame}[fragile]{Esempio Completo di CREATE TABLE}
		\begin{lstlisting}
			CREATE TABLE IF NOT EXISTS Studenti (
			Matricola CHAR(10),
			Nome VARCHAR(20) NOT NULL,
			Cognome VARCHAR(20) NOT NULL,
			CodiceFiscale CHAR(16) UNIQUE,
			DataNascita DATE,
			AnnoIscrizione INTEGER DEFAULT 1,
			Email VARCHAR(100),
			CONSTRAINT pk_matricola PRIMARY KEY (Matricola)
			);
		\end{lstlisting}
		
		\begin{block}{Elementi chiave}
			\begin{itemize}
				\item \texttt{NOT NULL}: impedisce valori nulli
				\item \texttt{UNIQUE}: valori univoci nella colonna
				\item \texttt{DEFAULT}: valore predefinito
				\item \texttt{PRIMARY KEY}: chiave primaria
			\end{itemize}
		\end{block}
	\end{frame}
	
	% --- SLIDE 10 ---
	\begin{frame}[fragile]{Valori di Default}
		\begin{block}{Definizione}
			I valori di default specificano cosa assegnare all'attributo quando non si indica un valore esplicitamente.
		\end{block}
		
		\textbf{Esempi:}
		\begin{lstlisting}
			CREATE TABLE Iscrizioni (
			ID INTEGER PRIMARY KEY,
			DataIscrizione DATE DEFAULT CURRENT_DATE,
			AnnoAccademico INTEGER DEFAULT 1,
			NumeroPatente CHAR(20) DEFAULT NULL,
			Stato VARCHAR(20) DEFAULT 'ATTIVO'
			);
		\end{lstlisting}
		
		\begin{alertblock}{Nota}
			Se non si specifica un valore di default, si assume \texttt{NULL} (se ammesso).
		\end{alertblock}
	\end{frame}
	
	% ===============================================
	% SEZIONE 4: VINCOLI INTRA-RELAZIONALI
	% ===============================================
	\section{Vincoli Intra-relazionali}
	
	% --- SLIDE 11 ---
	\begin{frame}{Tipologie di Vincoli}
		\begin{center}
			\begin{tikzpicture}[
				level 1/.style={sibling distance=4cm, level distance=1.5cm},
				level 2/.style={sibling distance=2cm, level distance=1.5cm},
				every node/.style={draw, rectangle, rounded corners, fill=blue!10, minimum width=3cm, align=center}
				]
				\node {Vincoli d'Integrità}
				child { node {Vincoli\\Intra-relazionali}
					child { node[fill=green!10] {NOT NULL} }
					child { node[fill=green!10] {UNIQUE} }
					child { node[fill=green!10] {PRIMARY KEY} }
					child { node[fill=green!10] {CHECK} }
				}
				child { node {Vincoli\\Inter-relazionali}
					child { node[fill=orange!10] {FOREIGN KEY} }
					child { node[fill=orange!10] {REFERENCES} }
				};
			\end{tikzpicture}
		\end{center}
	\end{frame}
	
	% --- SLIDE 12 ---
	\begin{frame}[fragile]{Vincolo NOT NULL}
		\begin{block}{Scopo}
			Vieta la presenza di valori nulli in quella colonna.
		\end{block}
		
		\textbf{Sintassi:}
		\begin{lstlisting}
			CREATE TABLE Studenti (
			Matricola CHAR(10),
			Nome VARCHAR(20) NOT NULL DEFAULT 'N.D.',
			Cognome VARCHAR(20) NOT NULL,
			Email VARCHAR(100),  -- può essere NULL
			PRIMARY KEY (Matricola)
			);
		\end{lstlisting}
		
		\begin{alertblock}{Importante}
			\begin{itemize}
				\item Il vincolo NOT NULL si applica a \textbf{singole colonne}
				\item Il valore di default viene specificato \textbf{dopo} il vincolo NOT NULL
				\item Non si può specificare NOT NULL per coppie di colonne
			\end{itemize}
		\end{alertblock}
	\end{frame}
	
	% --- SLIDE 13 ---
	\begin{frame}[fragile]{Vincolo UNIQUE}
		\begin{block}{Scopo}
			Garantisce che non esistano due righe con gli stessi valori per l'attributo specificato.
		\end{block}
		
		\textbf{UNIQUE su singola colonna:}
		\begin{lstlisting}
			CREATE TABLE Studenti (
			Matricola CHAR(10),
			CodiceFiscale CHAR(16) UNIQUE,
			Email VARCHAR(100) NOT NULL UNIQUE,
			Nome VARCHAR(20) NOT NULL,
			Cognome VARCHAR(20) NOT NULL,
			PRIMARY KEY (Matricola)
			);
		\end{lstlisting}
		
		\begin{block}{Nota su NULL}
			Il vincolo UNIQUE \textbf{non esclude} la presenza di più righe con valori NULL (che si assumono tutti diversi fra loro).
		\end{block}
	\end{frame}
	
	% --- SLIDE 14 ---
	\begin{frame}[fragile]{Vincolo UNIQUE su Più Colonne}
		\textbf{Sintassi 1: Senza nome vincolo}
		\begin{lstlisting}
			CREATE TABLE Studenti (
			Matricola CHAR(10),
			Nome VARCHAR(20) NOT NULL,
			Cognome VARCHAR(20) NOT NULL,
			AnnoIscrizione INTEGER,
			UNIQUE(Nome, Cognome),
			PRIMARY KEY (Matricola)
			);
		\end{lstlisting}
		
		\textbf{Sintassi 2: Con nome vincolo}
		\begin{lstlisting}
			CREATE TABLE Studenti (
			Matricola CHAR(10),
			Nome VARCHAR(20) NOT NULL,
			Cognome VARCHAR(20) NOT NULL,
			AnnoIscrizione INTEGER,
			CONSTRAINT uq_nome_cognome UNIQUE(Nome, Cognome),
			PRIMARY KEY (Matricola)
			);
		\end{lstlisting}
	\end{frame}
	
	% --- SLIDE 15 ---
	\begin{frame}[fragile]{Ordine dei Vincoli}
		\begin{block}{Regola}
			Quando i vincoli sono espressi su una singola colonna, l'ordine è:
			\begin{center}
				\texttt{tipo} $\rightarrow$ \texttt{NOT NULL} $\rightarrow$ \texttt{DEFAULT} $\rightarrow$ \texttt{UNIQUE}
			\end{center}
		\end{block}
		
		\textbf{Esempio corretto:}
		\begin{lstlisting}
			CREATE TABLE Studenti (
			Matricola CHAR(10),
			Nome VARCHAR(20) NOT NULL DEFAULT 'N.D.' UNIQUE,
			Cognome VARCHAR(20) NOT NULL,
			AnnoIscrizione INTEGER,
			PRIMARY KEY (Matricola)
			);
		\end{lstlisting}
		
		\begin{alertblock}{Attenzione}
			Un ordine errato può causare errori di sintassi.
		\end{alertblock}
	\end{frame}
	
	% --- SLIDE 16 ---
	\begin{frame}[fragile]{Vincolo PRIMARY KEY}
		\begin{block}{Scopo}
			Identifica la chiave primaria della tabella (implica NOT NULL e UNIQUE).
		\end{block}
		
		\textbf{Può esserci un solo vincolo di PRIMARY KEY per tabella!}
		
		\textbf{Sintassi 1: Chiave primaria singola}
		\begin{lstlisting}
			CREATE TABLE Studenti (
			Matricola CHAR(10) PRIMARY KEY,
			Nome VARCHAR(20) NOT NULL,
			Cognome VARCHAR(20) NOT NULL
			);
		\end{lstlisting}
		
		\textbf{Sintassi 2: Chiave primaria composta}
		\begin{lstlisting}
			CREATE TABLE Veicoli (
			Targa CHAR(10),
			CodiceProprietario CHAR(20) NOT NULL,
			PRIMARY KEY (Targa, CodiceProprietario)
			);
		\end{lstlisting}
	\end{frame}
	
	% --- SLIDE 17 ---
	\begin{frame}[fragile]{PRIMARY KEY - Tre Modalità}
		\textbf{Modalità 1: Inline (solo chiave singola)}
		\begin{lstlisting}
			CREATE TABLE Veicoli (
			Targa CHAR(10) PRIMARY KEY,
			CodiceProprietario CHAR(20) NOT NULL
			);
		\end{lstlisting}
		
		\textbf{Modalità 2: A livello di tabella}
		\begin{lstlisting}
			CREATE TABLE Veicoli (
			Targa CHAR(10),
			CodiceProprietario CHAR(20) NOT NULL,
			PRIMARY KEY (Targa, CodiceProprietario)
			);
		\end{lstlisting}
		
		\textbf{Modalità 3: Con nome vincolo}
		\begin{lstlisting}
			CREATE TABLE Veicoli (
			Targa CHAR(10),
			CodiceProprietario CHAR(20) NOT NULL,
			CONSTRAINT pk_veicoli PRIMARY KEY (Targa, CodiceProprietario)
			);
		\end{lstlisting}
	\end{frame}
	
	% --- SLIDE 18 ---
	\begin{frame}[fragile]{Vincolo CHECK}
		\begin{block}{Scopo}
			Verifica generiche condizioni sui valori di una o più colonne.
		\end{block}
		
		\textbf{CHECK su singola colonna:}
		\begin{lstlisting}
			CREATE TABLE Esami (
			Matricola CHAR(10),
			Corso VARCHAR(20) UNIQUE,
			Voto INTEGER CHECK (Voto >= 18 AND Voto <= 30),
			PRIMARY KEY (Corso, Matricola)
			);
		\end{lstlisting}
		
		\textbf{CHECK su più colonne:}
		\begin{lstlisting}
			CREATE TABLE Fatture (
			ID INTEGER PRIMARY KEY,
			ImportoLordo DECIMAL(10,2),
			Netto DECIMAL(10,2),
			Ritenute DECIMAL(10,2),
			CHECK (ImportoLordo = Netto + Ritenute)
			);
		\end{lstlisting}
	\end{frame}
	
	% --- SLIDE 19 ---
	\begin{frame}[fragile]{CHECK - Esempi Avanzati}
		\textbf{Con nome vincolo:}
		\begin{lstlisting}
			CREATE TABLE Esami (
			Matricola CHAR(10),
			Corso VARCHAR(20),
			Voto INTEGER,
			Lode BOOLEAN DEFAULT FALSE,
			CONSTRAINT ck_voto CHECK (Voto >= 18 AND Voto <= 30),
			CONSTRAINT ck_lode CHECK (Lode = FALSE OR Voto = 30),
			PRIMARY KEY (Corso, Matricola)
			);
		\end{lstlisting}
		
		\begin{block}{Nota}
			Il vincolo è violato se esiste \textbf{almeno una tupla} che rende falsa la condizione.
		\end{block}
		
		\textbf{Altri esempi:}
		\begin{lstlisting}
			CHECK (DataFine > DataInizio)
			CHECK (Stipendio > 0)
			CHECK (Email LIKE '%@%.%')
		\end{lstlisting}
	\end{frame}
	
	% ===============================================
	% SEZIONE 5: VINCOLI INTER-RELAZIONALI
	% ===============================================
	\section{Vincoli Inter-relazionali}
	
	% --- SLIDE 20 ---
	\begin{frame}{Integrità Referenziale}
		\begin{block}{Definizione}
			Vincolo che crea un legame tra valori di attributi in tabelle diverse.
		\end{block}
		
		\begin{center}
			\begin{tikzpicture}[
				table/.style={rectangle, draw=blue!50, fill=blue!10, thick, minimum width=3cm, minimum height=2cm},
				arrow/.style={->, >=stealth, thick, blue!70}
				]
				% Tabella interna
				\node[table] (interno) at (0,0) {
					\textbf{Studente}\\[0.2cm]
					\texttt{PK} Matricola\\
					\texttt{FK} CodiceEsame\\
					Nome
				};
				
				% Tabella esterna
				\node[table] (esterno) at (6,0) {
					\textbf{Esame}\\[0.2cm]
					\texttt{PK} Codice\\
					Nome\\
					Crediti
				};
				
				% Freccia
				\draw[arrow] (interno.east) -- (esterno.west) node[midway, above] {riferisce};
				\node[below=0.5cm of interno] {\small Tabella interna/referente};
				\node[below=0.5cm of esterno] {\small Tabella esterna/riferita};
			\end{tikzpicture}
		\end{center}
		
		\begin{alertblock}{Regola}
			L'attributo della tabella esterna deve essere soggetto a vincolo UNIQUE (di solito PRIMARY KEY).
		\end{alertblock}
	\end{frame}
	
	% --- SLIDE 21 ---
	\begin{frame}[fragile]{FOREIGN KEY - Concetto}
		\begin{block}{Vincolo di Foreign Key}
			Impone che, per ogni tupla, il valore dell'attributo della tabella interna, se diverso da NULL, deve essere uguale a un valore dell'attributo della tabella esterna.
		\end{block}
		
		\textbf{Schema esempio:}
		\begin{lstlisting}
			-- Tabella esterna (riferita)
			CREATE TABLE Dipartimenti (
			NomeDipartimento CHAR(15) PRIMARY KEY,
			Sede VARCHAR(50)
			);
			
			-- Tabella interna (referente)
			CREATE TABLE Impiegati (
			Matricola CHAR(6) PRIMARY KEY,
			Nome VARCHAR(50) NOT NULL,
			Cognome VARCHAR(50) NOT NULL,
			Dipartimento CHAR(15),
			FOREIGN KEY (Dipartimento) 
			REFERENCES Dipartimenti(NomeDipartimento)
			);
		\end{lstlisting}
	\end{frame}
	
	% --- SLIDE 22 ---
	\begin{frame}[fragile]{Vincolo REFERENCES}
		\begin{block}{Uso}
			Permette di specificare vincoli di colonna per singoli attributi.
		\end{block}
		
		\textbf{Sintassi compatta:}
		\begin{lstlisting}
			CREATE TABLE Impiegati (
			Matricola CHAR(6) PRIMARY KEY,
			Cognome VARCHAR(50) NOT NULL,
			Nome VARCHAR(50) NOT NULL,
			Dipartimento CHAR(15) 
			REFERENCES Dipartimenti(NomeDipartimento),
			Stipendio DECIMAL(10,2)
			);
		\end{lstlisting}
		
		\begin{block}{Quando usare REFERENCES}
			\begin{itemize}
				\item Quando la foreign key è composta da \textbf{un solo attributo}
				\item Per una sintassi più compatta e leggibile
			\end{itemize}
		\end{block}
	\end{frame}
	
	% --- SLIDE 23 ---
	\begin{frame}[fragile]{FOREIGN KEY per Più Attributi}
		\begin{block}{Vincolo FOREIGN KEY}
			Necessario quando si referenziano più attributi contemporaneamente.
		\end{block}
		
		\begin{lstlisting}
			CREATE TABLE Anagrafica (
			Nome VARCHAR(50),
			Cognome VARCHAR(50),
			DataNascita DATE,
			PRIMARY KEY (Nome, Cognome)
			);
			
			CREATE TABLE Impiegati (
			Matricola CHAR(6) PRIMARY KEY,
			NomeDip VARCHAR(50),
			CognomeDip VARCHAR(50),
			Stipendio DECIMAL(10,2),
			FOREIGN KEY (NomeDip, CognomeDip) 
			REFERENCES Anagrafica(Nome, Cognome)
			);
		\end{lstlisting}
	\end{frame}
	
	% --- SLIDE 24 ---
	\begin{frame}[fragile]{FOREIGN KEY con Nome Vincolo}
		\begin{block}{Best Practice}
			Dare un nome al vincolo facilita successive modifiche e debugging.
		\end{block}
		
		\begin{lstlisting}
			CREATE TABLE Impiegati (
			Matricola CHAR(6) PRIMARY KEY,
			NomeDip VARCHAR(50),
			CognomeDip VARCHAR(50),
			Dipartimento CHAR(15),
			CONSTRAINT fk_anagrafica 
			FOREIGN KEY (NomeDip, CognomeDip) 
			REFERENCES Anagrafica(Nome, Cognome),
			CONSTRAINT fk_dipartimento 
			FOREIGN KEY (Dipartimento) 
			REFERENCES Dipartimenti(NomeDipartimento)
			);
		\end{lstlisting}
		
		\begin{alertblock}{Vantaggio}
			Il nome permette di rimuovere facilmente il vincolo con \texttt{ALTER TABLE ... DROP CONSTRAINT nome\_vincolo}
		\end{alertblock}
	\end{frame}
	
	% ===============================================
	% SEZIONE 6: POLITICHE DI REAZIONE
	% ===============================================
	\section{Politiche di Reazione}
	
	% --- SLIDE 25 ---
	\begin{frame}{Politiche di Reazione - Introduzione}
		\begin{block}{Problema}
			Cosa succede quando si modifica o cancella una riga nella tabella esterna che è referenziata da una foreign key?
		\end{block}
		
		\begin{columns}[T]
			\column{0.48\textwidth}
			\textbf{Tabella interna:}
			\begin{itemize}
				\item Inserimento: sempre controllato
				\item Modifica: sempre controllata
				\item \textcolor{red}{Nessuna politica necessaria}
			\end{itemize}
			
			\column{0.48\textwidth}
			\textbf{Tabella esterna:}
			\begin{itemize}
				\item Modifica: \textcolor{blue}{politica definibile}
				\item Cancellazione: \textcolor{blue}{politica definibile}
				\item 4 opzioni disponibili
			\end{itemize}
		\end{columns}
		
		\vspace{0.3cm}
		
		\begin{alertblock}{Default}
			Se non specificata, la politica predefinita è \texttt{NO ACTION} (o \texttt{RESTRICT}).
		\end{alertblock}
	\end{frame}
	
	% --- SLIDE 26 ---
	\begin{frame}[fragile]{Sintassi Politiche di Reazione}
		\begin{block}{Sintassi}
			\begin{lstlisting}
				FOREIGN KEY (colonna) REFERENCES tabella(colonna)
				ON DELETE { NO ACTION | CASCADE | SET NULL | SET DEFAULT }
				ON UPDATE { NO ACTION | CASCADE | SET NULL | SET DEFAULT }
			\end{lstlisting}
		\end{block}
		
		\textbf{Le quattro politiche:}
		\begin{enumerate}
			\item \texttt{NO ACTION} (o \texttt{RESTRICT}): blocca l'operazione
			\item \texttt{CASCADE}: propaga l'operazione
			\item \texttt{SET NULL}: imposta NULL
			\item \texttt{SET DEFAULT}: imposta valore di default
		\end{enumerate}
		
		\begin{alertblock}{MySQL}
			La politica \texttt{SET DEFAULT} \textbf{non è supportata} in MySQL.
		\end{alertblock}
	\end{frame}
	
	% --- SLIDE 27 ---
	\begin{frame}{Politiche ON UPDATE}
		\begin{block}{NO ACTION / RESTRICT}
			Se si tenta di aggiornare un valore referenziato, l'operazione viene \textbf{bloccata} e viene generato un errore.
		\end{block}
		
		\begin{block}{CASCADE}
			Se si aggiorna un valore nella tabella esterna, tutti i valori corrispondenti nella tabella interna vengono \textbf{aggiornati automaticamente} al nuovo valore.
		\end{block}
		
		\begin{block}{SET NULL}
			Se si aggiorna un valore nella tabella esterna, i valori corrispondenti nella tabella interna vengono impostati a \texttt{NULL}.
		\end{block}
		
		\begin{block}{SET DEFAULT}
			I valori corrispondenti nella tabella interna vengono impostati al \textbf{valore di default} (non supportato in MySQL).
		\end{block}
	\end{frame}
	
	% --- SLIDE 28 ---
	\begin{frame}{Politiche ON DELETE}
		\begin{block}{NO ACTION / RESTRICT}
			Se si tenta di cancellare una riga referenziata, l'operazione viene \textbf{bloccata} e viene generato un errore.
		\end{block}
		
		\begin{block}{CASCADE}
			Se si cancella una riga nella tabella esterna, tutte le righe corrispondenti nella tabella interna vengono \textbf{cancellate automaticamente}.
		\end{block}
		
		\begin{block}{SET NULL}
			Se si cancella una riga nella tabella esterna, i valori della foreign key nella tabella interna vengono impostati a \texttt{NULL}.
		\end{block}
		
		\begin{block}{SET DEFAULT}
			I valori della foreign key nella tabella interna vengono impostati al \textbf{valore di default} (non supportato in MySQL).
		\end{block}
	\end{frame}
	
	% --- SLIDE 29 ---
	\begin{frame}[fragile]{Esempio CASCADE}
		\begin{lstlisting}
			CREATE TABLE Dipartimenti (
			IdReparto INTEGER PRIMARY KEY,
			Nome VARCHAR(50)
			);
			
			CREATE TABLE Impiegati (
			Matricola CHAR(6) PRIMARY KEY,
			Nome VARCHAR(50),
			IdReparto INTEGER,
			CONSTRAINT fk_reparto FOREIGN KEY (IdReparto)
			REFERENCES Dipartimenti(IdReparto)
			ON DELETE CASCADE
			ON UPDATE CASCADE
			);
		\end{lstlisting}
		
		\begin{exampleblock}{Comportamento}
			\begin{itemize}
				\item Se si cancella un dipartimento, vengono cancellati \textbf{tutti gli impiegati} di quel dipartimento
				\item Se si modifica l'ID di un dipartimento, viene aggiornato automaticamente in tutti gli impiegati
			\end{itemize}
		\end{exampleblock}
	\end{frame}
	
	% --- SLIDE 30 ---
	\begin{frame}[fragile]{Esempio SET NULL}
		\begin{lstlisting}
			CREATE TABLE Impiegati (
			Matricola CHAR(6) PRIMARY KEY,
			Nome VARCHAR(50),
			IdReparto INTEGER,
			CONSTRAINT fk_reparto FOREIGN KEY (IdReparto)
			REFERENCES Dipartimenti(IdReparto)
			ON DELETE SET NULL
			ON UPDATE NO ACTION
			);
		\end{lstlisting}
		
		\begin{exampleblock}{Comportamento}
			\begin{itemize}
				\item Se si cancella un dipartimento, il campo \texttt{IdReparto} degli impiegati viene impostato a \texttt{NULL}
				\item Le modifiche all'ID del dipartimento sono \textbf{bloccate} se ci sono impiegati collegati
			\end{itemize}
		\end{exampleblock}
		
		\begin{alertblock}{Requisito}
			La colonna \texttt{IdReparto} deve \textbf{permettere NULL}.
		\end{alertblock}
	\end{frame}
	
	% --- SLIDE 31 ---
	\begin{frame}[fragile]{Esempio Completo con Politiche}
		\begin{lstlisting}
			CREATE TABLE Studenti (
			Matricola CHAR(10) PRIMARY KEY,
			Nome VARCHAR(50) NOT NULL,
			Cognome VARCHAR(50) NOT NULL
			);
			
			CREATE TABLE Iscrizioni (
			CodIscrizione CHAR(20),
			Matricola CHAR(10),
			CorsoID INTEGER,
			CONSTRAINT pk_iscrizioni 
			PRIMARY KEY (CodIscrizione, Matricola),
			CONSTRAINT fk_studenti 
			FOREIGN KEY (Matricola) 
			REFERENCES Studenti(Matricola)
			ON DELETE CASCADE       -- cancellazione in cascata
			ON UPDATE NO ACTION     -- modifiche non permesse
			);
		\end{lstlisting}
	\end{frame}
	
	% ===============================================
	% SEZIONE 7: MODIFICA DELLO SCHEMA
	% ===============================================
	\section{Modifica dello Schema}
	
	% --- SLIDE 32 ---
	\begin{frame}[fragile]{DROP TABLE}
		\begin{block}{Scopo}
			Rimuove completamente una tabella dal database.
		\end{block}
		
		\textbf{Sintassi:}
		\begin{lstlisting}
			DROP TABLE [IF EXISTS] NomeTabella [RESTRICT | CASCADE];
		\end{lstlisting}
		
		\begin{columns}[T]
			\column{0.48\textwidth}
			\textbf{RESTRICT:}
			\begin{itemize}
				\item Cancella solo se vuota
				\item Nessun oggetto dipendente
				\item (MySQL: non implementato)
			\end{itemize}
			
			\column{0.48\textwidth}
			\textbf{CASCADE:}
			\begin{itemize}
				\item Cancella tabella e dati
				\item Cancella viste dipendenti
				\item (MySQL: non implementato)
			\end{itemize}
		\end{columns}
		
		\vspace{0.3cm}
		
		\begin{alertblock}{MySQL}
			In MySQL, \texttt{DROP TABLE} cancella sempre la tabella e tutti i suoi dati, ignorando RESTRICT/CASCADE.
		\end{alertblock}
	\end{frame}
	
	% --- SLIDE 33 ---
	\begin{frame}[fragile]{DROP TABLE - Esempi}
		\textbf{Esempi pratici:}
		\begin{lstlisting}
			-- Cancellazione semplice
			DROP TABLE Studenti;
			
			-- Cancellazione sicura (no errore se non esiste)
			DROP TABLE IF EXISTS StudentiTemp;
			
			-- Cancellazione di più tabelle
			DROP TABLE IF EXISTS Tabella1, Tabella2, Tabella3;
		\end{lstlisting}
		
		\begin{alertblock}{Attenzione alle Foreign Key}
			Se altre tabelle hanno foreign key che riferiscono la tabella da cancellare:
			\begin{itemize}
				\item L'operazione viene bloccata
				\item Soluzione: disabilitare temporaneamente i controlli
			\end{itemize}
		\end{alertblock}
		
		\begin{lstlisting}
			SET FOREIGN_KEY_CHECKS = 0;
			DROP TABLE TabellaEsterna;
			SET FOREIGN_KEY_CHECKS = 1;
		\end{lstlisting}
	\end{frame}
	
	% --- SLIDE 34 ---
	\begin{frame}{DROP TABLE vs DELETE FROM}
		\begin{center}
			\begin{tabular}{|p{0.45\textwidth}|p{0.45\textwidth}|}
				\hline
				\rowcolor{sqlblue!20}
				\textbf{DROP TABLE Tabella} & \textbf{DELETE FROM Tabella} \\
				\hline
				Elimina la \textbf{struttura} & Elimina solo i \textbf{dati} \\
				\hline
				Cancella lo schema & Mantiene lo schema \\
				\hline
				Rimuove vincoli e indici & Mantiene vincoli e indici \\
				\hline
				Cancella viste dipendenti & Mantiene le viste \\
				\hline
				\textbf{Irreversibile} & Può essere annullato (ROLLBACK) \\
				\hline
				Più veloce & Più lento \\
				\hline
			\end{tabular}
		\end{center}
		
		\begin{exampleblock}{Quando usare cosa?}
			\begin{itemize}
				\item \texttt{DROP TABLE}: quando non serve più la tabella
				\item \texttt{DELETE FROM}: quando si vogliono cancellare solo i dati
			\end{itemize}
		\end{exampleblock}
	\end{frame}
	
	% --- SLIDE 35 ---
	\begin{frame}[fragile]{ALTER TABLE - Introduzione}
		\begin{block}{Scopo}
			Permette di modificare la struttura di una tabella esistente.
		\end{block}
		
		\textbf{Operazioni possibili:}
		\begin{itemize}
			\item Aggiungere colonne
			\item Rimuovere colonne
			\item Modificare colonne
			\item Aggiungere vincoli
			\item Rimuovere vincoli
		\end{itemize}
		
		\textbf{Sintassi base:}
		\begin{lstlisting}
			ALTER TABLE NomeTabella 
			ADD [COLUMN] NomeColonna Definizione;
			
			ALTER TABLE NomeTabella 
			DROP [COLUMN] NomeColonna;
		\end{lstlisting}
	\end{frame}
	
	% --- SLIDE 36 ---
	\begin{frame}[fragile]{ALTER TABLE - Aggiungere Colonne}
		\textbf{Aggiunta colonna semplice:}
		\begin{lstlisting}
			ALTER TABLE Studenti 
			ADD COLUMN Sesso CHAR(1);
		\end{lstlisting}
		
		\textbf{Aggiunta con vincoli e default:}
		\begin{lstlisting}
			ALTER TABLE Studenti 
			ADD COLUMN DataIscrizione DATE DEFAULT CURRENT_DATE;
			
			ALTER TABLE Studenti 
			ADD COLUMN Email VARCHAR(100) UNIQUE;
			
			ALTER TABLE Studenti 
			ADD COLUMN Telefono VARCHAR(15) NOT NULL DEFAULT 'N.D.';
		\end{lstlisting}
		
		\begin{alertblock}{Nota}
			Le nuove colonne assumono valore \texttt{NULL} o il valore di default per tutte le righe esistenti.
		\end{alertblock}
	\end{frame}
	
	% --- SLIDE 37 ---
	\begin{frame}[fragile]{ALTER TABLE - Rimuovere Colonne}
		\textbf{Rimozione colonna:}
		\begin{lstlisting}
			ALTER TABLE Studenti 
			DROP COLUMN AnnoIscrizione;
		\end{lstlisting}
		
		\begin{alertblock}{Attenzione!}
			\begin{itemize}
				\item La cancellazione è \textbf{irreversibile}
				\item Se la colonna è referenziata da foreign key, l'operazione \textbf{fallisce}
				\item Tutti i dati nella colonna vengono \textbf{persi}
			\end{itemize}
		\end{alertblock}
		
		\textbf{Esempio che fallisce:}
		\begin{lstlisting}
			CREATE TABLE BaseTabella (
			ID INTEGER PRIMARY KEY,
			Nome VARCHAR(50) UNIQUE
			);
			
			-- Fallisce se Nome è referenziato
			ALTER TABLE BaseTabella DROP COLUMN Nome;
		\end{lstlisting}
	\end{frame}
	
	% --- SLIDE 38 ---
	\begin{frame}[fragile]{ALTER TABLE - Gestione Vincoli (1)}
		\textbf{Aggiungere vincolo di PRIMARY KEY:}
		\begin{lstlisting}
			ALTER TABLE Studenti 
			ADD PRIMARY KEY (Matricola);
		\end{lstlisting}
		
		\textbf{Rimuovere PRIMARY KEY:}
		\begin{lstlisting}
			ALTER TABLE Studenti 
			DROP PRIMARY KEY;
		\end{lstlisting}
		
		\textbf{Aggiungere UNIQUE:}
		\begin{lstlisting}
			ALTER TABLE Studenti 
			ADD CONSTRAINT uq_email UNIQUE (Email);
		\end{lstlisting}
		
		\textbf{Rimuovere UNIQUE:}
		\begin{lstlisting}
			ALTER TABLE Studenti 
			DROP INDEX uq_email;
			-- oppure
			ALTER TABLE Studenti 
			DROP KEY uq_email;
		\end{lstlisting}
	\end{frame}
	
	% --- SLIDE 39 ---
	\begin{frame}[fragile]{ALTER TABLE - Gestione Vincoli (2)}
		\textbf{Aggiungere FOREIGN KEY:}
		\begin{lstlisting}
			ALTER TABLE Impiegati 
			ADD CONSTRAINT fk_dipartimento 
			FOREIGN KEY (IdReparto) 
			REFERENCES Dipartimenti(IdReparto)
			ON DELETE CASCADE;
		\end{lstlisting}
		
		\textbf{Rimuovere FOREIGN KEY:}
		\begin{lstlisting}
			ALTER TABLE Impiegati 
			DROP FOREIGN KEY fk_dipartimento;
		\end{lstlisting}
		
		\begin{alertblock}{Importante}
			\begin{itemize}
				\item Quando si aggiunge un vincolo, \textbf{deve essere soddisfatto} dall'istanza corrente
				\item Usare nomi espliciti per i vincoli facilita la loro rimozione
			\end{itemize}
		\end{alertblock}
	\end{frame}
	
	% --- SLIDE 40 ---
	\begin{frame}[fragile]{ALTER TABLE - Esempio Completo}
		\begin{lstlisting}
			-- Creazione tabella base
			CREATE TABLE Studenti (
			Matricola CHAR(10),
			Nome VARCHAR(50)
			);
			
			-- Aggiunta PRIMARY KEY
			ALTER TABLE Studenti ADD PRIMARY KEY (Matricola);
			
			-- Aggiunta colonne
			ALTER TABLE Studenti ADD COLUMN Cognome VARCHAR(50);
			ALTER TABLE Studenti ADD COLUMN Email VARCHAR(100) UNIQUE;
			
			-- Aggiunta vincolo NOT NULL (MySQL: modifica colonna)
			ALTER TABLE Studenti MODIFY Cognome VARCHAR(50) NOT NULL;
			
			-- Aggiunta FOREIGN KEY
			ALTER TABLE Studenti 
			ADD COLUMN CorsoID INTEGER,
			ADD CONSTRAINT fk_corso 
			FOREIGN KEY (CorsoID) REFERENCES Corsi(ID);
		\end{lstlisting}
	\end{frame}
	
	% ===============================================
	% SEZIONE 8: VISTE (VIEWS)
	% ===============================================
	\section{Viste (Views)}
	
	% --- SLIDE 41 ---
	\begin{frame}[fragile]{Viste - Introduzione}
		\begin{block}{Definizione}
			Le viste sono \textbf{tabelle virtuali} il cui contenuto dipende dal contenuto di altre tabelle.
		\end{block}
		
		\textbf{Sintassi:}
		\begin{lstlisting}
			CREATE [OR REPLACE] VIEW NomeVista [(ListaAttributi)] 
			AS SelectSQL;
		\end{lstlisting}
		
		\textbf{Esempio base:}
		\begin{lstlisting}
			CREATE VIEW ImpiegatiCostosi (Matricola, Nome, Cognome) 
			AS SELECT Matricola, Nome, Cognome
			FROM Impiegati
			WHERE StipendioAnnuale > 30000;
		\end{lstlisting}
		
		\begin{block}{Nota}
			La lista degli attributi può essere omessa. In tal caso, lo schema è quello della SELECT.
		\end{block}
	\end{frame}
	
	% --- SLIDE 42 ---
	\begin{frame}[fragile]{Utilizzo delle Viste}
		\textbf{Una volta creata, la vista si usa come una normale tabella:}
		\begin{lstlisting}
			-- Interrogazione della vista
			SELECT * FROM ImpiegatiCostosi;
			
			-- Join con la vista
			SELECT IC.Nome, IC.Cognome, D.Nome AS Dipartimento
			FROM ImpiegatiCostosi IC
			JOIN Dipartimenti D ON IC.IdReparto = D.IdReparto;
			
			-- Filtri sulla vista
			SELECT * FROM ImpiegatiCostosi
			WHERE Nome LIKE 'M%'
			ORDER BY Cognome;
		\end{lstlisting}
		
		\begin{exampleblock}{Vantaggio}
			Le viste permettono di semplificare query complesse e riutilizzarle facilmente.
		\end{exampleblock}
	\end{frame}
	
	% --- SLIDE 43 ---
	\begin{frame}[fragile]{Viste - Esempi Pratici}
		\textbf{Vista per impiegati recenti:}
		\begin{lstlisting}
			CREATE VIEW ImpiegatiRecenti (Matricola, Nome, Cognome, DataAssunzione)
			AS SELECT Matricola, Nome, Cognome, DataAssunzione
			FROM Impiegati
			WHERE DataAssunzione > '2020-01-01';
		\end{lstlisting}
		
		\textbf{Vista con join:}
		\begin{lstlisting}
			CREATE VIEW ImpiegatiMilanesi 
			AS SELECT I.Matricola, I.Nome, I.Cognome, I.StipendioAnnuale
			FROM Impiegati I
			JOIN Reparti R ON I.IdReparto = R.IdReparto
			WHERE R.Citta = 'Milano';
		\end{lstlisting}
		
		\textbf{Uso della vista:}
		\begin{lstlisting}
			-- Media stipendi impiegati milanesi
			SELECT AVG(StipendioAnnuale) FROM ImpiegatiMilanesi;
		\end{lstlisting}
	\end{frame}
	
	% --- SLIDE 44 ---
	\begin{frame}[fragile]{Viste - Esempi con Aggregazione}
		\textbf{Vista con GROUP BY:}
		\begin{lstlisting}
			CREATE VIEW StatisticheDipartimenti
			AS SELECT 
			D.Nome AS Dipartimento,
			COUNT(*) AS NumeroImpiegati,
			AVG(I.StipendioAnnuale) AS StipendioMedio,
			MAX(I.StipendioAnnuale) AS StipendioMax
			FROM Dipartimenti D
			JOIN Impiegati I ON D.IdReparto = I.IdReparto
			GROUP BY D.IdReparto, D.Nome;
		\end{lstlisting}
		
		\textbf{Interrogazione della vista:}
		\begin{lstlisting}
			-- Dipartimenti con più di 10 impiegati
			SELECT * FROM StatisticheDipartimenti
			WHERE NumeroImpiegati > 10;
			
			-- Dipartimenti ordinati per stipendio medio
			SELECT * FROM StatisticheDipartimenti
			ORDER BY StipendioMedio DESC;
		\end{lstlisting}
	\end{frame}
	
	% --- SLIDE 45 ---
	\begin{frame}{Utilizzi Tipici delle Viste}
		\begin{enumerate}
			\item \textbf{Sicurezza:}
			\begin{itemize}
				\item Impedire ad alcuni utenti l'accesso completo ai dati
				\item Esempio: vista che nasconde stipendi e dati sensibili
			\end{itemize}
			
			\vspace{0.2cm}
			
			\item \textbf{Convenienza:}
			\begin{itemize}
				\item Semplificare la scrittura delle query
				\item Quando una query complessa compare più volte
			\end{itemize}
			
			\vspace{0.2cm}
			
			\item \textbf{Indipendenza logica:}
			\begin{itemize}
				\item Isolare le applicazioni dalla struttura fisica del DB
				\item Se cambia lo schema, le applicazioni continuano a funzionare
			\end{itemize}
			
			\vspace{0.2cm}
			
			\item \textbf{Riutilizzo del codice:}
			\begin{itemize}
				\item Definire una volta, usare ovunque
				\item Manutenzione centralizzata della logica di business
			\end{itemize}
		\end{enumerate}
	\end{frame}
	
	% --- SLIDE 46 ---
	\begin{frame}[fragile]{Modifica e Cancellazione Viste}
		\textbf{Modificare una vista esistente:}
		\begin{lstlisting}
			CREATE OR REPLACE VIEW ImpiegatiCostosi 
			AS SELECT Matricola, Nome, Cognome, StipendioAnnuale
			FROM Impiegati
			WHERE StipendioAnnuale > 40000;  -- soglia modificata
		\end{lstlisting}
		
		\textbf{Cancellare una vista:}
		\begin{lstlisting}
			DROP VIEW [IF EXISTS] NomeVista;
		\end{lstlisting}
		
		\textbf{Esempi:}
		\begin{lstlisting}
			DROP VIEW ImpiegatiCostosi;
			
			DROP VIEW IF EXISTS ImpiegatiRecenti;
			
			-- Cancellazione multipla
			DROP VIEW IF EXISTS Vista1, Vista2, Vista3;
		\end{lstlisting}
		
		\begin{alertblock}{Nota}
			Cancellare una vista non elimina i dati sottostanti, solo la definizione della vista.
		\end{alertblock}
	\end{frame}
	
	% ===============================================
	% SEZIONE 9: DATA CONTROL LANGUAGE (DCL)
	% ===============================================
	\section{Data Control Language (DCL)}
	
	% --- SLIDE 47 ---
	\begin{frame}[fragile]{DCL - Introduzione}
		\begin{block}{Definizione}
			Il DCL è il linguaggio per gestire i permessi e l'accesso al database.
		\end{block}
		
		\textbf{Comandi principali:}
		\begin{itemize}
			\item \texttt{CREATE USER}: crea nuovi utenti
			\item \texttt{DROP USER}: elimina utenti
			\item \texttt{GRANT}: concede privilegi
			\item \texttt{REVOKE}: revoca privilegi
		\end{itemize}
		
		\vspace{0.3cm}
		
		\textbf{Creazione utente:}
		\begin{lstlisting}
			CREATE USER 'mario'@'%' IDENTIFIED BY 'password123';
			CREATE USER 'admin'@'localhost' IDENTIFIED BY 'admin_pwd';
		\end{lstlisting}
		
		\textbf{Cancellazione utente:}
		\begin{lstlisting}
			DROP USER 'mario'@'%';
		\end{lstlisting}
	\end{frame}
	
	% --- SLIDE 48 ---
	\begin{frame}[fragile]{GRANT - Concessione Privilegi}
		\textbf{Sintassi base:}
		\begin{lstlisting}
			GRANT privilegi ON oggetto 
			TO utente 
			[WITH GRANT OPTION];
		\end{lstlisting}
		
		\textbf{Esempi:}
		\begin{lstlisting}
			-- Tutti i privilegi su una tabella
			GRANT ALL PRIVILEGES ON Impiegati TO 'mario'@'%';
			
			-- Privilegi specifici
			GRANT SELECT, INSERT ON Studenti TO 'utente1'@'localhost';
			
			-- Su tutto il database
			GRANT SELECT ON UniversitaDB.* TO 'lettore'@'%';
			
			-- Con possibilità di delegare
			GRANT SELECT ON Impiegati TO 'admin'@'%' 
			WITH GRANT OPTION;
		\end{lstlisting}
		
		\begin{block}{WITH GRANT OPTION}
			Permette all'utente di concedere a sua volta i privilegi ricevuti ad altri utenti.
		\end{block}
	\end{frame}
	
	% --- SLIDE 49 ---
	\begin{frame}[fragile]{REVOKE - Revoca Privilegi}
		\textbf{Sintassi:}
		\begin{lstlisting}
			REVOKE privilegi ON oggetto FROM utente;
		\end{lstlisting}
		
		\textbf{Esempi:}
		\begin{lstlisting}
			-- Revoca tutti i privilegi
			REVOKE ALL PRIVILEGES ON Impiegati FROM 'mario'@'%';
			
			-- Revoca privilegi specifici
			REVOKE INSERT, UPDATE ON Studenti FROM 'utente1'@'localhost';
			
			-- Revoca su database
			REVOKE SELECT ON UniversitaDB.* FROM 'lettore'@'%';
		\end{lstlisting}
		
		\begin{alertblock}{Importante}
			\begin{itemize}
				\item Un utente può revocare \textbf{solo privilegi che lui ha concesso}
				\item La revoca non è automaticamente ricorsiva
			\end{itemize}
		\end{alertblock}
	\end{frame}
	
	% --- SLIDE 50 ---
	\begin{frame}[fragile]{Privilegi Disponibili}
		\begin{columns}[T]
			\column{0.48\textwidth}
			\textbf{Privilegi sui dati:}
			\begin{itemize}
				\item \texttt{SELECT}
				\item \texttt{INSERT}
				\item \texttt{UPDATE}
				\item \texttt{DELETE}
			\end{itemize}
			
			\textbf{Privilegi sulla struttura:}
			\begin{itemize}
				\item \texttt{CREATE}
				\item \texttt{ALTER}
				\item \texttt{DROP}
				\item \texttt{INDEX}
			\end{itemize}
			
			\column{0.48\textwidth}
			\textbf{Privilegi amministrativi:}
			\begin{itemize}
				\item \texttt{ALL PRIVILEGES}
				\item \texttt{GRANT OPTION}
				\item \texttt{CREATE USER}
				\item \texttt{RELOAD}
				\item \texttt{SHUTDOWN}
			\end{itemize}
			
			\textbf{Privilegi speciali:}
			\begin{itemize}
				\item \texttt{EXECUTE} (procedure)
				\item \texttt{REFERENCES}
				\item \texttt{TRIGGER}
			\end{itemize}
		\end{columns}
	\end{frame}
	
	% --- SLIDE 51: Riepilogo ---
	\begin{frame}{Riepilogo Costrutti DDL}
		\begin{center}
			\begin{tabular}{|l|l|}
				\hline
				\rowcolor{sqlblue!20}
				\textbf{Operazione} & \textbf{Comando} \\
				\hline
				Creare schema & \texttt{CREATE SCHEMA} \\
				\hline
				Creare tabella & \texttt{CREATE TABLE} \\
				\hline
				Modificare tabella & \texttt{ALTER TABLE} \\
				\hline
				Cancellare tabella & \texttt{DROP TABLE} \\
				\hline
				Creare vista & \texttt{CREATE VIEW} \\
				\hline
				Cancellare vista & \texttt{DROP VIEW} \\
				\hline
				Concedere privilegi & \texttt{GRANT} \\
				\hline
				Revocare privilegi & \texttt{REVOKE} \\
				\hline
			\end{tabular}
		\end{center}
		
		\vspace{0.5cm}
		
		\begin{block}{Ricorda}
			Il DDL modifica la \textbf{struttura} del database, non i dati. Le operazioni sono \textbf{permanenti} e vanno eseguite con attenzione.
		\end{block}
	\end{frame}
	
	% --- SLIDE 52: Bibliografia ---
	\begin{frame}{Bibliografia e Riferimenti}
		\begin{itemize}
			\item SQL Standard - ISO/IEC 9075
			\item MySQL Reference Manual - \url{https://dev.mysql.com/doc/}
			\item PostgreSQL Documentation - \url{https://www.postgresql.org/docs/}
			\item Atzeni, Ceri, Paraboschi, Torlone - "Basi di Dati"
			\item Elmasri, Navathe - "Fundamentals of Database Systems"
		\end{itemize}
		
		\vspace{0.5cm}
		
		\begin{center}
			\Large{\textbf{Grazie per l'attenzione!}}
			
			\vspace{0.3cm}
			
			\normalsize
			Per domande o chiarimenti:\\
			\texttt{massimo.tivoli@iisfermi.edu.it}
		\end{center}
	\end{frame}
	
\end{document}