\documentclass[aspectratio=169,11pt]{beamer}
\usepackage[utf8]{inputenc}
\usepackage[italian]{babel}
\usepackage{amsmath}
\usepackage{amssymb}
\usepackage{graphicx}
\usepackage{booktabs}
\usepackage{array}
\usepackage{multirow}
\usepackage{xcolor}
\usepackage{tikz}
\usetikzlibrary{shapes,arrows,positioning}

% Tema
\usetheme{Madrid}
\usecolortheme{whale}

% Colori personalizzati
\definecolor{darkblue}{RGB}{0,51,102}
\definecolor{lightblue}{RGB}{102,178,255}
\definecolor{orange}{RGB}{255,128,0}

\setbeamercolor{structure}{fg=darkblue}
\setbeamercolor{frametitle}{bg=darkblue,fg=white}

% Informazioni
\title{Modello Relazionale: Algebra Relazionale}
\author{Prof. Fedeli Massimo - Tutti i diritti riservati}
\institute{ITS 4.0}

\begin{document}

% Slide 1: Titolo
\begin{frame}
\titlepage
\end{frame}

% Slide 2: Indice
\begin{frame}{Indice}
\tableofcontents
\end{frame}

% SEZIONE 1: INTRODUZIONE
\section{Introduzione al Modello Relazionale}

% Slide 3
\begin{frame}{Database Relazionali}
\begin{block}{Definizione}
Un \textbf{database relazionale} è una collezione di \textcolor{orange}{tabelle}
\end{block}

\begin{center}
\begin{tikzpicture}[node distance=1.5cm]
\node[draw, rectangle, fill=orange!30, minimum width=4cm, minimum height=1cm] (db) {Database relazionali};
\node[draw, rectangle, fill=lightblue!30, below of=db, minimum width=3cm] (tab) {Tabelle};

\draw[->, thick] (db) -- node[right] {è una collezione di} (tab);
\end{tikzpicture}
\end{center}
\end{frame}

% Slide 4
\begin{frame}{Operazioni sulle Tabelle}
\begin{columns}
\begin{column}{0.5\textwidth}
\textbf{Operazioni di base:}
\begin{itemize}
\item Selezione
\item Proiezione
\end{itemize}
\textcolor{darkblue}{(operazioni unarie)}
\end{column}

\begin{column}{0.5\textwidth}
\textbf{Operazioni complesse:}
\begin{itemize}
\item Prodotto cartesiano
\item Unione
\item Differenza
\end{itemize}
\textcolor{darkblue}{(operazioni binarie)}
\end{column}
\end{columns}

\vspace{0.5cm}
\begin{alertblock}{Operazioni derivate}
Intersezione, Congiunzione (Join)
\end{alertblock}
\end{frame}


% Slide 6
\begin{frame}{Storia del Modello Relazionale}
\begin{block}{E.F. Codd (1970)}
Il modello relazionale è stato introdotto formalmente da Edgar F. Codd nel 1970
\end{block}

\textbf{Caratteristiche principali:}
\begin{itemize}
\item Descrizione semplice ma rigorosa dei dati
\item Basato sul concetto di \textbf{relazione matematica}
\item Fondamento teorico: teoria degli insiemi e logica dei predicati
\end{itemize}
\end{frame}

% Slide 7
\begin{frame}{Rappresentazione dei Dati}
\begin{itemize}
\item I dati sono rappresentati come \textbf{tabelle bidimensionali}
\item Ogni tabella rappresenta un'entità del mondo reale
\item Le righe sono \textbf{tuple} (record)
\item Le colonne sono \textbf{attributi} (campi)
\end{itemize}

\vspace{0.5cm}
\begin{exampleblock}{Esempio}
Tabella STUDENTI: ogni riga rappresenta uno studente
\end{exampleblock}
\end{frame}

% SEZIONE 2: LINGUAGGI DI INTERROGAZIONE
\section{Linguaggi di Interrogazione}

% Slide 8
\begin{frame}{Linguaggi di Interrogazione}
\textbf{Due famiglie principali:}

\begin{enumerate}
\item \textbf{Algebra Relazionale}
\begin{itemize}
\item Linguaggio procedurale
\item Notazione algebrica
\item Usa 5 operatori fondamentali
\end{itemize}

\item \textbf{Calcolo Relazionale}
\begin{itemize}
\item Linguaggio dichiarativo
\item Notazione logica
\item Base per SQL
\end{itemize}
\end{enumerate}
\end{frame}

% Slide 9
\begin{frame}{Algebra Relazionale}
\begin{block}{Definizione}
Linguaggio procedurale con notazione algebrica che utilizza \textcolor{orange}{5 operatori} fondamentali
\end{block}

\textbf{Caratteristiche:}
\begin{itemize}
\item Usa simboli matematici
\item Permette di formulare interrogazioni complesse
\item Fornisce il pieno potere espressivo
\end{itemize}

\vspace{0.3cm}
\centering
\Large \textcolor{darkblue}{Usa simboli matematici! \(\sigma, \pi, \times, \cup, -\)}
\end{frame}

% Slide 10
\begin{frame}{Calcolo Relazionale e SQL}
\begin{block}{Calcolo Relazionale}
Linguaggio di alto livello con caratteristiche dichiarative
\end{block}

\textbf{SQL (Structured Query Language):}
\begin{itemize}
\item Parzialmente dichiarativo
\item L'utente specifica \textbf{cosa} vuole ottenere
\item Non specifica \textbf{come} ottenerlo
\item Il DBMS decide la strategia di esecuzione
\end{itemize}

\begin{exampleblock}{Ricorda}
SQL è il linguaggio che studieremo nella prossima unità!
\end{exampleblock}
\end{frame}

% Slide 11
\begin{frame}{Dichiarativo vs Procedurale}
\begin{columns}
\begin{column}{0.48\textwidth}
\textbf{Linguaggio Dichiarativo}
\begin{itemize}
\item[\textcolor{green}{\checkmark}] Specifica il risultato
\item[\textcolor{green}{\checkmark}] Non i passaggi
\item[\textcolor{green}{\checkmark}] Più semplice
\end{itemize}
\end{column}

\begin{column}{0.48\textwidth}
\textbf{Linguaggio Procedurale}
\begin{itemize}
\item Specifica i passaggi
\item Sequenza di operazioni
\item Più dettagliato
\end{itemize}
\end{column}
\end{columns}

\vspace{0.5cm}
\begin{alertblock}{SQL}
SQL è dichiarativo: dici \textbf{cosa} vuoi, non \textbf{come} ottenerlo
\end{alertblock}
\end{frame}

% SEZIONE 3: ALGEBRA RELAZIONALE - OPERATORI
\section{Algebra Relazionale: Operatori}

% Slide 12
\begin{frame}{I 5 Operatori Fondamentali}
\begin{center}
\begin{tabular}{cll}
\toprule
\textbf{Simbolo} & \textbf{Nome} & \textbf{Tipo} \\
\midrule
\(\sigma\) & Selezione & Unario \\
\(\pi\) & Proiezione & Unario \\
\(\times\) & Prodotto cartesiano & Binario \\
\(-\) & Differenza & Binario \\
\(\cup\) & Unione & Binario \\
\bottomrule
\end{tabular}
\end{center}

\vspace{0.5cm}
\textbf{Proprietà importante:} L'algebra è \textcolor{orange}{chiusa} \\
\(\Rightarrow\) Ogni operazione restituisce una relazione!
\end{frame}

% Slide 13
\begin{frame}{Operatori Unari vs Binari}
\begin{block}{Operatori Unari}
Si applicano a \textbf{una sola} relazione e restituiscono una relazione
\end{block}

\begin{block}{Operatori Binari}
Si applicano a \textbf{due} relazioni e restituiscono una relazione
\end{block}

\vspace{0.5cm}
\begin{center}
\begin{tikzpicture}
\node[draw, circle, fill=lightblue!30] (r1) {R};
\node[draw, circle, fill=orange!30, right=3cm of r1] (r2) {R};
\node[draw, circle, fill=orange!30, right=0.5cm of r2] (s) {S};
\node[below=0.5cm of r1] {Unario};
\node[below=0.5cm of r2] {Binario};
\end{tikzpicture}
\end{center}
\end{frame}

% Slide 14
\begin{frame}{Operatori Derivati}
\textbf{Dai 5 operatori di base derivano:}

\begin{itemize}
\item \textbf{Intersezione} (\(\cap\))
\item \textbf{Congiunzione (Join)} (\(\bowtie\))
\begin{itemize}
\item Natural Join
\item Left Join
\item Right Join
\item Full Join
\end{itemize}
\end{itemize}

\vspace{0.3cm}
\begin{alertblock}{Importante}
Questi operatori sono molto utili nella pratica!
\end{alertblock}
\end{frame}

% Slide 15
\begin{frame}{Tabelle come Insiemi}
\begin{block}{Concetto chiave}
Le tabelle relazionali sono \textbf{insiemi} e le righe sono \textbf{elementi}
\end{block}

\textbf{Conseguenze:}
\begin{itemize}
\item Possiamo applicare operazioni insiemistiche
\item Unione, intersezione, differenza
\item Il risultato è sempre una relazione
\end{itemize}

\begin{center}
\textcolor{darkblue}{\large Tabella = Insieme di tuple}
\end{center}
\end{frame}

% SEZIONE 4: ESEMPIO PRATICO
\section{Esempio di Database}

% Slide 16
\begin{frame}{Database di Esempio}
\begin{block}{Scenario}
Database scolastico con tre tabelle: \texttt{libri}, \texttt{docenti}, \texttt{materie}
\end{block}

\begin{center}
\begin{tikzpicture}[node distance=2cm]
\node[draw, rectangle, fill=lightblue!30] (lib) {libri};
\node[draw, rectangle, fill=lightblue!30, below left=1cm and -0.5cm of lib] (doc) {docenti};
\node[draw, rectangle, fill=lightblue!30, below right=1cm and -0.5cm of lib] (mat) {materie};

\draw[->, thick] (lib) -- (mat);
\draw[->, thick] (doc) -- (mat);
\end{tikzpicture}
\end{center}
\end{frame}

% Slide 17
\begin{frame}[fragile]{Tabella LIBRI}
\small
\begin{center}
\begin{tabular}{|l|l|l|l|l|l|}
\hline
\textbf{isbn} & \textbf{titolo} & \textbf{autore} & \textbf{editore} & \textbf{id\_materia} & \textbf{anno} \\
\hline
11223344 & Il linguaggio C & Camagni & Hoepli & MA01 & 2020 \\
11223355 & Il Novecento & Verdi & Perinazzi & MA04 & 2021 \\
11223366 & Java & Nikolassy & Hoepli & MA01 & 2022 \\
\hline
\end{tabular}
\end{center}

\vspace{0.3cm}
\textbf{Chiave primaria:} isbn \\
\textbf{Chiave esterna:} id\_materia
\end{frame}

% Slide 18
\begin{frame}[fragile]{Tabella DOCENTI}
\small
\begin{center}
\begin{tabular}{|l|l|l|l|l|}
\hline
\textbf{matricola} & \textbf{cognome} & \textbf{nome} & \textbf{id\_materia} & \textbf{telefono} \\
\hline
10100 & Verdi & Mario & MA01 & 111.1234567 \\
20100 & Bianchi & Filippo & MA02 & 222.1234567 \\
30100 & Neri & Antonio & MA01 & 333.1234567 \\
\hline
\end{tabular}
\end{center}

\vspace{0.3cm}
\textbf{Chiave primaria:} matricola \\
\textbf{Chiave esterna:} id\_materia
\end{frame}

% Slide 19
\begin{frame}[fragile]{Tabella MATERIE}
\begin{center}
\begin{tabular}{|l|l|}
\hline
\textbf{id\_materia} & \textbf{nome} \\
\hline
MA01 & Informatica \\
MA02 & Fisica \\
MA03 & Italiano \\
MA04 & Storia \\
\hline
\end{tabular}
\end{center}

\vspace{0.5cm}
\textbf{Chiave primaria:} id\_materia
\end{frame}

% SEZIONE 5: SELEZIONE
\section{Operatore di Selezione}

% Slide 20
\begin{frame}{Selezione (\(\sigma\)) - Introduzione}
\begin{block}{Definizione}
La \textbf{selezione} restituisce un sottoinsieme di tuple che soddisfano una condizione
\end{block}

\textbf{Caratteristiche:}
\begin{itemize}
\item Operatore unario (una relazione → una relazione)
\item Seleziona righe (tuple) in base a una condizione
\item Simbolo: \(\sigma\) (sigma)
\end{itemize}

\begin{center}
\Large \textcolor{orange}{\(\sigma_{\text{condizione}}(R)\)}
\end{center}
\end{frame}

% Slide 21
\begin{frame}{Sintassi della Selezione}
\textbf{Struttura generale:}

\begin{center}
\Large \(\sigma_{\text{<condizione booleana>}}(R)\)
\end{center}

\vspace{0.5cm}
\textbf{La condizione può essere:}
\begin{itemize}
\item \texttt{attributo = valore}
\item \texttt{attributo > valore}
\item \texttt{attributo <> valore}
\item Condizioni complesse con AND, OR, NOT
\end{itemize}
\end{frame}

% Slide 22
\begin{frame}{Esempio di Selezione}
\textbf{Richiesta:} Selezionare i libri con editore "Hoepli"

\vspace{0.3cm}
\textbf{In algebra relazionale:}
\begin{center}
\Large \(\sigma_{\text{editore = "Hoepli"}}(\text{libri})\)
\end{center}

\vspace{0.3cm}
\textbf{Risultato:}
\small
\begin{center}
\begin{tabular}{|l|l|l|l|l|l|}
\hline
\textbf{isbn} & \textbf{titolo} & \textbf{autore} & \textbf{editore} & \textbf{id\_materia} & \textbf{anno} \\
\hline
11223344 & Il linguaggio C & Camagni & \textcolor{orange}{Hoepli} & MA01 & 2020 \\
11223366 & Java & Nikolassy & \textcolor{orange}{Hoepli} & MA01 & 2022 \\
\hline
\end{tabular}
\end{center}
\end{frame}

% Slide 23
\begin{frame}{Selezione con Condizioni Multiple}
\textbf{Operatori logici disponibili:}
\begin{itemize}
\item AND (congiunzione)
\item OR (disgiunzione)
\item NOT (negazione)
\end{itemize}

\vspace{0.5cm}
\textbf{Esempio:}
\begin{center}
\(\sigma_{\text{autore = "Camagni" AND editore = "Hoepli"}}(\text{libri})\)
\end{center}

\begin{alertblock}{Proprietà Commutativa}
Più selezioni consecutive possono essere combinate con AND
\end{alertblock}
\end{frame}

% Slide 24
\begin{frame}{Proprietà Commutativa della Selezione}
\textbf{Le selezioni godono della proprietà commutativa:}

\begin{center}
\(\sigma_{\text{cond1}}(\sigma_{\text{cond2}}(R)) = \sigma_{\text{cond1 AND cond2}}(R)\)
\end{center}

\vspace{0.5cm}
\textbf{Vantaggio:}
\begin{itemize}
\item Possiamo combinare più condizioni
\item L'ordine non importa
\item Ottimizzazione delle query
\end{itemize}
\end{frame}

% SEZIONE 6: PROIEZIONE
\section{Operatore di Proiezione}

% Slide 25
\begin{frame}{Proiezione (\(\pi\)) - Introduzione}
\begin{block}{Definizione}
La \textbf{proiezione} restituisce un sottoinsieme di attributi (colonne)
\end{block}

\textbf{Caratteristiche:}
\begin{itemize}
\item Operatore unario
\item Seleziona colonne (attributi)
\item Simbolo: \(\pi\) (pi greco)
\item Elimina duplicati automaticamente
\end{itemize}

\begin{center}
\Large \textcolor{orange}{\(\pi_{\text{lista attributi}}(R)\)}
\end{center}
\end{frame}

% Slide 26
\begin{frame}{Sintassi della Proiezione}
\textbf{Struttura generale:}

\begin{center}
\Large \(\pi_{\text{attr}_1, \text{attr}_2, \ldots, \text{attr}_n}(R)\)
\end{center}

\vspace{0.5cm}
\textbf{Note importanti:}
\begin{itemize}
\item Il numero di attributi è il \textcolor{orange}{grado della proiezione}
\item Gli attributi sono separati da virgola
\item L'ordine degli attributi è significativo
\end{itemize}
\end{frame}

% Slide 27
\begin{frame}{Esempio di Proiezione (1)}
\textbf{Richiesta:} Selezionare solo il titolo dei libri

\vspace{0.3cm}
\textbf{In algebra relazionale:}
\begin{center}
\Large \(\pi_{\text{titolo}}(\text{libri})\)
\end{center}

\vspace{0.3cm}
\textbf{Risultato:}
\begin{center}
\begin{tabular}{|l|}
\hline
\textbf{titolo} \\
\hline
Il linguaggio C \\
Il Novecento \\
Java \\
\hline
\end{tabular}
\end{center}

\small Grado della proiezione: 1
\end{frame}

% Slide 28
\begin{frame}{Esempio di Proiezione (2)}
\textbf{Richiesta:} Selezionare titolo e autore dei libri

\vspace{0.3cm}
\textbf{In algebra relazionale:}
\begin{center}
\Large \(\pi_{\text{titolo, autore}}(\text{libri})\)
\end{center}

\vspace{0.3cm}
\textbf{Risultato:}
\begin{center}
\begin{tabular}{|l|l|}
\hline
\textbf{titolo} & \textbf{autore} \\
\hline
Il linguaggio C & Camagni \\
Il Novecento & Verdi \\
Java & Nikolassy \\
\hline
\end{tabular}
\end{center}

\small Grado della proiezione: 2
\end{frame}

% Slide 29
\begin{frame}{Combinazione Selezione e Proiezione}
\textbf{Possiamo combinare i due operatori!}

\vspace{0.3cm}
\textbf{Esempio:} Titolo e anno dei libri di Camagni

\begin{center}
\Large \(\pi_{\text{titolo, anno}}(\sigma_{\text{autore = "Camagni"}}(\text{libri}))\)
\end{center}

\vspace{0.3cm}
\textbf{Esecuzione:}
\begin{enumerate}
\item Prima: selezione (filtra le righe)
\item Poi: proiezione (seleziona le colonne)
\end{enumerate}

\begin{alertblock}{Importante}
L'ordine delle operazioni è significativo!
\end{alertblock}
\end{frame}

% Slide 30
\begin{frame}{Esempio Completo}
\textbf{Passi dell'elaborazione:}

\begin{enumerate}
\item \textbf{Selezione:} \\
\texttt{LibriDiCamagni} \(\leftarrow \sigma_{\text{autore = "Camagni"}}(\text{libri})\)

\item \textbf{Proiezione:} \\
\texttt{Risultato} \(\leftarrow \pi_{\text{titolo, anno}}(\text{LibriDiCamagni})\)
\end{enumerate}

\vspace{0.5cm}
\textbf{Risultato finale:}
\begin{center}
\begin{tabular}{|l|c|}
\hline
\textbf{titolo} & \textbf{anno} \\
\hline
Il linguaggio C & 2020 \\
\hline
\end{tabular}
\end{center}
\end{frame}

% SEZIONE 7: RIDENOMINAZIONE
\section{Operatore di Ridenominazione}

% Slide 31
\begin{frame}{Ridenominazione (\(\rho\))}
\begin{block}{Definizione}
La \textbf{ridenominazione} consente di rinominare gli attributi per eliminare ambiguità
\end{block}

\textbf{Sintassi:}
\begin{center}
\Large \(\rho_{\text{nuovoNome} \leftarrow \text{vecchioNome}}(\text{tabella})\)
\end{center}

\vspace{0.3cm}
\textbf{Utilizzo:}
\begin{itemize}
\item Chiarire risultati intermedi
\item Evitare conflitti di nomi
\item Preparare join complesse
\end{itemize}
\end{frame}

% Slide 32
\begin{frame}{Esempio di Ridenominazione}
\textbf{Scenario:} Rinominare "nome" in "disciplina" nella tabella materie

\vspace{0.3cm}
\begin{center}
\Large \(\rho_{\text{disciplina} \leftarrow \text{nome}}(\text{materie})\)
\end{center}

\vspace{0.3cm}
\textbf{Prima:}
\begin{tabular}{|l|l|}
\hline
id\_materia & \textcolor{red}{nome} \\
\hline
\end{tabular}

\textbf{Dopo:}
\begin{tabular}{|l|l|}
\hline
id\_materia & \textcolor{green}{disciplina} \\
\hline
\end{tabular}
\end{frame}

% SEZIONE 8: OPERATORI BINARI
\section{Operatori Binari}

% Slide 33
\begin{frame}{Operatori Binari - Introduzione}
\begin{block}{Requisito}
Le due relazioni devono avere la \textcolor{orange}{stessa struttura}
\end{block}

\textbf{I tre operatori insiemistici:}
\begin{enumerate}
\item \textbf{Unione} (\(\cup\))
\item \textbf{Intersezione} (\(\cap\))
\item \textbf{Differenza} (\(-\))
\end{enumerate}

\vspace{0.3cm}
Più l'operatore:
\begin{itemize}
\item \textbf{Prodotto Cartesiano} (\(\times\))
\end{itemize}
\end{frame}

% Slide 34
\begin{frame}{Tabelle di Esempio}
\textbf{Per gli esempi useremo:}

\begin{columns}
\begin{column}{0.48\textwidth}
\texttt{alunni\_ripetenti}
\small
\begin{tabular}{|l|l|l|}
\hline
\textbf{matr.} & \textbf{cognome} & \textbf{nome} \\
\hline
10100 & Verdi & Mario \\
20100 & Bianchi & Filippo \\
30100 & Neri & Antonio \\
\hline
\end{tabular}
\end{column}

\begin{column}{0.48\textwidth}
\texttt{alunni\_motorizzati}
\small
\begin{tabular}{|l|l|l|}
\hline
\textbf{matr.} & \textbf{cognome} & \textbf{nome} \\
\hline
10100 & Verdi & Mario \\
40100 & Gialli & Anna \\
50100 & Rossi & Pina \\
\hline
\end{tabular}
\end{column}
\end{columns}

\vspace{0.5cm}
\small \textit{Stesso schema: matricola, cognome, nome}
\end{frame}

% Slide 35
\begin{frame}{Unione (\(\cup\))}
\begin{block}{Definizione}
Restituisce \textbf{tutte le tuple} presenti in almeno una delle due relazioni
\end{block}

\textbf{Sintassi:}
\begin{center}
\Large \(R \cup S\)
\end{center}

\vspace{0.3cm}
\textbf{Proprietà:}
\begin{itemize}
\item Elimina automaticamente i duplicati
\item Commutativa: \(R \cup S = S \cup R\)
\item Le relazioni devono avere la stessa struttura
\end{itemize}
\end{frame}

% Slide 36
\begin{frame}{Esempio di Unione}
\textbf{Operazione:}
\begin{center}
\texttt{alunni\_ripetenti} \(\cup\) \texttt{alunni\_motorizzati}
\end{center}

\vspace{0.3cm}
\textbf{Risultato:}
\small
\begin{center}
\begin{tabular}{|l|l|l|}
\hline
\textbf{matricola} & \textbf{cognome} & \textbf{nome} \\
\hline
10100 & Verdi & Mario \\
20100 & Bianchi & Filippo \\
30100 & Neri & Antonio \\
40100 & Gialli & Anna \\
50100 & Rossi & Pina \\
\hline
\end{tabular}
\end{center}

\small \textcolor{orange}{5 righe totali (Verdi Mario compare una sola volta)}
\end{frame}

% Slide 37
\begin{frame}{Intersezione (\(\cap\))}
\begin{block}{Definizione}
Restituisce le tuple che sono presenti in \textbf{entrambe} le relazioni
\end{block}

\textbf{Sintassi:}
\begin{center}
\Large \(R \cap S\)
\end{center}

\vspace{0.3cm}
\textbf{Proprietà:}
\begin{itemize}
\item Solo le righe comuni
\item Commutativa: \(R \cap S = S \cap R\)
\item Operatore derivato (non fondamentale)
\end{itemize}
\end{frame}

% Slide 38
\begin{frame}{Esempio di Intersezione}
\textbf{Operazione:}
\begin{center}
\texttt{alunni\_ripetenti} \(\cap\) \texttt{alunni\_motorizzati}
\end{center}

\vspace{0.3cm}
\textbf{Risultato:}
\begin{center}
\begin{tabular}{|l|l|l|}
\hline
\textbf{matricola} & \textbf{cognome} & \textbf{nome} \\
\hline
10100 & Verdi & Mario \\
\hline
\end{tabular}
\end{center}

\vspace{0.5cm}
\textcolor{orange}{Solo 1 riga: quella presente in entrambe le tabelle}
\end{frame}

% Slide 39
\begin{frame}{Differenza (\(-\))}
\begin{block}{Definizione}
Restituisce le tuple presenti nella \textbf{prima} relazione ma \textbf{non} nella seconda
\end{block}

\textbf{Sintassi:}
\begin{center}
\Large \(R - S\)
\end{center}

\vspace{0.3cm}
\textbf{Proprietà:}
\begin{itemize}
\item \textcolor{red}{NON commutativa:} \(R - S \neq S - R\)
\item L'ordine è importante!
\end{itemize}
\end{frame}

% Slide 40
\begin{frame}{Esempio di Differenza}
\textbf{Operazione 1:}
\texttt{alunni\_ripetenti} \(-\) \texttt{alunni\_motorizzati}

\small
\begin{tabular}{|l|l|l|}
\hline
20100 & Bianchi & Filippo \\
30100 & Neri & Antonio \\
\hline
\end{tabular}

\vspace{0.5cm}
\textbf{Operazione 2:}
\texttt{alunni\_motorizzati} \(-\) \texttt{alunni\_ripetenti}

\small
\begin{tabular}{|l|l|l|}
\hline
40100 & Gialli & Anna \\
50100 & Rossi & Pina \\
\hline
\end{tabular}

\vspace{0.3cm}
\normalsize \textcolor{red}{Risultati diversi! L'ordine conta!}
\end{frame}

% SEZIONE 9: PRODOTTO CARTESIANO
\section{Prodotto Cartesiano e Join}

% Slide 41
\begin{frame}{Prodotto Cartesiano (\(\times\))}
\begin{block}{Definizione}
Produce una nuova tupla dalla combinazione di \textbf{ogni tupla} di R con \textbf{ogni tupla} di S
\end{block}

\textbf{Sintassi:}
\begin{center}
\Large \(R \times S\)
\end{center}

\textbf{Caratteristiche:}
\begin{itemize}
\item Combina tutte le righe possibili
\item Genera molte tuple "inutili"
\item Base per la JOIN
\end{itemize}
\end{frame}

% Slide 42
\begin{frame}{Esempio di Prodotto Cartesiano}
\textbf{Operazione:} \texttt{libri} \(\times\) \texttt{materie}

\vspace{0.3cm}
\textbf{Risultato parziale:}
\tiny
\begin{tabular}{|l|l|l|l|l|l|l|l|}
\hline
\textbf{isbn} & \textbf{titolo} & \textbf{id\_mat.} & \textbf{anno} & \textbf{ID\_mat.} & \textbf{nome} \\
\hline
11223344 & Il linguaggio C & MA01 & 2020 & \textcolor{orange}{MA01} & Informatica \\
11223344 & Il linguaggio C & MA01 & 2020 & \textcolor{red}{MA02} & Fisica \\
\ldots & \ldots & \ldots & \ldots & \ldots & \ldots \\
\hline
\end{tabular}

\vspace{0.3cm}
\normalsize
Se \texttt{libri} ha 3 righe e \texttt{materie} ha 4 righe: \\
\textcolor{orange}{Risultato = 3 × 4 = 12 righe!}
\end{frame}

% Slide 43
\begin{frame}{Problema del Prodotto Cartesiano}
\begin{alertblock}{Attenzione!}
Il prodotto cartesiano da solo non è molto utile!
\end{alertblock}

\textbf{Problemi:}
\begin{itemize}
\item Genera troppe tuple
\item La maggior parte sono "senza senso"
\item Combinazioni non significative
\end{itemize}

\vspace{0.5cm}
\textbf{Soluzione:}
\begin{center}
\Large Prodotto cartesiano + Selezione = \textcolor{orange}{JOIN}
\end{center}
\end{frame}

% SEZIONE 10: JOIN
% Slide 44
\begin{frame}{Congiunzione o Join (\(\bowtie\))}
\begin{block}{Definizione}
La \textbf{Join} combina tuple di due relazioni che hanno \textbf{un attributo in comune}
\end{block}

\textbf{Sintassi:}
\begin{center}
\Large \(R \bowtie_{\text{condizione}} S\)
\end{center}

\vspace{0.3cm}
\textbf{Pronuncia:} Il simbolo \(\bowtie\) si pronuncia "bowtie" (cravatta a farfalla)


\end{frame}

% Slide 45
\begin{frame}{Esempio di Join}
\textbf{Operazione:}
\begin{center}
\texttt{libri} \(\bowtie_{\text{id\_materia = ID\_materia}}\) \texttt{materie}
\end{center}

\vspace{0.3cm}
\textbf{Equivale a:}
\begin{enumerate}
\item Prodotto cartesiano
\item Selezione con condizione di uguaglianza
\item Elimina colonne duplicate
\end{enumerate}

\begin{alertblock}{Equi-Join}
Quando la condizione è un'uguaglianza, si chiama \textbf{equi-join}
\end{alertblock}
\end{frame}

% Slide 46
\begin{frame}{Natural Join (\(*\))}
\begin{block}{Natural Join}
Join automatica sugli attributi con lo \textbf{stesso nome}
\end{block}

\textbf{Sintassi:}
\begin{center}
\Large \(R * S\)
\end{center}

\vspace{0.3cm}
\textbf{Esempio:}
\begin{center}
\texttt{libri} \(* \) \texttt{materie}
\end{center}

\small Unisce automaticamente su \texttt{id\_materia}! \\
\textcolor{orange}{Nota:} Gli attributi comuni appaiono una sola volta
\end{frame}

% Slide 47
\begin{frame}{Left Join}
\begin{block}{Left Join}
Conserva \textbf{tutte le righe} della tabella di \textcolor{orange}{sinistra}
\end{block}

\textbf{Esempio:}
\begin{center}
\texttt{linee} = \(\bowtie_{\text{id\_autista}}\) \texttt{autisti}
\end{center}

\textbf{Risultato:}
\begin{itemize}
\item Tutte le linee, anche senza autista
\item Campi mancanti: NULL
\end{itemize}
\end{frame}

% Slide 48
\begin{frame}{Right Join}
\begin{block}{Right Join}
Conserva \textbf{tutte le righe} della tabella di \textcolor{orange}{destra}
\end{block}

\textbf{Esempio:}
\begin{center}
\texttt{linee} \(\bowtie =_{\text{id\_autista}}\) \texttt{autisti}
\end{center}

\textbf{Risultato:}
\begin{itemize}
\item Tutti gli autisti, anche senza linea
\item Campi mancanti: NULL
\end{itemize}
\end{frame}

% Slide 49
\begin{frame}{Full Join}
\begin{block}{Full Join}
Conserva \textbf{tutte le righe} di \textcolor{orange}{entrambe} le tabelle
\end{block}

\textbf{Sintassi:}
\begin{center}
\texttt{linee} = \(\bowtie =_{\text{id\_autista}}\) \texttt{autisti}
\end{center}

\textbf{Risultato:}
\begin{itemize}
\item Tutte le linee (anche senza autista)
\item Tutti gli autisti (anche senza linea)
\item Campi mancanti: NULL
\end{itemize}
\end{frame}

% Slide 50
\begin{frame}{Riepilogo delle Join}
\begin{center}
\begin{tabular}{ll}
\toprule
\textbf{Tipo di Join} & \textbf{Cosa conserva} \\
\midrule
Equi-Join / Natural Join & Solo righe corrispondenti \\
Left Join & Tutte le righe di sinistra \\
Right Join & Tutte le righe di destra \\
Full Join & Tutte le righe di entrambe \\
\bottomrule
\end{tabular}
\end{center}

\vspace{0.5cm}
\begin{alertblock}{In SQL}
Queste operazioni corrispondono a: \\
INNER JOIN, LEFT JOIN, RIGHT JOIN, FULL OUTER JOIN
\end{alertblock}
\end{frame}

% Slide finale
\begin{frame}
\begin{center}
{\Huge Grazie per l'attenzione!}

\vspace{1cm}

{\Large Domande?}

\vspace{1.5cm}

\textit{IIS Fermi Sacconi Ceci} \\
\textit{Informatica - 5B Inf.} \\
\textit{A.S. 2025/26}
\end{center}
\end{frame}

\end{document}
