\documentclass[a4paper,11pt]{article}
\usepackage[utf8]{inputenc}
\usepackage[italian]{babel}
\usepackage{geometry}
\usepackage{enumitem}
\usepackage{amsmath}
\usepackage{amssymb}
\usepackage{fancyhdr}

\geometry{a4paper, margin=2.5cm}

\pagestyle{fancy}
\fancyhf{}
\lhead{Algebra Relazionale - Database Cinema}
\rhead{IIS Fermi Sacconi Cpia}
\cfoot{\thepage}

\title{\textbf{Esercizi di Algebra Relazionale}\\Database Cinematografico}
\author{Prof. Fedeli Massimo}
\date{}

\begin{document}

\maketitle

\section*{Schema del Database}

\subsection*{Movies}
\begin{itemize}[noitemsep]
    \item Movie\_ID (number) - chiave primaria
    \item Title (string)
    \item ReleaseDate (date)
    \item Genre\_ID (number) - chiave esterna verso Genres
    \item Budget (number)
    \item OpeningWeek (number)
    \item Profit (number)
    \item Runtime (number)
    \item Certificate (number)
    \item SequelOf (number) - chiave esterna verso Movies
    \item Distribution (string)
\end{itemize}

\subsection*{Genres}
\begin{itemize}[noitemsep]
    \item Genre\_ID (number) - chiave primaria
    \item Name (string)
\end{itemize}

\subsection*{Persons}
\begin{itemize}[noitemsep]
    \item Person\_ID (number) - chiave primaria
    \item Firstname (string)
    \item Lastname (string)
\end{itemize}

\subsection*{PersonsMovies}
\begin{itemize}[noitemsep]
    \item Person\_ID (number) - chiave esterna verso Persons
    \item Movie\_ID (number) - chiave esterna verso Movies
    \item Role (string) - esempio: "Director", "Actor", "Producer"
\end{itemize}

\vspace{1cm}

\section*{Esercizi}

\subsection*{Esercizi Base (Selezione e Proiezione)}

\begin{enumerate}[leftmargin=*]

\item \textbf{Titoli dei film}\\
Estrarre i titoli di tutti i film presenti nel database.

\vspace{0.5cm}

\item \textbf{Film d'azione}\\
Trovare tutti i film del genere "Action" (assumendo Genre\_ID = 1). Mostrare Movie\_ID e Title.

\vspace{0.5cm}

\item \textbf{Film con budget elevato}\\
Estrarre titolo e budget dei film con budget superiore a 100 milioni.

\vspace{0.5cm}

\item \textbf{Film lunghi}\\
Trovare i film con durata (Runtime) superiore a 150 minuti. Mostrare Title e Runtime.

\vspace{0.5cm}

\item \textbf{Nomi completi delle persone}\\
Estrarre il nome completo (Firstname e Lastname) di tutte le persone nel database.

\vspace{0.5cm}

\subsection*{Esercizi con Join}

\item \textbf{Film con genere}\\
Estrarre titolo del film e nome del genere per tutti i film.

\vspace{0.5cm}

\item \textbf{Registi e i loro film}\\
Trovare il nome completo dei registi (Role = "Director") e i titoli dei film che hanno diretto.

\vspace{0.5cm}

\item \textbf{Attori in film d'azione}\\
Estrarre il nome degli attori (Role = "Actor") che hanno recitato in film d'azione (Genre Name = "Action").

\vspace{0.5cm}

\item \textbf{Film commedia con cast}\\
Per tutti i film di genere "Comedy", mostrare il titolo del film, il nome della persona e il suo ruolo.

\vspace{0.5cm}

\item \textbf{Sequel}\\
Trovare i titoli dei film che sono sequel di altri film, mostrando sia il titolo del sequel che il titolo del film originale.

\vspace{0.5cm}

\subsection*{Esercizi con Operazioni Insiemistiche}

\item \textbf{Persone che sono sia attori che registi}\\
Trovare le persone che hanno lavorato sia come attore che come regista (in film diversi o nello stesso).

\vspace{0.5cm}

\item \textbf{Film drammatici o thriller}\\
Estrarre i titoli dei film che sono del genere "Drama" oppure "Thriller".

\vspace{0.5cm}

\item \textbf{Registi che non hanno mai recitato}\\
Trovare i registi che non hanno mai lavorato come attori.

\vspace{0.5cm}

\subsection*{Esercizi Avanzati}

\item \textbf{Film profittevoli}\\
Trovare i film dove il profitto (Profit) è maggiore del budget (Budget). Mostrare Title, Budget e Profit.

\vspace{0.5cm}

\item \textbf{Distributori di film d'azione}\\
Estrarre i nomi dei distributori (Distribution) che hanno distribuito almeno un film d'azione.

\vspace{0.5cm}

\item \textbf{Persone coinvolte in film con alto incasso}\\
Trovare il nome delle persone che hanno lavorato (in qualsiasi ruolo) in film con OpeningWeek superiore a 50 milioni.

\vspace{0.5cm}

\item \textbf{Film dello stesso genere di un film specifico}\\
Dato il film "Inception" (assumendo Movie\_ID = 10), trovare tutti gli altri film dello stesso genere.

\vspace{0.5cm}

\item \textbf{Collaborazioni tra registi e attori}\\
Trovare le coppie regista-attore che hanno collaborato (lavorato insieme nello stesso film), mostrando i loro nomi e il titolo del film.

\vspace{0.5cm}

\item \textbf{Generi senza film lunghi}\\
Trovare i generi che non hanno film con durata superiore a 180 minuti.

\vspace{0.5cm}

\item \textbf{Film di maggior successo per genere}\\
Per ogni genere, trovare il film con il profitto più alto. Mostrare il nome del genere, il titolo del film e il profitto.\\
\textit{Suggerimento: questo richiede operazioni complesse con join e confronti}

\vspace{0.5cm}

\end{enumerate}

\section*{Note}

\begin{itemize}
    \item Utilizzare la notazione dell'algebra relazionale con i simboli: $\sigma$ (selezione), $\pi$ (proiezione), $\bowtie$ (join), $\cup$ (unione), $-$ (differenza), $\cap$ (intersezione), $\rho$ (ridenominazione)
    \item Per i join naturali si può usare il simbolo $\bowtie$ senza condizione
    \item Per i theta-join specificare la condizione: $\bowtie_{\text{condizione}}$
    \item Prestare attenzione alla corretta sintassi e all'ordine delle operazioni
\end{itemize}

\end{document}
