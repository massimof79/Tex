\documentclass[a4paper,11pt]{article}
\usepackage[utf8]{inputenc}
\usepackage[italian]{babel}
\usepackage{geometry}
\usepackage{enumitem}
\usepackage{amsmath}
\usepackage{amssymb}
\usepackage{fancyhdr}
\usepackage{xcolor}

\geometry{a4paper, margin=2.5cm}

\pagestyle{fancy}
\fancyhf{}
\lhead{Algebra Relazionale - Database Cinema}
\rhead{IIS Fermi Sacconi Ceci - Soluzioni}
\cfoot{\thepage}

\title{\textbf{Esercizi di Algebra Relazionale}\\Database Cinematografico\\[0.3cm]\large Con Soluzioni}
\author{Prof. Fedeli Massimo\\}
\date{}

\begin{document}

\maketitle

\section*{Schema del Database}

\subsection*{Movies}
\begin{itemize}[noitemsep]
    \item Movie\_ID (number) - chiave primaria
    \item Title (string)
    \item ReleaseDate (date)
    \item Genre\_ID (number) - chiave esterna verso Genres
    \item Budget (number)
    \item OpeningWeek (number)
    \item Profit (number)
    \item Runtime (number)
    \item Certificate (number)
    \item SequelOf (number) - chiave esterna verso Movies
    \item Distribution (string)
\end{itemize}

\subsection*{Genres}
\begin{itemize}[noitemsep]
    \item Genre\_ID (number) - chiave primaria
    \item Name (string)
\end{itemize}

\subsection*{Persons}
\begin{itemize}[noitemsep]
    \item Person\_ID (number) - chiave primaria
    \item Firstname (string)
    \item Lastname (string)
\end{itemize}

\subsection*{PersonsMovies}
\begin{itemize}[noitemsep]
    \item Person\_ID (number) - chiave esterna verso Persons
    \item Movie\_ID (number) - chiave esterna verso Movies
    \item Role (string) - esempio: "Director", "Actor", "Producer"
\end{itemize}

\vspace{1cm}

\section*{Esercizi e Soluzioni}

\subsection*{Esercizi Base (Selezione e Proiezione)}

\begin{enumerate}[leftmargin=*]

\item \textbf{Titoli dei film}\\
Estrarre i titoli di tutti i film presenti nel database.

\vspace{0.3cm}
\textcolor{blue}{\textbf{Soluzione:}}
$$\pi_{\text{Title}}(\text{Movies})$$

\vspace{0.5cm}

\item \textbf{Film d'azione}\\
Trovare tutti i film del genere "Action" (assumendo Genre\_ID = 1). Mostrare Movie\_ID e Title.

\vspace{0.3cm}
\textcolor{blue}{\textbf{Soluzione:}}
$$\pi_{\text{Movie\_ID, Title}}(\sigma_{\text{Genre\_ID} = 1}(\text{Movies}))$$

\vspace{0.5cm}

\item \textbf{Film con budget elevato}\\
Estrarre titolo e budget dei film con budget superiore a 100 milioni.

\vspace{0.3cm}
\textcolor{blue}{\textbf{Soluzione:}}
$$\pi_{\text{Title, Budget}}(\sigma_{\text{Budget} > 100000000}(\text{Movies}))$$

\vspace{0.5cm}

\item \textbf{Film lunghi}\\
Trovare i film con durata (Runtime) superiore a 150 minuti. Mostrare Title e Runtime.

\vspace{0.3cm}
\textcolor{blue}{\textbf{Soluzione:}}
$$\pi_{\text{Title, Runtime}}(\sigma_{\text{Runtime} > 150}(\text{Movies}))$$

\vspace{0.5cm}

\item \textbf{Nomi completi delle persone}\\
Estrarre il nome completo (Firstname e Lastname) di tutte le persone nel database.

\vspace{0.3cm}
\textcolor{blue}{\textbf{Soluzione:}}
$$\pi_{\text{Firstname, Lastname}}(\text{Persons})$$

\vspace{0.5cm}

\subsection*{Esercizi con Join}

\item \textbf{Film con genere}\\
Estrarre titolo del film e nome del genere per tutti i film.

\vspace{0.3cm}
\textcolor{blue}{\textbf{Soluzione:}}
$$\pi_{\text{Title, Name}}(\text{Movies} \bowtie_{\text{Movies.Genre\_ID} = \text{Genres.Genre\_ID}} \text{Genres})$$

\textit{Oppure con join naturale se i nomi degli attributi coincidono:}
$$\pi_{\text{Title, Name}}(\text{Movies} \bowtie \text{Genres})$$

\vspace{0.5cm}

\item \textbf{Registi e i loro film}\\
Trovare il nome completo dei registi (Role = "Director") e i titoli dei film che hanno diretto.

\vspace{0.3cm}
\textcolor{blue}{\textbf{Soluzione:}}
\begin{align*}
\text{Temp} &= \sigma_{\text{Role} = \text{"Director"}}(\text{PersonsMovies})\\
\text{Temp2} &= \text{Temp} \bowtie_{\text{Temp.Person\_ID} = \text{Persons.Person\_ID}} \text{Persons}\\
\text{Result} &= \pi_{\text{Firstname, Lastname, Title}}(\text{Temp2} \bowtie_{\text{Temp2.Movie\_ID} = \text{Movies.Movie\_ID}} \text{Movies})
\end{align*}

\textit{Forma compatta:}
$$\pi_{\text{Firstname, Lastname, Title}}((\sigma_{\text{Role} = \text{"Director"}}(\text{PersonsMovies}) \bowtie \text{Persons}) \bowtie \text{Movies})$$

\vspace{0.5cm}

\item \textbf{Attori in film d'azione}\\
Estrarre il nome degli attori (Role = "Actor") che hanno recitato in film d'azione (Genre Name = "Action").

\vspace{0.3cm}
\textcolor{blue}{\textbf{Soluzione:}}
\begin{align*}
\text{ActionMovies} &= \pi_{\text{Movie\_ID}}(\text{Movies} \bowtie_{\text{Movies.Genre\_ID} = \text{Genres.Genre\_ID}} \sigma_{\text{Name} = \text{"Action"}}(\text{Genres}))\\
\text{Actors} &= \sigma_{\text{Role} = \text{"Actor"}}(\text{PersonsMovies})\\
\text{ActionActors} &= \text{Actors} \bowtie_{\text{Actors.Movie\_ID} = \text{ActionMovies.Movie\_ID}} \text{ActionMovies}\\
\text{Result} &= \pi_{\text{Firstname, Lastname}}(\text{ActionActors} \bowtie_{\text{ActionActors.Person\_ID} = \text{Persons.Person\_ID}} \text{Persons})
\end{align*}

\vspace{0.5cm}

\item \textbf{Film commedia con cast}\\
Per tutti i film di genere "Comedy", mostrare il titolo del film, il nome della persona e il suo ruolo.

\vspace{0.3cm}
\textcolor{blue}{\textbf{Soluzione:}}
\begin{align*}
\text{Comedy} &= \text{Movies} \bowtie_{\text{Movies.Genre\_ID} = \text{Genres.Genre\_ID}} \sigma_{\text{Name} = \text{"Comedy"}}(\text{Genres})\\
\text{ComedyCast} &= \text{Comedy} \bowtie_{\text{Comedy.Movie\_ID} = \text{PersonsMovies.Movie\_ID}} \text{PersonsMovies}\\
\text{Result} &= \pi_{\text{Title, Firstname, Lastname, Role}}(\text{ComedyCast} \bowtie_{\text{ComedyCast.Person\_ID} = \text{Persons.Person\_ID}} \text{Persons})
\end{align*}

\vspace{0.5cm}

\item \textbf{Sequel}\\
Trovare i titoli dei film che sono sequel di altri film, mostrando sia il titolo del sequel che il titolo del film originale.

\vspace{0.3cm}
\textcolor{blue}{\textbf{Soluzione:}}
\begin{align*}
\text{Sequels} &= \sigma_{\text{SequelOf IS NOT NULL}}(\text{Movies})\\
\text{Originals} &= \rho_{\text{Orig}(\text{Movie\_ID2, Title2, ...})}(\text{Movies})\\
\text{Result} &= \pi_{\text{Title, Title2}}(\text{Sequels} \bowtie_{\text{Sequels.SequelOf} = \text{Orig.Movie\_ID2}} \text{Orig})
\end{align*}

\textit{Nota: Orig.Title2 rappresenta il titolo del film originale}

\vspace{0.5cm}

\subsection*{Esercizi con Operazioni Insiemistiche}

\item \textbf{Persone che sono sia attori che registi}\\
Trovare le persone che hanno lavorato sia come attore che come regista (in film diversi o nello stesso).

\vspace{0.3cm}
\textcolor{blue}{\textbf{Soluzione:}}
\begin{align*}
\text{Directors} &= \pi_{\text{Person\_ID}}(\sigma_{\text{Role} = \text{"Director"}}(\text{PersonsMovies}))\\
\text{Actors} &= \pi_{\text{Person\_ID}}(\sigma_{\text{Role} = \text{"Actor"}}(\text{PersonsMovies}))\\
\text{Both} &= \text{Directors} \cap \text{Actors}\\
\text{Result} &= \pi_{\text{Firstname, Lastname}}(\text{Both} \bowtie_{\text{Both.Person\_ID} = \text{Persons.Person\_ID}} \text{Persons})
\end{align*}

\vspace{0.5cm}

\item \textbf{Film drammatici o thriller}\\
Estrarre i titoli dei film che sono del genere "Drama" oppure "Thriller".

\vspace{0.3cm}
\textcolor{blue}{\textbf{Soluzione:}}
\begin{align*}
\text{Drama} &= \pi_{\text{Title}}(\text{Movies} \bowtie \sigma_{\text{Name} = \text{"Drama"}}(\text{Genres}))\\
\text{Thriller} &= \pi_{\text{Title}}(\text{Movies} \bowtie \sigma_{\text{Name} = \text{"Thriller"}}(\text{Genres}))\\
\text{Result} &= \text{Drama} \cup \text{Thriller}
\end{align*}

\textit{Alternativa con OR:}
$$\pi_{\text{Title}}(\text{Movies} \bowtie \sigma_{\text{Name} = \text{"Drama"} \lor \text{Name} = \text{"Thriller"}}(\text{Genres}))$$

\vspace{0.5cm}

\item \textbf{Registi che non hanno mai recitato}\\
Trovare i registi che non hanno mai lavorato come attori.

\vspace{0.3cm}
\textcolor{blue}{\textbf{Soluzione:}}
\begin{align*}
\text{AllDirectors} &= \pi_{\text{Person\_ID}}(\sigma_{\text{Role} = \text{"Director"}}(\text{PersonsMovies}))\\
\text{AllActors} &= \pi_{\text{Person\_ID}}(\sigma_{\text{Role} = \text{"Actor"}}(\text{PersonsMovies}))\\
\text{OnlyDirectors} &= \text{AllDirectors} - \text{AllActors}\\
\text{Result} &= \pi_{\text{Firstname, Lastname}}(\text{OnlyDirectors} \bowtie \text{Persons})
\end{align*}

\vspace{0.5cm}

\subsection*{Esercizi Avanzati}

\item \textbf{Film profittevoli}\\
Trovare i film dove il profitto (Profit) è maggiore del budget (Budget). Mostrare Title, Budget e Profit.

\vspace{0.3cm}
\textcolor{blue}{\textbf{Soluzione:}}
$$\pi_{\text{Title, Budget, Profit}}(\sigma_{\text{Profit} > \text{Budget}}(\text{Movies}))$$

\vspace{0.5cm}

\item \textbf{Distributori di film d'azione}\\
Estrarre i nomi dei distributori (Distribution) che hanno distribuito almeno un film d'azione.

\vspace{0.3cm}
\textcolor{blue}{\textbf{Soluzione:}}
\begin{align*}
\text{ActionMovies} &= \text{Movies} \bowtie \sigma_{\text{Name} = \text{"Action"}}(\text{Genres})\\
\text{Result} &= \pi_{\text{Distribution}}(\text{ActionMovies})
\end{align*}

\vspace{0.5cm}

\item \textbf{Persone coinvolte in film con alto incasso}\\
Trovare il nome delle persone che hanno lavorato (in qualsiasi ruolo) in film con OpeningWeek superiore a 50 milioni.

\vspace{0.3cm}
\textcolor{blue}{\textbf{Soluzione:}}
\begin{align*}
\text{HighGrossing} &= \sigma_{\text{OpeningWeek} > 50000000}(\text{Movies})\\
\text{InvolvedPersons} &= \text{HighGrossing} \bowtie \text{PersonsMovies}\\
\text{Result} &= \pi_{\text{Firstname, Lastname}}(\text{InvolvedPersons} \bowtie \text{Persons})
\end{align*}

\vspace{0.5cm}

\item \textbf{Film dello stesso genere di un film specifico}\\
Dato il film "Inception" (assumendo Movie\_ID = 10), trovare tutti gli altri film dello stesso genere.

\vspace{0.3cm}
\textcolor{blue}{\textbf{Soluzione:}}
\begin{align*}
\text{InceptionGenre} &= \pi_{\text{Genre\_ID}}(\sigma_{\text{Movie\_ID} = 10}(\text{Movies}))\\
\text{SameGenre} &= \text{Movies} \bowtie_{\text{Movies.Genre\_ID} = \text{InceptionGenre.Genre\_ID}} \text{InceptionGenre}\\
\text{Result} &= \pi_{\text{Title}}(\sigma_{\text{Movie\_ID} \neq 10}(\text{SameGenre}))
\end{align*}

\vspace{0.5cm}

\item \textbf{Collaborazioni tra registi e attori}\\
Trovare le coppie regista-attore che hanno collaborato (lavorato insieme nello stesso film), mostrando i loro nomi e il titolo del film.

\vspace{0.3cm}
\textcolor{blue}{\textbf{Soluzione:}}
\begin{align*}
\text{Directors} &= \rho_{\text{D}(\text{Person\_ID\_D, Movie\_ID, Role\_D})}(\sigma_{\text{Role} = \text{"Director"}}(\text{PersonsMovies}))\\
\text{Actors} &= \rho_{\text{A}(\text{Person\_ID\_A, Movie\_ID, Role\_A})}(\sigma_{\text{Role} = \text{"Actor"}}(\text{PersonsMovies}))\\
\text{Collab} &= \text{Directors} \bowtie_{\text{D.Movie\_ID} = \text{A.Movie\_ID}} \text{Actors}\\
\text{DirectorNames} &= \rho_{\text{DN}(\text{Person\_ID\_D, FirstnameD, LastnameD})}(\text{Persons})\\
\text{ActorNames} &= \rho_{\text{AN}(\text{Person\_ID\_A, FirstnameA, LastnameA})}(\text{Persons})\\
\text{WithNames} &= ((\text{Collab} \bowtie \text{DirectorNames}) \bowtie \text{ActorNames}) \bowtie \text{Movies}\\
\text{Result} &= \pi_{\text{FirstnameD, LastnameD, FirstnameA, LastnameA, Title}}(\text{WithNames})
\end{align*}

\vspace{0.5cm}

\item \textbf{Generi senza film lunghi}\\
Trovare i generi che non hanno film con durata superiore a 180 minuti.

\vspace{0.3cm}
\textcolor{blue}{\textbf{Soluzione:}}
\begin{align*}
\text{AllGenres} &= \pi_{\text{Genre\_ID}}(\text{Genres})\\
\text{LongMovies} &= \sigma_{\text{Runtime} > 180}(\text{Movies})\\
\text{GenresWithLong} &= \pi_{\text{Genre\_ID}}(\text{LongMovies})\\
\text{GenresWithoutLong} &= \text{AllGenres} - \text{GenresWithLong}\\
\text{Result} &= \pi_{\text{Name}}(\text{GenresWithoutLong} \bowtie \text{Genres})
\end{align*}

\vspace{0.5cm}

\item \textbf{Film di maggior successo per genere}\\
Per ogni genere, trovare il film con il profitto più alto. Mostrare il nome del genere, il titolo del film e il profitto.

\vspace{0.3cm}
\textcolor{blue}{\textbf{Soluzione:}}

\textit{Questo è l'esercizio più complesso. Richiede un approccio in più passi:}

\begin{align*}
\text{MoviesWithGenre} &= \text{Movies} \bowtie \text{Genres}\\
\text{M1} &= \rho_{\text{M1}(\text{Movie\_ID1, Title1, Genre\_ID1, Profit1, Name1, ...})}(\text{MoviesWithGenre})\\
\text{M2} &= \rho_{\text{M2}(\text{Movie\_ID2, Title2, Genre\_ID2, Profit2, Name2, ...})}(\text{MoviesWithGenre})\\
\text{NotMax} &= \pi_{\text{Movie\_ID1}}(\sigma_{\text{M1.Genre\_ID1} = \text{M2.Genre\_ID2} \land \text{M1.Profit1} < \text{M2.Profit2}}(\text{M1} \times \text{M2}))\\
\text{AllMovies} &= \pi_{\text{Movie\_ID}}(\text{Movies})\\
\text{MaxMovies} &= \text{AllMovies} - \text{NotMax}\\
\text{Result} &= \pi_{\text{Name, Title, Profit}}(\text{MaxMovies} \bowtie \text{MoviesWithGenre})
\end{align*}

\textit{Spiegazione: Si confronta ogni film con tutti gli altri dello stesso genere. Se esiste un film con profitto maggiore nello stesso genere, quello non è il massimo. I film che non risultano "non massimi" sono i film con profitto massimo per il loro genere.}

\vspace{0.5cm}

\end{enumerate}

\section*{Note Tecniche}

\subsection*{Operatori dell'Algebra Relazionale}

\begin{itemize}
    \item $\sigma$ (sigma) - \textbf{Selezione}: filtra le righe secondo una condizione
    \item $\pi$ (pi) - \textbf{Proiezione}: seleziona colonne specifiche
    \item $\bowtie$ (bowtie) - \textbf{Join}: combina relazioni correlate
    \item $\times$ (times) - \textbf{Prodotto cartesiano}: combina ogni riga di una relazione con ogni riga di un'altra
    \item $\cup$ (union) - \textbf{Unione}: combina tuple di due relazioni (elimina duplicati)
    \item $\cap$ (intersection) - \textbf{Intersezione}: tuple presenti in entrambe le relazioni
    \item $-$ (minus) - \textbf{Differenza}: tuple nella prima relazione ma non nella seconda
    \item $\rho$ (rho) - \textbf{Ridenominazione}: rinomina relazioni o attributi
\end{itemize}

\subsection*{Consigli per la Risoluzione}

\begin{enumerate}
    \item Identificare quali relazioni sono necessarie
    \item Applicare prima le selezioni per ridurre i dati
    \item Eseguire i join necessari
    \item Applicare le proiezioni alla fine per selezionare gli attributi richiesti
    \item Per query complesse, usare risultati intermedi con nomi significativi
    \item La ridenominazione è essenziale quando si devono fare self-join o confronti
\end{enumerate}

\end{document}
