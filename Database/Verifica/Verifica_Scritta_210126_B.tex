\documentclass{article}
\usepackage[top=1.5cm, bottom=1.5cm]{geometry}
\usepackage[utf8]{inputenc}
\usepackage[italian]{babel}
\usepackage{enumitem}
\usepackage{amsmath}

\title{Verifica scritta di informatica - Classe 4 AIQ}
\author{Prof. Fedeli Massimo}
\date{Mercoledì 21 Gennaio 2026}

\title{Verifica scritta di informatica - Classe 4 AIQ}
\author{Prof. Fedeli Massimo}
\date{Mercoledì 21 Gennaio 2026}
\begin{document}
	\maketitle
\begin{flushleft}
		\textbf{Compito B - Cognome e Nome}: 
\end{flushleft}
								
	\section*{Descrizione del Sistema}
	
	Si realizzi il diagramma ER della seguente realtà aeroportuale. 
	Un aeroporto internazionale gestisce voli di diverse compagnie aeree, ognuna caratterizzata da un codice IATA, nome, paese di origine e anno di fondazione.  Il personale aeroportuale comprende piloti, assistenti di volo, controllori di volo e personale di terra, tutti identificati da un codice dipendente, nome, cognome e data di assunzione. 
	L'aeroporto è organizzato in diversi terminal, ognuno identificato da un codice e caratterizzato da tipologia (nazionale/internazionale), anno di apertura e capienza massima di passeggeri. Ogni terminal dispone di almeno un gate, caratterizzato da numero identificativo, tipo (imbarco standard/imbarco rapido) e stato operativo (attivo/in manutenzione). Alcuni gate possono essere condivisi tra più compagnie aeree in base agli accordi commerciali. 
	
	I piloti sono caratterizzati dalla loro abilitazione (monomotore, plurimotore, jet regionale, jet internazionale) e dal numero di ore di volo accumulate. Ogni pilota può essere qualificato per pilotare uno o più modelli di aeromobile.  Gli assistenti di volo sono assegnati a una sola compagnia aerea e sono caratterizzati dalle lingue parlate (almeno una) e dal livello di anzianità (junior, senior, capo cabina). 
	
	I controllori di volo gestiscono uno o più settori di spazio aereo. I settori sono caratterizzati da un codice identificativo, una descrizione geografica e un livello di complessità (basso/medio/alto). I settori con complessità "alta" richiedono almeno due controllori simultanei, mentre quelli a complessità bassa o media possono essere gestiti da un singolo controllore. 
	
	Il personale di terra non appartiene a una compagnia specifica ma lavora per l'aeroporto stesso, è caratterizzato dalla mansione (check-in, sicurezza, bagagli, manutenzione) e dal terminal di assegnazione prevalente. 
	
	\section*{Svolgimento}
		\begin{itemize}[noitemsep]
		\item Realizzare lo schema concettuale definitivo spiegando come è stata effettuata la ristrutturazione qualora sia stata necessaria. 4 pt
		\item Scrivere lo schema logico mettendo bene in evidenza chiavi primarie e chiavi esterne. 1 pt 
		\item Scrivere il codice SQL DDL.  3.5 pt
		\item Scrivere le opportune politiche di reazione motivando la scelta.  1.5 pt
	\end{itemize}
	
	
	\vspace{2cm}

\end{document}