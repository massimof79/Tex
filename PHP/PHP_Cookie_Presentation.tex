\documentclass[aspectratio=169]{beamer}
\usepackage[utf8]{inputenc}
\usepackage[italian]{babel}
\usepackage{graphicx}
\usepackage{listings}
\usepackage{xcolor}
\usepackage{tikz}
\usepackage{hyperref}
\usepackage{array}
\usepackage{booktabs}
\usetikzlibrary{shapes,arrows,positioning,calc}

% Tema e colori
\usetheme{Madrid}
\usecolortheme{default}
\setbeamercolor{structure}{fg=blue!70!black}
\setbeamercolor{alerted text}{fg=red!80!black}

% Configurazione per il codice PHP
\lstdefinestyle{phpstyle}{
    language=PHP,
    basicstyle=\ttfamily\small,
    keywordstyle=\color{blue}\bfseries,
    commentstyle=\color{green!60!black}\itshape,
    stringstyle=\color{red},
    showstringspaces=false,
    numbers=left,
    numberstyle=\tiny\color{gray},
    frame=single,
    breaklines=true,
    captionpos=b,
    tabsize=4,
    morekeywords={setcookie, time}
}

\lstdefinestyle{httpstyle}{
    basicstyle=\ttfamily\footnotesize,
    keywordstyle=\color{blue}\bfseries,
    commentstyle=\color{green!60!black}\itshape,
    stringstyle=\color{red},
    showstringspaces=false,
    frame=single,
    breaklines=true,
    backgroundcolor=\color{gray!10}
}

\title{Cookie in PHP}
\subtitle{Gestione della Persistenza nelle Applicazioni Web}
\author{Prof. Fedeli Massimo IIS E. Fermi Sacconi Cpia }
\institute{Ascoli Piceno}
\date{\today}

\begin{document}

% Slide 1: Titolo
\begin{frame}
\titlepage
\end{frame}

% Slide 2: Indice
\begin{frame}{Indice}
\tableofcontents
\end{frame}

\section{Il Protocollo HTTP}

% Slide 3: Introduzione al Protocollo HTTP
\begin{frame}{Il Protocollo HTTP}
\begin{block}{Hyper Text Transfer Protocol}
Il World Wide Web per il trasferimento di dati ipertestuali si basa sul protocollo applicativo \textbf{HTTP} che utilizza l'architettura client/server.
\end{block}

\vspace{0.3cm}

\begin{columns}
\column{0.5\textwidth}
\textbf{Caratteristiche principali:}
\begin{itemize}
    \item Protocollo di livello applicativo
    \item Basato su TCP/IP
    \item Scambio di messaggi testuali
    \item RFC 1945 (HTTP/1.0)
    \item RFC 2616 (HTTP/1.1)
\end{itemize}

\column{0.5\textwidth}
\begin{center}
\begin{tikzpicture}[scale=0.8]
    \node[draw, rectangle, minimum width=2cm, minimum height=1cm, fill=blue!20] (client) at (0,0) {Client};
    \node[draw, rectangle, minimum width=2cm, minimum height=1cm, fill=green!20] (server) at (5,0) {Server};
    \draw[->,thick] (client.east) -- node[above] {Request} (server.west);
    \draw[->,thick] (server.west) -- node[below] {Response} (client.east);
\end{tikzpicture}
\end{center}
\end{columns}
\end{frame}

% Slide 4: Struttura dei Messaggi HTTP
\begin{frame}{Struttura dei Messaggi HTTP}
\begin{block}{Tipologie di Messaggi}
I messaggi HTTP sono composti da caratteri ASCII quindi leggibili e terminati da CR+LF (Carriage Return + Line Feed) 
\end{block}

	\begin{itemize}
	\item 	Una riga vuota (ovvero due CR+LF di fila) indica che l' intestazionè è finita e che sta per iniziare il "\textbf{corpo}" del messaggio (ad esempio il codice HTML della pagina o i dati di un modulo).
	
	\item 	Un esempio pratico:
	\subitem Ecco come appare una richiesta per il sito esempio.it:
	
	\subitem GET /index.html HTTP/1.1\textbf{[CRLF]} Host: esempio.it[CRLF] Accept: text/html\textbf{[CRLF] [CRLF] }<-- Riga vuota (fine intestazioni)
	
	\item 	Senza quei segnali CR+LF, il server vedrebbe tutto come un unico blocco di testo confuso (GET /index.html HTTP/1.1Host: esempio.it...) e restituirebbe un errore.
	
	\end{itemize}
\end{frame}


\begin{frame}{Struttura dei Messaggi HTTP}
	\begin{block}{Tipologie di Messaggi}
		I messaggi HTTP  possono essere messaggi di richiesta o di risposta:
	\end{block}
		\vspace{0.5cm}
	\begin{columns}
		\column{0.5\textwidth}
		\textbf{REQUEST (Richiesta)}
		\begin{itemize}
			\item Inviata dal client
			\item Richiede una risorsa
			\item Contiene metodi (GET, POST, ecc.)
		\end{itemize}
		\column{0.5\textwidth}
		\textbf{RESPONSE (Risposta)}
		\begin{itemize}
			\item Inviata dal server
			\item Fornisce la risorsa richiesta
			\item Contiene codici di stato
		\end{itemize}
	\end{columns}
	
	\vspace{0.5cm}
	\begin{alertblock}{Importante}
		La comunicazione avviene mediante TCP/IP utilizzando gli indirizzi IP dei computer che ospitano client e server.
	\end{alertblock}
\end{frame}





% Slide 5: Meccanismo di Comunicazione HTTP
\begin{frame}{Meccanismo di Comunicazione HTTP}
\begin{block}{Sequenza di Operazioni}
\end{block}

\begin{tikzpicture}[node distance=1cm, auto]
    \tikzstyle{process} = [rectangle, rounded corners, minimum width=3cm, minimum height=0.8cm, text centered, draw=black, fill=blue!20]
    \tikzstyle{arrow} = [thick,->,>=stealth]
    
    \node (step1) [process] {1. Apertura connessione TCP};
    \node (step2) [process, below of=step1] {2. Browser richiede risorsa};
    \node (step3) [process, below of=step2] {3. Server invia risposta};
    \node (step4) [process, below of=step3] {4. Chiusura connessione};
    
    \draw [arrow] (step1) -- (step2);
    \draw [arrow] (step2) -- (step3);
    \draw [arrow] (step3) -- (step4);
\end{tikzpicture}

\begin{alertblock}{Nota}
Ogni richiesta di pagina richiede una nuova connessione TCP indipendente.
\end{alertblock}
\end{frame}

% Slide 6: Esempio di Connessione HTTP
\begin{frame}[fragile]{Esempio di Connessione HTTP}
\begin{block}{Fasi della Connessione}
\end{block}

\begin{enumerate}
    \item \textbf{Analisi URL}: Il browser estrae il dominio dall'URL
    \begin{lstlisting}[style=httpstyle]
http://www.esempio.it/pagina.html
    \end{lstlisting}
    
    \item \textbf{Connessione TCP}: Il client inizia una connessione verso il server sulla porta 80
    
    \item \textbf{Invio Richiesta}: Il client invia una richiesta GET attraverso il socket TCP
    \begin{lstlisting}[style=httpstyle]
GET /pagina.html HTTP/1.1
Host: www.esempio.it
    \end{lstlisting}
    
    \item \textbf{Risposta Server}: Il server incapsula la risorsa nella risposta HTTP e la invia al client.
\end{enumerate}
\end{frame}

% Slide 7: Completamento della Connessione
\begin{frame}{Completamento della Connessione HTTP}
\begin{block}{Fasi Finali}
\end{block}

\begin{enumerate}
    \setcounter{enumi}{4}
    \item \textbf{Chiusura Connessione}: Il server richiede al TCP di chiudere la connessione dopo l'invio della risposta
    
    \item \textbf{Terminazione TCP}: La connessione si conclude dopo il riscontro del client
    
    \item \textbf{Parsing HTML}: Il client estrae il file HTML e identifica gli oggetti referenziati (immagini, CSS, JavaScript)
    
    \item \textbf{Richieste Multiple}: I passi precedenti vengono ripetuti per ogni risorsa referenziata (eventualmente aprendo connessioni in parallelo)
\end{enumerate}

\begin{alertblock}{Prestazioni}
L'apertura di multiple connessioni parallele migliora le prestazioni nel caricamento delle pagine web moderne.
\end{alertblock}
\end{frame}

\section{Formato dei Messaggi HTTP}

% Slide 8: Struttura del Messaggio HTTP
\begin{frame}[fragile]{Formato dei Messaggi HTTP}
\begin{block}{Componenti Principali}
\end{block}

\begin{columns}
\column{0.5\textwidth}
\textbf{Struttura:}
\begin{enumerate}
    \item \textbf{START-LINE}\\
    Riga di richiesta/risposta
    \item \textbf{HEADER}\\
    Intestazione HTTP
    \item \textbf{BODY}\\
    Corpo HTTP (opzionale)
\end{enumerate}

\column{0.5\textwidth}
\begin{lstlisting}[style=httpstyle]
GET /index.html HTTP/1.1
Host: www.example.com
User-Agent: Mozilla/5.0
Accept: text/html
[riga vuota]
[eventuale body]
\end{lstlisting}
\end{columns}

\vspace{0.3cm}
\begin{block}{Metodi HTTP Principali}
\textbf{GET}, \textbf{POST}, HEAD, PUT, DELETE, TRACE, CONNECT, OPTIONS
\end{block}
\end{frame}

% Slide 9: START-LINE della Richiesta
\begin{frame}[fragile]{START-LINE - Riga di Richiesta}
\begin{block}{Elementi della Prima Riga}
La START-LINE della richiesta contiene tre elementi:
\end{block}

\begin{center}
\begin{tikzpicture}
    \node[draw, rectangle, fill=blue!20, minimum width=2.5cm, minimum height=1cm] (method) at (0,0) {METODO};
    \node[draw, rectangle, fill=green!20, minimum width=2.5cm, minimum height=1cm] (path) at (3.5,0) {PERCORSO};
    \node[draw, rectangle, fill=orange!20, minimum width=2.5cm, minimum height=1cm] (version) at (7,0) {VERSIONE};
    
    \node[below=0.3cm of method] {GET, POST, ...};
    \node[below=0.3cm of path] {/index.html};
    \node[below=0.3cm of version] {HTTP/1.1};
\end{tikzpicture}
\end{center}

\vspace{0.3cm}
\begin{exampleblock}{Esempio}
\begin{lstlisting}[style=httpstyle]
GET /sistemi/index.html HTTP/1.1
POST /form/login.php HTTP/1.1
\end{lstlisting}
\end{exampleblock}
\end{frame}

\begin{frame}{Formato dei Messaggi HTTP}
	\begin{block}{} % <--- Aggiunto il titolo
		\centering % <--- Centra l'immagine nel blocco
		\includegraphics[width=0.8\textwidth]{header.png}
	\end{block}
		Nella sezione General, visibile tramite i webtools del browser troviamo tra le altre informazioni anche il contenuto della start line.
\end{frame}

% Slide 10: Header HTTP della Richiesta
\begin{frame}[fragile]{Header HTTP - Intestazione}
\begin{block}{Campi dell'Header}
L'intestazione contiene informazioni aggiuntive sulla richiesta/risposta
\end{block}




\begin{columns}
\column{0.5\textwidth}
\textbf{Informazioni nell'Header:}
\begin{itemize}
    \item Tipo di browser
    \item Data e ora
    \item Cookie
    \item Codifica
    \item Lingua preferita
    \item Tipo di connessione
\end{itemize}

\column{0.5\textwidth}
\begin{lstlisting}[style=httpstyle, basicstyle=\ttfamily\tiny]
Host: www.example.com
User-Agent: Mozilla/5.0
Accept: text/html
Accept-Language: it-IT
Accept-Encoding: gzip
Connection: keep-alive
Cookie: session=abc123
\end{lstlisting}
\end{columns}

\vspace{0.3cm}
\begin{alertblock}{Body del Messaggio}
Il corpo può essere omesso nella richiesta, mentre è sempre presente nella risposta con la pagina HTML.
\end{alertblock}
\end{frame}

% Slide 11: Esempio Completo di Richiesta HTTP
\begin{frame}[fragile]{Esempio Completo - Richiesta HTTP}
\begin{lstlisting}[style=httpstyle, basicstyle=\ttfamily\scriptsize]
GET /index.html HTTP/1.1
Host: www.example.com
User-Agent: Mozilla/5.0 (Windows NT 10.0; Win64; x64)
Accept: text/html,application/xhtml+xml
Accept-Language: it-IT,it;q=0.9,en;q=0.8
Accept-Encoding: gzip, deflate, br
Connection: keep-alive
Cache-Control: max-age=0
Cookie: sessionID=xyz789; user=mario

\end{lstlisting}

\begin{itemize}
    \item \textbf{Linea 1}: Metodo GET, risorsa richiesta, versione HTTP
    \item \textbf{Linee 2-9}: Header con metadati della richiesta
    \item \textbf{Linea 10}: Riga vuota che separa header da body
    \item \textbf{Body}: Assente nelle richieste GET
\end{itemize}
\end{frame}

% Slide 12: Esempio Completo di Risposta HTTP
\begin{frame}[fragile]{Esempio Completo - Risposta HTTP}
\begin{lstlisting}[style=httpstyle, basicstyle=\ttfamily\scriptsize]
HTTP/1.1 200 OK
Date: Mon, 15 Jan 2024 10:30:00 GMT
Server: Apache/2.4.41
Content-Type: text/html; charset=UTF-8
Content-Length: 1234
Connection: keep-alive
Set-Cookie: sessionID=abc123; Path=/; HttpOnly

<!DOCTYPE html>
<html>
<head><title>Pagina di Esempio</title></head>
<body><h1>Benvenuto!</h1></body>
</html>
\end{lstlisting}

\begin{itemize}
    \item \textbf{Linea 1}: Versione HTTP e codice di stato
    \item \textbf{Linee 2-7}: Header con metadati della risposta
    \item \textbf{Linee 9-13}: Body con il contenuto HTML
\end{itemize}
\end{frame}

% Slide 13: Metodo POST e Body
\begin{frame}[fragile]{Metodo POST - Invio Dati}
\begin{block}{Utilizzo del Body}
Il corpo del messaggio contiene i dati di un form \textbf{SOLO} con il metodo POST
\end{block}

\begin{lstlisting}[style=httpstyle, basicstyle=\ttfamily\small]
POST /login.php HTTP/1.1
Host: www.example.com
Content-Type: application/x-www-form-urlencoded
Content-Length: 35

username=mario&password=secretpass
\end{lstlisting}

\vspace{0.3cm}
\begin{columns}
\column{0.5\textwidth}
\textbf{GET vs POST:}
\begin{itemize}
    \item GET: parametri nell'URL
    \item POST: parametri nel body
    \item POST: più sicuro e capiente
\end{itemize}

\column{0.5\textwidth}
\begin{alertblock}{Sicurezza}
Il metodo POST è preferibile per dati sensibili (password, dati personali)
\end{alertblock}
\end{columns}
\end{frame}

% Slide 1: Introduzione e Identificazione
\begin{frame}{Significato dei campi dell'Header - 1. Identificazione e Controllo}
	Gli header di identificazione definiscono l'origine e la destinazione della richiesta.
	\begin{itemize}
		\item \textbf{Host:} Dominio del server (es. \texttt{google.com}). Obbligatorio in HTTP/1.1.
		\item \textbf{User-Agent:} Identifica browser e sistema operativo del client.
		\item \textbf{Referer:} URL della pagina precedente che ha generato il link.
		\item \textbf{Origin:} Indica la provenienza di una richiesta (fondamentale per la sicurezza CORS).
	\end{itemize}
\end{frame}

% Slide 2: Negoziazione del Contenuto
\begin{frame}{Significato dei campi dell'Header - 2. Negoziazione del Contenuto}
	Permettono al client di comunicare al server i formati preferiti per la risposta.
	\begin{block}{Campi "Accept"}
		\begin{itemize}
			\item \textbf{Accept:} Tipi MIME supportati (es. \texttt{text/html}, \texttt{application/json}).
			\item \textbf{Accept-Encoding:} Algoritmi di compressione (es. \texttt{gzip}, \texttt{br}).
			\item \textbf{Accept-Language:} Lingue preferite (es. \texttt{it-IT}, \texttt{en-US}).
			\item \textbf{Accept-Charset:} Set di caratteri (es. \texttt{utf-8}).
		\end{itemize}
	\end{block}
\end{frame}

% Slide 3: Sicurezza e Stato (Cookies)
\begin{frame}{Significato dei campi dell'Header -  3. Sicurezza e Gestione Sessione}
	Gestiscono l'accesso alle risorse protette e il mantenimento dello stato.
	\begin{description}
		\item[Authorization] Credenziali per l'accesso (es. \texttt{Bearer token} o \texttt{Basic auth}).
		\item[Cookie] Invia al server i token di sessione precedentemente memorizzati.
		\item[Upgrade-Insecure-Requests] Segnala al server la preferenza per connessioni cifrate (HTTPS).
		\item[DNT (Do Not Track)] Esprime la volontà dell'utente di non essere tracciato.
	\end{description}
\end{frame}

% Slide 4: Ottimizzazione e Caching
\begin{frame}{Significato dei campi dell'Header -  4. Header Condizionali e Caching}
	Fondamentali per ridurre il traffico dati e migliorare le prestazioni.
	\begin{itemize}
		\item \textbf{Cache-Control:} Direttive per la cache (es. \texttt{no-cache}, \texttt{max-age}).
		\item \textbf{If-None-Match / If-Modified-Since:} La risorsa viene inviata solo se è cambiata rispetto alla versione in cache (basato su ETag o data).
		\item \textbf{Connection:} Gestione della persistenza TCP (es. \texttt{keep-alive}).
	\end{itemize}
\end{frame}

% Slide 5: Header di Entità (Body) e Moderni
\begin{frame}{5. Dati dell'Entità e Client Hints}
	Usati quando il client invia dati (es. POST) e per la privacy moderna.
	\begin{itemize}
		\item \textbf{Content-Type:} Formato dei dati inviati (es. \texttt{multipart/form-data}).
		\item \textbf{Content-Length:} Dimensione in byte del corpo della richiesta.
		\item \textbf{Sec-CH-UA:} (Moderno) Versione sicura dello User-Agent per proteggere la privacy.
	\end{itemize}
	\vfill
	\centering
	\textbf{Nota:} Tutti gli header seguono la struttura \texttt{Chiave: Valore}.
\end{frame}


% Slide 14: Codici di Stato HTTP
\begin{frame}{Codici di Stato HTTP}
\begin{block}{Categorie dei Codici di Stato}
Il server risponde con un codice numerico che indica l'esito della richiesta
\end{block}

\begin{table}
\small
\begin{tabular}{cl}
\toprule
\textbf{Codice} & \textbf{Significato} \\
\midrule
\textbf{1xx} & Informational - Richiesta ricevuta \\
\textbf{2xx} & Success - Richiesta completata con successo \\
& 200 OK - Richiesta riuscita \\
\textbf{3xx} & Redirection - Ulteriori azioni necessarie \\
& 301 Moved Permanently - Risorsa spostata \\
\textbf{4xx} & Client Error - Errore nella richiesta \\
& 404 Not Found - Risorsa non trovata \\
\textbf{5xx} & Server Error - Errore del server \\
& 500 Internal Server Error \\
\bottomrule
\end{tabular}
\end{table}
\end{frame}

\section{Header HTTP e Cookie}

% Slide 15: Elementi dell'Header HTTP
\begin{frame}{Elementi dell'Header HTTP}
\begin{block}{Meta-informazioni}
Ogni volta che un utente visita una pagina web, browser e server si scambiano meta-informazioni mediante l'header HTTP
\end{block}

\vspace{0.3cm}
\textbf{Campi dell'Header:}
\begin{itemize}
    \item Ogni riga è chiamata "\textbf{Campo dell'Header}"
    \item Formato: \texttt{Nome: Valore}
    \item Separatore: due punti (\texttt{:})
    \item Circa 100 campi disponibili
    \item 30 campi per la richiesta
    \item 30 campi per la risposta
    \item Altri campi non standardizzati
\end{itemize}
\end{frame}

% Slide 16: Principali Campi dell'Header
\begin{frame}[fragile]{Principali Campi dell'Header}
\begin{block}{Significato dei Campi Comuni}
\end{block}

\begin{table}
\scriptsize
\begin{tabular}{p{3.5cm}p{7cm}}
\toprule
\textbf{Campo} & \textbf{Descrizione} \\
\midrule
\texttt{HTTP/1.1} & Versione del protocollo HTTP utilizzata \\
\texttt{200 OK} & Codice di stato (ricezione e accettazione) \\
\texttt{Content-Encoding} & Tipo di codifica del file \\
\texttt{Content-Type} & Tipo MIME del contenuto \\
\texttt{Age, Cache-Control} & Informazioni sul caching \\
\texttt{Expires, Vary} & Gestione della cache \\
\texttt{ETag} & Versione del file (per validazione cache) \\
\texttt{Last-Modified} & Data ultima modifica \\
\texttt{Server} & Software del web server \\
\texttt{Content-Length} & Dimensione del file in byte \\
\bottomrule
\end{tabular}
\end{table}
\end{frame}

% Slide 17: Scopo dell'Header HTTP
\begin{frame}{Scopo dell'Header HTTP}
\begin{block}{Funzioni Principali}
Le informazioni dell'header servono per il coordinamento tra client e server
\end{block}

\vspace{0.3cm}
\textbf{Obiettivi dello Scambio di Header:}
\begin{itemize}
    \item \textbf{Comprensione formato}: Assicurare che il client comprenda la forma del file ricevuto
    \item \textbf{Validazione dimensione}: Verificare che la dimensione coincida con quella attesa
    \item \textbf{Gestione cache}: Ottimizzare le prestazioni memorizzando risorse
    \item \textbf{Negoziazione contenuto}: Selezionare la versione più adatta (lingua, formato)
    \item \textbf{Gestione sessioni}: Mantenere lo stato attraverso cookie
\end{itemize}

\vspace{0.3cm}
\begin{alertblock}{Estensibilità}
Esistono quasi 100 campi header, di cui solo una parte è standardizzata
\end{alertblock}
\end{frame}

% Slide 18: Campo Set-Cookie nell'Header
\begin{frame}[fragile]{Campo Set-Cookie - Risposta del Server}
\begin{block}{Creazione di Cookie}
Il campo \texttt{Set-Cookie} nella risposta del server richiede la memorizzazione di informazioni in un cookie sul client
\end{block}

\begin{lstlisting}[style=httpstyle]
HTTP/1.1 200 OK
Date: Mon, 15 Jan 2024 10:30:00 GMT
Server: Apache/2.4.41
Content-Type: text/html; charset=UTF-8
Set-Cookie: username=mario; Expires=Wed, 15-Jan-2025 10:30:00 GMT
Set-Cookie: sessionID=xyz789; Path=/; HttpOnly; Secure

<!DOCTYPE html>
...
\end{lstlisting}

\begin{alertblock}{Importante}
Il server può richiedere la creazione di multipli cookie nella stessa risposta
\end{alertblock}
\end{frame}

% Slide 19: Campo Cookie nell'Header
\begin{frame}[fragile]{Campo Cookie - Richiesta del Client}
\begin{block}{Invio Cookie al Server}
Il campo \texttt{Cookie} nella richiesta del client comunica il contenuto archiviato nei cookie
\end{block}

\begin{lstlisting}[style=httpstyle]
GET /profilo.php HTTP/1.1
Host: www.example.com
User-Agent: Mozilla/5.0
Cookie: username=mario; sessionID=xyz789
Accept: text/html

\end{lstlisting}

\vspace{0.3cm}
\begin{center}
\begin{tikzpicture}[node distance=2cm]
    \node[draw, circle, fill=blue!20] (client) {Client};
    \node[draw, circle, fill=green!20, right=of client] (server) {Server};
    \draw[->,thick, bend left=30] (server.north) to node[above] {\scriptsize Set-Cookie} (client.north);
    \draw[->,thick, bend left=30] (client.south) to node[below] {\scriptsize Cookie} (server.south);
\end{tikzpicture}
\end{center}
\end{frame}

\section{Persistenza in PHP}

% Slide 20: Problema della Persistenza
\begin{frame}{La Persistenza in PHP}
\begin{block}{Protocollo Stateless}
Il protocollo HTTP non permette al server di "riconoscere" un utente che si era precedentemente collegato
\end{block}

\vspace{0.3cm}
\textbf{Caratteristiche di HTTP:}
\begin{itemize}
    \item Ogni pagina richiede una connessione TCP \textbf{indipendente}
    \item Nessuna informazione viene mantenuta tra richieste successive
    \item Ogni coppia request/response è isolata dalle altre
    \item \alert{HTTP è un protocollo STATELESS}
\end{itemize}

\vspace{0.3cm}
\begin{exampleblock}{Conseguenza}
Le richieste dei client non lasciano alcuno stato nel server. Ciascuna coppia request\_client/response\_server è indipendente dalle altre.
\end{exampleblock}
\end{frame}

% Slide 21: Limiti del Protocollo HTTP
\begin{frame}{Limiti del Protocollo Stateless}
\begin{alertblock}{Problemi Applicativi}
Sin dall'inizio dell'espansione del Web, questa caratteristica del protocollo HTTP ha mostrato tutti i suoi limiti
\end{alertblock}

\vspace{0.3cm}
\textbf{Scenari Problematici:}
\begin{itemize}
    \item \textbf{Autenticazione}: Come mantenere l'utente loggato?
    \item \textbf{Carrello e-commerce}: Come ricordare i prodotti selezionati?
    \item \textbf{Preferenze}: Come salvare le impostazioni dell'utente?
    \item \textbf{Tracking}: Come seguire il percorso di navigazione?
\end{itemize}

\vspace{0.3cm}
\begin{block}{Soluzione}
Esistono tecniche che simulano lo stato in una tipica connessione client/server
\end{block}
\end{frame}

% Slide 22: Tecniche per la Persistenza
\begin{frame}{Tecniche per la Persistenza della Connessione}
\begin{block}{Il Problema}
Dopo che una pagina web viene inviata dal server al client, per uno script PHP non è più possibile accedere ai dati relativi alla pagina stessa
\end{block}

\vspace{0.3cm}
\textbf{Necessità degli Sviluppatori:}
\begin{itemize}
    \item Memorizzare informazioni persistenti per più pagine
    \item Mantenere traccia degli utenti loggati
    \item Conservare preferenze e impostazioni
\end{itemize}

\vspace{0.3cm}
\begin{exampleblock}{Soluzioni in PHP}
PHP mette a disposizione due metodi principali:
\begin{enumerate}
    \item \textbf{COOKIE} - Memorizzazione lato client
    \item \textbf{SESSIONI} - Memorizzazione lato server
\end{enumerate}
\end{exampleblock}
\end{frame}

\section{I Cookie}

% Slide 23: Cosa Sono i Cookie
\begin{frame}{I Cookie - Definizione}
\begin{block}{Definizione}
I cookie sono \textbf{file di testo} memorizzati sul client su richiesta esplicita del server
\end{block}

\vspace{0.3cm}
\begin{columns}
\column{0.5\textwidth}
\textbf{Caratteristiche:}
\begin{itemize}
    \item File di piccole dimensioni
    \item Memorizzati sul filesystem del client
    \item Associati a un dominio specifico
    \item Validità temporale configurabile
    \item Inviati automaticamente nelle richieste
\end{itemize}

\column{0.5\textwidth}
\textbf{Funzionamento:}
\begin{enumerate}
    \item Server richiede creazione via \texttt{Set-Cookie}
    \item Browser salva il cookie
    \item Nelle richieste successive il cookie viene inviato
    \item Server legge i dati del cookie
\end{enumerate}
\end{columns}
\end{frame}

% Slide 24: Meccanismo dei Cookie
\begin{frame}{Meccanismo dei Cookie}
\begin{block}{Processo di Gestione}
\end{block}

\begin{center}
\begin{tikzpicture}[node distance=1.2cm, scale=0.9, every node/.style={scale=0.9}]
    \tikzstyle{box} = [rectangle, rounded corners, minimum width=3cm, minimum height=0.8cm, text centered, draw=black, fill=blue!20]
    \tikzstyle{arrow} = [thick,->,>=stealth]
    
    \node (req1) [box] {1. Client richiede pagina};
    \node (resp1) [box, below of=req1] {2. Server invia Set-Cookie};
    \node (save) [box, below of=resp1] {3. Browser salva cookie};
    \node (req2) [box, below of=save] {4. Nuova richiesta con Cookie};
    \node (read) [box, below of=req2] {5. Server legge cookie};
    
    \draw [arrow] (req1) -- (resp1);
    \draw [arrow] (resp1) -- (save);
    \draw [arrow] (save) -- (req2);
    \draw [arrow] (req2) -- (read);
\end{tikzpicture}
\end{center}
\end{frame}

% Slide 25: Componenti di un Cookie
\begin{frame}{Componenti di un Cookie}
\begin{block}{Elementi Fondamentali}
I cookie sono caratterizzati da diversi parametri
\end{block}

\begin{table}
\small
\begin{tabular}{lp{7cm}}
\toprule
\textbf{Elemento} & \textbf{Descrizione} \\
\midrule
\textbf{Nome} & Identificatore univoco del cookie \\
\textbf{Valore} & Dati memorizzati nel cookie \\
\textbf{Scadenza} & Validità temporale (opzionale) \\
\textbf{Path} & Percorso del sito per cui è valido \\
\textbf{Domain} & Dominio per cui è valido \\
\textbf{Secure} & Trasmissione solo su HTTPS \\
\textbf{HttpOnly} & Non accessibile da JavaScript \\
\bottomrule
\end{tabular}
\end{table}

\begin{alertblock}{Nota}
Impostazioni di sicurezza adeguate proteggono i dati degli utenti
\end{alertblock}
\end{frame}

% Slide 26: Array \$_COOKIE
\begin{frame}[fragile]{Array Superglobale \$\_COOKIE}
\begin{block}{Accesso ai Cookie in PHP}
\texttt{\$\_COOKIE} è un array associativo che memorizza i valori dei cookie inviati dal client
\end{block}

\begin{lstlisting}[style=phpstyle]
<?php
// Lettura di un cookie
if (isset($_COOKIE['username'])) {
    $username = $_COOKIE['username'];
    echo "Benvenuto, " . $username;
} else {
    echo "Cookie non trovato";
}

// Visualizzare tutti i cookie
foreach ($_COOKIE as $nome => $valore) {
    echo "$nome: $valore<br>";
}
?>
\end{lstlisting}

\begin{alertblock}{Importante}
Ogni elemento dell'array è identificato da un'etichetta corrispondente al nome del cookie
\end{alertblock}
\end{frame}

% Slide 27: Creazione Cookie - Esempio Base
\begin{frame}[fragile]{Creazione di Cookie - Esempio Base}
\begin{block}{Funzione setcookie()}
\end{block}

\begin{lstlisting}[style=phpstyle]
<?php
$dato = "Questa stringa viene memorizzata nel cookie";

// Cookie senza scadenza (session cookie)
setcookie("Alfa", $dato);

// Cookie con scadenza di 30 giorni
setcookie("Beta", $dato, time() + 60*60*24*30);
?>
\end{lstlisting}

\vspace{0.3cm}
\begin{columns}
\column{0.5\textwidth}
\textbf{Cookie "Alfa":}
\begin{itemize}
    \item Senza scadenza
    \item Valido per la sessione
    \item Cancellato alla chiusura del browser
\end{itemize}

\column{0.5\textwidth}
\textbf{Cookie "Beta":}
\begin{itemize}
    \item Scadenza: 30 giorni
    \item Persistente
    \item Rimane dopo chiusura browser
\end{itemize}
\end{columns}
\end{frame}

% Slide 28: Calcolo della Scadenza
\begin{frame}[fragile]{Calcolo della Scadenza del Cookie}
\begin{block}{Funzione time()}
La funzione \texttt{time()} restituisce il timestamp Unix corrente (secondi dal 1 gennaio 1970)
\end{block}

\begin{lstlisting}[style=phpstyle]
<?php
// Cookie valido per 30 giorni
setcookie("Beta", $dato, time() + 60*60*24*30);

// Calcolo dettagliato:
// 60 secondi  = 1 minuto
// 60 minuti   = 1 ora      (60 * 60)
// 24 ore      = 1 giorno   (60 * 60 * 24)
// 30 giorni   = 1 mese     (60 * 60 * 24 * 30)

// Altri esempi di scadenza:
setcookie("temp", "valore", time() + 3600);      // 1 ora
setcookie("week", "valore", time() + 604800);    // 1 settimana
setcookie("year", "valore", time() + 31536000);  // 1 anno
?>
\end{lstlisting}
\end{frame}

% Slide 29: Funzione setcookie - Sintassi Completa
\begin{frame}[fragile]{Funzione setcookie() - Sintassi Completa}
\begin{block}{Prototipo Completo}
\end{block}

\begin{lstlisting}[style=phpstyle, basicstyle=\ttfamily\scriptsize]
setcookie(
    string $name,
    string $value = "",
    int $expires = 0,
    string $path = "",
    string $domain = "",
    bool $secure = false,
    bool $httponly = false
): bool
\end{lstlisting}

\begin{table}
\tiny
\begin{tabular}{lp{6cm}}
\toprule
\textbf{Parametro} & \textbf{Descrizione} \\
\midrule
\texttt{name} & Nome del cookie (obbligatorio) \\
\texttt{value} & Valore da memorizzare \\
\texttt{expires} & Timestamp di scadenza (0 = session cookie) \\
\texttt{path} & Percorso sul server (default: directory corrente) \\
\texttt{domain} & Dominio per cui è valido \\
\texttt{secure} & TRUE = solo HTTPS \\
\texttt{httponly} & TRUE = non accessibile da JavaScript \\
\bottomrule
\end{tabular}
\end{table}
\end{frame}

% Slide 30: Restrizioni di setcookie()
\begin{frame}[fragile]{Restrizioni della Funzione setcookie()}
\begin{alertblock}{IMPORTANTE}
I cookie devono essere inviati \textbf{prima} di qualsiasi output dello script!
\end{alertblock}

\begin{columns}
\column{0.5\textwidth}
\textbf{ERRATO:}
\begin{lstlisting}[style=phpstyle, basicstyle=\ttfamily\tiny]
<?php
echo "Pagina Web";
// ERRORE: header già inviati!
setcookie("test", "valore");
?>
\end{lstlisting}

\column{0.5\textwidth}
\textbf{CORRETTO:}
\begin{lstlisting}[style=phpstyle, basicstyle=\ttfamily\tiny]
<?php
// Prima setcookie
setcookie("test", "valore");
// Poi output
echo "Pagina Web";
?>
\end{lstlisting}
\end{columns}

\vspace{0.3cm}
\begin{block}{Spiegazione}
Questa è una restrizione del protocollo HTTP: l'header (che contiene Set-Cookie) deve precedere il body
\end{block}

\begin{exampleblock}{Soluzione}
Usare \texttt{ob\_start()} per il buffering dell'output se necessario
\end{exampleblock}
\end{frame}

% Slide 31: Esempio Completo di Gestione Cookie
\begin{frame}[fragile]{Esempio Completo - Gestione Cookie}
\begin{lstlisting}[style=phpstyle, basicstyle=\ttfamily\scriptsize]
<?php
// Imposta un cookie con tutte le opzioni
setcookie(
    "user_prefs",                    // nome
    "theme=dark&lang=it",            // valore
    time() + 86400 * 30,             // scadenza: 30 giorni
    "/",                             // path: tutto il sito
    "example.com",                   // domain
    true,                            // secure: solo HTTPS
    true                             // httponly: protetto da XSS
);

// Lettura del cookie
if (isset($_COOKIE['user_prefs'])) {
    parse_str($_COOKIE['user_prefs'], $preferences);
    echo "Tema: " . $preferences['theme'];
    echo "Lingua: " . $preferences['lang'];
}
?>
\end{lstlisting}
\end{frame}

% Slide 32: Eliminazione Cookie
\begin{frame}[fragile]{Eliminazione di un Cookie}
\begin{block}{Metodo}
Un cookie può essere eliminato impostando una scadenza \textbf{precedente} all'orario corrente
\end{block}

\begin{lstlisting}[style=phpstyle]
<?php
// Eliminazione di un cookie
setcookie("nome_cookie", "", time() - 3600);

// Oppure con data esplicita nel passato
setcookie("nome_cookie", "", 1);

// Importante: usare gli stessi parametri di path e domain
// utilizzati in fase di creazione
setcookie("user_prefs", "", time() - 3600, "/", "example.com");

// Verifica eliminazione
if (!isset($_COOKIE['nome_cookie'])) {
    echo "Cookie eliminato con successo";
}
?>
\end{lstlisting}

\begin{alertblock}{Attenzione}
Il cookie viene eliminato nella prossima richiesta, non immediatamente
\end{alertblock}
\end{frame}

% Slide 33: Vantaggi e Svantaggi dei Cookie
\begin{frame}{Cookie - Vantaggi e Svantaggi}
\begin{columns}
\column{0.5\textwidth}
\begin{block}{Vantaggi}
\begin{itemize}
    \item Semplici da implementare
    \item Non caricano il server
    \item Persistenti nel tempo
    \item Supportati da tutti i browser
    \item Permettono personalizzazione
\end{itemize}
\end{block}

\column{0.5\textwidth}
\begin{alertblock}{Svantaggi}
\begin{itemize}
    \item Memorizzati lato client
    \item \alert{Modificabili dall'utente}
    \item Limite di dimensione (4KB)
    \item Privacy concerns
    \item Possono essere disabilitati
    \item Non sicuri per dati sensibili
\end{itemize}
\end{alertblock}
\end{columns}

\vspace{0.5cm}
\begin{exampleblock}{Punto Debole Principale}
Le informazioni di stato vengono salvate in un file sul filesystem del client e sono quindi \textbf{potenzialmente modificabili} dall'utente
\end{exampleblock}
\end{frame}

\section{Normativa sui Cookie}

% Slide 34: Cookie e Normativa
\begin{frame}{Cookie in Base alla Normativa}
\begin{block}{Riferimenti Normativi}
\end{block}

\begin{itemize}
    \item \textbf{Provvedimento del Garante} per la Protezione dei Dati Personali (Gazzetta Ufficiale n. 126 del 3 giugno 2014)
    
    \item \textbf{Linee guida cookie e altri strumenti di tracciamento} – 10 giugno 2021 [9677876] (Gazzetta Ufficiale n. 163 del 9 luglio 2021)
\end{itemize}

\vspace{0.5cm}
\begin{block}{Classificazione}
La normativa distingue tra:
\begin{enumerate}
    \item \textbf{Cookie Tecnici}
    \item \textbf{Cookie di Profilazione}
\end{enumerate}
\end{block}
\end{frame}

% Slide 35: Cookie Tecnici
\begin{frame}{Cookie Tecnici}
\begin{block}{Definizione}
I cookie tecnici sono quelli che facilitano la navigazione permettendo al sito di offrire alcune funzionalità
\end{block}

\vspace{0.3cm}
\textbf{Esempi di Cookie Tecnici:}
\begin{itemize}
    \item \textbf{Autenticazione}: Riconoscimento dell'utente loggato senza dover reinserire username e password
    \item \textbf{Carrello}: Mantenimento dei prodotti aggiunti anche dopo giorni
    \item \textbf{Preferenze lingua}: Memorizzazione della lingua scelta
    \item \textbf{Impostazioni visualizzazione}: Font size, tema, layout
\end{itemize}

\vspace{0.3cm}
\begin{exampleblock}{Caratteristica Importante}
I cookie tecnici \textbf{non richiedono} il consenso esplicito dell'utente
\end{exampleblock}
\end{frame}

% Slide 36: Categorie di Cookie Tecnici
\begin{frame}{Categorie di Cookie Tecnici}
\begin{block}{Tre Tipologie}
\end{block}

\begin{enumerate}
    \item \textbf{Cookie di navigazione o di sessione}
    \begin{itemize}
        \item Garantiscono la normale navigazione e fruizione del sito web
        \item Permettono acquisti e autenticazione ad aree riservate
        \item Durata limitata alla sessione
    \end{itemize}
    
    \item \textbf{Cookie analytics}
    \begin{itemize}
        \item Assimilati ai cookie tecnici se usati dal gestore del sito
        \item Raccolgono informazioni aggregate sul numero di utenti
        \item Analizzano come gli utenti visitano il sito
    \end{itemize}
    
    \item \textbf{Cookie di funzionalità}
    \begin{itemize}
        \item Permettono navigazione personalizzata
        \item Memorizzano criteri selezionati (lingua, prodotti)
        \item Migliorano il servizio offerto all'utente
    \end{itemize}
\end{enumerate}
\end{frame}

% Slide 37: Cookie di Profilazione
\begin{frame}{Cookie di Profilazione}
\begin{block}{Definizione Normativa}
"I Cookie di profilazione hanno come scopo la creazione di un profilo del navigatore"
\end{block}

\vspace{0.3cm}
\textbf{Finalità:}
\begin{itemize}
    \item Comprendere il comportamento sul sito
    \item Identificare interessi e orientamenti
    \item Raccogliere e incrociare informazioni (proprie o di terze parti)
    \item Capire chi è l'utente e cosa gli interessa
\end{itemize}

\vspace{0.3cm}
\textbf{Utilizzo:}
\begin{itemize}
    \item Invio di messaggi pubblicitari personalizzati
    \item Vendita di servizi o prodotti mirati
    \item Personalizzazione navigazione oltre il minimo necessario
\end{itemize}

\vspace{0.3cm}
\begin{alertblock}{IMPORTANTE}
I cookie di profilazione \textbf{RICHIEDONO} il consenso esplicito dell'utente
\end{alertblock}
\end{frame}

% Slide 38: Cookie Banner e Consenso
\begin{frame}{Cookie Banner e Consenso}
\begin{block}{Obbligo di Informativa}
I siti web devono informare gli utenti sull'uso dei cookie e richiedere il consenso
\end{block}

\vspace{0.3cm}
\textbf{Requisiti del Banner:}
\begin{itemize}
    \item \textbf{Chiaro e visibile} al primo accesso
    \item \textbf{Informazioni complete} sui cookie utilizzati
    \item \textbf{Scelta granulare} per diverse tipologie di cookie
    \item \textbf{Possibilità di rifiuto} facilmente accessibile
    \item Link alla \textbf{Cookie Policy} completa
\end{itemize}

\vspace{0.3cm}
\begin{exampleblock}{Best Practice}
\begin{itemize}
    \item Cookie tecnici: attivati automaticamente
    \item Cookie di profilazione: richiedono consenso esplicito
    \item Permettere personalizzazione delle scelte
    \item Garantire facile revoca del consenso
\end{itemize}
\end{exampleblock}
\end{frame}

% Slide 39: Conclusioni e Alternative
\begin{frame}{Conclusioni e Alternative ai Cookie}
\begin{block}{Recap Cookie}
I cookie rappresentano una soluzione semplice ma con limitazioni di sicurezza
\end{block}

\vspace{0.3cm}
\textbf{Quando Usare i Cookie:}
\begin{itemize}
    \item Dati non sensibili
    \item Preferenze utente
    \item Tracking analytics
    \item Personalizzazione base
\end{itemize}

\vspace{0.3cm}
\begin{alertblock}{Alternative più Sicure}
\begin{itemize}
    \item \textbf{SESSIONI PHP}: Dati memorizzati lato server
    \item \textbf{Database}: Persistenza a lungo termine
    \item \textbf{Local Storage}: HTML5, maggiore capacità
    \item \textbf{Session Storage}: Dati temporanei nel browser
    \item \textbf{Token JWT}: Autenticazione stateless sicura
\end{itemize}
\end{alertblock}
\end{frame}

% Slide 40: Esempio Pratico Completo
\begin{frame}[fragile]{Esempio Pratico - Sistema di Login}
\begin{lstlisting}[style=phpstyle, basicstyle=\ttfamily\tiny]
<?php
// login.php - Gestione login con cookie "Ricordami"

if ($_SERVER['REQUEST_METHOD'] == 'POST') {
    $username = $_POST['username'];
    $password = $_POST['password'];
    $remember = isset($_POST['remember']);
    
    // Verifica credenziali (esempio semplificato)
    if ($username == 'admin' && $password == 'pass123') {
        
        // Cookie di sessione standard
        setcookie('logged_in', 'true', 0, '/', '', true, true);
        setcookie('username', $username, 0, '/', '', true, true);
        
        // Se "Ricordami" è selezionato, cookie persistente
        if ($remember) {
            $expire = time() + (86400 * 30); // 30 giorni
            setcookie('remember_token', 
                     hash('sha256', $username . time()),
                     $expire, '/', '', true, true);
        }
        
        header('Location: dashboard.php');
        exit;
    }
}
?>
\end{lstlisting}
\end{frame}

% Slide 41: Riepilogo Finale
\begin{frame}{Riepilogo Finale}
\begin{block}{Concetti Chiave}
\end{block}

\begin{enumerate}
    \item \textbf{HTTP è Stateless}: Non mantiene stato tra richieste
    \item \textbf{Cookie}: File di testo sul client per mantenere stato
    \item \textbf{setcookie()}: Funzione PHP per creare cookie
    \item \textbf{\$\_COOKIE}: Array per leggere cookie
    \item \textbf{Scadenza}: Gestibile con timestamp Unix
    \item \textbf{Sicurezza}: Usare Secure e HttpOnly
    \item \textbf{Normativa}: Distinguere tecnici vs profilazione
    \item \textbf{Limitazioni}: Modificabili dall'utente, 4KB max
    \item \textbf{Alternative}: Sessioni PHP per dati sensibili
\end{enumerate}

\vspace{0.3cm}
\begin{alertblock}{Ricorda}
Usa i cookie per dati non sensibili e considera sempre la privacy degli utenti
\end{alertblock}
\end{frame}

\end{document}
