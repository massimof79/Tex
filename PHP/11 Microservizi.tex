\documentclass[a4paper,12pt]{article}
\usepackage[italian]{babel}
\usepackage{graphicx}
\usepackage{hyperref}
\usepackage{geometry}
\geometry{margin=2.5cm}

\title{Architettura a Microservizi}
\author{Prof. Fedeli Massimo}
\date{}

\begin{document}
	
	\maketitle
	\newpage
	
	\section{Perché studiare i microservizi}
	Oggi molte delle applicazioni che utilizziamo quotidianamente (streaming, e-commerce, social, servizi online) sono costruite utilizzando l’architettura a microservizi.
	
	Comprendere questo modello significa capire come vengono progettati i sistemi software moderni e scalabili.
	
	\section{Dal monolite ai microservizi}
	Immaginiamo una applicazione per la gestione di una palestra.
	
	Un sistema monolitico contiene:
	
	\begin{itemize}
		\item gestione iscrizioni
		\item gestione pagamenti
		\item gestione corsi
		\item gestione certificati medici
	\end{itemize}
	
	Tutto è contenuto in un unico programma.
	
	Problemi:
	\begin{itemize}
		\item difficile da modificare
		\item difficile da scalare
		\item un errore blocca tutto il sistema
	\end{itemize}
	
	\section{La soluzione: microservizi}
	Si suddivide il sistema in servizi separati:
	
	\begin{itemize}
		\item Servizio Iscrizioni
		\item Servizio Pagamenti
		\item Servizio Corsi
		\item Servizio Certificati
	\end{itemize}
	
	Ogni servizio è indipendente.
	
	\section{Cosa è un microservizio}
	Un microservizio è un programma autonomo che:
	
	\begin{itemize}
		\item svolge una sola funzione
		\item comunica con altri servizi
		\item può essere aggiornato senza fermare tutto il sistema
	\end{itemize}
	
	Esempio:
	Il servizio pagamenti gestisce solo i pagamenti.
	
	\section{Architettura generale}
	Una applicazione a microservizi include:
	
	\begin{itemize}
		\item servizi indipendenti
		\item API Gateway
		\item database separati
		\item comunicazione via rete
	\end{itemize}
	
	\section{API Gateway}
	È il punto di accesso unico.
	
	Esempio:
	Uno studente accede all'app della palestra.
	
	La richiesta passa dall’API Gateway che decide:
	\begin{itemize}
		\item quale servizio coinvolgere
		\item come gestire la risposta
	\end{itemize}
	
	\section{Esempio reale}
	Sistema di e-commerce:
	
	\begin{itemize}
		\item Servizio Ordini
		\item Servizio Catalogo
		\item Servizio Pagamenti
		\item Servizio Spedizioni
	\end{itemize}
	
	Se aumenta il traffico sugli ordini, si scala solo quel servizio.
	
	\section{Autonomia dei servizi}
	Ogni servizio può:
	
	\begin{itemize}
		\item usare tecnologie diverse
		\item avere il proprio database
		\item essere sviluppato da team diversi
	\end{itemize}
	
	\section{Comunicazione tra servizi}
	Due modalità principali:
	
	\subsection{Sincrona}
	Un servizio chiama direttamente un altro.
	
	Esempio:
	Ordini chiede al servizio pagamenti di verificare il pagamento.
	
	\subsection{Asincrona}
	Un servizio invia un evento.
	
	Esempio:
	Pagamento completato → evento inviato → spedizioni si attiva.
	
	\section{Gestione dei dati}
	Ogni microservizio ha il proprio database.
	
	Vantaggi:
	\begin{itemize}
		\item maggiore indipendenza
		\item migliore scalabilità
	\end{itemize}
	
	\section{Problema delle transazioni}
	Se più servizi collaborano, non si può usare una transazione unica.
	
	Soluzione: Saga.
	
	Esempio:
	Ordine → pagamento → spedizione.
	
	Se fallisce la spedizione, si annulla il pagamento.
	
	\section{Scalabilità}
	Si può aumentare solo il servizio necessario.
	
	Esempio:
	Durante i saldi aumenta il servizio ordini.
	
	\section{Disponibilità}
	Se un servizio si blocca, gli altri continuano a funzionare.
	
	\section{Migrazione dal monolite}
	Si usa il Pattern Strangler.
	
	Passaggi:
	\begin{itemize}
		\item si inserisce un API Gateway
		\item si sostituiscono gradualmente le funzioni
	\end{itemize}
	
	\section{DDD e microservizi}
	Il Domain Driven Design aiuta a dividere il sistema.
	
	Esempio palestra:
	\begin{itemize}
		\item dominio clienti
		\item dominio pagamenti
		\item dominio corsi
	\end{itemize}
	
	\section{Sfide}
	\begin{itemize}
		\item maggiore complessità
		\item gestione rete
		\item gestione eventi
	\end{itemize}
	
	\section{Quando usarli}
	Non sempre sono la soluzione migliore.
	
	Sono utili quando:
	\begin{itemize}
		\item il sistema cresce
		\item servono aggiornamenti frequenti
		\item serve alta disponibilità
	\end{itemize}
	
	\section{Conclusione}
	I microservizi rappresentano un modello moderno per costruire software scalabile, modulare e aggiornabile.
	
	\section{Esercizio per studenti}
	Progetta un sistema a microservizi per una scuola.
	
	Individua:
	\begin{itemize}
		\item i servizi
		\item i dati
		\item le comunicazioni
	\end{itemize}
	
\end{document}