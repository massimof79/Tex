\textbackslash{}documentclass[10pt]\{beamer\}

\% -------------------------------------------------
\% Pacchetti base
\% -------------------------------------------------
\textbackslash{}usepackage[utf8]\{inputenc\}
\textbackslash{}usepackage[T1]\{fontenc\}
\textbackslash{}usepackage[italian]\{babel\}

\textbackslash{}usepackage\{listings\}
\textbackslash{}usepackage\{xcolor\}
\textbackslash{}usepackage\{graphicx\}
\textbackslash{}usepackage\{tikz\}
\textbackslash{}usepackage\{hyperref\}

\textbackslash{}usetikzlibrary\{shapes,arrows,positioning\}

\% -------------------------------------------------
\% Tema Beamer
\% -------------------------------------------------
\textbackslash{}usetheme\{Madrid\}
\textbackslash{}usecolortheme\{seahorse\}

\% -------------------------------------------------
\% Impostazioni per il codice PHP
\% -------------------------------------------------
\% -------------------------------------------------
\% LISTINGS: configurazione PHP a prova di studente
\% -------------------------------------------------
\textbackslash{}lstdefinestyle\{php\}\{
language=PHP,
basicstyle=\textbackslash{}ttfamily\textbackslash{}small,
keywordstyle=\textbackslash{}color\{blue\}\textbackslash{}bfseries,
commentstyle=\textbackslash{}color\{green!50!black\}\textbackslash{}itshape,
stringstyle=\textbackslash{}color\{red!70!black\},
showstringspaces=false,
breaklines=true,
breakatwhitespace=true,
frame=single,
backgroundcolor=\textbackslash{}color\{gray!5\},
numbers=left,
numberstyle=\textbackslash{}tiny\textbackslash{}color\{gray\},
stepnumber=1,
tabsize=2,
keepspaces=true,
columns=fullflexible,
mathescape=false,      \% ignora \$
texcl=false,           \% ignora commenti TeX
escapeinside=\{(*@\}\{@*)\} \% eventuale escape manuale
\}



\textbackslash{}title\{Funzioni del Linguaggio PHP\}
\textbackslash{}subtitle\{Guida per studenti del quinto anno\}
\textbackslash{}author\{Prof. Fedeli Massimo\textbackslash{}\textbackslash{}IIS E. Fermi -- Sacconi Ceci\textbackslash{}\textbackslash{}Ascoli Piceno\}
\textbackslash{}date\{Tutti i diritti riservati\}

\% -------------------------------------------------
\% Documento
\% -------------------------------------------------
\textbackslash{}begin\{document\}

\textbackslash{}begin\{frame\}
\textbackslash{}titlepage
\textbackslash{}end\{frame\}

\% -----------------------------------
\% SLIDE 1: Introduzione
\% -----------------------------------
\textbackslash{}begin\{frame\}\{Benvenuti nel mondo di PHP\}
PHP è come una cassetta degli attrezzi per sviluppatori web: contiene tutto ciò che serve per costruire siti dinamici e interattivi.

\textbackslash{}vspace\{0.5cm\}

Oggi esploreremo insieme le \textbackslash{}textbf\{funzioni predefinite\} di PHP, strumenti già pronti che ci risparmiano ore di lavoro e ci permettono di concentrarci sulla logica dell’applicazione.

\textbackslash{}vspace\{0.5cm\}

\textbackslash{}begin\{center\}
\textbackslash{}textit\{``Perché reinventare la ruota quando PHP ce l’ha già pronta?''\}
\textbackslash{}end\{center\}
\textbackslash{}end\{frame\}

\% -----------------------------------
\% SLIDE 2: Librerie e Funzioni
\% -----------------------------------
\textbackslash{}begin\{frame\}\{Il tesoro nascosto: le librerie PHP\}
Immaginate di dover scrivere da zero il codice per ordinare una lista di nomi o per calcolare quanti giorni mancano al vostro compleanno. Sarebbe lungo e complicato.

\textbackslash{}vspace\{0.3cm\}

PHP mette a disposizione \textbackslash{}textbf\{migliaia di funzioni\} pronte all’uso per:
\textbackslash{}begin\{itemize\}
\textbackslash{}item Gestire variabili e verificarne l’esistenza
\textbackslash{}item Manipolare array
\textbackslash{}item Elaborare stringhe di testo
\textbackslash{}item Eseguire calcoli matematici
\textbackslash{}item Gestire date e orari
\textbackslash{}item Interagire con file, database e immagini
\textbackslash{}end\{itemize\}

\textbackslash{}vspace\{0.3cm\}

Il nostro compito è imparare a conoscerle e usarle correttamente.
\textbackslash{}end\{frame\}

\% -----------------------------------
\% SLIDE 3: Variabili
\% -----------------------------------
\textbackslash{}begin\{frame\}\{Tenere sotto controllo le variabili\}
PHP fornisce alcune funzioni fondamentali:

\textbackslash{}begin\{itemize\}
\textbackslash{}item \textbackslash{}texttt\{isset(\textbackslash{}\$var)\} – verifica se la variabile esiste
\textbackslash{}item \textbackslash{}texttt\{empty(\textbackslash{}\$var)\} – verifica se è vuota
\textbackslash{}item \textbackslash{}texttt\{unset(\textbackslash{}\$var)\} – elimina la variabile
\textbackslash{}end\{itemize\}

\textbackslash{}begin\{lstlisting\}
\$nome = ``Mario'';
if (isset(\$nome)) \{
echo ``Benvenuto, \$nome!'';
\} else \{
echo ``Nome non definito!'';
\}
\textbackslash{}end\{lstlisting\}
\textbackslash{}end\{frame\}

\% -----------------------------------
\% SLIDE 4: Operatore ??
\% -----------------------------------
\textbackslash{}begin\{frame\}\{L’operatore ?? (null coalescing)\}
Prima di PHP 7:

\textbackslash{}begin\{lstlisting\}
\$utente = isset(\$\_GET['nome']) ? \$\_GET['nome'] : 'Ospite';
\textbackslash{}end\{lstlisting\}

Con PHP 7 e successivi:

\textbackslash{}begin\{lstlisting\}
\$utente = \$\_GET['nome'] ?? 'Ospite';
\textbackslash{}end\{lstlisting\}
\textbackslash{}end\{frame\}

\% -----------------------------------
\% SLIDE 5: Array
\% -----------------------------------
\textbackslash{}begin\{frame\}\{Gli array\}
Un array è un contenitore che può memorizzare più valori.

\textbackslash{}begin\{itemize\}
\textbackslash{}item \textbackslash{}texttt\{count()\}
\textbackslash{}item \textbackslash{}texttt\{array\textbackslash{}\_push()\}
\textbackslash{}item \textbackslash{}texttt\{sort()\}
\textbackslash{}item \textbackslash{}texttt\{array\textbackslash{}\_keys()\}, \textbackslash{}texttt\{array\textbackslash{}\_values()\}
\textbackslash{}end\{itemize\}
\textbackslash{}end\{frame\}

\% -----------------------------------
\% SLIDE 6: Array – Esempio
\% -----------------------------------
\textbackslash{}begin\{frame\}\{Esempio di array\}
\textbackslash{}begin\{lstlisting\}
\$frutti = [``mela'', ``banana'', ``arancia''];
array\_push(\$frutti, ``kiwi'');
sort(\$frutti);
print\_r(\$frutti);
\textbackslash{}end\{lstlisting\}
\textbackslash{}end\{frame\}

\% -----------------------------------
\% SLIDE 7: Array associativi
\% -----------------------------------
\textbackslash{}begin\{frame\}\{Array associativi\}
\textbackslash{}begin\{lstlisting\}
\$persona = [
``nome'' => ``Luca'',
``eta'' => 17,
``classe'' => ``5B'',
``citta'' => ``Ascoli Piceno''
];

echo ``Ciao '' . \$persona[``nome''] . ``!'';
\textbackslash{}end\{lstlisting\}
\textbackslash{}end\{frame\}

\% -----------------------------------
\% SLIDE 8: Date e ora
\% -----------------------------------
\textbackslash{}begin\{frame\}\{Gestire date e orari\}
\textbackslash{}begin\{itemize\}
\textbackslash{}item \textbackslash{}texttt\{date()\}
\textbackslash{}item \textbackslash{}texttt\{time()\}
\textbackslash{}item \textbackslash{}texttt\{strtotime()\}
\textbackslash{}end\{itemize\}
\textbackslash{}end\{frame\}

\% -----------------------------------
\% SLIDE 9: Date – Esempi
\% -----------------------------------
\textbackslash{}begin\{frame\}\{Esempi di formattazione\}
\textbackslash{}begin\{lstlisting\}
echo date(``d/m/Y'');
echo date(``H:i:s'');
\textbackslash{}end\{lstlisting\}
\textbackslash{}end\{frame\}

\% -----------------------------------
\% SLIDE 10: Calcolo età
\% -----------------------------------
\textbackslash{}begin\{frame\}\{Calcolo dell’età\}
\textbackslash{}begin\{lstlisting\}
\$data\_nascita = ``2008-05-20'';
\$oggi = new DateTime();
\$nascita = new DateTime(\$data\_nascita);
\$eta = \$oggi->diff(\$nascita)->y;
echo ``Hai \$eta anni.'';
\textbackslash{}end\{lstlisting\}
\textbackslash{}end\{frame\}

\% -----------------------------------
\% SLIDE 11: Stringhe
\% -----------------------------------
\textbackslash{}begin\{frame\}\{Funzioni per le stringhe\}
\textbackslash{}begin\{itemize\}
\textbackslash{}item \textbackslash{}texttt\{strlen()\}
\textbackslash{}item \textbackslash{}texttt\{strpos()\}
\textbackslash{}item \textbackslash{}texttt\{str\textbackslash{}\_replace()\}
\textbackslash{}item \textbackslash{}texttt\{strtolower()\}, \textbackslash{}texttt\{strtoupper()\}
\textbackslash{}item \textbackslash{}texttt\{trim()\}
\textbackslash{}item \textbackslash{}texttt\{substr()\}
\textbackslash{}end\{itemize\}
\textbackslash{}end\{frame\}

\% -----------------------------------
\% SLIDE 12: Stringhe – Esempio
\% -----------------------------------
\textbackslash{}begin\{frame\}\{Pulizia del testo\}
\textbackslash{}begin\{lstlisting\}
\$testo = ``  Benvenuti su PHP!  '';
\$testo = trim(\$testo);
\$testo = strtolower(\$testo);
echo strlen(\$testo);
\textbackslash{}end\{lstlisting\}
\textbackslash{}end\{frame\}

\% -----------------------------------
\% SLIDE 13: Sicurezza
\% -----------------------------------
\textbackslash{}begin\{frame\}\{Protezione da XSS\}
\textbackslash{}begin\{lstlisting\}
\$input = ``<script>alert('Attacco');</script>'';
echo htmlspecialchars(\$input);
\textbackslash{}end\{lstlisting\}

\textbackslash{}begin\{alertblock\}\{Regola fondamentale\}
Non fidarti mai dell’input dell’utente.
\textbackslash{}end\{alertblock\}
\textbackslash{}end\{frame\}

\% -----------------------------------
\% SLIDE 14: Concatenazione
\% -----------------------------------
\textbackslash{}begin\{frame\}\{Concatenazione di stringhe\}
\textbackslash{}begin\{lstlisting\}
\$nome = ``Giulia'';
echo ``Ciao, '' . \$nome . ``!'';
echo ``Ciao, \$nome!'';
\textbackslash{}end\{lstlisting\}
\textbackslash{}end\{frame\}

\% -----------------------------------
\% SLIDE 15: Numeri
\% -----------------------------------
\textbackslash{}begin\{frame\}\{Funzioni matematiche\}
\textbackslash{}begin\{itemize\}
\textbackslash{}item \textbackslash{}texttt\{round()\}
\textbackslash{}item \textbackslash{}texttt\{floor()\}
\textbackslash{}item \textbackslash{}texttt\{ceil()\}
\textbackslash{}item \textbackslash{}texttt\{rand()\}
\textbackslash{}item \textbackslash{}texttt\{sqrt()\}
\textbackslash{}item \textbackslash{}texttt\{number\textbackslash{}\_format()\}
\textbackslash{}end\{itemize\}
\textbackslash{}end\{frame\}

\% -----------------------------------
\% SLIDE 16: Numeri – Esempio
\% -----------------------------------
\textbackslash{}begin\{frame\}\{Esempio numerico\}
\textbackslash{}begin\{lstlisting\}
\$valore = 3.789;
echo number\_format(\$valore, 2, ',', '.');
\textbackslash{}end\{lstlisting\}
\textbackslash{}end\{frame\}

\% -----------------------------------
\% SLIDE 17: Conversione tipi
\% -----------------------------------
\textbackslash{}begin\{frame\}\{Conversione dei tipi\}
\textbackslash{}begin\{lstlisting\}
\$numero = ``123'';
\$intero = (int)\$numero;
echo \$intero + 10;
\textbackslash{}end\{lstlisting\}
\textbackslash{}end\{frame\}

\% -----------------------------------
\% SLIDE 18: Esercizio
\% -----------------------------------
\textbackslash{}begin\{frame\}\{Esercizio completo\}
\textbackslash{}begin\{lstlisting\}
\$nome = \$\_GET['nome'] ?? 'Anonimo';
\$anno = \$\_GET['anno'] ?? 2008;
\$eta = date('Y') - \$anno;
echo ``Ciao \$nome, hai circa \$eta anni.'';
\textbackslash{}end\{lstlisting\}
\textbackslash{}end\{frame\}

\% -----------------------------------
\% SLIDE 19: Conclusione
\% -----------------------------------
\textbackslash{}begin\{frame\}\{Conclusione\}
\textbackslash{}begin\{itemize\}
\textbackslash{}item Variabili
\textbackslash{}item Array
\textbackslash{}item Stringhe
\textbackslash{}item Date
\textbackslash{}item Sicurezza
\textbackslash{}end\{itemize\}

\textbackslash{}begin\{center\}
\textbackslash{}Large Pratica costante = competenza reale
\textbackslash{}end\{center\}
\textbackslash{}end\{frame\}

\% -----------------------------------
\% SLIDE 20: Risorse
\% -----------------------------------
\textbackslash{}begin\{frame\}\{Risorse utili\}
\textbackslash{}begin\{itemize\}
\textbackslash{}item \textbackslash{}url\{https://www.php.net/manual/it/\}
\textbackslash{}item \textbackslash{}url\{https://www.w3schools.com/php/\}
\textbackslash{}item \textbackslash{}url\{https://phptherightway.com/\}
\textbackslash{}end\{itemize\}
\textbackslash{}end\{frame\}

\textbackslash{}end\{document\}
