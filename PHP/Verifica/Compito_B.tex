\documentclass[a4paper,12pt]{article}
\usepackage[italian]{babel}
\usepackage[utf8]{inputenc}
\usepackage[T1]{fontenc}
\usepackage{geometry}
\usepackage{setspace}
\geometry{margin=2.5cm}
\onehalfspacing
\title{Verifica di TPSIT - 4AIQ - Compito B} 
\author{Prof. Fedeli Massimo}
\date{23/02/26}
\begin{document}
	
	\maketitle
	
	\section*{Contesto}
	
	Hai a disposizione il seguente schema logico svolgi l'esercizio 1 o l'esercizio 2 a tua scelta:
	
	\begin{verbatim}
		BIBLIOTECHE(id PK, nome, citta)
		UTENTI(id PK, cognome, nome, email, tessera UNIQUE, data_iscrizione)
		LIBRI(id PK, isbn UNIQUE, titolo, autore, anno, collocazione, biblioteca_id FK)
		PRESTITI(id PK, utente_id FK, libro_id FK, data_prestito,
		data_scadenza, data_restituzione)
	\end{verbatim}
	
	\section*{Esercizio 1}
	
	\subsection*{Obiettivo}
	
	Realizzare una piccola applicazione web che consenta di inserire un nuovo utente
	nella tabella \texttt{UTENTI} tramite una pagina web (client) e un endpoint PHP
	(backend) che salvi i dati su MySQL.
	
	\subsection*{Vincoli e requisiti tecnici}
	
	\begin{itemize}
		\item Frontend: una pagina \texttt{index.html} con un form e uno script
		JavaScript. \textbf{3 pt}
		\item Backend: un file PHP (esempio: \texttt{api\_utenti\_create.php}) che
		riceva i dati e inserisca la riga nel database. \textbf{3 pt}
	\end{itemize}
	
	Requisiti:
	\begin{itemize}
		\item Comunicazione: il client deve inviare i dati al backend usando
		\texttt{fetch} in POST, in formato JSON.
		\item Sicurezza: il backend deve usare query parametrizzate (PDO con prepared
		statements). 
	\end{itemize}
	
	\section*{Esercizio 2}
	
	\subsection*{Obiettivo}
	
	Realizzare una piccola applicazione web che consenta di visualizzare l'elenco dei
	prestiti di tutti gli utenti, includendo anche le informazioni minime dell'utente
	e del libro.
	
	\subsection*{Vincoli e requisiti tecnici}
	
	\begin{itemize}
		\item Frontend: una pagina HTML (può essere la stessa \texttt{index.html} o
		una pagina separata, ad esempio \texttt{prestiti.html}) che mostri i
		risultati in una tabella e uno script JavaScript. \textbf{4 pt}
		\item Backend: un file PHP (esempio: \texttt{api\_prestiti\_list.php}) che
		legga i dati dal database e restituisca un JSON. \textbf{4 pt}
	\end{itemize}
	
	Requisiti:
	\begin{itemize}
		\item Comunicazione: il client deve richiedere i dati al backend usando
		\texttt{fetch} ricevere dati in formato SON.
		\item Query: il backend deve eseguire una SELECT con JOIN tra:
		\begin{itemize}
			\item \texttt{PRESTITI} e \texttt{UTENTI}
			\item \texttt{PRESTITI} e \texttt{LIBRI}
			\item \texttt{LIBRI} e \texttt{BIBLIOTECHE}
		\end{itemize}
		\item Sicurezza: il backend deve usare PDO e prepared statements anche per
		eventuali parametri di filtro.
	\end{itemize}
	
	\subsection*{Ottimizzazione}
	
	Filtrare l'elenco per intervallo di date del prestito usando una select popolata
	tramite script PHP. \textbf{2 pt}
	
	\begin{itemize}
		\item Il frontend deve includere due campi data (\texttt{data\_da},
		\texttt{data\_a}) e un pulsante di ricerca.
		\item Il backend deve validare le date e applicare il filtro nella clausola
		WHERE tramite parametri bindati.
		\item Se i parametri sono assenti, devono essere restituiti tutti i prestiti.
	\end{itemize}
	
\end{document}