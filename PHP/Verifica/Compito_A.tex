\documentclass[a4paper,12pt]{article}
\usepackage[italian]{babel}
\usepackage[utf8]{inputenc}
\usepackage[T1]{fontenc}
\usepackage{geometry}
\usepackage{setspace}
\geometry{margin=2.5cm}
\onehalfspacing
\title{Verifica di TPSIT -  Classe 4AIQ - Compito A}
\author{Prof. Fedeli Massimo}
\date{23/02/2026}
\begin{document}
	
	\maketitle
	
	\section*{Contesto}
	
Dato il seguente schema di database svolgi a tua scelta l'esercizio 1 o l'esercizio 2
	
	\begin{verbatim}
		ABBBONAMENTI(id PK, nome, tipo, durata_giorni, ingressi)
		CLIENTI(id PK, cognome, nome, indirizzo, telefono, codice_fiscale,
		numero_tessera, email, fototessera, data_creazione,
		data_nascita, luogo_nascita, citta, cap, provincia, note)
		CORSI(id PK, nome, descrizione, codice_tornello)
		ISCRIZIONI(id PK, cliente_id FK, data_iscrizione,
		data_scadenza_iscrizione, certificato_medico,
		data_scadenza_certificato, file_certificato,
		corso_id FK, limitazioni_accesso)
	\end{verbatim}
	
	\section*{Esercizio 1}
	
	\subsection*{Obiettivo}
	
	Realizzare una piccola applicazione web che consenta di inserire un nuovo cliente nella tabella \texttt{CLIENTI} tramite una pagina html (client) e un endpoint PHP (backend) che salvi i dati su MySQL.
	
	\subsection*{Vincoli e requisiti tecnici}
	
	\begin{itemize}
		\item Frontend: una pagina index.html con un form e uno script JavaScript. \textbf{3 pt}
		\item Backend: un file PHP (esempio: \texttt{api\_clienti\_create.php}) che riceve i dati e inserisce la riga nel database. \textbf{3 pt}
		\item NB. Comunicazione: il client deve inviare i dati al backend usando fetch in POST, in formato JSON.
		\item NB. Sicurezza: il backend deve usare query parametrizzate (PDO con prepared statements). Vietato concatenare stringhe in SQL.
	\end{itemize}
	
	
	\section*{Esercizio 2}
	
	\subsection*{Obiettivo}
	
	Realizzare una piccola applicazione web che consenta di visualizzare l'elenco delle iscrizioni di tutti i clienti, includendo anche le informazioni anagrafiche minime del cliente e, se presente, il corso associato.
	
	\subsection*{Vincoli e requisiti tecnici}
	
	\begin{itemize}
		\item Frontend: una pagina HTML (può essere la stessa \texttt{index.html} o una pagina separata, ad esempio \texttt{iscrizioni.html}) che mostri i risultati in una tabella e uno script JavaScript.  \textbf{4 pt}
		\item Backend: un file PHP (esempio: \texttt{api\_iscrizioni\_list.php}) che legga i dati dal database e restituisca un JSON. \textbf{4 pt}
		\item N.B. Comunicazione: il client deve richiedere i dati al backend usando \texttt{fetch}  ricevere i dati in formato JSON.
		\item N.B. Query: il backend deve eseguire una SELECT con JOIN
		\item Sicurezza: il backend deve usare PDO e prepared statements anche per eventuali parametri di filtro.
	\end{itemize}
	
	\subsection*{Ottimizzazione}
	
	Filtrare l'elenco per data di iscrizione e per tipologia di corso usando una select popolata tramite script php. \textbf{2 pt}
	
	\begin{itemize}
		\item Il frontend deve includere uno o due campi data (es. \texttt{data\_da}, \texttt{data\_a}) e un pulsante di ricerca.
		\item Il backend deve validare le date e applicare il filtro nella clausola WHERE tramite parametri bindati.
		\item Se i parametri sono assenti, devono essere restituite tutte le iscrizioni.
	\end{itemize}
	
\end{document}