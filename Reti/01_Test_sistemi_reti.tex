\documentclass[a4paper,11pt]{article}
\usepackage[utf8]{inputenc}
\usepackage[italian]{babel}
\usepackage{geometry}
\usepackage{fancyhdr}
\usepackage{amsmath}
\usepackage{graphicx}
\usepackage{enumitem}
\usepackage{array}
\usepackage{multicol}

\geometry{a4paper, margin=2cm}

\pagestyle{fancy}
\fancyhf{}
\fancyhead[L]{Test Fondamenti di Reti - IIS Fermi Sacconi Ceci}
\fancyhead[R]{A.S. 2025/26}
\fancyfoot[C]{\thepage\ di \pageref{LastPage}}

\title{\textbf{Test a Risposta Multipla\\Fondamenti di Reti}}
\author{Prof. Fedeli Massimo - IIS Fermi Sacconi Ceci}
\date{A.S. 2025/26}

\begin{document}

\maketitle
\thispagestyle{empty}

\vspace{1cm}

\noindent\textbf{Nome e Cognome:} \underline{\hspace{8cm}}\\[0.5cm]
\noindent\textbf{Classe:} \underline{\hspace{3cm}} \textbf{Data:} \underline{\hspace{3cm}}

\vspace{1cm}

\noindent\textbf{Istruzioni:}
\begin{itemize}
    \item Il test contiene 70 domande a risposta multipla
    \item Per ogni domanda, segnare una sola risposta
    \item Utilizzare il riquadro in fondo per riportare le risposte
    \item Tempo a disposizione: 90 minuti
\end{itemize}

\newpage

\section{Reti di Calcolatori e Concetti Generali}

\begin{enumerate}

\item Quale delle seguenti affermazioni descrive meglio una rete di calcolatori?
\begin{enumerate}[label=\alph*)]
    \item Un insieme di computer che condividono lo stesso sistema operativo
    \item Un software distribuito per la gestione di database
    \item Sistemi interconnessi che comunicano tramite canali di trasmissione
    \item Dispositivi che utilizzano esclusivamente il protocollo TCP/IP
\end{enumerate}

\item I principali vantaggi delle reti di calcolatori includono:
\begin{enumerate}[label=\alph*)]
    \item Maggiore consumo energetico e necessità di personale specializzato
    \item Solo la possibilità di navigare su Internet
    \item Eliminazione completa dei rischi di sicurezza
    \item Condivisione di risorse, affidabilità e scalabilità
\end{enumerate}

\item Una rete MAN (Metropolitan Area Network) copre tipicamente:
\begin{enumerate}[label=\alph*)]
    \item Un'area geografica di circa 10 metri (una stanza)
    \item Un'area urbana o metropolitana (10-100 km)
    \item Un singolo edificio o campus (fino a 1 km)
    \item Continenti diversi (oltre 1000 km)
\end{enumerate}

\item In una topologia a stella:
\begin{enumerate}[label=\alph*)]
    \item Ogni nodo è collegato a un nodo centrale (hub o switch)
    \item Tutti i nodi condividono un unico cavo bus
    \item I nodi sono collegati in cerchio senza gerarchia
    \item Ogni nodo ha connessione diretta con tutti gli altri nodi
\end{enumerate}

\item Qual è lo svantaggio principale della topologia a bus?
\begin{enumerate}[label=\alph*)]
    \item Richiede troppi cavi rispetto alle altre topologie
    \item Il costo degli switch centrali è proibitivo
    \item Se il cavo principale si guasta, tutta la rete smette di funzionare
    \item La velocità di trasmissione è troppo bassa
\end{enumerate}

\item In una topologia ad anello (ring):
\begin{enumerate}[label=\alph*)]
    \item Esiste sempre un nodo centrale che coordina il traffico
    \item I dati viaggiano in entrambe le direzioni simultaneamente
    \item Ogni nodo è collegato a esattamente due nodi adiacenti formando un circuito chiuso
    \item Tutti i nodi trasmettono contemporaneamente senza collisioni
\end{enumerate}

\item Cosa si intende per "larghezza di banda" (bandwidth) in una rete?
\begin{enumerate}[label=\alph*)]
    \item Il numero massimo di utenti che possono connettersi
    \item La distanza fisica massima coperta dalla rete
    \item La capacità di trasmissione dati, misurata in bit/secondo
    \item La lunghezza totale dei cavi utilizzati
\end{enumerate}

\item Un hub in una rete locale:
\begin{enumerate}[label=\alph*)]
    \item Filtra i pacchetti basandosi sulla tabella di routing
    \item Memorizza gli indirizzi MAC e commuta i frame intelligentemente
    \item Cripta automaticamente tutto il traffico di rete
    \item Replica il segnale ricevuto su tutte le porte (broadcasting)
\end{enumerate}

\item Quale dispositivo opera a livello 2 (Data Link) del modello OSI?
\begin{enumerate}[label=\alph*)]
    \item Router
    \item Gateway applicativo
    \item Switch
    \item Repeater
\end{enumerate}

\item La latenza in una rete indica:
\begin{enumerate}[label=\alph*)]
    \item La quantità di dati trasmessi al secondo
    \item Il ritardo temporale nella trasmissione dei dati
    \item Il rapporto tra pacchetti persi e pacchetti inviati
    \item La capacità massima del canale di comunicazione
\end{enumerate}

\end{enumerate}

\section{Internet e Architetture di Rete}

\begin{enumerate}[resume]

\item La principale caratteristica di Internet è:
\begin{enumerate}[label=\alph*)]
    \item Utilizzare esclusivamente connessioni in fibra ottica
    \item Essere gestita da un'unica organizzazione mondiale
    \item Garantire sempre la massima velocità di trasmissione
    \item Interconnettere reti eterogenee usando protocolli standard
\end{enumerate}

\item Il termine "backbone" in Internet si riferisce a:
\begin{enumerate}[label=\alph*)]
    \item Le dorsali ad alta velocità che interconnettono le reti principali
    \item I server DNS radice distribuiti nel mondo
    \item I computer client degli utenti finali
    \item I firewall di protezione delle reti aziendali
\end{enumerate}

\item Nella commutazione di circuito:
\begin{enumerate}[label=\alph*)]
    \item Ogni pacchetto può seguire un percorso diverso
    \item I dati sono divisi in pacchetti con header indipendenti
    \item Il percorso viene stabilito prima della trasmissione e rimane dedicato
    \item Non è necessario prenotare risorse di rete
\end{enumerate}

\item Un vantaggio della commutazione di pacchetto rispetto a quella di circuito è:
\begin{enumerate}[label=\alph*)]
    \item Garanzia assoluta della banda per ogni comunicazione
    \item Latenza sempre costante e prevedibile
    \item Maggiore efficienza nell'uso delle risorse di rete
    \item Impossibilità di congestione nei router
\end{enumerate}

\item Cosa contiene l'intestazione (header) di un pacchetto IP?
\begin{enumerate}[label=\alph*)]
    \item Solo i dati utili da trasmettere
    \item Esclusivamente l'indirizzo del destinatario
    \item La password di autenticazione dell'utente
    \item Indirizzi IP sorgente e destinazione, TTL, checksum, ecc.
\end{enumerate}

\item Il campo TTL (Time To Live) in un pacchetto IP serve a:
\begin{enumerate}[label=\alph*)]
    \item Indicare la velocità di trasmissione richiesta
    \item Evitare che i pacchetti circolino indefinitamente nella rete
    \item Stabilire la priorità del pacchetto rispetto ad altri
    \item Memorizzare il tempo di creazione del pacchetto
\end{enumerate}

\item In Internet, il termine "hop" indica:
\begin{enumerate}[label=\alph*)]
    \item Il numero di byte trasmessi in un secondo
    \item Un errore di trasmissione che richiede ritrasmissione
    \item Il passaggio di un pacchetto attraverso un router
    \item La distanza fisica tra due dispositivi
\end{enumerate}

\item Quando un router riceve un pacchetto:
\begin{enumerate}[label=\alph*)]
    \item Lo memorizza permanentemente nel disco
    \item Lo cripta per sicurezza prima di inoltrarlo
    \item Consulta la tabella di routing e lo inoltra verso la destinazione
    \item Lo replica su tutte le interfacce di uscita
\end{enumerate}

\item La funzione principale di un gateway è:
\begin{enumerate}[label=\alph*)]
    \item Interconnettere reti con protocolli diversi (livello applicativo)
    \item Amplificare il segnale elettrico su lunghe distanze
    \item Filtrare automaticamente lo spam e i virus
    \item Assegnare indirizzi IP dinamici ai client
\end{enumerate}

\item Cosa si intende per "best effort" nel servizio offerto da IP?
\begin{enumerate}[label=\alph*)]
    \item Garanzia assoluta di consegna entro un tempo massimo
    \item Il protocollo cerca di consegnare i pacchetti ma non garantisce successo
    \item Priorità massima assegnata a tutti i pacchetti di rete
    \item Cifratura obbligatoria di tutti i dati trasmessi
\end{enumerate}

\end{enumerate}

\section{Architetture a Livelli e Protocolli}

\begin{enumerate}[resume]

\item Il vantaggio principale dell'architettura a livelli è:
\begin{enumerate}[label=\alph*)]
    \item Maggiore velocità di trasmissione dei dati
    \item Riduzione dei costi hardware necessari
    \item Eliminazione completa degli errori di rete
    \item Modularità e indipendenza tra i livelli
\end{enumerate}

\item Nel modello OSI, il livello di presentazione si occupa di:
\begin{enumerate}[label=\alph*)]
    \item Instradamento dei pacchetti tra sottoreti
    \item Formattazione, codifica e crittografia dei dati
    \item Gestione fisica del mezzo trasmissivo
    \item Controllo del flusso tra applicazioni
\end{enumerate}

\item Il livello di sessione nel modello OSI gestisce:
\begin{enumerate}[label=\alph*)]
    \item La conversione dei bit in segnali elettrici
    \item L'apertura, gestione e chiusura delle connessioni tra applicazioni
    \item L'assegnazione degli indirizzi IP agli host
    \item L'instradamento dei pacchetti tra reti diverse
\end{enumerate}

\item Quanti livelli dell'OSI sono presenti nell'architettura TCP/IP?
\begin{enumerate}[label=\alph*)]
    \item Tutti e 7 i livelli dell'OSI
    \item 3 livelli (applicazione, rete, fisico)
    \item 4 livelli principali (applicazione, trasporto, rete, collegamento)
    \item 5 livelli (include anche presentazione e sessione)
\end{enumerate}

\item Il protocollo TCP opera a quale livello?
\begin{enumerate}[label=\alph*)]
    \item Livello applicativo (insieme a HTTP e FTP)
    \item Livello di rete (insieme a IP e ICMP)
    \item Livello di trasporto
    \item Livello fisico e data link
\end{enumerate}

\item La principale differenza tra TCP e UDP è:
\begin{enumerate}[label=\alph*)]
    \item UDP è più lento ma più affidabile di TCP
    \item TCP è orientato alla connessione e affidabile, UDP no
    \item TCP non garantisce l'ordine dei pacchetti, UDP sì
    \item Non esiste differenza sostanziale, sono intercambiabili
\end{enumerate}

\item Il protocollo UDP:
\begin{enumerate}[label=\alph*)]
    \item Garantisce sempre la consegna ordinata dei dati
    \item Richiede una fase di handshake a tre vie
    \item È più leggero e veloce ma non garantisce affidabilità
    \item Cripta automaticamente tutti i dati trasmessi
\end{enumerate}

\item Il three-way handshake in TCP serve per:
\begin{enumerate}[label=\alph*)]
    \item Chiudere correttamente una connessione esistente
    \item Crittografare la comunicazione end-to-end
    \item Stabilire una connessione affidabile tra client e server
    \item Frammentare i dati in pacchetti di dimensione fissa
\end{enumerate}

\item L'incapsulamento dei dati nel modello a livelli significa:
\begin{enumerate}[label=\alph*)]
    \item Ogni livello aggiunge il proprio header ai dati del livello superiore
    \item I dati vengono compressi per ridurne la dimensione
    \item I dati vengono eliminati se contengono errori
    \item I dati vengono divisi in parti uguali
\end{enumerate}

\item Il numero di porta in TCP/UDP identifica:
\begin{enumerate}[label=\alph*)]
    \item L'indirizzo fisico della scheda di rete
    \item Il router da utilizzare per l'instradamento
    \item L'applicazione o servizio specifico sull'host
    \item Il tipo di cavo utilizzato per la connessione
\end{enumerate}

\end{enumerate}

\section{Indirizzi IP}

\begin{enumerate}[resume]

\item Qual è la dimensione di un indirizzo IPv6?
\begin{enumerate}[label=\alph*)]
    \item 32 bit (come IPv4)
    \item 64 bit (doppio di IPv4)
    \item 128 bit
    \item 256 bit
\end{enumerate}

\item Un indirizzo IPv4 192.168.1.1 appartiene alla classe:
\begin{enumerate}[label=\alph*)]
    \item Classe A (primo otteto 0-127)
    \item Classe B (primo otteto 128-191)
    \item Classe D (multicast)
    \item Classe C (primo otteto 192-223)
\end{enumerate}

\item L'indirizzo 10.0.0.1 è:
\begin{enumerate}[label=\alph*)]
    \item Un indirizzo pubblico instradabile su Internet
    \item Un indirizzo multicast per gruppi
    \item Un indirizzo privato di classe A (non instradabile)
    \item Un indirizzo di loopback riservato
\end{enumerate}

\item Gli indirizzi della classe E (240-255.X.X.X) sono:
\begin{enumerate}[label=\alph*)]
    \item Riservati per usi sperimentali e futuri
    \item Utilizzati per reti aziendali di grandi dimensioni
    \item Dedicati al broadcasting globale
    \item Assegnati alle reti domestiche
\end{enumerate}

\item L'indirizzo 127.0.0.1 è noto come:
\begin{enumerate}[label=\alph*)]
    \item Gateway predefinito della rete locale
    \item Indirizzo di broadcast per la subnet
    \item Loopback o localhost (interfaccia virtuale locale)
    \item Server DNS primario del sistema
\end{enumerate}

\item Qual è il range di indirizzi privati di classe B?
\begin{enumerate}[label=\alph*)]
    \item 10.0.0.0 - 10.255.255.255
    \item 192.168.0.0 - 192.168.255.255
    \item 172.16.0.0 - 172.31.255.255
    \item 169.254.0.0 - 169.254.255.255
\end{enumerate}

\item Il NAT (Network Address Translation):
\begin{enumerate}[label=\alph*)]
    \item Cripta i dati per sicurezza nella trasmissione
    \item Converte indirizzi privati in pubblici e viceversa
    \item Assegna automaticamente nomi di dominio agli host
    \item Instrada i pacchetti tra diverse sottoreti
\end{enumerate}

\item Quanti indirizzi può teoricamente indirizzare IPv4?
\begin{enumerate}[label=\alph*)]
    \item Circa 16 milioni (2$^{24}$)
    \item Circa 4 miliardi (2$^{32}$)
    \item Circa 65.000 (2$^{16}$)
    \item Infiniti con il CIDR
\end{enumerate}

\item L'indirizzo MAC (Media Access Control) è:
\begin{enumerate}[label=\alph*)]
    \item Un indirizzo IP di classe speciale
    \item Un protocollo di rete di livello 3
    \item L'indirizzo fisico univoco della scheda di rete (48 bit)
    \item Un sistema operativo per Apple
\end{enumerate}

\item La differenza principale tra indirizzo IP e MAC è:
\begin{enumerate}[label=\alph*)]
    \item L'IP è logico e modificabile, il MAC è fisico e permanente
    \item Il MAC è più lungo dell'IP in termini di bit
    \item Non esiste differenza, sono sinonimi
    \item L'IP è permanente, il MAC cambia dinamicamente
\end{enumerate}

\end{enumerate}

\section{Subnetting e Notazioni}

\begin{enumerate}[resume]

\item Una netmask 255.255.255.128 corrisponde a:
\begin{enumerate}[label=\alph*)]
    \item /24 (classe C standard)
    \item /25 (un bit di subnet)
    \item /26 (due bit di subnet)
    \item /23 (supernetting)
\end{enumerate}

\item Con una netmask /28, quanti host utilizzabili possiamo avere?
\begin{enumerate}[label=\alph*)]
    \item 16 (tutti gli indirizzi)
    \item 14 (escludendo rete e broadcast)
    \item 30 (come una /27)
    \item 32 (confondendo con /27)
\end{enumerate}

\item L'indirizzo di broadcast di una rete 192.168.1.0/24 è:
\begin{enumerate}[label=\alph*)]
    \item 192.168.1.0 (indirizzo di rete)
    \item 192.168.1.1 (primo host)
    \item 192.168.1.254 (ultimo host utilizzabile)
    \item 192.168.1.255 (tutti i bit host a 1)
\end{enumerate}

\item Dato l'indirizzo 172.16.50.100 con netmask 255.255.255.240 (/28), l'indirizzo di rete è:
\begin{enumerate}[label=\alph*)]
    \item 172.16.50.0 (ignorando la netmask)
    \item 172.16.50.96 (100 AND 240 = 96)
    \item 172.16.50.64 (subnet precedente)
    \item 172.16.50.112 (subnet successiva)
\end{enumerate}

\item Una rete /26 può ospitare al massimo:
\begin{enumerate}[label=\alph*)]
    \item 64 host (tutti gli indirizzi)
    \item 126 host (come una /25)
    \item 62 host (2$^6$ - 2)
    \item 32 host (metà di 64)
\end{enumerate}

\item Per dividere una rete di classe C in 4 sottoreti uguali, quale netmask si usa?
\begin{enumerate}[label=\alph*)]
    \item 255.255.255.128 (/25, crea 2 subnet)
    \item 255.255.255.192 (/26, crea 4 subnet)
    \item 255.255.255.224 (/27, crea 8 subnet)
    \item 255.255.255.240 (/28, crea 16 subnet)
\end{enumerate}

\item Dato l'IP 10.20.30.45/27, quale NON è un indirizzo valido nella stessa subnet?
\begin{enumerate}[label=\alph*)]
    \item 10.20.30.33 (nella subnet 10.20.30.32/27)
    \item 10.20.30.40 (nella subnet 10.20.30.32/27)
    \item 10.20.30.62 (nella subnet 10.20.30.32/27)
    \item 10.20.30.50 (nella subnet 10.20.30.64/27)
\end{enumerate}

\item Il CIDR (Classless Inter-Domain Routing) permette di:
\begin{enumerate}[label=\alph*)]
    \item Usare esclusivamente le classi A, B, C tradizionali
    \item Allocare indirizzi in modo flessibile senza vincoli di classe
    \item Eliminare completamente il routing su Internet
    \item Convertire automaticamente IPv4 in IPv6
\end{enumerate}

\item Una rete 192.168.10.0/29 ha:
\begin{enumerate}[label=\alph*)]
    \item 8 indirizzi totali, 6 host utilizzabili (2$^3$ - 2)
    \item 16 indirizzi totali, 14 host utilizzabili
    \item 32 indirizzi totali, 30 host utilizzabili
    \item 4 indirizzi totali, 2 host utilizzabili
\end{enumerate}

\item Per calcolare il numero di host utilizzabili data una netmask /n, si usa la formula:
\begin{enumerate}[label=\alph*)]
    \item 2$^n$ - 2 (confondendo n con i bit host)
    \item 2$^{32}$ - n - 2
    \item 2$^{(32-n)}$ - 2 (bit host = 32 - n)
    \item n$^2$ - 2
\end{enumerate}

\item Gli indirizzi APIPA (169.254.x.x) vengono assegnati automaticamente quando:
\begin{enumerate}[label=\alph*)]
    \item Il server DHCP non è disponibile o non risponde
    \item La velocità della rete supera 1 Gbps
    \item Si utilizza esclusivamente IPv6
    \item Il router principale è configurato correttamente
\end{enumerate}

\item Una subnet mask 255.255.248.0 corrisponde a:
\begin{enumerate}[label=\alph*)]
    \item /19 (240 = 11110000 in binario, errore comune)
    \item /20 (248 = 11111000, quindi 8+8+5 = 21, errore)
    \item /21 (248 = 11111000, quindi 8+8+5 = 21 bit a 1)
    \item /22 (252 = 11111100, non 248)
\end{enumerate}

\item Se vogliamo creare almeno 10 sottoreti da una rete di classe C, quanti bit dobbiamo "prendere in prestito" dalla parte host?
\begin{enumerate}[label=\alph*)]
    \item 2 bit (creano solo 4 subnet, insufficiente)
    \item 3 bit (creano solo 8 subnet, insufficiente)
    \item 4 bit (creano 16 subnet, sufficienti)
    \item 5 bit (creano 32 subnet, eccessive ma valide)
\end{enumerate}

\item Due host con IP 192.168.5.50/26 e 192.168.5.100/26:
\begin{enumerate}[label=\alph*)]
    \item Sono nella stessa subnet (50 e 100 in 192.168.5.0-63)
    \item Sono in subnet diverse (50 in .0-.63, 100 in .64-.127)
    \item Non possono mai comunicare tra loro
    \item Hanno entrambi indirizzi di rete invalidi
\end{enumerate}

\item Il primo indirizzo utilizzabile in una rete 10.0.0.0/8 è:
\begin{enumerate}[label=\alph*)]
    \item 10.0.0.0 (indirizzo di rete, non utilizzabile)
    \item 10.0.0.1 (primo host valido)
    \item 10.0.0.2 (secondo host)
    \item 10.1.0.0 (confondendo con ottetti)
\end{enumerate}

\end{enumerate}

\section{Protocolli Applicativi e Servizi}

\begin{enumerate}[resume]

\item Il protocollo HTTP (HyperText Transfer Protocol) opera sulla porta TCP:
\begin{enumerate}[label=\alph*)]
    \item 21 (FTP)
    \item 25 (SMTP)
    \item 80 (HTTP standard)
    \item 443 (HTTPS)
\end{enumerate}

\item HTTPS (HTTP Secure) utilizza la porta TCP:
\begin{enumerate}[label=\alph*)]
    \item 80 (HTTP non sicuro)
    \item 8080 (HTTP alternativo)
    \item 22 (SSH)
    \item 443 (HTTPS standard)
\end{enumerate}

\item Il protocollo FTP (File Transfer Protocol) utilizza due porte TCP:
\begin{enumerate}[label=\alph*)]
    \item 20 (dati) e 21 (controllo)
    \item 80 (HTTP) e 443 (HTTPS)
    \item 25 (SMTP) e 110 (POP3)
    \item 53 (DNS) e 67 (DHCP)
\end{enumerate}

\item SMTP (Simple Mail Transfer Protocol) è utilizzato per:
\begin{enumerate}[label=\alph*)]
    \item Ricevere e scaricare email dal server
    \item Trasferire file tra client e server
    \item Inviare email dal client al server o tra server
    \item Navigare su pagine web
\end{enumerate}

\item POP3 e IMAP sono protocolli utilizzati per:
\begin{enumerate}[label=\alph*)]
    \item Inviare email ai destinatari
    \item Ricevere e gestire email dal server
    \item Trasferire file di grandi dimensioni
    \item Configurare automaticamente la rete
\end{enumerate}

\item La porta TCP 22 è utilizzata dal protocollo:
\begin{enumerate}[label=\alph*)]
    \item HTTP (navigazione web)
    \item FTP (trasferimento file)
    \item SMTP (invio email)
    \item SSH (Secure Shell, accesso remoto sicuro)
\end{enumerate}

\item Il DHCP (Dynamic Host Configuration Protocol) serve principalmente a:
\begin{enumerate}[label=\alph*)]
    \item Tradurre nomi di dominio in indirizzi IP
    \item Instradare i pacchetti tra reti diverse
    \item Assegnare automaticamente configurazioni di rete (IP, gateway, DNS, ecc.)
    \item Crittografare il traffico di rete
\end{enumerate}

\item Un server DNS (Domain Name System) traduce:
\begin{enumerate}[label=\alph*)]
    \item Indirizzi IP in indirizzi MAC fisici
    \item Numeri di porta in nomi di servizio
    \item Nomi di dominio in indirizzi IP (e viceversa)
    \item Protocolli in applicazioni specifiche
\end{enumerate}

\item Quale tipo di record DNS associa un nome di dominio a un indirizzo IPv4?
\begin{enumerate}[label=\alph*)]
    \item MX record (mail exchanger)
    \item CNAME record (canonical name, alias)
    \item NS record (name server)
    \item A record (address, IPv4)
\end{enumerate}

\item Il protocollo ICMP (Internet Control Message Protocol) è utilizzato principalmente da:
\begin{enumerate}[label=\alph*)]
    \item Browser web per caricare pagine HTML
    \item Comandi diagnostici come ping e traceroute
    \item Client email per inviare messaggi
    \item Server FTP per trasferire file
\end{enumerate}

\item Un proxy server svolge la funzione di:
\begin{enumerate}[label=\alph*)]
    \item Router di livello 3 per instradamento
    \item Assegnare dinamicamente indirizzi IP
    \item Intermediario tra client e server, con possibilità di caching
    \item Crittografare obbligatoriamente tutto il traffico
\end{enumerate}

\item Il protocollo Telnet:
\begin{enumerate}[label=\alph*)]
    \item È sicuro perché cripta tutto il traffico
    \item Trasferisce file in modo affidabile
    \item Gestisce la posta elettronica in uscita
    \item Permette accesso remoto ma trasmette in chiaro (non sicuro)
\end{enumerate}

\item Una VPN (Virtual Private Network) serve principalmente a:
\begin{enumerate}[label=\alph*)]
    \item Aumentare la velocità di connessione a Internet
    \item Assegnare automaticamente indirizzi IP privati
    \item Creare tunnel sicuri e crittografati attraverso reti pubbliche
    \item Bloccare automaticamente virus e malware
\end{enumerate}

\item Un firewall svolge principalmente la funzione di:
\begin{enumerate}[label=\alph*)]
    \item Aumentare la banda disponibile sulla rete
    \item Assegnare indirizzi IP agli host della rete
    \item Filtrare il traffico di rete secondo regole di sicurezza
    \item Convertire protocolli incompatibili
\end{enumerate}

\item ARP (Address Resolution Protocol) serve a:
\begin{enumerate}[label=\alph*)]
    \item Convertire indirizzi IP in nomi di dominio leggibili
    \item Instradare pacchetti tra sottoreti diverse
    \item Convertire indirizzi IP in indirizzi MAC (livello 2)
    \item Assegnare indirizzi IP in modo dinamico
\end{enumerate}

\end{enumerate}

\newpage

\section*{Riquadro Risposte}

\noindent Riportare le risposte segnando la lettera corrispondente alla risposta scelta:

\vspace{0.5cm}

\begin{center}
\begin{tabular}{|c|c||c|c||c|c||c|c|}
\hline
\textbf{N.} & \textbf{Risp.} & \textbf{N.} & \textbf{Risp.} & \textbf{N.} & \textbf{Risp.} & \textbf{N.} & \textbf{Risp.} \\
\hline
1 & & 19 & & 37 & & 55 & \\
\hline
2 & & 20 & & 38 & & 56 & \\
\hline
3 & & 21 & & 39 & & 57 & \\
\hline
4 & & 22 & & 40 & & 58 & \\
\hline
5 & & 23 & & 41 & & 59 & \\
\hline
6 & & 24 & & 42 & & 60 & \\
\hline
7 & & 25 & & 43 & & 61 & \\
\hline
8 & & 26 & & 44 & & 62 & \\
\hline
9 & & 27 & & 45 & & 63 & \\
\hline
10 & & 28 & & 46 & & 64 & \\
\hline
11 & & 29 & & 47 & & 65 & \\
\hline
12 & & 30 & & 48 & & 66 & \\
\hline
13 & & 31 & & 49 & & 67 & \\
\hline
14 & & 32 & & 50 & & 68 & \\
\hline
15 & & 33 & & 51 & & 69 & \\
\hline
16 & & 34 & & 52 & & 70 & \\
\hline
17 & & 35 & & 53 & & \\
\hline
18 & & 36 & & 54 & & \\
\hline
\end{tabular}
\end{center}

\vspace{1cm}

\noindent\textbf{Punteggio:} \underline{\hspace{3cm}} / 70

\label{LastPage}

\end{document}
